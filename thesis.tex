\documentclass{kththesis}

\usepackage{blindtext} % This is just to get some nonsense text in this template, can be safely removed

\usepackage{csquotes} % Recommended by biblatex
\usepackage{biblatex}
\addbibresource{references.bib} % The file containing our references, in BibTeX format


\title{This is the English title}
\alttitle{Detta är den svenska översättningen av titeln}
\author{Ellen Arvidsson}
\email{magarv@kth.se}
\supervisor{Philipp Haller}
\examiner{Mads Dam}
\programme{Master in Computer Science}
\school{School of Electrical Engineering and Computer Science}
\date{\today}


\begin{document}

% Frontmatter includes the titlepage, abstracts and table-of-contents
\frontmatter

\titlepage

\begin{abstract}
  English abstract goes here.

  \blindtext
\end{abstract}


\begin{otherlanguage}{swedish}
  \begin{abstract}
    Träutensilierna i ett tryckeri äro ingalunda en oviktig faktor,
    för trevnadens, ordningens och ekonomiens upprätthållande, och
    dock är det icke sällan som sorgliga erfarenheter göras på grund
    af det oförstånd med hvilket kaster, formbräden och regaler
    tillverkas och försäljas Kaster som äro dåligt hopkomna och af
    otillräckligt.
  \end{abstract}
\end{otherlanguage}


\tableofcontents


% Mainmatter is where the actual contents of the thesis goes
\mainmatter


\chapter{Introduction}
\label{cha:introduction}

%We use the \emph{biblatex} package to handle our references.  We therefore use the
%command \texttt{parencite} to get a reference in parenthesis, like this
%\parencite{heisenberg2015}.  It is also possible to include the author
%as part of the sentence using \texttt{textcite}, like talking about
%the work of \textcite{einstein2016}.

\chapter{Background}
\label{cha:background}

\section{Language Syntax}
\label{sec:language_syntax}

% Describe the foundations of syntaxes for languages: BNF, ANF

\section{Language Semantics}
\label{sec:language_semantics}

% Describe small step operational semantics

\section{Type Systems}
\label{sec:type_systems}

% Describe basic type systems and properties that should hold e.g. 
% TODO Preservation and Progress


\chapter{Related Work}
\label{cha:related_work}

\section{LVars}
\label{sec:lvars}

% Should describe the basic ideas of LVars and also mention that there are
% problems with the proof and hint of bigger problems.

\section{Reactive Async}
\label{sec:reactive_async}

% Should describe reactive async. There are multiple issues with this system,
% e.g. that great care has to be taken in order to make the system
% deterministic, e.g. make sure that only certain types of operations are
% allowed in the callbacks.

\section{LaCasa}
\label{sec:lacasa}

% Describes the basic ideas of lacasa and the idea of using OCAP constraints to % assure determinism

\section{Spores}
\label{sec:spores}

% Describe the basic ideas of spores: (dis)allow certain types of captures to
% enforce certain properties



\chapter{Challenges of Deterministic Concurrency}
\label{cha:challenges}

% Describe the problem of arbitrary reading data from an object that is
% concurrently being changed. Describe the idea of threshold reads and how it
% can be utilized to get determinism.

\chapter{Core Language}
\label{cha:core_language}

% Introduce the core language with syntax, type system and state properties like
% WT heap, isolation, well typed state

\chapter{Preservation and Progress}
\label{cha:preservation_and_progress}

% Write about the basic ideas behind the proof of preservation and progress

\chapter{Proof of Determinism}
\label{cha:proof_of_determinism}

% Describe:
% Theorem of determinism
% Lemmas leading up to the proof of this

\chapter{Discussion \& Conclusion}
\label{cha:disc_concl}

% Discuss the addition of quiescence in order to actually use the result of the
% computation. Also discuss the efficiency of our system. In order to make it
% usable we need to replace the threshold checks with some callback that spawns
% the required callbacks. This of course is also required to not depend on any
% shared mutable state.
% Discuss threshold reads: Could introduce a get stmt. Draw some
% conclusions of the problematic nature of deterministic concurrency and that
% great care has to be taken when trying to prove determinism for so called
% "deterministic-by-design" systems.
% SHould also

\printbibliography[heading=bibintoc] % Print the bibliography (and make it appear in the table of contents)

\appendix

\chapter{Full Proofs}

\end{document}
