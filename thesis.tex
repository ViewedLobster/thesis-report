\documentclass{kththesis}

\usepackage{csquotes} % Recommended by biblatex
\usepackage{biblatex}
\addbibresource{references.bib} % The file containing our references, in BibTeX format

\usepackage{mathtools, graphicx, hyperref, amssymb, algorithm2e}
\usepackage[utf8]{inputenc}
\usepackage[T1]{fontenc}
\usepackage{listings}
\usepackage{bcprules}
\usepackage{multicol}
\usepackage{tikz}
\usepackage{mathrsfs}
\usepackage{amsthm}
\usepackage{scrextend}
\usepackage{stmaryrd}
\usepackage{wasysym}
\usepackage{float}
\usepackage[labelfont=bf]{caption}
\usepackage{subcaption}
\usepackage{enumitem}

\usepackage{inconsolata}

\setlist[description]{listparindent=\parindent}

% tikz packages
\usetikzlibrary{arrows.meta,decorations.pathreplacing}

% Configuration of font and stuff for listings
\usepackage{color}
\newcommand{\codestyle}{\small\ttfamily}
\newcommand{\nstyle}{\small\ttfamily\color{grey}}
\definecolor{grey}{rgb}{0.5,0.5,0.5}
\lstset{
%  frame=tb,
  language=scala,
%  aboveskip=3mm,
%  belowskip=3mm,
%  lineskip=-0.1em,
  showstringspaces=false,
  columns=[c]fixed,
  basewidth={0.5em,0.4em},
  mathescape=true,
  numbers=left,
  numberstyle=\nstyle,
  basicstyle=\codestyle
}  

% Other listings-related settings from
% http://lampsvn.epfl.ch/trac/scala/export/26099/scala-tool-support/trunk/src/latex/scaladoc.sty
% activate the language and predefine settings
%\lstset{
%    language=Scala,%
%    xleftmargin=4mm,%
%    aboveskip=3mm,%
%    belowskip=3mm,%
%    fontadjust=true,%
%    columns=[c]fixed,%
%    keepspaces=true,%
%    basewidth={0.58em, 0.53em},%
%    tabsize=2,%
%    basicstyle=\renewcommand{\baselinestretch}{0.95}\ttfamily,%
%    commentstyle=\itshape,%
%    keywordstyle=\bfseries,%
%    mathescape=true,%
%    escapechar=¤,%
%    captionpos=b,%
%    framerule=0.3pt,%
%    firstnumber=0,%
%    numbersep=1.5mm,%
%    numberstyle=\tiny,%
%}
%
%\lstdefinestyle{floating}{%
%    xleftmargin=10pt,%
%    xrightmargin=5pt,%
%    aboveskip=4mm,%
%    belowskip=4mm,%
%    fontadjust=true,%
%    columns=[c]flexible,%
%    keepspaces=true,%
%    basewidth={0.5em, 0.425em},%
%    tabsize=2,%
%    basicstyle=\renewcommand{\baselinestretch}{0.95}\ttfamily,%
%    commentstyle=\rm,%
%    keywordstyle=\bfseries,%
%    mathescape=true,%
%    captionpos=b,%
%    framerule=0.3pt,%
%    firstnumber=0,%
%    numbersep=1.5mm,%
%    numberstyle=\tiny,%
%    float=tbp,%
%    frame=tblr,%
%    framesep=5pt,%
%    framexleftmargin=3pt,%
%    abovecaptionskip=\smallskipamount,%
%    belowcaptionskip=\smallskipamount,%
%} % to define: caption, label


% Definitions of commands
%% This file is for definitions of commands

\newcommand{\image}{\text{Im }}
\newcommand{\reals}{\mathbb{R}}
\newcommand{\integers}{\mathbb{Z}}
\newcommand{\ratio}{\mathbb{Q}}
\newcommand{\ilc}[1]{\lstinline[keepspaces=true]$#1$}
\newcommand{\ilclang}[2]{\lstinline[keepspaces=true,language=#1]$#2$}
\newcommand{\lacasa}{\textsc{LaCasa}}
\newcommand{\CLCone}{\textsc{CLC}$^1$}
\newcommand{\scrule}[3]{\infrule[\textsc{#1}]{#2}{#3}}
\newcommand{\scax}[2]{\infax[\textsc{#1}]{#2}}
\newcommand{\stof}{<:}
\newcommand{\lub}{\sqcup}

\newcommand{\LatType}{\mbox{$\mathcal{L}$}}
\newcommand{\AnyRefType}{\mbox{$\mathsf{AnyRef}$}}
\newcommand{\CellType}{\mbox{$\mathsf{Cell}$}}
\newcommand{\NullType}{\mbox{$\mathsf{Null}$}}

\newcommand{\LatVals}{\mbox{$\mathscr{L}$}}

% Commands for writing language syntax

\newcommand{\ClassDef}[4]{\mbox{\texttt{class}~$#1$~\texttt{extends}~$#2$~\texttt{\{}$#3$~$#4$\texttt{\}}}}
\newcommand{\VarDecl}[2]{\mbox{\texttt{var}~$#1$~\texttt{:}~$#2$}}
\newcommand{\MethodDef}[5]{\mbox{\texttt{def}~$#1$\texttt{(}$#2$~\texttt{:}~$#3$\texttt{)}~\texttt{:}~$#4$~\texttt{=}~$#5$}}
\newcommand{\Let}[3]{\mbox{\texttt{let}~$#1$~\texttt{=}~$#2$~\texttt{in}~$#3$}}
\newcommand{\NullVal}{\mbox{\texttt{null}}}
\newcommand{\FSel}[2]{\mbox{$#1$\texttt{.}$#2$}}
\newcommand{\FAss}[3]{\mbox{$#1$\texttt{.}$#2$~\texttt{=}~$#3$}}
\newcommand{\New}[1]{\mbox{\texttt{new}~$#1$}}
\newcommand{\NewCell}{\mbox{\texttt{new}~\texttt{Cell}}}
\newcommand{\Call}[3]{\mbox{$#1$\texttt{.}$#2$\texttt{(}$#3$\texttt{)}}}
\newcommand{\Put}[2]{\mbox{$#1$~\texttt{put}~$#2$}}
\newcommand{\When}[3]{\mbox{\texttt{when}~$#1$~\texttt{pass}~$#2$~\texttt{then}~$#3$}}
\newcommand{\Capt}[2]{\mbox{$#1$~\texttt{=}~$#2$}}
\newcommand{\CB}[3]{\mbox{($#1$, $#2$~$\Rightarrow$~$#3$)}}
\newcommand{\This}{\mbox{\texttt{this}}}


% Typing relation
\newcommand{\TypeRel}[4]{\mbox{$#1;#2\vdash#3:#4$}}

\newcommand{\RuleSpace}{\vspace{0.8em}}


\newcommand{\nocap}{\mbox{$\epsilon$}}
\newcommand{\ocap}{\mbox{$ocap$}}

\newcommand{\LaCasa}{\mbox{LaCasa}}
\newcommand{\RACL}{\mbox{RACL}}

% Reduction relations
\newcommand{\FRedTo}{\mbox{$\rightarrow$}}
\newcommand{\FSRedTo}{\mbox{~$\twoheadrightarrow$~}}
\newcommand{\ProcsRedTo}{\mbox{~$\Rrightarrow$~}}

\newcommand{\Frame}[3]{\mbox{$\langle #1, #2 \rangle^{#3}$}}
\newcommand{\sFrame}[2]{\Frame{#1}{#2}{s}}
\newcommand{\sframe}[1]{\mbox{$\langle #1 \rangle^s$}}
\newcommand{\xframe}[1]{\mbox{$\langle #1 \rangle$}}

%\newcommand{\Obj}[2]{\mbox{$\langle #1, #2 \rangle$}}
\newcommand{\Obj}[1]{\mbox{$\langle #1 \rangle$}}
\newcommand{\Cell}[1]{\mbox{$\langle \CellType{}, #1 \rangle$}}
\newcommand{\Error}{\mbox{{\bf error}}}

\newcommand{\typeOf}{\mbox{$\mathrm{typeOf}$}}
\newcommand{\dom}{\mbox{$\mathrm{dom}$}}
\newcommand{\fields}{\mbox{$\mathrm{fields}$}}
\newcommand{\ftype}{\mbox{$\mathrm{ftype}$}}
\newcommand{\ErrType}{\mbox{$T_{\text{err}}$}}

\newcommand{\isolated}{\mbox{$\mathrm{isolated}$}}
\newcommand{\accRoot}{\mbox{$\mathrm{accRoot}$}}
\newcommand{\accRoots}{\mbox{$\mathrm{accRoots}$}}
\newcommand{\csep}{\mbox{$\mathrm{csep}$}}
\newcommand{\isolation}{\mbox{$\mathrm{isolation}$}}
\newcommand{\reach}{\mbox{$\mathrm{reach}$}}
\newcommand{\ocrloc}{\mbox{$\mathrm{ocr}$}}
\newcommand{\ocr}{\mbox{\bf ocr}}
\newcommand{\gsep}{\mbox{\bf gsep}}
\newcommand{\stateok}{\mbox{\bf ok}}
\newcommand{\tsep}{\:\:}
\newcommand{\reachable}{\mbox{$\mathrm{reachable}$}}

\newcommand{\noSpawn}{\mbox{$\mathrm{noSpawn}$}}


\newcommand{\Graph}{\ensuremath{\mathscr{G}}}
\newcommand{\FieldNames}{\ensuremath{\mathscr{F}}}
\newcommand{\Values}{\ensuremath{\mathscr{V}}}


% Declarations of theorems and alike
\theoremstyle{definition}
\newtheorem{definition}{Definition}[chapter]

\theoremstyle{theorem}
\newtheorem{theorem}{Theorem}
\newtheorem*{theorem*}{Theorem}
\newtheorem{proposition}[definition]{Proposition}%[chapter]
\newtheorem*{proposition*}{Proposition}
\newtheorem{lemma}[definition]{Lemma}
\newtheorem{corollary}[definition]{Corollary}
\newtheorem*{claim}{Claim}

\theoremstyle{remark}
\newtheorem*{remark}{Remark}
\newtheorem*{note}{Note}
\newtheorem*{notation}{Notation}



\title{Concurrent Determinism Using Lattices and the Object Capability Model}
\alttitle{Determinism i parallelliserade program med hjälp av gitterstrukturer och objektsförmågor}
\author{Ellen Arvidsson}
\email{magarv@kth.se}
\supervisor{Philipp Haller}
\examiner{Mads Dam}
\programme{Master in Computer Science}
\school{School of Electrical Engineering and Computer Science}
\date{\today}

\widowpenalty10000
\clubpenalty10000

\begin{document}

% Frontmatter includes the titlepage, abstracts and table-of-contents
\frontmatter

\titlepage

\begin{abstract}
  Parallelization is an important part of modern data systems. However, the
  non-determinism of thread scheduling introduces the difficult problem of
  considering all different execution orders when constructing an algorithm.
  Therefore deterministic-by-design concurrent systems are attractive.  A new
  approach called LVars consists of using data which is part of a lattice, with
  a predefined join operation. Updates to shared data are carried out using the
  join operation and thus the updates commute. Together with limiting the reads
  of shared data, this guarantees a deterministic result. The Reactive Async
  framework follows a similar approach but has several aspects which can cause a
  non-deterministic result. The goal with this thesis is to explore how we can
  ammend Reactive Async in order to guarantee a deterministic result. First an
  exploration into the subtleties of lattice based data combined with
  concurrency is made.  Then a formal model based on a simple object-oriented
  language is constructed.  The constructed small-step operational semantics and
  type system are shown to guarantee a form of determinism. This shows that
  LVars-similar system can be implemented in an object-oriented setting.
  Furthermore the work can act as a basis for future revisions of Reactive Async
  and similar frameworks.
\end{abstract}


\begin{otherlanguage}{swedish}
  \begin{abstract}
    Parallellisering är en viktig del i moderna datasystem. Flertrådade
    applikationer innebär dock en svårighet i och med att programmerare måste ta
    alla exekveringsordningar i beaktning. Därför är beräkningsmodeller vars
    resultat är garanterat deterministiskt en attraktiv utväg. En ny modell,
    kallad LVars, använder gitterstrukturer tillsammans med en
    supremum-operation för att garantera att uppdateringar av delad data
    kommuterar. Detta tillsammans med begränsningar av läsning av datan
    garanterar ett deterministiskt resultat. Reactive Async är ett
    programmeringsramverk som följer en liknande strategi. Det finns dock flera
    delar i dess konstruktion som i en oförsiktig programmerares händer kan
    orsaka att ett programs resultat blir icke-deterministiskt. Målet med detta
    examensarbete är att utforska vilka modifikationer som skulle kunna göras av
    Reactive Async för att garantera determinism. Först görs en undersökning av de mer
    svårförståeliga delarna i kombinationen av gitterbaserad data med
    flertrådad exekvering. Sedan kostrueras en formell beräkningsmodell baserad på
    ett enkelt objekt-orienterat språk. Konstruktionens småstegade operationella
    semantik tillsammans med dess typsystem visas kunna garantera en form av
    determinism. Detta visar att ett system liknande LVars kan implementeras i
    ett objekt-orienterat språk. Därmed skulle detta arbete kunna ligga till
    grund för framtida versioner av Reactive Async.
  \end{abstract}
\end{otherlanguage}


\tableofcontents


% Mainmatter is where the actual contents of the thesis goes
\mainmatter


%We use the \emph{biblatex} package to handle our references.  We therefore use the
%command \texttt{parencite} to get a reference in parenthesis, like this
%\parencite{heisenberg2015}.  It is also possible to include the author
%as part of the sentence using \texttt{textcite}, like talking about
%the work of \textcite{einstein2016}.

\chapter{Introduction \& Motivation}
\label{cha:introduction}

The main approach to achieving concurrency in general applications has long been
to use locks and similar constructs in order to locally synchronize data
accesses. The major drawback with this is that it demands a lot of effort from
the programmer in order to achieve good parallelization, even of simple tasks.
Furthermore the likelihood of concurrency related errors such as deadlocks, data
races and livelocks naturally becomes higher. 

Concurrent deterministic programs are concurrent programs which always produce
the same results for the same inputs. This is attractive since it provides
reproducible computations.  Deterministic-by-design concurrent programming
models are systems which guarantees this property at compile time and thus
significantly simplifies concurrent programming. This greatly eases effort
from the programmer. 

A novel deterministic-by-design model call LVish ~\parencite{kuper2013lvars,
kuper2014freeze} has been introduced.  It uses data types that fulfill the
condition of being part of a lattice, or more explicitly, having some partial
order defined on its values. By defining a join operation on the values and only
allowing writes to the data in form of this operation, you can achieve concurrent
determinism with very high parallelism using event based computation.  This is
also the basic approach of the Reactive Async
project~\parencite{conf/scala/HallerGES16}, which has shown potential to speed up
e.g.\ static analysis of source code. The main problem with Reactive Async as
of now, is that it does not include any construct to ensure that data races
between event handlers do not occur. Futhermore, there are also more fundamental
problems with callback spawning, which could also lead to non-determinism.

\section{Research Question \& Goal}%
\label{sec:goal}


This thesis aims to construct a formal model which could later be adopted by
future revisions of Reactive Async. Constructing a model similar to that
of LVish but in an object-oriented setting, the final goal
will be to mathematically prove a form of determinism for the system.

Doing this will hopefully give us an answer to the research question:

\begin{quotation}
  How can we enforce a deterministic concurrent model similar to LVish in an
  object-oriented setting?
\end{quotation}

\section{Methodology}%
\label{sec:methodology}

We use the theory of programming language semantics and type systems to build a
simple object-oriented language. The type system incorporates theory that
relates to the object capability model from computer security, also utilized in
previous work by \textcite{conf/oopsla/HallerL16}.  After defining the property
of a well-typed program state, we prove that this is preserved during execution.
Furthermore, we prove that an execution of a well-typed program will not get
stuck unless there is a null-pointer exception. Finally, we prove a form of
determinism for the system.

\section{Contributions}%
\label{sec:contributions}

With this work, the following technical contributions are made:
\begin{itemize}
  \item Identification and exemplification of  problems with previous work on
    deterministic concurrency.
  \item Formalization of a core object-oriented language for deterministic
    concurrency.
  \item Design and formalization of an accompanying type system.
  \item A proof of soundness for the core language and type system.
  \item A proof of quasi-determinism, a form of determinism, for our
    formalized system.
\end{itemize}

\section{Sustainability \& Ethical Aspects}%
\label{sec:sustainability_&_ethical_aspects}

While this work migh have consequences with regards to sustainability, these are
hard to overlook due to the theoretical nature of the work. The same applies to
ethical aspects. Therefore the discussion of these are not really applicable.

\section{Report Structure}%
\label{sec:report_structure}

In Chapter~\ref{cha:background} some technical background is described. This
includes some mathematical definitions, an introduction to programming language
formalization and an informal description of the object capability
model. Chapter~\ref{cha:related_work} describes some selected related work, upon
which our model will build. In Chapter~\ref{cha:challenges} we describe and
exemplify some problems with some existing deterministic-by-design concurrent
systems. In Chapter~\ref{cha:core_language} we introduce our core language and
formalization. Chapter~\ref{cha:properties_of_racl} describes the theorems and
proofs of soundness and quasi-determinism. Finally, in
Chapter~\ref{cha:discussion_and_conclusion} we discuss extensions and
implementation of our formal system, as well as conclude the report.





\chapter{Background} \label{cha:background}

This chapter is a brief introduction to language and type system theory. In
section~\ref{sec:language_syntax} some basics of language syntax will be
explained. Section~\ref{sec:language_semantics} gives a brief introduction to
programming language semantics. Finally section~\ref{sec:type_systems} explains
some basics of programming language type systems.

% TODO give a reference
As a basis the well known While language will be used as an example language.

% TODO add a section about basic math, like lattices join operations, partial
% maps

\section{Language Syntax} \label{sec:language_syntax}

% Describe the foundations of syntaxes for languages: BNF, ANF
A language syntax is a formal way to describe the structure of programs written
in a programming language. It gives us a basis on which to build the language
semantics and type systems as will be clear later. In this section the
Backus-Naur-form (BNF) will be introduced. 

\subsection{The Backus-Naur-form} \label{sub:the_backus_naur_form} 

Backus-Naur-form (BNF) is a very common way to describe the syntax of
programming languages. BNF describes a syntax using rules of the form
\begin{equation*}
  s ::= \text{<expression>}.
\end{equation*}
The the left side symbol $s$ is refered to as a non-terminal. <expression> could
also be a list of the
possible forms of $s$, separated by $|$, i.e. 
\begin{equation*}
  s ::= \text{<expression1>} \: | \dots | \: \text{<expressionN>}
\end{equation*}
The expressions can themselves contain non-terminals and terminals.
Terminals are symbols which does not occur on the left side of any rule.

% TODO add reference to Semantics with Applications An Appetizer
The While language can be written in BNF as described in~\cite{} % FIXME reference
For reference we give a similar description here with the addition of a very
simple type system.

% TODO add reference to Semantics with Applications An Appetizer
Furthermore we let $\WhVar$ be the set of all variable names and $\WhVal$ the
set of all values which for the While language will be the integers $\integers$,
and the boolean values $\WhTrue$ and $\WhFalse$. 
\begin{equation*}
  \WhVal = \integers \cup \left\{ \WhTrue, \WhFalse \right\}
\end{equation*}
Note that in~\cite{} % FIXME reference
the conversion between the syntactical (numerals\slash boolean literals) and semantic
(integers\slash boolean values) versions of values are explicit whereas here we will
make this conversion implicit.

\begin{figure}[h]
  \centering
  $\begin{array}{lll@{\hspace{4mm}}l}
    c &::= &  & \mbox{Program code} \\
    &  & \varepsilon & \mbox{Empty code} \\
    &| & s; c' & \mbox{Statement concatenation} \\
    s &::= & & \mbox{Program statement} \\
    &  & \WhDecl{x}{\tau} & \mbox{Variable declaration} \\
    &| & \WhAssign{x}{e} & \mbox{Assignment} \\
    &| & \WhSkip & \mbox{Skip} \\
    &| & \WhIf{e}{c_1}{c_2} & \mbox{If branch} \\
    &| & \WhWhile{e}{c} & \mbox{While loop} \\
    &&&\\
    e & ::= & & \mbox{Expressions} \\
    & & x & \mbox{Variable} \\
    &| & v & \mbox{Value literal} \\
    &| & \WhAdd{e_1}{e_2} & \mbox{Addition} \\
    &| & \WhTimes{e_1}{e_2} & \mbox{Multiplication} \\
    &| & \WhMinus{e_1}{e_2} & \mbox{Subtraction} \\
    &| & \WhEq{e_1}{e_2} & \mbox{Equality comparison} \\
    &| & \WhLeq{e_1}{e_2} & \mbox{$\leq$ comparison} \\
    &| & \WhLnot{e} & \mbox{Logical not} \\
    &| & \WhLand{e_1}{e_2} & \mbox{Logical and} \\
    &&&\\
    \tau & ::= & & \mbox{Types} \\
    & & \WhBool & \mbox{Boolean type} \\
    &| & \WhInt & \mbox{Integer type} \\
    &&&\\
    v & ::= & & \mbox{Values} \\
    &  & n & \mbox{Integer} \\
    &| & \WhTrue & \mbox{Boolean true} \\
    &| & \WhFalse & \mbox{Boolean false} \\
  \end{array}$
  \caption{Grammar of While}
  \label{fig:while_grammar}
\end{figure}

\section{Small-step Operational Semantics} \label{sec:language_semantics}

% Describe small step operational semantics and give example with while lang
In order to prove properties of a program you must be able to formally describe
the "meaning" of the program, i.e. to describe how programs statements modify
state and what execution steps can be taken. A common approach to this is
small-step operational semantics, also called structural operational semantics
(SOS). Other approaches includes big-step operational semantics or natural
semantics, and denotational semantics. Since SOS models concurrency well this is
the only one described here.

Small-step operational semantics defines rules which can be used to derive
single execution steps (or transitions) between program states. We exemplify
this with the While language defined in the above section. In order to define
rules for state transitions a state must be defined. Therefore we make the
following definition for the While language.

\begin{definition}
  A \emph{state} $S$ for the While language is a value of the form
  \begin{equation*}
    \WhState{V, c}  
  \end{equation*}
  where $c$ is a statement and $V$ is a partial map
  \begin{equation*}
    V: \WhVar \rightharpoonup \WhVal
  \end{equation*}
\end{definition}

We can now define rules for state transitions (see Figure~\ref{fig:while_sos}).
All rules are of the form
\infrule[Rule name]
{\text{precondition}}
{S \rightarrow S'}
Which basically says "if precondition holds then we can step from $S$
to $S'$".  If there are no preconditions we just write
\infax[Rule name]
{S \rightarrow S'}

\begin{figure}[h]
  \centering
  \infax[WhE-Decl1]
  {\WhState{V, \WhDecl{x}{\WhInt};c} \: \rightarrow \: \WhState{V[ x \mapsto 0],
  c}}
  
  \RuleSpace

  \infax[WhE-Decl2]
  {\WhState{V, \WhDecl{x}{\WhBool};c} \: \rightarrow \: \WhState{V[ x \mapsto
  \WhFalse], c}}
  
  \RuleSpace

  \infax[WhE-Assign]
  {\WhState{V, \WhAssign{x}{e};c} \: \rightarrow \: \WhState{V[ x \mapsto
  \WhEval{V}(e)], c}}

  \RuleSpace

  \infax[WhE-Skip]
  {\WhState{V, \WhSkip; c} \: \rightarrow \: \WhState{V, c}}

  \RuleSpace

  \infrule[WhE-IfTrue]
  {\WhEval{V}(e) = \WhTrue}
  {\WhState{V, \WhIf{e}{c_1}{c_2};c} \: \rightarrow \: \WhState{V, c_1;c}}

  \RuleSpace 

  \infrule[WhE-IfFalse]
  {\WhEval{V}(e) = \WhTrue}
  {\WhState{V, \WhIf{e}{c_1}{c_2}} \: \rightarrow \: \WhState{V, c_2;c}}
  
  \RuleSpace

  \infax[WhE-While]
  {\WhState{V, \WhWhile{e}{c'};c} \: \rightarrow \: \WhState{V,
  \WhIf{e}{(\WhConc{c'}{\WhWhile{e}{c'}})}{\WhSkip};c}}

  \caption{Small-step operational semantics for While}
  \label{fig:while_sos}
\end{figure}

Most of the rules are very intuitive. E.g. rules {\sc WhE-Decl1} and {\sc
WhE-Decl2} extends the state map $V$ with $x$ and a default value corresponding
to the declared type. Rule {\sc WhE-Assign} evaluates the expression $e$ and
maps $x$ to the result. Rules {\sc WhE-IfTrue} and {\sc WhE-IfFalse} evaluates
its expression $e$ and if the result is $\WhTrue$ or $\WhFalse$ they prepend the
code of the corresponding branch to the rest of the program. We can also see
that {\sc WhE-While} is just an expansion of a while loop into an if statement.
Finally we note that we can not step from $\WhState{V, \varepsilon}$ which means
that this is a halting state.

The SOS rules of While heavily relies on the expression evaluation function
$\WhEval{\cdot}(\cdot)$. This is a function which takes a state map $V$ and an
expression $e$ and returns a value in $\WhVal$. We define this recursively as follows
\begin{equation*}
  \begin{array}{lll}
    \WhEval{V}(v) &=& v \\
    \WhEval{V}(x) &=& V(x) \\
    \WhEval{V}(e_1 + e_2) &=& \WhEval{V}(e_1) + \WhEval{V}(e_2) \\
    \WhEval{V}(e_1 * e_2) &=& \WhEval{V}(e_1) * \WhEval{V}(e_2) \\
    \WhEval{V}(e_1 - e_2) &=& \WhEval{V}(e_1) - \WhEval{V}(e_2) \\
    \WhEval{V}(e_1 = e_2) &=& \WhEval{V}(e_1) = \WhEval{V}(e_2) \\
    \WhEval{V}(e_1 \leq e_2) &=& \WhEval{V}(e_1) \leq \WhEval{V}(e_2) \\
    \WhEval{V}(e_1 \land e_2) &=& \WhEval{V}(e_1) \land \WhEval{V}(e_2) \\
    \WhEval{V}(\lnot e_1) &=& \lnot \WhEval{V}(e_1) \\
  \end{array}
\end{equation*}
Here we implicitly rely on that the operations $+, -, *, =, \leq$ are only
defined for integer values, and that $\lnot, \land$ are only defined for boolean
values.  If we encounter values where an operation is undefined the value for
$\WhEval{V}(e)$ is undefined for the corresponding state map $V$ and expression
$e$. Since the definition is recursive, the occurrence of an undefined value
should also propagate upwards, e.g. if $\WhEval{V}(e_1)$ is undefined
then $\WhEval{V}(e_1 + e_2)$ is undefined aswell. This is implicit in our
definition.

The evaluation function $\WhEval{\cdot}(\cdot)$ together with the derivation
rules in Figure~\ref{fig:while_sos} gives us a complete description of what
steps are allowed. It is implicit in the description that if the evaluation
function is undefined at some point in the derivation of a step, execution
cannot progress and we get stuck.

\section{Type Systems} \label{sec:type_systems}

A type system is a mathematical construct that consists of elements called types
and a set of rules that assign types to parts of a programming language, e.g.
statements, variables and expressions. Type systems are most commonly used to
prevent programming errors such as feeding a data structure to a function for
which it was not made, and this can be enforced both using static checking at
compile time or dynamic checking at runtime. Type systems are found in many
modern programming languages such as Java, Scala, Haskell, Python and C++.

The focus here will be on static type systems. We will explain the basics using
our previous example of the While language. Then some uses and extensions will be
explained.

\subsection{A Type System for While}
\label{sub:a_type_system_for_while}

For the While language we have the following types:
\begin{equation*}
  \WhInt \andalso \WhBool \andalso \WhVoid
\end{equation*}
We call the set of all types 
\begin{equation*}
  \WhTpe = \left\{ \WhInt, \WhBool, \WhVoid \right\}.
\end{equation*}
Our typing rules will define a relation of the form
\begin{equation} \label{eq:tpe_sys1}
  \Gamma \vdash r : \tau.
\end{equation}
Here $\Gamma$ is a \emph{typing environment}, i.e. a partial map
\begin{equation*}
  \Gamma: \WhVar \rightharpoonup \WhTpe,
\end{equation*}
$r$ is either code or an expression and $\tau \in \WhTpe$.
Equation \eqref{eq:tpe_sys1} basically reads "$r$ is typed as $\tau$ under the
typing environment $\Gamma$." A program $c$ is typeable if
\begin{equation*}
  \emptyset \vdash c : \WhVoid
\end{equation*}
where $\emptyset$ is the empty typing environment, i.e. $\emptyset(x)$ is
undefined for all $x \in \WhVar$.

\begin{notation}
  Sometimes you would like to remap a key of, or extend a partial map $M$. One notation
  for this commonly used is 
  \begin{equation*}
    M[k \mapsto v],
  \end{equation*}
  which means that $M(l) = M[k \mapsto v](l)$ for all $l \neq k$ and
  $M(k) = v$.
  For typing environments we will also use the notation
  \begin{equation*}
    \Gamma, x: \tau
  \end{equation*}
  which is equivalent to $\Gamma[x \mapsto \tau]$.
\end{notation}

\begin{figure}[h]
  \infax[WhT-Empty]
  {\Gamma \vdash \varepsilon : \WhVoid}

  \RuleSpace

  \infrule[WhT-Decl]
  {\Gamma(x) \text{ undefined} \andalso \Gamma, x: \tau \vdash c : \WhVoid}
  {\Gamma \vdash \WhDecl{x}{\tau}; c : \WhVoid}

  \RuleSpace 

  \infrule[WhT-Assign]
  {(x: \tau) \in \Gamma \andalso \Gamma \vdash e: \tau \andalso \Gamma \vdash c
  : \WhVoid}
  {\Gamma \vdash \WhAssign{x}{e}; c : \WhVoid}

  \RuleSpace

  \infrule[WhT-Skip]
  {\Gamma \vdash c : \WhVoid}
  {\Gamma \vdash \WhSkip; c : \WhVoid}

  \RuleSpace

  \infrule[WhT-If]
  {\Gamma \vdash e : \WhBool \andalso \Gamma \vdash c_1: \WhVoid \\
  \Gamma \vdash c_2 : \WhVoid \andalso \Gamma \vdash c: \WhVoid}
  {\Gamma \vdash \WhIf{e}{c_1}{c_2}; c : \WhVoid}

  \RuleSpace

  \infrule[WhT-While]
  {\Gamma \vdash e: \WhBool \andalso \Gamma \vdash c' : \WhVoid \andalso \Gamma
  \vdash c : \WhVoid}
  {\Gamma \vdash \WhWhile{e}{c'}; c : \WhVoid}

  \caption{While program code typing rules.}
  \label{fig:while_code_tpe}
\end{figure}

Typing rules for program code are defined in Figure~\ref{fig:while_code_tpe}.
Generally While program code can only be typed as $\WhVoid$. Intuitively this is
because the execution of program code does not have a resulting value. Instead
it can only modify the state map $V$. The rules for typing program code should
not be hard to understand. E.g. {\sc WhT-Assign} states that for code that
starts with an assignment the expression in the assignment must be of the same
type as the variable. {\sc WhT-If} requires that the expression is typeable as
$\WhBool$ and that both branches should be typeable as $\WhVoid$.

\begin{figure}[h]
  \begin{multicols}{3}
  \infax[WhT-Num]
  {\Gamma \vdash n : \WhInt}

  \infax[WhT-True]
  {\Gamma \vdash \WhTrue: \WhBool}

  \infax[WhT-False]
  {\Gamma \vdash \WhFalse: \WhBool}
  \end{multicols}

  \RuleSpace

  \infrule[WhT-Var]
  {(x: \tau) \in \Gamma}
  {\Gamma \vdash x: \tau}

  \RuleSpace

  \infrule[WhT-Add]
  {\Gamma \vdash e_1: \WhInt \andalso \Gamma \vdash e_2: \WhInt}
  {\Gamma \vdash e_1 + e_2: \WhInt}

  \RuleSpace

  \infrule[WhT-Times]
  {\Gamma \vdash e_1: \WhInt \andalso \Gamma \vdash e_2: \WhInt}
  {\Gamma \vdash e_1 * e_2: \WhInt}

  \RuleSpace

  \infrule[WhT-Minus]
  {\Gamma \vdash e_1: \WhInt \andalso \Gamma \vdash e_2: \WhInt}
  {\Gamma \vdash e_1 - e_2: \WhInt}

  \RuleSpace 

  \infrule[WhT-Eq]
  {\Gamma \vdash e_1: \WhInt \andalso \Gamma \vdash e_2: \WhInt}
  {\Gamma \vdash e_1 = e_2: \WhBool}

  \RuleSpace 

  \infrule[WhT-Leq]
  {\Gamma \vdash e_1: \WhInt \andalso \Gamma \vdash e_2: \WhInt}
  {\Gamma \vdash e_1 \leq e_2: \WhBool}

  \RuleSpace

  \infrule[WhT-Not]
  {\Gamma \vdash e : \WhBool}
  {\Gamma \vdash \lnot e: \WhBool}

  \RuleSpace 

  \infrule[WhT-And]
  {\Gamma \vdash e_1: \WhBool \andalso \Gamma \vdash e_2: \WhBool}
  {\Gamma \vdash e_1 \land e_2: \WhBool}

  \caption{While expression typing rules.}
  \label{fig:while_expr_tpe}
\end{figure}

Finally the rules for typing expressions are found in
Figure~\ref{fig:while_expr_tpe}. The possible types for an expression are
$\WhInt$ and $\WhBool$. The rules themselves should not be difficult to
understand.


\subsection{Uses \& Extensions}
\label{sub:uses_and_extensions}

So what are type systems used for? As mentioned earlier, commonly they are put
in place to prevent errors such as uncompatible data structures being fed as arguments
to a function which cannot handle such properly. E.g. in Java you cannot feed a
List to a method which is denoted to take an Integer as an argument. This will
result in a compiler error.

\subsection{Preservation \& Progress}
\label{sub:preservation_&_progress}

A formal type system can furthermore be used to prove properties like
\emph{preservation} and \emph{progress} for programs which are properly type
checked. These are important properties since they can beforehand ensure that
programs terminate properly, or at least that e.g. if it terminates erroneously it
must have been the result of a null-pointer exception.

As an example we can state preservation and progress properties of While as follows. Let
$\Gamma(V)$ be defined as 
\[
  \Gamma(V)(x) = \begin{cases}
    \WhInt & \text{ if } V(x) \in \integers \\
    \WhBool & \text{ if } V(x) \in \left\{\WhTrue, \WhFalse \right\} \\
    \text{undefined} & \text{ otherwise }
  \end{cases}
\]
\begin{proposition}{(Preservation of While)} 
  Let
  \begin{equation*}
    \WhState{V, c} \: \rightarrow \: \WhState{V', c'} \andalso \Gamma(V) \vdash
    c: \WhVoid.
  \end{equation*}
  Then 
  \begin{equation}
    \Gamma(V') \vdash c': \WhVoid
  \end{equation}
\end{proposition}
\begin{proposition}{(Progress of While)}
  Let $\WhState{V, c}$ be a state and let
  \begin{equation*}
    \Gamma(V) \vdash c: \WhVoid.
  \end{equation*}
  Then either $c = \varepsilon$ or there is a state $\WhState{V', c'}$ such that
  \begin{equation*}
    \WhState{V, c} \: \rightarrow \: \WhState{V', c'}.
  \end{equation*}
\end{proposition}


\subsection{Extensions}
\label{sub:extensions}

The example type system for While is of course very simple. This is mostly due
to the simplicity of the language itself, more complicated languages have more
complicated type systems and a type system is oftenmost designed in together 
with the language itself. 

One common construct in type systems is \emph{subtyping}. This introduces a a
(semi)-lattice with based on a partial order relation denoted $\stof$ between
types and it oftenmost is used to say that "if $\sigma \stof \tau$ then values
of type $\sigma$ can be used in the same way as values of type $\tau$". This is
the case for Java where a class $C$ can extend a class $D$. This for simply
stated means that the fields and methods of $D$ are also availible in $C$. We
will see an example of how subtyping can be used in
chapter~\ref{cha:core_language}. 

In chapter~\ref{cha:core_language} the typing relation will also include an
effect $a$. In short this means that the typing rules are more or less strict
depending on the value of $a$.



\chapter{Related Work}\label{cha:related_work}

In this chapter four major pieces of major work is going to be described.
Section~\ref{sec:lvars} describes the LVars system and LVish, the extension
of LVars which introduces the concepts of quiescence and freezing. 
Section~\ref{sec:reactive_async} describes Reactive Async, a programming model
inspired by the extended LVars system. In section~\ref{sec:lacasa} the LaCasa
system is introduced, and finally the concept of spores is briefly introduced in
section~\ref{sec:spores}.

\section{LVars}\label{sec:lvars}

LVars~\parencite{kuper2013lvars} is a programming model that was introduced as a
solution to the problem of guaranteed deterministic concurrent programs. It
generalizes the concept of write-once data
structures~\parencite{nikhil1989structures}, also called IVars, with the ability
to write more than once but limiting update operations to being monotonic. I.e.
the values taken by LVars are part of a programmer specified lattice and all
updates are done through a join operation of the old and new values. This
ensures that writes commute~\parencite{kuper2013lvars}.

\subsection{Stores \& Lattice}%
\label{sub:stores_and_lattice}

At the foundation of the LVars system lays lattices. The values resulting from a
computation is going to be elements from a lattice $\LVarsLat$, specified by the
programmer. These lattice values are stored in a \emph{store}.  This is a set of
pairs consisting of a location and a lattice element. For a location there is
exactly one value. Letting $\LVarsLoc$ be the set of locations, a store $S$ can
be represented using a partial map
\begin{equation*}
  S: \LVarsLoc \rightharpoonup \LVarsLat.
\end{equation*}

\subsection{LVars Operations}%
\label{sub:lvars_operations}

The LVars model supports three main operations
\begin{itemize}
  \item Extending the store with a new location. This takes a fresh location and
    sets its value to $\bot_{\LVarsLat}$, the bottom element of $\LVarsLat$.
  \item Updating a store location, also called a \emph{put} operation. This
    operation takes a store location $r$ and a lattice value $l$. Given a store
    $S$ this updates location $r$ with $l \sqcup S(r)$. To ensure determinism,
    any put that takes a store location to $\top_{\LVarsLat}$ results in an
    error.
  \item A read operation also referred to as \emph{get}. This operation is
    further specified with a threshold set, i.e. a set of lattice values, and a
    store location. The operation blocks until the store location has passed one
    of the values in the threshold set, upon which it returns this value. In
    order to ensure determinism the elements in the threshold set are required
    to be mutually \emph{incompatible}. Two elements $l_1, l_2$ are incompatible if
    \begin{equation*}
      l_1 \sqcup l_2 = \top_{\LVarsLat}
    \end{equation*}
    where $\top$ is the top element of $\LVarsLat$.
\end{itemize}

% TODO brief introduction to lvars:
% Programming model to ensure deterministic concurrent execution
% Quasi determinism (maybe later)
% stores and lattice variables
% put and get operations

\subsection{LVish: Extending LVars}%
\label{sub:lvish_extending_lvars}

% TODO describe lvish, the lvars extension:
% freezing
% quasi determinism
% quiescence
% indicate that there are problems

\begin{equation}
\end{equation}

% Should describe the basic ideas of LVars and also mention that there are
% problems with the proof and hint of bigger problems.

\section{Reactive Async}\label{sec:reactive_async}

% Should describe reactive async. There are multiple issues with this system,
% e.g. that great care has to be taken in order to make the system
% deterministic, e.g. make sure that only certain types of operations are
% allowed in the callbacks.

\section{LaCasa}\label{sec:lacasa}

% Describes the basic ideas of lacasa and the idea of using OCAP constraints to 
% assure determinism

\section{Spores}\label{sec:spores}

% Describe the basic ideas of spores: (dis)allow certain types of captures to
% enforce certain properties





\chapter{Challenges of Deterministic Concurrency}
\label{cha:challenges}

% Describe the problem of arbitrary reading data from an object that is
% concurrently being changed. Describe the idea of threshold reads and how it
% can be utilized to get determinism.

% TODO describe problems inherent with concurrent determinism
% TODO problems inherent in reactive async
% TODO problem inherent with freeze in LVish

% problems of reactive async includes:
% access to shared state is permitted within callbacks
% even then semantics of callback thread spawning is flawed

% Should we do these in the challenges? Prolly yes.
% TODO explain the problem of callback spawning
% TODO explain the problem of shared state

As noted in the introduction, achieving a high level of parallelization and
ensuring determinism is a difficult task. The inherent non-determinism of
concurrent operations leaves the programmer with a hard task, i.e. to consider
all paths of execution and making sure that they all lead to the same result.

Systems like Reactive Async and LVish both have the ambition of moving this
burden off the programmer and onto the programming model and type system. Both
however have problems. In this chapter we describe these problems in an informal
way. As a running example we will use the lattice of integers $(\integers,
\leq)$ with a bottom value of $-\infty$.

\section{Problems of Reactive Async}%
\label{sec:problems_of_reactive_async}

Reactive Async has two main problems. To highlight these we use a similar
example as the one in Figure~\ref{fig:ra_example}. Figure~\ref{fig:ra_example2}
describes a dependency graph which is a little more complex.

\begin{figure}
  \centering
  \begin{tikzpicture}
    \node[circle,draw] (c1) at (0, 1) {$c_1$};
    \node[circle,draw] (c2) at (0,-1) {$c_2$};
    \node[circle,draw] (c3) at (2, 0) {$c_3$};
    \node[circle,draw] (c4) at (4, 0) {$c_4$};
    \draw[-{Latex[length=0.3cm]}] (c1) -- node[above] {$f_1$} (c3);
    \draw[-{Latex[length=0.3cm]}] (c2) -- node[above] {$f_2$} (c3);
    \draw[-{Latex[length=0.3cm]}] (c3) -- node[above] {$f_3$} (c4);
  \end{tikzpicture}
  \caption{Reactive Async problematic example.}
  \label{fig:ra_example2}
\end{figure}

\subsection{Sharing State}%
\label{sub:sharing_state}

The first problem of Reactive Async is that it is not prohibited for callbacks
to share mutable state. To exemplify this let $f_1$ and $f_2$ be as in
Figure~\ref{fig:ra_fun_shared_state}. This is written with Scala function syntax
and could be seen as a simplified version of Reactive Async. Say both are
scheduled to run and that the initial value of \ilc{Global.someInt} is $0$.
Also, let the initial cell value of all cells be $0$.  

Remember that, in Reactive Async the return value of a dependency function will
be used to update the dependent cell with through a put operation. Then it
should be clear that depending on the order in which $f_1$ and $f_2$ executes
$c_3$ will have a different final cell value. This is due to that the callbacks
share mutable state through the \ilc{Global} object.

This problem can be remedied with the help of the OCAP model. By ensuring that
callbacks threads are isolated in the reference graph we ensure that they do not
share any state.

\begin{figure}
  \begin{subfigure}[b]{0.5\textwidth}
    \begin{lstlisting}
(l) => {
  Global.someInt = 1
  0
}
    \end{lstlisting}
    \caption{Code of $f_1$.}
  \end{subfigure}
  ~
  \begin{subfigure}[b]{0.5\textwidth}
    \begin{lstlisting}
(l) => {
  if (Global.someInt == 1)
    1
  else 0
}
    \end{lstlisting}
    \caption{Code of $f_2$.}
  \end{subfigure}
  \caption{Two callback functions sharing state.}
  \label{fig:ra_fun_shared_state}
\end{figure}


\subsection{Callback Spawning Semantics}%
\label{sub:callback_spawning_semantics}

Remedying the above problem does not however, yield a deterministic system.
There is a flaw inherent in how Reactive Async decides to spawn a callback
thread. A thread is only spawned when a cell value actually changes, which means
that all put operations does not necessarily spawn a callback. To see that this
is a problem, let $f_1$, $f_2$ and $f_3$ be as in
Figure~\ref{fig:ra_fun_callback_spawn} and \ref{fig:ra_fun_callback_spawn}. Let
all cells have an initial value of $0$, and let both $f_1$ and $f_2$ be
scheduled to run.

Say $f_1$ runs before $f_2$. The result will be a put of $2$ to cell $c_3$.
Since $2 \sqcup 0 = 2$, the cell value of $c_3$ will be changed to $2$. Thus it
will spawn the callback $f_3(2)$. This callback will result in a put of $1$ to
$c_4$, i.e., $c_4$ will have its cell value updated to $1$. Running $f_2$ now
results in the value of $c_3$ being updated to $3$ which will spawn the callback
$f_3(3)$. The resulting put of $0$ to $c_4$ will not update $c_4$ since $0 \leq
1$. The final value of $c_4$ will in this case be $1$.

Now let $f_2$ execute before $f_3$. This results in a put of $3$ to $c_3$, which
results in $c_3$ being updated to $3$. The callback $f_3(3)$ will be spawned and
result in a put of $0$ to $c_4$. Since $0 \sqcup 0 = 0$ this will not change the
value of $c_4$. Now, if $f_1$ executes, it will result in a put of $2$ to $c_3$.
Since the cell value of $c_3$ is already $3$ and $2 \sqcup 3 = 3$, this will not
result in an update of $c_3$. Thus the callback $f_3(2)$ is never spawned. The
final cell value of $c_4$ for this execution order is $0$, clearly different
from before.
\begin{figure}
  \begin{minipage}{0.5\textwidth}
    \begin{subfigure}[b]{\linewidth}
      \begin{lstlisting}
(l) => 2
      \end{lstlisting}
      \caption{Code of $f_1$.}
    \end{subfigure}
    
    \begin{subfigure}[b]{\linewidth}
      \begin{lstlisting}
(l) => 3
      \end{lstlisting}
      \caption{Code of $f_2$.}
    \end{subfigure}
  \end{minipage}
  ~
  \begin{minipage}{0.5\textwidth}
    \begin{subfigure}[b]{\linewidth}
      \begin{lstlisting}
(l) => {
  if (l == 2)
    1
  else 0
}
      \end{lstlisting}
      \caption{Code of $f_3$.}
    \end{subfigure}
  \end{minipage}
  \caption{Trivial callback functions breaking determinism using callback spawning semantics.}
  \label{fig:ra_fun_callback_spawn}
\end{figure}

The problem of callback spawning can be remedied with the use of the threshold
sets of LVish. Instead of only spawning a callback thead when a cell value
changes, a callback will be spawned for each lattice value passed in the
specified threshold set.

% INTRO
% As noted in the introduction, a 
% there are problems inherent in both reactive async and LVish. 

% TODO describe the reactive async problems first

\section{LVish Freezing Problem}%
\label{sec:a_problem_of_lvish}

As hinted in the end of Section~\ref{sec:lvars}, the proof of quasi determinism
of LVish is flawed. While this leaves the quasi determinism of LVish an open
problem this is not really interesting. This is due to the fact that by adding a
simple if-statement to the language, we can easily construct a counterexample.

The core language of LVish is very minimal with a very implicit way to implement
threads. We will therefore resort to writing the counterexample in a language
more akin to the system of Reactive Async, with the modification of having all
put operations being explicit in code and requiring all callbacks to have an
associated threshold set, similar to LVish. It can easily be translated into a
language and semantics of LVish (with the extension of an if-statement).

Say we have two cells $c_1$ and $c_2$ and two callbacks $f_1$ and $f_2$. Let the
lattice be $(\integers, \leq)$ and let the initial values of both cells be the
bottom element $0$ and both associated freeze bits are $\LVarsFalse$. The
callbacks are registered to some auxilliary cell $c$. 

\begin{table}
  \centering
  \begin{tabular}{c|c|c}
    Callback & Cell & Threshold Set \\
    \hline
    $f_1$ & $c$ & $\left\{ 0 \right\}$ \\
    $f_2$ & $c$ & $\left\{ 0 \right\}$ \\
  \end{tabular}
  \caption{Cell registration and threshold sets for $f_1$ and $f_2$.}
  \label{tab:cellreg}
\end{table}

The callback thread spawning semantics are as follows. If the value of a cell $c$
with a registerered callback $f$ passes $l \in T$, $T$ being the threshold set
assiciated with $f$, the callback $f(l)$ will be spawned eventually. This is
equivalent with the semantics of LVish.

There are two main operations used in our callback definitions. The freeze
statement \ilc{frz(...)} takes one argument of cell type and freezes the cell
specified, i.e. sets its associated freeze bit to true. Furthermore it returns
the associated cell value. The put operation \ilc{put(...)} takes two arguments,
one of cell type and one lattice value. This updates the cell value with the
specified one using the join operation. If the freeze bit is true and the join
result is not the same as the current value, the put operation results in a step
to $\Error$. The return value of \ilc{put} is the cell reference itself. We can
see an example of both in Figure~\ref{fig:ex_cell_op}.

% TODO change update to change above in callback spawning semantics.
\begin{figure}
  \centering
  \begin{subfigure}[t]{0.4\textwidth}
    \begin{lstlisting}[numbers=none,mathescape=true]
x = put($c$, l)
    \end{lstlisting}
    \caption{Example put operation. We put the lattice value assigned to \ilc{l} into
    cell $c$.}
  \end{subfigure}
  \quad
  \begin{subfigure}[t]{0.4\textwidth}
    \begin{lstlisting}[numbers=none,mathescape=true]
x = frz($c$)
    \end{lstlisting}
    \caption{Example freeze operation. We freeze the cell $c$ and assigns its cell
    value in variable \ilc{x}}
  \end{subfigure}
  \caption{Example cell operations.}
  \label{fig:ex_cell_op}
\end{figure}

Now let $f_1$ and $f_2$ be as in Figure~\ref{fig:lvish_fun_breaking}.
Furthermore let both $f_1$ and $f_2$ be scheduled (due to some put of integer
value greater than $0$ to auxilliary cell $c$).

If $f_1$ executes before $f_2$ we end up with an execution as described by
Table~\ref{stab:f_1exec}. It is clear that we never step to $\Error$.
If $f_2$ executes before $f_1$ we can describe the execution with
Table~\ref{stab:f_2exec}. Clearly the results differ and are non-erroneous, a
non-quasi-deterministic result.

\begin{figure}
  \centering
  \begin{subfigure}[t]{0.4\textwidth}
    \begin{lstlisting}[mathescape=true]
(l) => {
  x = frz($c_1$)
  if ( x == 0 )
    put($c_2$, 1)
}
    \end{lstlisting}
    \caption{Callback code for $f_1$.}
  \end{subfigure}
  \quad
  \begin{subfigure}[t]{0.4\textwidth}
    \begin{lstlisting}[mathescape=true]
(l) => {
  x = frz($c_2$)
  if ( x == 0 )
    put($c_1$, 1)
}
    \end{lstlisting}
    \caption{Callback code for $f_2$.}
  \end{subfigure}
  \caption{Callback code to break determinism in LVish.}
  \label{fig:lvish_fun_breaking}
\end{figure}

\begin{table}
  \centering
  \begin{subtable}[t]{\textwidth}
    \centering
    \begin{tabular}{l|c|c|c|c}
      Execution point & $c_1$ cell val & $c_1$ frz bit & $c_2$ cell val & $c_2$
      frz bit \\
      \hline
      $f_1$ start & $0$ & \LVarsFalse & $0$ & \LVarsFalse \\
      $f_1$ termination & $0$ & \LVarsTrue & $1$ & \LVarsFalse \\
      $f_2$ start & $0$ & \LVarsTrue & $1$ & \LVarsFalse \\
      $f_2$ termination & $0$ & \LVarsTrue & $1$ & \LVarsTrue \\
    \end{tabular}
    \caption{$f_1$ executes before $f_2$.}
    \label{stab:f_1exec}
  \end{subtable}

  \vspace{0.5em}

  \begin{subtable}[t]{\textwidth}
    \centering
    \begin{tabular}{l|c|c|c|c}
      Execution point & $c_1$ cell val & $c_1$ frz bit & $c_2$ cell val & $c_2$
      frz bit \\
      \hline
      $f_2$ start & $0$ & \LVarsFalse & $0$ & \LVarsFalse \\
      $f_2$ termination & $1$ & \LVarsFalse & $0$ & \LVarsTrue \\
      $f_1$ start & $1$ & \LVarsFalse & $0$ & \LVarsTrue \\
      $f_1$ termination & $1$ & \LVarsTrue & $0$ & \LVarsTrue \\
    \end{tabular}
    \caption{$f_2$ executes before $f_1$.}
    \label{stab:f_2exec}
  \end{subtable}
  \caption{Two different execution paths contradicting LVish quasi-determinism.}
\end{table}

This shows that as soon as we introduce the if-statement, a basic part of any
programming language, we can easily break determinism. The problem is inherent
in the freeze statement. By freezing a variable we could attain information
about the value and freeze bit of another cell. In the example above, by
freezing cell $c_1$ ($c_2$) and inspecting its value we can attain information
about wether $c_2$ ($c_1$) has been frozen or not. By conditioning on this
information we can skip a put operation that would otherwise lead to an error.

A simple solution to this problem is to simply remove the capability of freezing
a cell and this is the approach the system presented in
chapter~\ref{cha:core_language} will take. It should be possible to add other
forms of freezing and will be discussed in
chapter~\ref{cha:discussion_and_conclusion}.


% TODO then note that the LVish proof is flawed and even if you could find a
% proof it is not really interesting since if you introduce the if statement you
% could easily create counterexamples.

\begin{equation*}
\end{equation*}


\chapter{Core Language}%
\label{cha:core_language}

In this chapter a basic core language is introduced. It will build a lot upon
the core language of LaCasa and incorporate features of LVish. In the end we will
have an object oriented language incorporating many of the features of LVish with
a type system enforcing OCAP properties, much like the system of LaCasa.

\section{Syntax}
\label{sec:syntax}

The language which we shall call Reactive Async Core Language (RACL) has a big
similarity with the core language from LaCasa. Many expressions are the same
except for a few removals and additions. The language grammar is defined in
Figure~\ref{fig:racl_grammar}. It is a simple object oriented language which is
parameterized on the lattice \LatVals{}.

\begin{figure}
  \centering
  $\begin{array}{lll@{\hspace{4mm}}l}
    p &::= &\overline{cd}~\overline{vd}~t  & \mbox{Program} \\
    cd &::= &\texttt{class}~C~\texttt{extends}~D~\{\overline{fd}~\overline{md}\}
    & \mbox{Class} \\
    vd,fd &::= &\texttt{var}~f~:~\tau & \mbox{Variable/Field} \\
    md &::= &\texttt{def}~m(x: \sigma):~\tau = t & \mbox{Method}\\
    &&&\\
    \sigma,\tau & ::= & & \mbox{Types} \\
    & & C, D & \mbox{Class types} \\ 
    &| & \CellType & \mbox{Cell type} \\
    %&| & \NullType & \mbox{Null type} \\
    &| & \LatType & \mbox{Lattice value type} \\
    &&&\\
    t &::=& & \mbox{Terms} \\
    & & x & \mbox{Variable} \\
    &|& \texttt{let}~x = e~\texttt{in}~t &\mbox{Let binding} \\
    &&&\\
    e &::=& & \mbox{Expression} \\
    & & l & \mbox{Lattice value} \\
    &|& \texttt{null} & \mbox{Null reference} \\
    &|& x &\mbox{Variable} \\
    &|& x.f &\mbox{Field selection} \\
    &|& x.f = y & \mbox{Field assignment} \\
    &|& \texttt{new}~C & \mbox{Class instance creation} \\
    &|& \texttt{new}~\texttt{Cell} & \mbox{Cell instance creation} \\
    &|& x.m(y) & \mbox{Method invocation} \\
    &|& x~\texttt{put}~y & \mbox{Cell value update} \\
    &|& \texttt{when}~x~\texttt{pass}~y~\texttt{then}~(\overline{cap}, z
    \Rightarrow t) & \mbox{Dependency creation} \\
    &&&\\
    cap & ::= & x = y & \mbox{Variable capture} \\
  \end{array}$
  \caption{Grammar of RACL}
  \label{fig:racl_grammar}
\end{figure}

We can see that a program $p$ consists of a sequence of class definitions
$\overline{cd}$, a sequence of variable declarations $\overline{vd}$ and a term
$t$. A class definition $cd$ consists of a name specifier $C$, inheritance
specifier $D$, field declarations $\overline{fd}$ and method definitions
$\overline{md}$. Variable and field declarations both have the same form
consisting of a name specifier $f$ and a type $\tau$. Method definitions are
also standard, with one thing to note that all methods takes exactly one input.
More complicated inputs can be constructed using a container class.

The type specifiers $\sigma$ and $\tau$ can take on the values as specified.
Note that we have the special \CellType{} and \LatType{} types which are meant
to represent the cells of reactive async and values from the used lattice. Types
will be discussed more in Section~\ref{sec:type_system}.

As with LaCasa, the terms of RACL are written in A-normal form (i.e.\ every
subexpression is named). Most of the expressions should be self-explanatory.
However, note for example that we have a separate instance creation expression
for cells, an expression for updating the value of a cell aswell as an
expression for creating dependencies between cells. The dependency creation
expression is probably the most interesting since it mimics the syntax of
spores~\parencite{conf/ecoop/MillerHO14}. All captured variables are clearly
stated in the sequence of captures $\overline{cap}$. The semantics of 
expressions will be explained next.

\section{Semantics}%
\label{sec:semantics}

In this section the semantics of \RACL{} will be introduced. First a brief
overview is made and then a few of the more interesting or non-standard
execution rules will be explained.

In short, the following definitions says that the state of the execution of a
\RACL{} program consists of a \emph{heap} and a \emph{thread set}. The heap is
represented by a partial map from object identifiers $\OIDs$ to objects
$\Objects$. The thread set is a set of threads each of which consists of call
frame stacks. Each call frame holds a local variable environment and a term to
be executed. Steps between states can be made accoring to rules on either frame,
frame stack or thread set level. All steps taken on lower levels are propagated
to thread set level using auxilliary rules. After the following definitions we
describe the execution rules in more detail.

\subsection{Semantical Definitions}%
\label{sub:semantical_definitions}

\begin{definition}[Sets] We define the following sets.
  \begin{itemize}
    \item We let $\VarNames$ be the set of all allowed variable names.
    \item We let $\FieldNames$ be the set of all allowed field names.
    \item We let $\LatVals$ be the set of all lattice elements.
    \item We let $\OIDs$ be the set of all object identifiers.
    \item We let $\TIDs$ be the set of all thread identifiers.
    \item We let $\NullVal$ be the special null value.
    \item We let $\Values$ be the set of all possible runtime values
      \begin{equation*}
        \Values = \LatVals \cup \OIDs \cup \left\{ \NullVal \right\}.
      \end{equation*}
    \item We let $\ocapstats$ be the set of OCAP statuses
      \begin{equation*}
        \ocapstats = \left\{ \ocap, \nocap \right\}.
      \end{equation*}
  \end{itemize}
\end{definition}

\begin{definition}[Heap Objects]\label{def:heap_obj}
  We let $\Objects$ be the set of all \emph{heap objects}, i.e. objects of the
  form
  \begin{equation*}
    \Obj{C, FM} \text{ or } \Cell{DEP, l}.
  \end{equation*}
  In $\Obj{C, FM}$, $FM$ is a partial map
  \begin{equation*}
    FM: \FieldNames \rightharpoonup \Values.
  \end{equation*}
  In $\Cell{DEP, l}$, $l \in \LatVals$ and $DEP$ is a set of elements of the
  form
  \begin{equation*}
    (l' , (L_{\text{env}}, z \Rightarrow t))^\iota
  \end{equation*}
  where $l' \in \LatVals$, $L_{\text{env}}$ is a local environment, $z \in
  \VarNames$, $t$ is a term and $\iota \in \TIDs$ is a unique thread identifier. The
  uniqueness of $\iota$ is global to the state of a program.
\end{definition}

\begin{definition}
  A \emph{heap} $H$ is a partial map
  \begin{equation*}
    H: \OIDs \rightharpoonup \Objects.
  \end{equation*}
  A heap must always contain the global object as specified in
  definition~\ref{def:state_zero}.
\end{definition}

\begin{definition}
  A \emph{local environment} $L$ is a partial map
  \begin{equation*}
    L: \VarNames \rightharpoonup \Values.
  \end{equation*}
\end{definition}

\begin{definition}\label{def:thread_sets}
  A \emph{frame} $F$ is an object of the form
  \begin{equation*}
    \sframe{L, t}
  \end{equation*}
  where $L$ is a local environment, $t$ is a term and $s \in \VarNames$ is a
  return tag.

  A \emph{frame stack} $FS$ is a finitely large stack of frames. We can write
  all of these recursively as either the empty stack $FS = \varepsilon$, or as
  $FS = F \circ GS$ where $GS$ is also a frame stack.

  A \emph{thread set} $P$ is a finite set 
  \begin{equation*}
    P = \left\{ FS_i|_{a_i}^{\iota_i} \right\}_{i = 1}^{n}
  \end{equation*}
  of non-empty frame stacks $FS_i$ tagged with a unique thread id $\iota_i \in \TIDs$ and an
  OCAP status $a_i \in \ocapstats$. Note that $P$ can be the empty set.
\end{definition}

\begin{definition}
  We define a \emph{state} to be either the error state $\Error$ or a pair $H,
  P$, where $H$ is a heap and $P$ is a thread set.
\end{definition}

\begin{definition}
  The set of all valid types $\ValidTypes$ for a given program $p$ consists of
  $\AnyRefType$, $\NullType$, $\CellType$, $\LatType$ and all defined
  class types in $p$, $\ClassTypes$. I.e.
  \begin{equation*}
    \ValidTypes = \ClassTypes \cup \left\{ \AnyRefType, \NullType, \CellType,
    \LatType \right\}
  \end{equation*}
\end{definition}

\begin{definition}
  The \emph{default value function} $\default: \ValidTypes \to \Values$ is
  defined as follows.
  \begin{equation*}
    \default(\tau) =
    \begin{cases}
      \bot_{\LatVals} & \text{ if } \tau = \LatType \\
      \NullVal        & \text{ otherwise } 
    \end{cases}
  \end{equation*}
\end{definition}

\begin{definition} \label{def:state_zero}
  Execution of a program starts from state $S_0 = H_0, P_0$. $H_0$ and
  $P_0$ are defined as follows for a program $p =
  \overline{cd}~\overline{vd}~t$.
  \begin{equation*}
    \begin{gathered}
      H_0 = [o_g \mapsto \Obj{C_g, FM_g}] \\
      FM_g = [f \mapsto \default(\sigma): (\VarDecl{f}{\sigma}) \in \overline{vd}]
    \end{gathered}
  \end{equation*}
  where $o_g$ is a reserved object identifier for the \emph{global object} $\Obj{C_g,
  FM_g}$. $C_g$ is the \emph{global class name}.  
  \begin{equation*}
    \begin{gathered}
      P_0 = \left\{ \xframe{L_0, t} \circ \varepsilon
      |_{\nocap}^{\iota_{\text{main}}} \right\} \andalso
      L_0 = [ \Global \mapsto o_g ].
    \end{gathered}
  \end{equation*}
  where $\iota_{\text{main}}$ is a reserved thread identifier.
\end{definition}

% TODO define \Gamma_0

\subsection{Reduction Rules}%
\label{sub:reduction_rules}

An execution step from state $S$ to $S'$ in RACL is expressed using the relation
\begin{equation*}
  S \Rrightarrow S'.
\end{equation*}
RACL reduction rules are expressed at three different levels:
Thread set, frame stack and frame level. 
Therefore we will also see the relations 
\begin{equation*}
  H, FS \;\FSRedTo\; H', FS' \quad \text{ and } \quad H, F\; \FRedTo\; H', F'
\end{equation*}
This is to allow expression of e.g. single frame execution,
thread creation and method calls as will be explained below. 
A reduction on frame or frame stack level are propagated up to thread set level
with the rules \EFProp{} and \EFSProp{} which are defined in
figures~\ref{fig:fs_red_rules} and~\ref{fig:threads_red_rules}.

In Figure~\ref{fig:frame_red_rules}, reduction rules for single frames are
defined. These are rules that advance the state of a single frame $F$ and
possibly changes the heap $H$. We have very simple ones like the rule {\sc
E-LVal}, which assigns the specified lattice value to a local variable, and the
rule {\sc E-Var} which assigns the value of one local variable to an other.
The rules \ESelect{} and \EAssign{} respectively fetches and sets the value of a
field $f$ of an object on the heap. Rules \ENew{} and \ENewCell{} creates a new
object on the heap of either class or cell type and assigns the corresponding
new object identifier to a local variable. The rule \EPut{} updates the cell
value of a cell object on the heap through a join operation.  Finally the rule
\EWhen{} is responsible for adding a new dependency callback. Looking at
this rule more closely we see that it captures the variables specified by
$\overline{cap}$ and creates a local environment $L_{\text{env}}$. This is then
stored, together with the closure $w \Rightarrow t'$, a threshold value $l'$ and
a fresh thread identifier $\iota$, in the new dependency set $DEP'$. \EWhen{} 
mimics the capture semantics of spores~\parencite{conf/ecoop/MillerHO14}.
Note that all rules in Figure~\ref{fig:frame_red_rules} assigns something to the
variable $x$.

In Figure~\ref{fig:error_red_rules}, rules that lead to error are defined. These
are all what would be called null-pointer exceptions in a language like Java.
That is, that we try to access an object through the $\NullVal$ identifier.
Errors are then propagated to frame stack and then thread set level using rules
\EErrorFS{} and \EErrorP{} from figures~\ref{fig:fs_red_rules} and
\ref{fig:threads_red_rules}.

In Figure~\ref{fig:fs_red_rules}, rules for frame stack reductions are defined.
Here we find the aforementioned \EFProp{} and \EErrorFS{} together with rules
for method calls and returns. \ECall{} handles method calls by creating a new
frame with the corresponding term and local environment. Note that the new local
environment differs depending on the OCAP status of the working thread. If the
thread is tagged with $a = \nocap$, the new frame also has access to the global
environment, while if $a = \ocap$ it will not. Note also that the new frame
stack is tagged with the variable name $x$, the variable to which the method
result will be assigned to with rule \ERet{}. \ERet{} corresponds to method
return and utilizes the aforementioned variable name tag. Note that none of
these rules changes the thread identifier or OCAP status of a thread.

Finally, in Figure~\ref{fig:threads_red_rules}, the rules for thread set
reductions are defined. Here we find \EFSProp{} and \EErrorP{} which were
mentioned above. Furthermore we have rule \ESpawn{}, which spawns a new callback
thread for the threshold value $l$ in the cell specified by $o$. \ETerm{} is a
rule to remove any thread stack finished with its execution.

\begin{figure}[h]
  \scax{E-Null}
  {H, \sFrame{L}{\Let{x}{\NullVal}{t}} \; \FRedTo \; H, \sFrame{L[x \mapsto
  \NullVal]}{t}}

  \RuleSpace{}

  \scax{E-LVal}
  {H, \sFrame{L}{\Let{x}{l}{t}} \; \FRedTo \; H, \sFrame{L[x \mapsto
  l]}{t}}

  \RuleSpace{}

  \scax{E-Var}
  {H, \sFrame{L}{\Let{x}{y}{t}} \; \FRedTo \; H, \sFrame{L[x \mapsto
  L(y)]}{t}}

  \RuleSpace{}

  \scrule{E-Select}
  {L(y) = o \andalso H(o) = \Obj{C, FM} \andalso f \in \dom(FM)}
  {H, \sFrame{L}{\Let{x}{ \FSel{y}{f} }{t}} \; \FRedTo  \\ 
  H, \sFrame{L[x \mapsto FM(f)]}{t}}

  \RuleSpace{}

  \scrule{E-Assign}
  {L(y) = o \andalso H(o) = \Obj{C, FM} \andalso f \in \dom(FM) \\
  FM' = FM[f \mapsto L(z)] \andalso H' = H[o \mapsto \Obj{C, FM'}]}
  {H, \sFrame{L}{\Let{x}{\FAss{y}{f}{z}}{t}} \; \FRedTo \\
   H', \sFrame{L[x \mapsto L(z)]}{t}}

  \RuleSpace{}

  \scrule{E-New}
  {o\text{ fresh object identifier } \\
  FM = [f \mapsto \default(\sigma): (\VarDecl{f}{\sigma}) \in \fdecls(C)] \\
  H' = H[o \mapsto \Obj{C, FM}]}
  {H, \sFrame{L}{ \Let{x}{\New{C}}{t} } \; \FRedTo \\
  H', \sFrame{L[x \mapsto o]}{t}}
  
  \RuleSpace{}

  \scrule{E-NewCell}
  {o\text{ fresh object identifier } \andalso
  H' = H[o \mapsto \Cell{\emptyset, \bot_{\LatVals{}}}]}
  {H, \sFrame{L}{ \Let{x}{\NewCell}{t} } \; \FRedTo \\
  H', \sFrame{L[x \mapsto o]}{t}}

  \RuleSpace{}
  
  \scrule{E-Put}
  {L(y) = o \andalso H(o) = \Cell{DEP, l} \\
  L(z) = l' \andalso c' = \Cell{DEP, l \sqcup l'} \\
  H' = H[o \mapsto c']}
  {H, \sFrame{L}{\Let{x}{\Put{y}{z}}{t}} \; \FRedTo \\
  H', \sFrame{L[x \mapsto L(y)]}{t}}

  \RuleSpace{}

  \scrule{E-When}
  {L(y) = o \andalso H(o) = \Cell{DEP, l} \andalso L(z) = l' \\
  L_{\text{env}} = [u \mapsto L(u') : (\Capt{u}{u'}) \in \overline{cap}]
  \andalso cb = (L_{\text{env}}, w \Rightarrow t') \\
  \iota\text{ fresh thread identifier } \andalso DEP' = DEP \cup (l', cb)^\iota \\
  H' = H[o \mapsto \Cell{DEP', l}] }
  { H, \sFrame{L}{ \Let{x}{ \When{y}{z}{ (\overline{cap}, w \Rightarrow t') }}{t} } \\ \FRedTo \;
  H', \sFrame{L[x \mapsto L(y)]}{t} }
  \caption{\RACL{} single frame reduction rules.}
  \label{fig:frame_red_rules}
\end{figure}


\begin{figure}
  \scrule{E-NullSelect}
  {L(y) = \NullVal}
  {H, \sFrame{L, \Let{x}{\FSel{y}{f}}{t} } \; \FRedTo \; \Error}

  \RuleSpace{}

  \scrule{E-NullAssign}
  {L(y) = \NullVal}
  {H, \sFrame{L, \Let{x}{\FAss{y}{f}{z}}{t} } \; \FRedTo \; \Error}

  \RuleSpace{}

  \scrule{E-NullCall}
  {L(y) = \NullVal}
  {H, \sFrame{L, \Let{x}{\Call{y}{m}{z}}{t} } \; \FRedTo \; \Error}

  \RuleSpace{}

  \scrule{E-NullPut}
  {L(y) = \NullVal}
  {H, \sFrame{L, \Let{x}{\Put{y}{z}}{t} } \; \FRedTo \; \Error}

  \RuleSpace{}

  \scrule{E-NullWhen}
  {L(y) = \NullVal}
  {H, \sFrame{L}{\Let{x}{  \When{y}{z}{ (\overline{cap}, w \Rightarrow t')}}{t}
  }  \\ \FRedTo \; \Error}
  \caption{\RACL{} error spawning rules.}
  \label{fig:error_red_rules}
\end{figure}

\begin{figure}
  \scrule{E-Call}
  {
    L(y) = o \andalso H(o) = \Obj{C, FM} \\
    \mbody(m, C) = w \to t' \\
    L_{\text{base}} =
    \begin{cases}
      \emptyset & \text{if } a = \ocap \\
      L_0 & \text{if } a = \nocap
    \end{cases} \\
    L' = L_{\text{base}}[\This \mapsto L(y), w \mapsto L(z)] 
  }
  {H, \sFrame{L}{ \Let{x}{ \Call{y}{m}{z} }{t} } \circ FS |_a^\iota \; \FSRedTo \\
  H, \Frame{L'}{t'}{x} \circ \sFrame{L}{t} \circ FS |_a^\iota}

  \RuleSpace{}

  \scax{E-Ret}
  {H, \Frame{L'}{x}{y} \circ \sframe{L, t} \circ FS |_a^\iota \; \FSRedTo \\
  H, \sFrame{L[y \mapsto L'(x)]}{t} \circ FS |_a^\iota }

  \RuleSpace{}

  \scrule{E-FProp}
  {H, F \; \FRedTo \; H', F'}
  {H, F \circ FS |_a^\iota \; \FSRedTo \; H', F' \circ FS |_a^\iota }

  \RuleSpace{}
  
  \scrule{E-ErrorFS}
  {H, F \; \FRedTo \; \Error }
  {H, F \circ FS |_a^\iota \; \FSRedTo \; \Error}

  \caption{\RACL{} frame stack reduction rules.}
  \label{fig:fs_red_rules}
\end{figure}

\begin{figure}
  \scrule{E-Spawn}
  {
    o \in \dom(H) \andalso H(o) = \Cell{DEP, l} \\ 
    l' \sqsubseteq l \andalso (l', cb)^\iota \in DEP \andalso cb = (L_{\text{env}}, z
    \Rightarrow t) \\
    L = L_{\text{env}}[z \mapsto l'] \\
    H' = H[o \mapsto \Cell{DEP - (l', cb)^\iota, l}]
  }
  {
    H,P \Rrightarrow H', P \cup \left\{ \Frame{L}{t}{-} \circ \varepsilon
    |_{\ocap}^{\iota} \right\}
  }

  \RuleSpace{}

  \scrule{E-Term}
  {P = P' \cup_D \left\{ \sFrame{L}{x} \circ \varepsilon |_a^\iota \right\} }
  {H,P \Rrightarrow H, P'}

  \RuleSpace{}

  \scrule{E-FSProp}
  {H, FS \twoheadrightarrow H', FS'}
  {H, P \cup_D \left\{ FS \right\} \Rrightarrow H', P \cup \left\{ FS' \right\} }

  \RuleSpace{}

  \scrule{E-ErrorP}
  {H, FS \twoheadrightarrow \Error}
  {H, P \cup_D \left\{ FS \right\} \Rrightarrow \Error }
  \caption{\RACL{} thread set reduction rules.}
  \label{fig:threads_red_rules}
\end{figure}


\section{Type System}
\label{sec:type_system}

\begin{figure}
  \begin{multicols}{2}

    \scax{ST-Top}
    {\tau \stof \RaclTop}

    \RuleSpace

    \scax{ST-Bot}
    {\RaclBot \stof \tau}

    \RuleSpace

    \scax{ST-Cell-AnyRef}
    {\CellType \stof \AnyRefType}

    \RuleSpace

    \scax{ST-C-AnyRef}
    {C \stof \AnyRefType}

    \RuleSpace

    \scax{ST-Null-Cell}
    {\NullType \stof \CellType}
    
    \RuleSpace

    \scax{ST-Null-C}
    {\NullType \stof C}
  \end{multicols}

  \RuleSpace

  \scrule{ST-C-D}
  {p \vdash \ClassDef{C}{D}{...}{...}}
  {C \stof D}

  \caption{Subtyping relation of RACL}
  \label{fig:def_stof}
\end{figure}

The type system will now be introduced starting with the types themselves.
The subtyping relation of RACL is defined in Figure~\ref{fig:def_stof}.  The
types and type lattice of RACL is summarized in Figure~\ref{fig:racl_typelat}.
Except for the standard types we see that the \CellType{} type is a subtype of
\AnyRefType{} like the class types. Intuitively this originates from that cell
objects are stored on the heap. We also have a separate lattice value type
\LatType{}.

\begin{figure}[]
  \centering
  \begin{tikzpicture} 
    \node (top) at (1,3) {$\RaclTop$};
    \node (anyref) at (2,2) {\AnyRefType};
    \node (cell) at (1,1) {\CellType};
    \node (classes) at (3,1) {$C,D$};
    \node (null) at (2,0) {\NullType};
    \node (lat) at (-1, 1) {\LatType};
    \node (bot) at (1, -1) {$\RaclBot$};
    \draw (top) -- (lat) -- (bot) -- (null) -- (classes) -- (anyref) -- (top);
    \draw (anyref) -- (cell) -- (null);
  \end{tikzpicture}
  \caption{Type lattice of RACL}
  \label{fig:racl_typelat}
\end{figure}

\subsection{Terms \& Expressions}%
\label{sub:terms_and_expressions}

The basic building blocks of our type system are the typing of expressions and
terms. Our typing relation is written
\begin{equation}
  \TypeRel{\Gamma}{a}{t}{\tau} \quad \text{ or } \quad
  \TypeRel{\Gamma}{a}{e}{\tau}. \notag
\end{equation}
Apart from the usual components, i.e., typing environment $\Gamma$, term $t$ or
expression $e$ and type $\tau$, we note that it includes a designator $a$ which
can take on the values \nocap{} and \ocap{}. The latter indicates that the term
or expression is typed under OCAP constraints. This is equivalent to the OCAP
typing of \LaCasa{}~\parencite{conf/oopsla/HallerL16}. 

All typing rules for terms and expressions can be found in
Figure~\ref{fig:expr_typing}. Most of the type system rules are standard. For
example, rule {\sc T-Let} types a let-term if the subexpression $e$ is typeable
as $\tau$ under $\Gamma; a$, and the subterm $t$ is typeable under the extended
environment $\Gamma, x: \tau$ and $a$. Rule {\sc T-Var} types a variable under
$\Gamma$ provided $x \in \dom(\Gamma)$. {\sc T-New} types $\New{C}$ under effect
$a = \ocap$ only if the class $C$ is typeable as $\ocap$. The rules for typing a
class as $\ocap$ is defined in Figure~\ref{fig:ocap_typing}. Typing rule {\sc
T-Call} states that a method call $\Call{x}{m}{y}$, is only typeable if the type
of $y$ is a subtype of the method parameter type $\sigma$. {\sc T-Put} types the
expression $\Put{x}{y}$ if $x$ is typeable as \CellType{} and $y$ is of the lattice
type \LatType.

As a final example, the rule {\sc T-When} describes typing of the dependency
creation expression. It says that in order to register a callback for lattice
value $y$ in cell $x$, first $x$ and $y$ must be typeable as $\CellType$ and
$\LatType$ respectively. Furthermore all captured variables in $\overline{cap}$
must be typeable as $\CellType$. This is to ensure that all objects shared
between threads are of cell type. This is similar to capturing constraints in
spores~\parencite{conf/ecoop/MillerHO14}. Finally, the term $t$ from callback
closure $z \Rightarrow t$ must be typeable in an environment consisting of the
captured variables and $z: \LatType$.


\begin{figure}[h!]
  \centering
  %\begin{multicols}{2}
    \scrule{T-Let}
    {\TypeRel{\Gamma}{a}{e}{\tau} \andalso \TypeRel{\Gamma,x:
    \tau}{a}{t}{\sigma}}
    {\TypeRel{\Gamma}{a}{ \Let{x}{e}{t} }{\sigma}}

    \vspace{0.5em}

    \scax{T-Null}{\TypeRel{\Gamma}{a}{\NullVal}{\NullType}}

    \vspace{0.5em}

    \scax{T-LVal}{\TypeRel{\Gamma}{a}{l}{\LatType}}
    
    \vspace{0.5em}

    \scrule{T-Var}{x \in \dom(\Gamma)}{\TypeRel{\Gamma}{a}{x}{\Gamma(x)}}

    \vspace{0.5em}

    \scrule{T-Select}
    {\TypeRel{\Gamma}{a}{x}{C} \andalso \ftype(f, C) = \tau }
    {\TypeRel{\Gamma}{a}{\FSel{x}{f}}{\tau}}
    
    \vspace{0.5em}

    \scrule{T-Assign}
    {\TypeRel{\Gamma}{a}{x}{C} \andalso \ftype(f, C) = \tau \\
    \TypeRel{\Gamma}{a}{y}{\tau'} \andalso \tau' \stof \tau }
    {\TypeRel{\Gamma}{a}{\FAss{x}{f}{y}}{\tau}}

    \vspace{0.5em}

    \scrule{T-New}
    {a = \ocap \Longrightarrow \ocap(C)}
    {\TypeRel{\Gamma}{a}{\New{C}}{C}}

    \vspace{0.5em}

    \scax{T-NewCell}{\TypeRel{\Gamma}{a}{\NewCell}{\CellType}}
    
    \vspace{0.5em}

    \scrule{T-Call}
    {\TypeRel{\Gamma}{a}{x}{C} \andalso \mtype(C,m) = \sigma \to \tau \\
    \TypeRel{\Gamma}{a}{y}{\sigma'} \andalso \sigma' \stof \sigma }
    {\TypeRel{\Gamma}{a}{\Call{x}{m}{y}}{\tau}}
    
    \vspace{0.5em}

    \scrule{T-Put}
    {\TypeRel{\Gamma}{a}{x}{\CellType} \andalso \TypeRel{\Gamma}{a}{y}{\LatType}}
    {\TypeRel{\Gamma}{a}{\Put{x}{y}}{\CellType}}
    
    \vspace{0.5em}

    \scrule{T-When}
    {\TypeRel{\Gamma}{a}{x}{\CellType} \andalso \TypeRel{\Gamma}{a}{y}{\LatType} \\
    \forall \Capt{u}{u'} \in \overline{cap}. \; \TypeRel{\Gamma}{a}{u'}{\CellType}\\
    \Gamma_{\text{cells}} = [u \mapsto \CellType : \Capt{u}{u'} \in \overline{cap}]\\
    \TypeRel{\Gamma_{\text{cells}}, z : \LatType}{\ocap}{t}{\sigma}}
    { \TypeRel{\Gamma}{a}{\When{x}{y}{ \CB{\overline{cap}}{z}{t}}}{\CellType} }

  %\end{multicols}
  \caption{\RACL{} typing rules for expressions and terms.}
  \label{fig:expr_typing}
\end{figure}

\subsection{Well Formed Programs}%
\label{sub:well_formed_programs}

Intuitively, a well formed program is a program which obeys our type system. We
define well-formedness in Figure~\ref{fig:wf_typing}. We also need the
definition of the global typing environment $\Gamma_0$. 
\begin{definition}
  Let the \emph{global typing environment} $\Gamma_0$ be defined as 
  \begin{equation*}
    \Gamma_0 = \Global : C_g
  \end{equation*}
\end{definition}
{\sc WF-Prog} says that in order for a program $p =
\overline{cd}~\overline{vd}~t$ to be well typed, all classes $\overline{cd}$,
global variables $\overline{vd}$ and the program term $t$ must be well formed.
Rule {\sc WF-Global} says that in order for a global variable declaration to be
well formed, the denoted type must either be $\LatType$, $\CellType$ or be of
class type $C$ where the class definition of $C$ is well formed. In order for a
class definition of $C$ to be well formed, rule {\sc WF-Class} declares that all
methods must be well formed under $C$, the extended class $D$ must either be
$\AnyRefType$ or must also be well formed. Furthermore, all methods must obey the
rules of overriding and $C$ cannot redeclare any fields that has been declared
in an extended class $D$. {\sc WF-Override} declares that for a method to be
correctly overriden it must either not be declared in any extended class or it
must have the same declared type as in the extended class. Finally {\sc
WF-Method} states that in order for a method to be well formed, its term must be
typable as a type $\tau'$ under an environment consisting of the global
environment, the class itself $\This : C$ and the method parameter $x: \sigma$.
Futhermore $\tau' \stof \tau$, where $\tau$ is the declared return type of the
method.



\begin{figure}[h]
  \scrule{WF-Prog}
  {p \vdash \overline{cd} \andalso p \vdash \overline{vd} \andalso
  \TypeRel{\Gamma_0}{\nocap}{t}{\tau}}
  {p \vdash \overline{cd} \: \overline{vd} \: t}

  \RuleSpace{}

  \scrule{WF-Global}
  {\sigma = \LatType \lor \sigma = \CellType ~ \lor  \\
  (\sigma = C \land p \vdash \ClassDef{C}{...}{...}{...})}
  {p \vdash \VarDecl{f}{\sigma}}
  
  \RuleSpace{}

  \scrule{WF-Class}
  {C \vdash \overline{md} \\ D = \AnyRefType{} \lor
  p \vdash \ClassDef{D}{...}{...}{...} \\
  \forall (\MethodDef{m}{...}{...}{...}{...}) \in \overline{md} . \: \override(m,
  C, D) \\
  \forall (\VarDecl{f}{\tau}) \in \overline{vd} . \: f \notin \fields(D) }
  {p \vdash \ClassDef{C}{D}{\overline{vd}}{\overline{md}}}

  \RuleSpace{}

  \scrule{WF-Override}
  {\mtype(m, D)\text{ not def. } \lor \mtype(m, C) = \mtype(m, D)}
  {\override(m, C, D)}
  
  \RuleSpace{}

  \scrule{WF-Method}
  { \TypeRel{\Gamma_0, \This:C, x : \sigma}{\nocap}{t}{\tau'} \\
  \tau' \stof \tau}
  {C \vdash \MethodDef{m}{x}{\sigma}{\tau}{t}}
  \caption{\RACL{} rules for well formedness of programs.}
  \label{fig:wf_typing}
\end{figure}

\subsection{OCAP Typing}%
\label{sub:ocap_typing}

In Figure~\ref{fig:ocap_typing} we find the rules for classifying a type as $\ocap$.
Immediately we see that unconditionally, both $\AnyRefType$ and $\CellType$ are
$\ocap$. To type a class $C$ as $\ocap$, {\sc OCAP-Class} says that apart from being
well formed, the superclass $D$ must be $\ocap$. Furthermore, all methods must
be typable under the special judgement $\vdash_{\ocap}$ and all fields must be
$\ocap$.  For a method to be typed under $\vdash_{\ocap}$, rule {\sc
OCAP-Method} declares that the method term must be typable under effect $\ocap$
without access to the global environment $\Gamma_0$. Being typable under effect
$\ocap$ means that expression $\New{C}$ is only allowed if $C$ is $\ocap$, as
stated in rule {\sc T-New}. In short, an ocap term can only instansiate ocap
classes. This part of the type system is similar to the corresponding part of
LaCasa~\parencite{conf/oopsla/HallerL16}.

\begin{figure}
  \scax{OCAP-AnyRef}
  {\ocap{(\AnyRefType{})}}

  \RuleSpace{}

  \scax{OCAP-Cell}
  {\ocap(\CellType)}

  \RuleSpace{}

  \scrule{OCAP-Class}
  {\ocap{(D)} \andalso C \vdash_{\ocap} \overline{md} \\
  \forall (\VarDecl{f}{\sigma}) \in \overline{vd}. \: ocap(\sigma)}
  {\ocap{(C)}}

  \RuleSpace{}

  \scrule{OCAP-Method}
  {\TypeRel{\This : C, x : \sigma}{\ocap}{t}{\tau'} \\
  \tau' \stof \tau}
  {C \vdash_{\ocap} \MethodDef{m}{x}{\sigma}{\tau}{t}}
  \caption{\RACL{} OCAP rules.}
  \label{fig:ocap_typing}
\end{figure}


\section{Properties}
\label{sec:properties}

In order to state and prove things such as progress and preservation we need a few more
definitions. Many of them are just auxilliary properties of things like states
and types and build up to the final definition of a well typed state.

\subsection{Well Typed Heap}%
\label{sub:well_typed_heap}

This first definition is a straightforward partial function defining the dynamic
type of a value.
\begin{definition}
  The partial function $\typeOf$ is defined as follows:
  \begin{equation}
    \typeOf{(k, H)} =
    \begin{cases}
      \LatType, &\text{ if }k \in \LatVals \\
      \NullType, &\text{ if }k = \NullVal \\
      \CellType, &\text{ if }k \in \dom{(H)}\text{ and } H(k) = \Cell{...} \\
      C, &\text{ if }k \in \dom{(H)}\text{ and } H(k) = \Obj{C, ...} \\
    \end{cases}
  \end{equation}
\end{definition}
In order to simplify the definition of a well typed heap we define the
following.
\begin{definition}
  The typing environment $\Gamma_{\CellType}(L)$ is defined as follows:
  \begin{equation}
    \Gamma_{\CellType}(L) = \left[(x: \CellType) : x \in \dom(L)\right]
  \end{equation}
\end{definition}
Next comes the definition of a well typed heap. To say that a heap is
well typed intuitively means that for all class objects, all field values are of
a subtype of the declared type. Furthermore, our definition includes a statement
about cell objects. We say that for all callbacks stored in dependency sets, all
captured values must be of cell type and the callback term must be typeable
under an environment containing the captured variables and closure parameter
$z$.
\begin{definition}[Well Typed Heap]
  A heap $H$ is well typed, written $\vdash{H}$, if
  $\forall o \in \dom{(H)}$:

  If $H(o) = \Obj{C, FM}$ then
  \begin{equation} \label{eq:defwth1}
    \begin{aligned}
      \forall f &\in \fields{(C)}. \\ 
      &f \in \dom{(FM)} \: \land \\ 
      &\typeOf{(FM(f), H)} \stof \ftype{(f, C)}
    \end{aligned}
  \end{equation}
  and if $H(o) = \Cell{DEP, l}$ then
  \begin{equation} \label{eq:defwth2}
    \begin{aligned}
      \forall (l'&, (L_{\text{env}}, z \Rightarrow t))^\iota \in DEP. \\
      &\forall (x \mapsto k) \in L_{\text{env}}.\: \typeOf{(k, H)} \stof
      \CellType \: \land \\
      &\TypeRel{\Gamma_{\CellType{}}(L_{\text{env}}), z:
      \LatType}{\ocap}{t}{\gamma} 
    \end{aligned}
  \end{equation}
\end{definition}

\subsection{Well Typed Threads}%
\label{sub:well_typed_threads}

A non erroneous state $S = H, P$ has two parts. In order to define and
prove preservation properties, we furthermore need to put restrictions on the
thread set $P$. We do this with the relation $H \vdash P$, saying $P$ is well
typed under heap $H$. This relation is defined in Figure~\ref{fig:ts_typing}.
{\sc T-Procs} and {\sc T-Empty} basically says that in order for a thread set
$P$ to be well typed under heap $H$, all threads needs to be well typed under
$H$ and its OCAP status $a$.  For a thread $GS|_a^\iota$ we write this as $H; a
\vdash GS$. This relation is defined in Figure~\ref{fig:fs_typing}. 

For $H; a \vdash GS$ to hold, either $FS = \varepsilon$ as in rule {\sc
T-FSEmpty1} (this is actually impossible since frame stacks of a state cannot be
empty, see definition~\ref{def:thread_sets}), or the term $t$ of top frame $F$
is typeable with some environment $\Gamma$ which is conformant with its local
variable map $L$ and $H$. This conformancy is expressed through the relation $H
\vdash \Gamma; L$ which is defined in Figure~\ref{fig:local_typing}. Simply
stated, it says that all types specified in $\Gamma$ aligns with the dynamic
types of the values in $L$. Furthermore {\sc T-FS1} declares that the rest of the frame
stack $FS$ must be typable under the judgement $\vdash^{x: \sigma}$. This
judgement is defined by rules {\sc T-FSEmpty2} and {\sc T-FS2}. The latter
reflects that the top frame returns some value to its underlying frame stack. It
is closely connected with the well-formedness of methods and call\slash return
execution semantics as defined by rules {\sc WF-Method} and \ECall{}\slash\ERet{}
respectively.

\subsection{Isolation}%
\label{sub:isolation}

The next two definitions are used when defining isolation of threads in the
reference graph of the heap. The ruling accRoot(o, FS) is defined in
Figure~\ref{fig:def_accroot}.
\begin{definition}[Class Object Separation]
  For any heap $H$ and heap references $o, o'$ we have class object separation,
  $\csep{(H, o, o')}$ iff
  \begin{align}
    \label{eq:csep_def}
    \forall q, q' &\in \dom{(H)}. \notag\\
    & \reach{(H, o, q)} \: \land \: \reach{(H, o', q')} \implies \notag\\ 
    &q \neq q' \: \lor \typeOf{(q, H)} = \CellType
  \end{align}
\end{definition}

\begin{definition}[Accessible Roots]
  For a heap $H$ and a frame stack $FS$ we define $\accRoots{(FS, H)}$ as
  \begin{equation}
    \accRoots{(FS, H)} = \left\{ o \in \dom{(H)}: \accRoot{(o, FS)} \right\}
  \end{equation}
\end{definition}

Isolation of threads is defined in Figure~\ref{fig:def_isolation}. Rule {\sc
ISO-FS} states that for two threads to be isolated with regards to heap $H$, all
accessible roots of the two threads must have class object separation, i.e. that
all objects that are reachable from both threads must be of type $\CellType$.
Rule {\sc ISO-Procs} states that we have isolation for a thread set $P$ under
heap $H$, if there is isolation between all pair of threads such that at least
one is $\ocap$.

\subsection{OCAP Reachability}%
\label{sub:ocap_reachability}

In order to prove preservation of isolation we also need to ensure that all
objects reachable from an $\ocap$-annotated thread can be typed as an $\ocap$
type. The next definition together with Figure~\ref{fig:def_ocapreach} defines
this property formally.
\begin{definition}[OCAP Reachability]
  For a heap $H$ and frame stack $FS$ we have $\ocrloc{(FS,H)}$ iff
  \begin{align}
    \label{eq:ocr_def}
    \forall o \in \: &\accRoots(FS,H), o' \in \dom(H). \notag\\
    &\reach{(H, o, o')} \implies \ocap{(\typeOf{(o', H)})}
  \end{align}
\end{definition}

\subsection{Global Object Separation}%
\label{sub:global_object_separation}

Another thing needed to prove preservation of isolation is global object
separation. Simply stated this means that no $\ocap$ thread can reach the global
object $o_g$ through heap references. This is formally defined in
Figure~\ref{fig:def_gsep}.

\subsection{No Thread Spawning}%
\label{sub:no_thread_spawning}

The next definition is needed to state and prove progress.
\begin{definition}[No Spawn]
  For any heap $H$ we have $\noSpawn(H)$ if and only if
  \begin{equation}
    \begin{aligned}
      \forall o &\in \dom(H). \\
        & H(o) = \Cell{DEP, l} \implies 
        \forall (l', cb)^\iota \in DEP. \: \lnot (l' \sqsubseteq l).
    \end{aligned}
  \end{equation}
\end{definition}

\subsection{Unique Main Thread}%
\label{sub:unique_main_thread}

In order to prove determinism we must be sure that there is at most one
non-$\ocap$ thread running. Otherwise these could interfere since there are no
constraints on whether these can share data. Therefore we define the following
property.
\begin{definition}[Unique Main Thread]
  For a thread $FS|_a^\iota$ let
  \begin{equation*}
    \chi_{\nocap}(a) =
    \begin{cases}
      1 & \text{ if } a = \nocap \\
      0 & \text{ o.w. }
    \end{cases}
  \end{equation*}
  For a thread set $P$ let
  \begin{equation*}
    \chi(P) = \sum_{FS|_a^\iota \in P} \chi_{\nocap}(a).
  \end{equation*}
  This is the number of non OCAP protected threads in $P$. Finally we define
  \begin{equation*}
    \uniqMain(P) \iff \chi(P) \leq 1
  \end{equation*}
\end{definition}


\subsection{Well Typed States}%
\label{sub:well_typed_states}

Finally we can define the notion of a well typed state. This combines many of
the properties already defined into one.
\begin{definition}[Well Typed States]
  For a state $S$ we say that it is well typed, written
  \begin{equation}
    \vdash S \tsep \stateok
  \end{equation}
  if $S = \Error$ or $S = H, P$ and
  \begin{equation*}
    \begin{gathered}
      \vdash H \andalso H \vdash P \andalso H \vdash P \tsep \ocr \\
      \isolation{(H, P)} \andalso H \vdash P \tsep \gsep \andalso \uniqMain(P)
    \end{gathered}
  \end{equation*}
  
\end{definition}

\begin{figure}
  \scrule{WF-EnvVar}
  {\typeOf{(L(x), H) \stof \Gamma{(x)}}}
  { H \vdash \Gamma; L; x }

  \RuleSpace{}

  \scrule{WF-Env}
  {\dom{(\Gamma)} \subseteq \dom{(L)} \\
    \forall x \in \dom{(\Gamma)}. \: H \vdash \Gamma; L; x 
  }
  { H \vdash \Gamma; L }

  \caption{Rules for classifying local environments $L$ as well typed.}
  \label{fig:local_typing}
\end{figure}

\begin{figure}
  \scax{T-FSEmpty1}
  {H; a \vdash \varepsilon}

  \RuleSpace{}

  \scax{T-FSEmpty2}
  {H; a \vdash^{x : \sigma} \varepsilon}

  \RuleSpace{}

  \scrule{T-FS1}
  { F = \Frame{L}{t}{x} \andalso H \vdash \Gamma; L \\
  \TypeRel{\Gamma}{a}{t}{\sigma'} \andalso \sigma' \stof \sigma \andalso H; a \vdash^{x: \sigma} FS }
  { H; a \vdash F \circ FS }
  
  \RuleSpace{}

  \scrule{T-FS2}
  { F = \Frame{L}{t}{y} \andalso H \vdash \Gamma; L \\
  \TypeRel{\Gamma, x: \tau}{a}{t}{\sigma'} \andalso \sigma' \stof \sigma 
  \andalso H; a \vdash^{y: \sigma} FS }
  { H; a \vdash^{x: \tau} F \circ FS }

  \caption{Rules for typing frame stacks under some heap $H$ and effect $a$.}
  \label{fig:fs_typing}
\end{figure}

\begin{figure}
  \scax{T-Empty}
  {H \vdash \emptyset}

  \RuleSpace{}

  \scrule{T-Procs}
  { H; a \vdash FS \andalso H \vdash P }
  { H \vdash P \cup \left\{FS |_a^\iota \right\} }

  \caption{Rules for typing thread sets and the special value \Error{} under some heap $H$.}
  \label{fig:ts_typing}
\end{figure}

\begin{figure}
  \scrule{ISO-FS}
  { 
    \forall o \in \accRoots(FS), o' \in \accRoots(GS) . \: \csep{(H, o, o')}
  }
  {
    \isolated{(H, FS, GS)}
  }
  
  \RuleSpace{}

  \scrule{ISO-Procs}
  {
    \forall FS|_a^\iota, GS_b^{\iota'} \in P \text{ where } FS|_a^\iota \neq
    GS|_b^{\iota'} . \\
    a = \ocap \: \lor \: b = \ocap \implies \isolated{(H, FS, GS)}
  }
  {
    \isolation{(H, P)}
  }
  \caption{Definition of isolation}
  \label{fig:def_isolation}
\end{figure}

\begin{figure}
  \scrule{Reach1}
  {o \in \dom{(H)}}
  {\reach{(H, o, o)}}

  \RuleSpace{}

  \scrule{Reach2}
  {
    o \in \dom{(H)} \andalso H(o) = \Obj{C, FM} \\
    o'' \in \image{(FM)} \andalso \reach{(H, o'', o')}
  }
  { \reach{(H, o, o')} }
  \caption{Definition of reach}
  \label{fig:def_reach}
\end{figure}

\begin{figure}
  \scrule{AR-F}
  { (x \mapsto o) \in L }
  { \accRoot{(o, \sFrame{L}{t})} }

  \RuleSpace{}

  \scrule{AR-FS}
  { \accRoot{(o, F)} \: \lor \: \accRoot{(o, FS)} }
  { \accRoot{(o, F \circ FS)} }
  \caption{Definition of accRoot}
  \label{fig:def_accroot}
\end{figure}

\begin{figure}
  \scrule{OCR-FS}
  { a = \ocap \implies \ocrloc(FS, H) }
  { H; a \vdash FS \tsep \ocr }

  \RuleSpace{}

  \scax{OCR-PEmpty}
  {H \vdash \emptyset \tsep \ocr}

  \RuleSpace{}

  \scrule{OCR-P}
  {H \vdash P \tsep \ocr \andalso H; a \vdash FS \tsep \ocr}
  {H \vdash P \cup \left\{ FS|_a^\iota \right\} \tsep \ocr}

  \caption{Definition of OCAP reachability}
  \label{fig:def_ocapreach}
\end{figure}

\begin{figure}
  \scrule{GSep-Threads}
  {\forall FS|_a^\iota \in P. \: a = \ocap \implies \forall o \in \accRoots{(FS, H)}. \: \csep{(H, o, o_g)} }
  {H \vdash P \tsep \gsep}

  \caption{Definition of global separation}
  \label{fig:def_gsep}
\end{figure}

% TODO include the \Gamma_{\CellType} definition somewhere


% Introduce the core language with syntax, type system and state properties like
% WT heap, isolation, well typed state




\chapter{Properties of RACL}
\label{cha:properties_of_racl}

% Write about the basic ideas behind the proof of preservation and progress

We are now finally ready to state and prove the progress and preservation
properties of RACL. These are then used to prove
quasi-determinism. 


\section{Preservation \& Progress}%
\label{sec:preservation_and_progress}

The preservation theorem states that if we have a well-typed state that can
step to another state, this will also be well-typed.

\begin{theorem}[Preservation]
  \label{thm:preservation}
  Let $S, S'$ be states such that $\vdash S \tsep~\stateok$ and $S \Rrightarrow
  S'$. Then $\vdash S' \tsep \stateok$.
\end{theorem}

The proof of preservation in its full can be found in
Appendix~\ref{cha:proof_of_pnp}. The part concerning preservation of isolation
is probably the most interesting since properties like well-typed heap and
threads are standard. The preservation proof of isolation is based on the
fact that, by OCAP reachability, all $\ocap$ threads can only reach $\ocap$
class objects. Combined with global object separation, this means that an
$\ocap$ thread can never access the global object or create non-$\ocap$ objects
through a $\New{...}$-expression. This implies that the thread follows the OCAP model
as described in Section~\ref{sec:the_object_capability_model}. Thus isolated
threads will stay isolated.

Theorem~\ref{thm:progress} states that a well-typed program never gets stuck
unless we encounter an error or finish execution properly.
\begin{theorem}[Progress]
  \label{thm:progress}
  Let $S$ be a state such that $\vdash S \tsep \stateok$. Then either 
  \begin{enumerate}
    \item $\exists S'$ s.t. $S \Rrightarrow S'$, 
    \item $S = H, \emptyset$ for some heap $H$ s.t. $\noSpawn{(H)}$ or
    \item $S = \Error$.
  \end{enumerate}
\end{theorem}

The proof of progress is by contradiction. The full proof can also be found in
Appendix~\ref{cha:proof_of_pnp}.
% Describe:
% Theorem of determinism
% Lemmas leading up to the proof of this

\section{Quasi-Determinism}%
\label{sec:quasi_determinism}

Having proved soundness of the type system, the next thing to prove is
quasi-determinism. In order to state this more formally, we need some
definitions.

We first define a relation on states and show that it is an equivalence
relation.
\begin{definition} \label{def:eqrel}
  Let $\simeq$ be a binary relation on the set of states $\States$.
  We let $S \simeq S'$ if
  \begin{equation*}
    S = S' = \Error
  \end{equation*}
  or if
  \begin{equation}
    S = H, P \andalso S' = H', P'
  \end{equation}
  and there exists bijections $g \in \OIDs \hookrightarrow \OIDs, h \in \TIDs
  \hookrightarrow \TIDs$ such that
  \begin{equation}
    H' = \pi(H, g) \andalso P' = \rho(P, g, h)
  \end{equation}
  
  The functions $\pi$ and $\rho$ replace all object and thread identifiers
  occuring in $H$ or $P$ according to the bijections $g$ and $h$. They are
  explicitly defined in Definition~\ref{def:pirho},
  Appendix~\ref{cha:proof_of_qd}.
\end{definition}



\begin{proposition} \label{prop:eqrel}
  $\simeq$ is an equivalence relation.
\end{proposition}

A proof sketch of this can be found in Appendix~\ref{cha:proof_of_qd}.
Similarly we can prove the following. 
\begin{proposition} \label{prop:eqrel_stateok}
  For any $S, S' \in \States$ such that $S \simeq S'$
  \begin{equation}
    \vdash S \tsep \stateok \iff \vdash S' \tsep \stateok
  \end{equation}
\end{proposition}
A proof sketch can also be found in Appendix~\ref{cha:proof_of_qd}.
We can also make the following observation
\begin{claim}
  A transition between two states can be uniquely identified by:
  \begin{itemize}
    \item A start state $S$
    \item A base rule name $R$, e.g., $R = \ENew$ or $R = \ESpawn$. We say base rule
      because in many cases the step involves more than one rule in its
      derivation tree of a complete state step. However the use of these extra
      rules are implied by which rule is used as the "base" of the tree. For
      example, using $\ECall$ to reduce a thread implies use of $\EFSProp$ to
      advance the state, and using $\ENew$ to advance one thread frame implies
      use of $\EFProp$ and $\EFSProp$ in order to advance the state.
    \item A thread identifier $\iota \in \TIDs$. In the cases where we advance or terminate
      a thread, e.g., $\ENull$ or $\ECall$, this is the thread identifier of
      that thread, as in Definition~\ref{def:thread_sets}. In the case where we
      spawn a new thread, this is the unique thread identifier as in
      Definition~\ref{def:heap_obj}.
    \item A fresh object or thread identifier $\beta$ which could be of value:
      \begin{itemize}
        \item Object identifier $o' \in \OIDs$ in the cases where the rule
          references a fresh object identifier, i.e. $\ENew$ and $\ENewCell$.
        \item Thread id $\iota' \in \TIDs$ for the case where a fresh thread id is
          referenced, i.e. $\EWhen$.
        \item $\smiley$, a default value for all other cases.
      \end{itemize}
  \end{itemize}
\end{claim}

Because of this claim we can make the following definition

\begin{definition} \label{def:trans_id}
  We write a \emph{transition identifier} as $R^{\iota, \beta}$. The
  \emph{application} of $R^{\iota, \beta}$ to state $S$ (if possible) is the
  use of a rule specified by $R$, $\iota$ and $\beta$ as described above, to
  step from $S$ to some state $S'$. We write this as
  \begin{equation*}
    S \Rrightarrow^{R^{\iota, \beta}} S'
  \end{equation*}
\end{definition}

\begin{remark}
  The rule name $R$ in the above definition is actually redundant since this is
  completely decided by the state $S$ and the thread identifier $\iota$.
\end{remark}

To simplify notation we make the following definition.

\begin{definition}
  A \emph{transition sequence} $\bar{R}$ is a finite length sequence of
  transition identifiers,
  \begin{equation*}
    \bar{R} = R_1^{\iota_1, \beta_1}, R_2^{\iota_2, \beta_2}, \dots,
    R_n^{\iota_n, \beta_n}.
  \end{equation*}
  The application of this sequence to a state $S$ is the consecutive application
  of
  \begin{equation*}
    R_i^{\iota_i, \beta_i} \andalso i = 1, \dots, n
  \end{equation*}
  beginning with state $S$. I.e.
  \begin{equation*} 
    S \Rrightarrow^{R_1^{\iota_1, \beta_1}} S_1 \Rrightarrow^{R^{\iota_2,
    \beta_2}} S_2  \: \dots \: S_{n-1} \Rrightarrow^{R_n^{\iota_n, \beta_n}} S_n
  \end{equation*}
  We shorten this to
  \begin{equation*}
    S \Rrightarrow^{\bar{R}} S_n.
  \end{equation*}
  We call $n$ the length of $\bar{R}$.
\end{definition}

We state the following proposition without proof, just for reference. This
follows from the observations made in the claim above.
\begin{proposition} \label{prop:uniq_trans}
  Let $H, P \Rrightarrow H', P'$. Then there is a unique transition identifier
  $R^{\alpha, \beta}$ such that $H, P \Rrightarrow^{R^{\alpha, \beta}} H', P'$.
\end{proposition}

We are now ready to state the last main theorem.

\begin{theorem} \label{thm:qd}
  Let $S, S', T, T'$ be well-typed states not equal to $\Error$ such that
  \begin{equation*}
    \begin{gathered}
      S \simeq S' \\
      S \Rrightarrow^{\bar{R}} T \andalso S' \Rrightarrow^{\bar{R'}} T',
    \end{gathered}
  \end{equation*}
  Let $n$ and $n'$ be the lengths of $\bar{R}$ and $\bar{R'}$ respectively.
  Furthermore we assume that neither $T$ or $T'$ can make a step. Then
  \begin{equation*}
    n = n' \quad \text{and} \quad T \simeq T'.
  \end{equation*}
\end{theorem}

The proof of Theorem~\ref{thm:qd} is based on induction on the length of the two
transition sequences. It can be seen as incrementally transforming $\bar{R'}$ to
$\bar{R}$ by moving transition identifiers to positions specified by $\bar{R}$.
Each intermediate sequence is shown to halt in the same state. The full proof
can be found in Appendix~\ref{cha:proof_of_qd}.

Finally, quasi-determinism follows from Theorem~\ref{thm:qd} and
proposition~\ref{prop:uniq_trans}:
\begin{corollary}[Quasi-Determinism]
  Let $S, S', T, T'$ be well-typed states not equal to $\Error$ such that
  \begin{equation*}
    \begin{gathered}
      S \simeq S' \\
      S \Rrightarrow^* T \andalso S' \Rrightarrow^* T',
    \end{gathered}
  \end{equation*}
  and neither $T$ or $T'$ can make a step.  Then
  \begin{equation*}
    T \simeq T'.
  \end{equation*}
\end{corollary}

\section{Summary}%
\label{sec:summary}

We have proved soundness for the formal model RACL. This means that, given that
a program is well typed under our type system, it will never reach an
error state unless it encounters a null-pointer exception. Furthermore, the
program will never halt unless it has reached an error state, or no threads are
running and no threads are waiting to spawn.

Furthermore we have proved that RACL posesses the property of quasi-determinism.
This means that all non-erroneous halting states of a well-typed program are
equivalent.

%
%\begin{remark}
%  You should be able to prove that our system posesses a property similar to the
%  confluence property proved by \textcite{confluence}. This should not be hard
%  using the 
%\end{remark}
%


\chapter{Discussion \& Conclusion}
\label{cha:discussion_and_conclusion}

In this chapter, RACL, the formal system of this work is discussed from two
angles: extensions and implementation. Finally the report is summarized with
conclusions.
% EXTENSIONS:
% TODO discuss quiescence: allows us to use the result and freeze values
% TODO discuss allowing gets
% TODO 

% Could this be in conclusions?
% GENERAL:
% TODO great care has to be taken when designing and proving things about
% deterministic-by-design systems

% IMPLEMENTATION
% Threshold set is not a viable approach and only an abstraction. In reality you
% would have to do something similar to the implementation of LVish: Atomic
% lattices. Have some
% function that generates callbacks: I.e. for the new value, spawn some new
% callbackthreads

\section{Possible Extensions}%
\label{sec:extensions}

Currently, RACL does not have any way to actually use the result of a
computation. That is, there is no way to access the final cell values. The
freeze operation of LVish~\parencite{kuper2014freeze} is flawed and allows us to
write non-deterministic programs.  However, as hinted in the end of
Section~\ref{sec:a_problem_of_lvish} there should be a way to freeze values and
access them without breaking determinism.  This uses a form of of quiescence,
the concept introduced by~\textcite{kuper2014freeze}. The difference is that
this new form of freezing needs to be done globally, and at a time where changes
to cell values are no longer going to take place. Quiescence allows us to do
this. For example, if no threads are running or waiting to spawn,
we can be sure that the cell values will not change. Therefore, with this
definition of quiescence, introducing a program statement which blocks until
quiescence and then freezes all cells, should allow us to access the result
without risking non-determinism. In order to ensure this does not cause
deadlocks, this \emph{on-quiescence-freeze} operation should only be allowed in
the sole main (non-$\ocap$) thread of RACL.

As mentioned in Section~\ref{sec:lvars}, LVars has an additional \emph{get}
operation, not modeled in RACL. Adopting the semantics of LVars, it should be
possible to extend RACL with this operation aswell.


\section{Implementation}%
\label{sec:implementation}

In order to implement RACL effectively, some changes have to be made in the way
callbacks are registered and spawned. This is mainly due to that the dependency
set sematics of our formal system are hard to implement effectively, since it
requires scanning the sets for threshold values that have been passed. The time
required for this operation is linear in the combined size of dependency sets.
The LVish implementation solved this by optimizing their library for so called
atomic lattices, i.e., lattices where all values can be written as a join of so
called \emph{atoms}. An atom $a$ is an element of the lattice $\LatVals$ where
$l \sqsubseteq a \implies l \in \{\bot, a\}$. Instead of using the threshold set
semantics, the LVish implementation uses the concept of


% TODO discussion: bring up fact of running callbacks twice does not
% affect result

% Discuss the addition of quiescence in order to actually use the result of the
% computation. Also discuss the efficiency of our system. In order to make it
% usable we need to replace the threshold checks with some callback that spawns
% the required callbacks. This of course is also required to not depend on any
% shared mutable state.
% Discuss threshold reads: Could introduce a get stmt. Draw some
% conclusions of the problematic nature of deterministic concurrency and that
% great care has to be taken when trying to prove determinism for so called
% "deterministic-by-design" systems.
% SHould also





\printbibliography[heading=bibintoc] % Print the bibliography (and make it appear in the table of contents)

\appendix

\chapter{Proof of Preservation and Progress}
\label{proof_of_pnp}

\begin{definition}{(Heap Induced Graph)}
  The induced graph of heap $H$, written $\Graph{(H)}$ is the directed graph
  $(V, E)$ such that $V = \dom{(H)}$ and 
  \begin{equation}
    E = \left\{ (o, o')_f \in \dom{(H)}^2 \times \FieldNames: 
      H(o) = \Obj{C, FM} \text{ and } FM(f) = o'
    \right\}
  \end{equation}
\end{definition}

\begin{definition}{(Graph Reachability)}
  We say an object $o'$ is reachable from $o$ in graph $G = (V,
  E)$, $\reach{(G, o, o')}$ if there is a finite sequence $o_0, \dots,
  o_n \in V$ such that $o = o_0, o_n = o'$ and
  \begin{equation}
    \forall i = 1,\dots, n-1. \: \exists f \in \FieldNames \text{ s.t. } (o_i,
    o_{i+1})_f \in E.
  \end{equation}
  The sequence $o_0, \dots, o_n$ is called a \emph{path}.
\end{definition}

\begin{definition}
  Given a graph $G = (V, E)$ and a set $O \subseteq V$, we define
  \begin{equation}
    \reachable{(O, G)} = \left\{ q \in V: \exists o\in O.\: \reach{(G, o, q)}
    \right\} \notag
  \end{equation}
\end{definition}

\begin{proposition}{(Reachability equivalence)}
  \label{prop:reacheq}
  For a heap $H$ 
  \begin{equation}
    \reach{(H, o, o')} \iff \reach{(\Graph{(H)}, o, o')}
  \end{equation}
\end{proposition}

\begin{proof}
  Both directions of implication are simple to prove.
  \begin{description}
    \item[Case $\implies$:] By induction on the shape of derivation tree.
    \item[Case $\impliedby$:] By induction on the path length in graph
      $\Graph{(H)}$.
  \end{description}
\end{proof}

% TODO change definition of ocrloc
\begin{proposition} For any heap $H$ and frame stack $FS$ we have
  \begin{flalign*}
    &\ocrloc{(FS, H)} &\\
    &\iff &\\
    &\begin{aligned}
    \forall o, q &\in \dom{(H)}. \\
    & \accRoot{(o, FS, H)} \land \reach{(H, o, q)} \implies \\
    & \ocap{(\typeOf{(o, H)})} 
    \end{aligned}&\\
    &\iff &\\
    &\begin{aligned}
    \forall q &\in \reachable{{(\accRoots{(FS,H)}, \Graph{(H)})}}. \notag\\
    & \ocap{(\typeOf{(q, H)})}
    \end{aligned}&
  \end{flalign*}
\end{proposition}

% TODO should I elaborate?
\begin{proof}
  Follows from definition of $\accRoot$, $\accRoots$ and $\reachable$.
\end{proof}

\begin{proposition} For any heap $H$ and object references $o, o'$ we have
  \begin{flalign*}
    &\csep{(H, o, o')} \iff &\\
    & \begin{aligned}
        \forall q \in \: &\reachable{(\{o\}, \Graph{(H)})} \cap \reachable{(\{o'\},
        \Graph{(H)})}. \\
        & \typeOf{(q, H)} \stof \CellType
    \end{aligned}&
  \end{flalign*}
\end{proposition}

% TODO proof 
\begin{proof}
  First of all we let $R = \reachable{(\{o\}, \Graph{(H)})}, R' =
  \reachable{(\{o'\}, \Graph{(H)})}$.
  We prove each direction of implication separately.
  \begin{description}
    \item[Case $\implies$:] Take any $q \in  R\cap R'$. By definition of
      $\reachable$, reachability equivalence (prop.~\ref{prop:reacheq}) and 
      definition of $\csep{(H, o, o')}$ we have $\typeOf{(q, H)} \stof \CellType$.
    \item[Case $\impliedby$:] Take any $q, q'\in \dom{(H)}$ such that
      $\reach{(H, o, q)}$ and $\reach{(H, o', q')}$. If $q \neq q$ we are done.
      Otherwise $q = q'$ and by definition of $\reachable$ $q \in R \cap R'$.
      Finally $\typeOf{(q, H)} \stof \CellType$ by assumption.
  \end{description}
\end{proof}

\begin{proposition} For any heap $H$ and frame stacks $FS, HS$ we have
  \begin{flalign*}
    &\isolated{(H, FS, HS)}  \iff &\\
    &\begin{aligned}
        \forall q \in \:&\reachable{(\accRoots{(FS, H)}, \Graph{(H)})} \cap \\
        & \reachable{(\accRoots{(HS, H)}, \Graph{(H)})}. \\
        & \typeOf{(q, H)} \stof \CellType
    \end{aligned}&
  \end{flalign*}
\end{proposition}

% TODO proof

\begin{proposition} \label{prop:simple_isolation_orc}
  Let $H, H'$ be heaps and $P, P'$ thread sets such that
  \begin{enumerate}
    \item $P = Q \cup_D \left\{ FS|_a^d \right\}$, $\isolation{(H, P)}$, $H
      \vdash P \tsep \ocr$ and $P' = Q \cup_D \left\{ FS'|_a^d \right\}$
    \item $\Graph{(H)} = \Graph{(H')}$
    \item $\forall HS|_b^e \in Q. \: \accRoots{(HS, H')} \subseteq \accRoots{(HS, H)}$
    \item $\accRoots{(FS', H')} \subseteq \accRoots{(FS, H)}$
    \item $\forall o \in \dom{(H)}. \: \typeOf{(o, H)} = \typeOf{(o, H')}$
  \end{enumerate}
  Then $\isolation{(H', P')}$ and $H' \vdash P' \tsep \ocr$.
\end{proposition}

\begin{remark}
  Note that $\Graph{(H)} = \Graph{(H')}$ implies $\dom{(H)} = \dom{(H')}$.
  Otherwise many of the preconditions above would not make sense.
\end{remark}

% TODO proof

\begin{proposition} \label{prop:simple_isolation_orc_gen}
  Let $H, H'$ be heaps and $P, P'$ thread sets such that
  \begin{enumerate}
    \item $P = Q \cup_D \left\{ FS|_a^d \right\}$, $\isolation{(H, P)}$, $H
      \vdash P \tsep \ocr$ and $P' = Q \cup_D \left\{ FS'|_a^d \right\}$
    \item $\Graph{(H)} = \Graph{(H')}$
    \item $\forall HS|_b^e \in Q.$ \\ 
      $\reachable{(\accRoots{(HS, H')}, \Graph{(H')})} \subseteq$ \\
      $\reachable{(\accRoots{(HS, H)}, \Graph{(H)})}$
    \item $\reachable{(\accRoots{(FS', H')}, \Graph{(H')})} \subseteq$ \\
      $\reachable{(\accRoots{(FS, H)}, \Graph{(H)})}$
    \item $\forall o \in \dom{(H)}. \: \typeOf{(o, H)} = \typeOf{(o, H')}$
  \end{enumerate}
  Then $\isolation{(H', P')}$ and $H' \vdash P' \tsep \ocr$.
\end{proposition}

\begin{remark}
  Note that this is a generalization of the previous proposition.
\end{remark}

% TODO proof (should just reference the previous one)

\begin{definition}
  Let $FS_g = \Frame{L_0}{global}{-} \circ \varepsilon$ bet the framestack where
  $L_0 = [global \mapsto o_g]$. Then for any thread set $P$, let $\tilde{P}$ be the
  thread set
  \begin{equation*}
    \tilde{P} = P \cup \left\{ FS_g|_{\nocap}^d \right\}
  \end{equation*}
  for some valid $d$.
\end{definition}

\begin{proposition} \label{prop:gsep_eq}
  For any heap $H$ and thread set $P$
  \begin{equation*}
    \isolation{(H, \tilde{P})} \iff \isolation{(H, P)} \text{ and } H \vdash P
    \tsep \gsep 
  \end{equation*}
\end{proposition}

%TODO check formulation of prop
\begin{proposition} \label{prop:gsep_simple}
  Let $H, P$ be such that $\vdash H$, $H \vdash P$ %and $\isolation{(H, P)}$ and
  and let $H', P'$ be such that $H, P \Rrightarrow H', P'$. Then if
  \begin{equation*}
    \isolation{(H,P)} \implies \isolation{(H', P')}
  \end{equation*}
  we have
  \begin{equation*}
    \isolation{(H, \tilde{P})} \implies \isolation{(H', \tilde{P'})}
  \end{equation*}
\end{proposition}

\begin{remark}
  This proposition can e.g. noteably be applied in the cases where we have
  applied prop.~\ref{prop:simple_isolation_orc}
  or~\ref{prop:simple_isolation_orc_gen} to prove isolation since the proof of
  these does not rely on global object separation.
\end{remark}

\begin{proposition}
  Let $H, H'$ be heaps. If $\dom{(H)} = \dom{(H')}$ and
  \begin{equation*}
    \forall o \in \dom{(H)}. \: \typeOf{(o, H)} = \typeOf{(o, H')}
  \end{equation*}
  then 
  \begin{equation*}
    \forall k \in \Values. \: \typeOf{(k, H)} = \typeOf{(k, H')}
  \end{equation*}
\end{proposition}

\begin{proof}
  Trivial.
\end{proof}

\begin{proposition} \label{prop:tprocseq}
  For any $H, P$ we have 
  \begin{equation}
    H \vdash P \iff \forall HS|_b^e \in P.\: H;b \vdash HS
  \end{equation}
\end{proposition}

\begin{proof} (Sketch) Done in each direction separately.
  \begin{description}
    \item[Case $\implies$:] By induction on the shape of the derivation tree.
    \item[Case $\impliedby$:] By induction on size of $P$.
  \end{description}
\end{proof}

% TODO define heap
\begin{proposition}
  Let $H, H'$ be heaps such that $\dom{(H)} \subseteq \dom{(H')}$ and 
  \begin{equation*}
    \forall o \in \dom{(H)}. \: \typeOf{(o, H)} = \typeOf{(o, H')}.
  \end{equation*}
  Then
  \begin{equation*}
    H; a \vdash FS \implies H'; a \vdash FS
  \end{equation*}
  and
  \begin{equation*}
    H; a \vdash^{x :\tau} FS \implies H'; a \vdash^{x: \tau} FS
  \end{equation*}
\end{proposition}

% TODO fix the statement
\begin{proposition}
  Let $HS|_b^e \in Q$ where $Q$ as in thm case {\sc E-Assign}. Then
  \begin{align*}
    &\reachable{(\accRoots{(FS',H')}, \Graph{(H')})} \subseteq \\
    &\reachable{(\accRoots{(FS,H)}, \Graph{(H)})}
  \end{align*}
  and if $a = \ocap$ or $b = \ocap$ then
  \begin{align*}
    &\reachable{(\accRoots{(HS,H')}, \Graph{(H')})} \subseteq \\
    &\reachable{(\accRoots{(HS,H)}, \Graph{(H)})}
  \end{align*}
\end{proposition}

\begin{corollary}
  If $b = \ocap$ then 
  \begin{align*}
    &\reachable{(\accRoots{(HS,H')}, \Graph{(H')})} \subseteq \\
    &\reachable{(\accRoots{(HS,H)}, \Graph{(H)})}
  \end{align*}
\end{corollary}

\begin{corollary}
  \begin{align*}
    &\reachable{(\accRoots{(FS',H')}, \Graph{(H')})} \subseteq \\
    &\reachable{(\accRoots{(FS,H)}, \Graph{(H)})}
  \end{align*}
\end{corollary}

\begin{proposition}
  Let $HS|_b^e \in Q$ where $Q$ as in thm case {\sc E-Assign}. Then
  \begin{align*}
    &\reachable{(\accRoots{(HS,H')}, \Graph{(H')})} \subseteq \\
    &\reachable{(\accRoots{(HS,H)}, \Graph{(H)})}\: \cup \\ 
    &\reachable{(\accRoots{(FS,H)}, \Graph{(H)})}
  \end{align*}
\end{proposition}

\begin{proposition}
  If 
  \[
    \forall o \in \dom{(H)} = \dom{(H')}. \: \typeOf{(o, H)} = \typeOf{(o,
  H')}
  \] 
  and $H \vdash \Gamma; L$ then $H' \vdash \Gamma; L$
\end{proposition}
\begin{proof}
  Trivial.
\end{proof}


% TODO 2.20 can it be included in 2.5 perhaps?

\section{Proof of preservation}
\label{sec:proof_of_preservation}

\begin{proof} 
  First of all we have that $S \neq \Error$ since no step can be made from this
  state. Thus $S = H, P$ and
  \begin{equation*}
    \vdash H \andalso H \vdash P \andalso H \vdash P \tsep \ocr \andalso
    \isolated{(H, P)} \andalso H \vdash P \tsep \gsep
  \end{equation*}
  We also assume we have $S' = H', P'$ since otherwise $S' = \Error$ and the
  theorem holds trivially. What remains is to prove
  \begin{equation*}
    \vdash H' \andalso H' \vdash P' \andalso H' \vdash P' \tsep \ocr \andalso
    \isolated{(H', P')} \andalso H' \vdash P' \tsep \gsep.
  \end{equation*}
  We proceed by cases.
  \begin{description}
    \item[Case {\sc E-FSProp}:] According to this rule $P = Q \cup_D \left\{
        FS|_a^d \right\}$ and $P' = Q \cup \left\{ FS'|_a^d \right\}$. We proceed by cases.
      \begin{description}
        \item[Case {\sc E-FProp}:] By the rule definition $FS|_a^d = F \circ
          GS|_a^d$ and $FS'|_a^d = F' \circ GS|_a^d$. Furthermore we can assume
          that $F = \sframe{L, t}$ and $F' = \sframe{L', t'}$.
          We proceed by cases.
          \begin{description}
            \item[Case {\sc E-Null}:] By this rule we have $t =
              \Let{x}{\NullVal}{t'}$, $H = H'$ and $L' = L[x \mapsto \NullVal]$.
              Immediatelly we have $\vdash H'$.

              By {\sc T-Procs} and proposition~\ref{prop:tprocseq} we have 
              \begin{equation} \label{eq:enull1}
               H \vdash Q  \andalso H; a \vdash FS 
              \end{equation}
              Trivially $H' \vdash Q$. 
              By~\eqref{eq:enull1} and rule {\sc T-FS1} we have
              \begin{equation} \label{eq:enull2}
                H \vdash \Gamma; L \andalso \TypeRel{\Gamma}{a}{t}{\sigma}
                \andalso H; a \vdash^{s: \sigma} GS.
              \end{equation}
              Let $\Gamma' = \Gamma, x: \NullType$. By {\sc T-Let}, {\sc
              T-Null} and \eqref{eq:enull2} we have
              \begin{equation} \label{eq:enull3}
                \TypeRel{\Gamma'}{a}{t'}{\sigma}.
              \end{equation}
              By {\sc T-FS1}, \eqref{eq:enull3} and \eqref{eq:enull2}
              \begin{equation}\label{eq:enull4}
                H;a \vdash FS'
              \end{equation}
              By \eqref{eq:enull4}, {\sc T-Procs} and $H = H'$ we get  
              \begin{equation}
                H' \vdash P'
              \end{equation}
              $H' \vdash P' \tsep \ocr$ and $\isolation{(H', P')}$ follows from
              proposition~\ref{prop:simple_isolation_orc}. Then $H' \vdash P'
              \tsep \gsep$ follows from propositions~\ref{prop:gsep_eq}
              and~\ref{prop:gsep_simple}.
            \item[Case {\sc E-LVal}:] Similarly.
            \item[Case {\sc E-Var}:] Similarly.
          \end{description}
      \end{description}
  \end{description}
\end{proof}













\chapter{Proof of Quasi-Determinism}
\label{cha:proof_of_qd}

\section{Preliminaries}
\label{sec:preliminaries}

\begin{definition}
  Let $M: A \rightharpoonup B, g: B \hookrightarrow B$. Then $\delta(M, g): A
  \rightharpoonup B$ and is defined as follows for any element $a \in A$.
  \begin{equation*}
    \delta(M, g)(a) = g(M(a))
  \end{equation*}
\end{definition}

\begin{definition}
  Let $g, h$ be as in definition \ref{def:pirho}. 
  \begin{equation*}
    \eta(\DEP, g, h) = \left\{ (l, (\delta(L_{\text{env}}, g), z \Rightarrow
    t))^{h(\iota)} \mid (l,
    (L_{\text{env}}, z \Rightarrow t))^\iota \in \DEP \right\}
  \end{equation*}
  I.e. we replace all object identifiers occuring in callback environments
  according to the replacement map $g$, and change the corresponding thread
  identifier $\iota$ to $h(\iota)$.
\end{definition}

\begin{definition} \label{def:pirho}
  Let $S = H, P$ and let 
  \begin{equation*}
    g \in \OIDs \hookrightarrow \OIDs \andalso h \in \TIDs \hookrightarrow \TIDs .
  \end{equation*}
  Then 
  \begin{equation}
    \pi(H, g, h) = H^* \in \Heaps
  \end{equation}
  where
  \begin{equation}
    \begin{gathered}
      H^*(o) =
      \begin{cases}
        \Obj{C, \FM^*}   & \text{ if } H(g^{-1}(o)) = \Obj{C, \FM} \\
        \Cell{\DEP^*, l} & \text{ if } H(g^{-1}(o)) = \Cell{\DEP, l}
      \end{cases} \\
      \FM^* = \delta(\FM, g) \andalso \DEP^* = \eta(\DEP, g, h).
    \end{gathered}
  \end{equation}

  Furthermore writing
  \begin{equation}
    P = \left\{ \FS_i|_{a_i}^{\iota_i} \right\}_{i = 1}^n
  \end{equation}
  we let
  \begin{equation*}
    \rho(P, g, h) = P^*
  \end{equation*}
  where
  \begin{equation*}
    \begin{gathered}
      \FS_i = \xframe{L_{m_i}, t_{m_i}}^{s_{m_i}} \circ \dots \circ \xframe{L_1,
      t_1}^{s_1} \circ \varepsilon \\
      \FS_i^* = \xframe{\delta(L_{m_i}, g), t_{m_i}}^{s_{m_i}} \circ \dots \circ
      \xframe{\delta(L_1,g), t_1}^{s_1} \circ \varepsilon \\
      P^* = \left\{ \FS_i^*|_{a_i}^{h(\iota_i)} \right\}_{i = 1}^n.
    \end{gathered}
  \end{equation*}
\end{definition}

\begin{remark}
  As noted earlier in Chapter~\ref{cha:properties_of_racl}, the $\pi$ and $\rho$
  functions more or less just replaces object and thread identifiers as
  specified by $g$ and $h$.
\end{remark}


\begin{proof}{(Proposition~\ref{prop:eqrel}, sketch)} 
  To prove that $\simeq$ is an equivalence relation we need to prove
  reflexivity, symmetry and transitivity.
  \begin{description}
    \item[Reflexivity:] Let $S$ be a state. If $S = \Error$ then reflexivity
      follows trivially. Thus let $S = H, P$. We need to prove that there are $g
      \in \OIDs \hookrightarrow \OIDs, h \in \TIDs \hookrightarrow \TIDs$ such
      that $H = \pi(H, g, h)$ and $P = \rho(P, g, h)$. This is easy since the
      identity functions
      \begin{equation*}
        g = \Id_{\OIDs} \andalso h = \Id_{\TIDs}
      \end{equation*}
      can be seen to fulfill this.

    \item[Symmetry:] Let $S \simeq S'$. If $S = \Error$ then $S' = \Error$ and
      we have $S' \simeq S$. If $S = H, P$ and $S' = H', P'$, we must have
      \begin{equation*}
        g \in \OIDs \hookrightarrow \OIDs \andalso h \in \TIDs
        \hookrightarrow \TIDs
      \end{equation*}
      such that
      \begin{equation*}
        H' = \pi(H, g, h) \andalso P' = \rho(P, g, h).
      \end{equation*}
      It can be verified that $g^{-1}$ and $h^{-1}$ are bijections such that
      \begin{equation*}
        H = \pi(H', g^{-1}, h^{-1}) \andalso P = \rho(P', g^{-1}, h^{-1}).
      \end{equation*}
      We thus get that $S' \simeq S$.

    \item[Transitivity:] Let $S \simeq S'$ and $S' \simeq S''$. The case where
      $S = \Error$ is trivial. Thus let $S = H,P; S' = H',P'; S'' = H'', P''$.
      We must have
      \begin{equation*}
        \begin{gathered}
          g\in \OIDs \hookrightarrow \OIDs \andalso h\in \TIDs
          \hookrightarrow \TIDs \\
          g'\in \OIDs \hookrightarrow \OIDs \andalso h'\in \TIDs
          \hookrightarrow \TIDs
        \end{gathered}
      \end{equation*}
      such that
      \begin{equation*}
        \begin{gathered}
          H' = \pi(H, g, h) \andalso P' = \rho(P, g, h)  \\
          H'' = \pi(H', g', h') \andalso P'' = \rho(P', g', h').
        \end{gathered}
      \end{equation*}
      It is easily shown that $g \circ g'$ and $h \circ h'$ are bijections such
      that 
      \begin{equation*}
        H'' = \pi(H', g \circ g', h \circ h') \andalso P'' = \rho(P', g \circ
        g', h \circ h').
      \end{equation*}
      Thus $S \simeq S''$.
  \end{description}
\end{proof}

\begin{proof}{(Proposition~\ref{prop:eqrel_stateok}, sketch)}
  Since $\simeq$ is an equivalence relation we only need to prove one direction.
  Thus assume
  \begin{equation}
    \vdash S \tsep \stateok.
  \end{equation}
  Letting $S = H, P$ and $S' = H', P'$, we need to show
  \begin{equation} 
    \begin{gathered}
      \vdash H' \andalso H' \vdash P' \andalso H' \vdash P' \tsep \ocr \\
      \isolation(H', P') \andalso H' \vdash P' \tsep \gsep \andalso \uniqMain(P').
    \end{gathered}
  \end{equation}
  Since the application of $\pi$ to $H$ amounts only to a renaming of the object
  and thread identifiers we have that $\Graph(H)$ is the same as $\Graph(H')$ up to a
  renaming of vertices. It does not change any types i.e. 
  \begin{equation} \label{eq:eqrel_stateok1}
    \typeOf(o, H) = \typeOf(g(o), H').
  \end{equation}
  Thus it is not hard to prove $\vdash H'$. 
  
  The application of $\rho$ similarly does the same kind of operation to $P$.
  Thus it is not hard proving $H' \vdash P'$ either.
  $H' \vdash P' \tsep \ocr$ follows from the graph equivalence mentioned above
  and \eqref{eq:eqrel_stateok1}. Using the same properties we can show
  $\isolation(H', P')$ and $H' \vdash P' \tsep \gsep$.

  Finally, since $\rho$ does not modify OCAP status of a thread, it is clear
  $\uniqMain(P')$ holds.
\end{proof}

\newpage
\section{Proof of Quasi-Determinism}
\label{sec:proof_of_quasi_determinism}

\begin{lemma} \label{lem:lemma1}
  Let $S$ be a state s.t. $\vdash S \tsep \stateok$ where the two transitions
  \begin{equation*}
    \begin{gathered}
      R_1^{\iota_1, \beta_1} \neq R_2^{\iota_2, \beta_2} \\
      \iota_1 \neq \iota_2 \andalso \beta_1 \neq \beta_2 \text{ or } \beta_1 =
      \beta_2 = - 
    \end{gathered}
  \end{equation*}
  are applicable.
  Let
  \begin{equation*}
    \bar{R}_1  = R_1^{\iota_1, \beta_1}, R_2^{\iota_2, \beta_2} \andalso
    \bar{R}_2 = R_2^{\iota_2, \beta_2}, R_1^{\iota_1, \beta_1}
  \end{equation*}
  Then both $\bar{R}_1, \bar{R}_2$ applicable to $S$ and
  \begin{equation*}
    \begin{gathered}
      S \Rrightarrow^{\bar{R}_1} S'_1 \andalso S \Rrightarrow^{\bar{R}_2} S'_2
      \\
      S'_1 = S'_2
    \end{gathered}
  \end{equation*}
\end{lemma}

\begin{figure}
  \centering
  \begin{tikzpicture}
    \node[circle,fill,minimum size=0.5cm,inner sep=0pt] (S) at (0,0) {};
    \node[below left=3pt] at (S) {$S$};
    \node[circle,fill,minimum size=0.5cm,inner sep=0pt] (S1) at (4,0) {};
    \node[below right=3pt] at (S1) {$S_1$};
    \node[circle,fill,minimum size=0.5cm,inner sep=0pt] (S2) at (0,4) {};
    \node[above left=3pt] at (S2) {$S_2$};
    \node[circle,fill,minimum size=0.5cm,inner sep=0pt] (S12) at (4,4) {};
    \node[above right=3pt] at (S12) {$S_{12} = S_{21}$};
    \draw[-{Latex[length=0.5cm]}] (S) -- node[below] {$R_1^{\iota_1, \beta_1}$} (S1);
    \draw[-{Latex[length=0.5cm]}] (S) -- node[left] {$R_2^{\iota_2, \beta_2}$} (S2);
    \draw[-{Latex[length=0.5cm]}] (S1) -- node[right] {$R_2^{\iota_2, \beta_2}$}
    (S12);
    \draw[-{Latex[length=0.5cm]}] (S2) -- node[above] {$R_1^{\iota_1, \beta_1}$}
    (S12);
  \end{tikzpicture}
  \caption{Both sequences $R_1^{\iota_1, \beta_1}, R_2^{\iota_2, \beta_2}$ and
  $R_2^{\iota_2, \beta_2}, R_1^{\iota_1, \beta_1}$ lead to the same end state.}
  \label{fig:lemma1_pf}
\end{figure}

\begin{proof}
  The can be done by cases on $R_1$ and $R_2$. However, there are a big number
  of cases, most of which are easy to verify. Thus we are only going to do a few
  here to convince the reader. The others can be done similarly. One thing to
  note is that the proofs are symmetric in $R_1$ and $R_2$, meaning that if we
  prove case $R_1 = R, R_2 = R'$ we have also proven case $R_1 = R', R_2 = R$.
  An illustration of what we are proving for each case can be found in
  Figure~\ref{fig:lemma1_pf}.

  \begin{description}
    \item[Case $R_1 = \EAssign, R_2 = \EAssign$:] We note that this means that
      $\beta_1 = \beta_2 = \smiley$. Furthermore by the rules being applicable
      to $S$ we have
      \begin{equation}
        \begin{gathered}
          S = H, P \andalso P = P_0 \cup_D \left\{ \FS_1|_{a_1}^{\iota_1},
          \FS_2|_{a_2}^{\iota_2} \right\} \\
          \FS_1 = \xframe{L_1, \Let{w_1}{\FAss{x_1}{f_1}{y_1}}{t'_1}}^{s_1} \circ \GS_1
          \\ 
          \FS_1 = \xframe{L_2, \Let{w_2}{\FAss{x_2}{f_2}{y_2}}{t'_2}}^{s_2}
          \circ \GS_2 \\
          L_1(x_1) = o_1 \andalso L_2(x_2) = o_2 \\
          H(o_1) = \Obj{C_1, \FM_1} \andalso H(o_2) = \Obj{C_2, \FM_2} \\
          f_1 \in \dom(\FM_1) \andalso f_2 \in \dom(\FM_2) \\
          S_1 = H_1, P_1 \andalso S \Rrightarrow^{R_1^{\iota_1, \smiley}} S_1
          \\
          P_1 = P_0 \cup_D \left\{ \FS'_1|_{a_1}^{\iota_1},
          \FS_2|_{a_2}^{\iota_2} \right\} \\
          \FS'_1 = \xframe{L'_1, t'_1}^{s_1} \circ \GS_1  \andalso L'_1 = L_1[w_1 \mapsto
          L_1(y_1)] \\
          \FM'_1 = \FM_1[f_1 \mapsto L_1(y_1)] \andalso H_1 = H[o_1 \mapsto
          \Obj{C_1, \FM'_1}]
          \\
          S_2 = H_2, P_2 \andalso S \Rrightarrow^{R_2^{\iota_2, \smiley}} S_2
          \\
          P_2 = P_0 \cup_D \left\{ \FS_1|_{a_1}^{\iota_1},
          \FS'_2|_{a_2}^{\iota_2} \right\} \\
          \FS'_2 = \xframe{L'_2, t'_2}^{s_2} \circ \GS_2  \andalso L'_2 = L_2[w_2 \mapsto
          L_2(y_2)] \\
          \FM'_2 = \FM_2[f_2 \mapsto L_2(y_2)] \andalso H_2 = H[o_2 \mapsto
          \Obj{C_2, \FM'_2}]
          \\
        \end{gathered}
      \end{equation}
      By $\vdash S \tsep \stateok$ we have
      \begin{equation}
        \uniqMain(P) \andalso \isolation(H, P).
      \end{equation}
      These two implies that
      \begin{equation}
        o_1 \neq o_2.
      \end{equation}
      Using this and inspecting $R_2^{\iota_2, \smiley}$ it is not hard to see
      that 
      \begin{equation}
        \begin{gathered}
          S_{12} = H_{12}, P_{12} \andalso S_1 \Rrightarrow^{R_2^{\iota_2,
          \smiley}} S_{12}
          \\
          P_{12} = P_0 \cup_D \left\{ \FS'_1|_{a_1}^{\iota_1},
          \FS'_2|_{a_2}^{\iota_2} \right\} \\
          H_{12} = H[o_1 \mapsto \Obj{C_1, \FM'_1}, o_2 \mapsto \Obj{C_2, \FM'_2}]
        \end{gathered}
      \end{equation}
      and similarly 
      \begin{equation}
        \begin{gathered}
          S_{21} = H_{21}, P_{21} \andalso S_2 \Rrightarrow^{R_1^{\iota_1,
          \smiley}} S_{21}
          \\
          P_{21} = P_0 \cup_D \left\{ \FS'_1|_{a_1}^{\iota_1},
          \FS'_2|_{a_2}^{\iota_2} \right\} \\
          H_{21} = H[o_1 \mapsto \Obj{C_1, \FM'_1}, o_2 \mapsto \Obj{C_2, \FM'_2}]
        \end{gathered}
      \end{equation}
      Clearly then $S_{12} = S_{21}$ and we are done.
      \begin{remark}
        The main point made here is that the two transition identifiers operate
        on different parts of the heap because of isolation of threads. Thus all
        cases where we have this property, e.g., $R_1 = \ESelect, R_2 =
        \EAssign$ follow similarly.
      \end{remark}
    \item[Case $R_1 = \EVar, R_2 = \EAssign$:] We first note that
      \begin{equation*}
        \beta_1 = \beta_2 = \smiley.
      \end{equation*}
      Since the transition identifiers are applicable to $S$ we have
      \begin{equation}
        \begin{gathered}
          S = H, P \andalso P = P_0 \cup_D \left\{ \FS_1|_{a_1}^{\iota_1},
          \FS_2|_{a_2}^{\iota_2} \right\} \\
          \FS_1 = \xframe{L_1, \Let{x_1}{y_1}{t'_1}}^{s_1} \circ \GS_1
          \\ 
          \FS_1 = \xframe{L_2, \Let{x_2}{\FAss{y_2}{f_2}{z_2}}{t'_2}}^{s_2}
          \circ \GS_2 \\
          L_2(y_2) = o_2 \andalso H(o_2) = \Obj{C_2, \FM_2} \\
          f_2 \in \dom(\FM_2) \\
          S_1 = H_1, P_1 \andalso S \Rrightarrow^{R_1^{\iota_1, \smiley}} S_1
          \\
          P_1 = P_0 \cup_D \left\{ \FS'_1|_{a_1}^{\iota_1},
          \FS_2|_{a_2}^{\iota_2} \right\} \\
          \FS'_1 = \xframe{L'_1, t'_1}^{s_1} \circ \GS_1  \andalso L'_1 = L_1[x_1 \mapsto
          L_1(y_1)] \\
          H_1 = H
          \\
          S_2 = H_2, P_2 \andalso S \Rrightarrow^{R_2^{\iota_2, \smiley}} S_2
          \\
          P_2 = P_0 \cup_D \left\{ \FS_1|_{a_1}^{\iota_1},
          \FS'_2|_{a_2}^{\iota_2} \right\} \\
          \FS'_2 = \xframe{L'_2, t'_2}^{s_2} \circ \GS_2  \andalso L'_2 = L_2[x_2 \mapsto
          L_2(z_2)] \\
          \FM'_2 = \FM_2[f_2 \mapsto L_2(y_2)] \andalso H_2 = H[o_2 \mapsto
          \Obj{C_2, \FM'_2}]
          \\
        \end{gathered}
      \end{equation}
      Furthermore it is not hard to see that 
      \begin{equation}
        \begin{gathered}
          S_{12} = H_{12}, P_{12} \andalso S_1 \Rrightarrow^{R_2^{\iota_2,
          \smiley}} S_{12}
          \\
          P_{12} = P_0 \cup_D \left\{ \FS'_1|_{a_1}^{\iota_1},
          \FS'_2|_{a_2}^{\iota_2} \right\} \\
          H_{12} = H[o_2 \mapsto \Obj{C_2, \FM'_2}]
        \end{gathered}
      \end{equation}
      and 
      \begin{equation}
        \begin{gathered}
          S_{21} = H_{21}, P_{21} \andalso S_2 \Rrightarrow^{R_1^{\iota_1,
          \smiley}} S_{21}
          \\
          P_{21} = P_0 \cup_D \left\{ \FS'_1|_{a_1}^{\iota_1},
          \FS'_2|_{a_2}^{\iota_2} \right\} \\
          H_{21} = H[o_2 \mapsto \Obj{C_2, \FM'_2}]
        \end{gathered}
      \end{equation}
      Clearly $S_{12} = S_{21}$ and we are done.
      \begin{remark}
        This case is very simple since one of the transition identifiers
        operate solely on the local state of a thread. All cases with this
        property follow similarly.
      \end{remark}
    \item[Case $R_1 = \EPut, R_2 = \EPut$:] Clearly
      \begin{equation*}
        \beta_1 = \beta_2 = \smiley.
      \end{equation*}
      Similarly to previous cases we have
      \begin{equation}
        \begin{gathered}
          S = H, P \andalso P = P_0 \cup_D \left\{ \FS_1|_{a_1}^{\iota_1},
          \FS_2|_{a_2}^{\iota_2} \right\} \\
          \FS_1 = \xframe{L_1, \Let{x_1}{\Put{y_1}{z_1}}{t'_1}}^{s_1} \circ \GS_1
          \\ 
          \FS_1 = \xframe{L_2, \Let{x_2}{\Put{y_2}{z_2}}{t'_2}}^{s_2}
          \circ \GS_2 \\
          L_1(y_1) = o_1 \andalso L_2(y_2) = o_2 \\
          L_1(z_1) = l_1 \andalso L_2(z_2) = l_2
        \end{gathered}
      \end{equation}
      Now we have two cases since isolation does not prevent sharing of
      references to $\CellType$ objects. If $o_1 \neq o_2$ we are done similarly
      to case $R_1 = \EAssign, R_2 = \EAssign$. If $o_1 = o_2 = o$ we proceed as
      follows.
      
      First of all,
      \begin{equation}
        H(o) = \Cell{\DEP, l}.
      \end{equation}
      Then
      \begin{equation}
        \begin{gathered}
          S_1 = H_1, P_1 \andalso S \Rrightarrow^{R_1^{\iota_1, \smiley}} S_1
          \\
          P_1 = P_0 \cup_D \left\{ \FS'_1|_{a_1}^{\iota_1},
          \FS_2|_{a_2}^{\iota_2} \right\} \\
          \FS'_1 = \xframe{L'_1, t'_1}^{s_1} \circ \GS_1  \andalso L'_1 = L_1[x_1 \mapsto
          L_1(y_1)] \\
          H_1 = H[o \mapsto \Cell{\DEP, l \sqcup l_1}]
        \end{gathered}
      \end{equation}
      and similarly
      \begin{equation}
        \begin{gathered}
          S_2 = H_2, P_2 \andalso S \Rrightarrow^{R_2^{\iota_2, \smiley}} S_2
          \\
          P_1 = P_0 \cup_D \left\{ \FS_1|_{a_1}^{\iota_1},
          \FS'_2|_{a_2}^{\iota_2} \right\} \\
          \FS'_2 = \xframe{L'_2, t'_2}^{s_2} \circ \GS_2  \andalso L'_2 = L_2[x_2 \mapsto
          L_2(y_2)] \\
          H_2 = H[o \mapsto \Cell{\DEP, l \sqcup l_2}]
        \end{gathered}
      \end{equation}
      We also have
      \begin{equation}
        \begin{gathered}
          S_{12} = H_{12}, P_{12} \andalso S_1 \Rrightarrow^{R_2^{\iota_2,
          \smiley}} S_{12}
          \\
          P_{12} = P_0 \cup_D \left\{ \FS'_1|_{a_1}^{\iota_1},
          \FS'_2|_{a_2}^{\iota_2} \right\} \\
          H_{12} = H[o \mapsto \Cell{\DEP, (l \sqcup l_1) \sqcup l_2}]
        \end{gathered}
      \end{equation}
      and
      \begin{equation}
        \begin{gathered}
          S_{21} = H_{21}, P_{21} \andalso S_2 \Rrightarrow^{R_1^{\iota_1,
          \smiley}} S_{21}
          \\
          P_{21} = P_0 \cup_D \left\{ \FS'_1|_{a_1}^{\iota_1},
          \FS'_2|_{a_2}^{\iota_2} \right\} \\
          H_{21} = H[o \mapsto \Cell{\DEP, (l \sqcup l_2) \sqcup l_1}]
        \end{gathered}
      \end{equation}
      By commutability of the least upper bound operation $\sqcup$ we have
      \begin{equation}
        (l \sqcup l_1) \sqcup l_2 = (l \sqcup l_2) \sqcup l_1
      \end{equation}
      and thus we have $S_{12} = S_{21}$.

    \item[Case $R_1 = \EWhen, R_2 = \EWhen$:] We have that
      \begin{equation}
        \beta_1 = \iota'_1 \andalso \beta_2 = \iota'_2 \andalso \iota'_1 \neq
        \iota'_2.
      \end{equation}
      Similarly to previous cases we have
      \begin{equation}
        \begin{gathered}
          S = H, P \andalso P = P_0 \cup_D \left\{ \FS_1|_{a_1}^{\iota_1},
          \FS_2|_{a_2}^{\iota_2} \right\} \\
          \FS_1 = \xframe{L_1, \Let{x_1}{\When{y_1}{z_1}{\dots}}{t'_1}}^{s_1} \circ \GS_1
          \\ 
          \FS_1 = \xframe{L_2, \Let{x_2}{\When{y_2}{z_2}{\dots}}{t'_2}}^{s_2}
          \circ \GS_2 \\
          L_1(y_1) = o_1 \andalso L_2(y_2) = o_2 \\
          L_1(z_1) = l_1 \andalso L_2(z_2) = l_2
        \end{gathered}
      \end{equation}
      Where we use $\dots$ for things that are not really vital.

      If $o_1 \neq o_2$ we are done similar to previous case. If $o_1 = o_2 = o$
      then we proceed as follows. We use the notation of
      $H_{\text{\#}}, P_{\text{\#}}$ similar to earlier cases. It
      is simple to see that
      \begin{equation}
        \begin{gathered}
          H_1 = H[o \mapsto \Cell{\DEP_1, l}] \andalso \DEP_1 = \DEP \cup \{ (l_1,
          \dots)^{\iota'_1} \} \\
          H_2 = H[o \mapsto \Cell{\DEP_2, l}] \andalso \DEP_2 = \DEP \cup \{ (l_2,
          \dots)^{\iota'_2} \} \\
          H_{12} = H[o \mapsto \Cell{\DEP_{12}, l}] \\ \DEP_{12} = \DEP \cup \{ (l_1,
          \dots)^{\iota'_1}, (l_2, \dots)^{\iota'_2} \} \\
          H_{21} = H[o \mapsto \Cell{\DEP_{21}, l}] \\ \DEP_{21} = \DEP \cup \{ (l_1,
          \dots)^{\iota'_1}, (l_2, \dots)^{\iota'_2} \} \\
        \end{gathered}
      \end{equation}
      Clearly $\DEP_{12} = \DEP_{21}$, which implies $H_{12} = H_{21}$.
      Similarly to earlier cases, $P_{12} = P_{21}$ and thus $S_{12} = S_{21}$.

    \item[Case $R_1 = \EWhen, R_2 = \ESpawn$:] First 
      \begin{equation}
        \beta_1 = \iota'_1 \andalso \beta_2 = \smiley.
      \end{equation}
      We have
      \begin{equation}
        \begin{gathered}
          S = H, P \andalso P = P_0 \cup_D \left\{ \FS_1|_{a_1}^{\iota_1}
          \right\} \\
          \FS_1 = \xframe{L_1, \Let{x_1}{\When{y_1}{z_1}{\dots}}{t'_1}}^{s_1} \circ \GS_1
          \\ 
          L_1(y_1) = o_1 \andalso L_1(z_1) = l_1 \\
          o_2 \in \dom(H) \andalso H(o_2) = \Cell{\DEP, l} \\
          (l_2, (L_{\text{env}}, z_2 \Rightarrow t''))^{\iota_2} \in \DEP
        \end{gathered}
      \end{equation}
      Similarly to previous cases we have the case where $o_1 \neq o_2$ which
      follows similarly to case $R_1 = \EAssign, R_2 = \EAssign$ and the case
      $o_1 = o_2 = o$ for which we proceed as follows.

      First we note that
      \begin{equation}
        H(o) = \Cell{\DEP, l}.
      \end{equation}
      We have that 
      \begin{equation} \label{eq:lem1_ewhen_espawn_1}
        \begin{gathered}    
          S_1 = H_1, P_1 \andalso S \Rrightarrow^{R_1^{\iota_1, \beta_1}} S_1
          \\
          P_1 = P_0 \cup_D \left\{ \FS'_1|_{a_1}^{\iota_1} \right\} \\
          \FS'_1 = \xframe{L'_1, t'_1}^{s_1} \circ \GS_1  \andalso L'_1 = L_1[x_1 \mapsto
          L_1(y_1)] \\
          H_1 = H[o \mapsto \Cell{\DEP_1, l}] \andalso \DEP_1 = \DEP \cup \{ (l_1,
          \dots)^{\iota'_1} \}
        \end{gathered}
      \end{equation}
      Stepping accoriding to $R_2^{\iota_2, \smiley}$ from $S$ gives
      \begin{equation} \label{eq:lem1_ewhen_espawn_2}
        \begin{gathered}
          S_2 = H_2, P_2 \andalso S \Rrightarrow^{R_2^{\iota_2, \smiley}} S_2
          \\
          P_2 = P_0 \cup_D \left\{ \FS_1|_{a_1}^{\iota_1},
          \FS_2|_{\ocap}^{\iota_2} \right\} \\
          \FS_2 = \xframe{L_{\text{env}}[z \mapsto l_2], t''}^{-} \circ
          \varepsilon \\
          H_2 = \Cell{\DEP_2, l} \andalso \DEP_2 = \DEP \setminus \{ (l_2,
          \dots)^{\iota_2} \}
        \end{gathered}
      \end{equation}
      Given \eqref{eq:lem1_ewhen_espawn_1} and \eqref{eq:lem1_ewhen_espawn_2} it
      is not hard to verify that
      \begin{equation} 
        \begin{gathered}
          S_{12} = H_{12}, P_{12} \andalso S_1 \Rrightarrow^{R_2^{\iota_2,
          \smiley}} S_{12}
          \\
          P_{12} = P_0 \cup_D \left\{ \FS'_1|_{a_1}^{\iota_1},
          \FS_2|_{\ocap}^{\iota_2} \right\} \\
          H_{12} = H[o \mapsto \Cell{\DEP_{12}, l}] \\ 
          \DEP_{12} = \DEP \cup \{ (l_1, \dots)^{\iota'_1} \} \setminus \{ (l_2,
          \dots)^{\iota_2} \}
        \end{gathered}
      \end{equation}
      \begin{equation} 
        \begin{gathered}
          S_{21} = H_{21}, P_{21} \andalso S_2 \Rrightarrow^{R_1^{\iota_1,
          \iota'_1}} S_{21}
          \\
          P_{21} = P_0 \cup_D \left\{ \FS'_1|_{a_1}^{\iota_1},
          \FS_2|_{\ocap}^{\iota_2} \right\} \\
          H_{21} = H[o \mapsto \Cell{\DEP_{21}, l}] \\ 
          \DEP_{21} = \DEP \setminus \{ (l_2, \dots)^{\iota_2} \} \cup \{ (l_1,
          \dots)^{\iota'_1} \} 
        \end{gathered}
      \end{equation}
      Since $\iota'_1$ is fresh it is clear that $\DEP_{12} = \DEP_{21}$ and
      therefore we have $S_{12} = S_{21}$.
      \begin{remark}
        The main point here is that $\iota'_1$ being a fresh thread identifier
        makes the removal and addition to the dependency set $\DEP$ commute.
      \end{remark}
  \end{description}
  It should not be hard to verify the other cases in a similar manner.
\end{proof}

\begin{definition}
  Let $R_1^{\iota_1, \beta_1}, R_2^{\iota_2, \beta_2}$ be two transition
  identifiers. We call $\beta_1$ and $\beta_2$ compatible if
  \begin{equation*}
    \beta_1 \neq \beta_2 \text{ or } \beta_1 = \beta_2 = \smiley.
  \end{equation*}
\end{definition}

\begin{lemma} \label{lem:lemma2}
  Let $S$ be a well typed state. Let
  \begin{equation*}
    \bar{R} = R_1^{\iota_1, \beta_1}, \dots, R_n^{\iota_n, \beta_n} \andalso
    S \Rrightarrow^{\bar{R}} S_{\bar{R}}
  \end{equation*}
  Let $R^{\iota, \beta}$ be a transition such that 
  \begin{equation*}
    S \Rrightarrow^{R^{\iota, \beta}} S_{R^{\iota, \beta}} \andalso
    S_{\bar{R}} \Rrightarrow^{R^{\iota, \beta}} S',
  \end{equation*}
  for some states $S_{R^{\iota, \beta}}$ and $S'$, and
  \begin{equation} \label{eq:lemma2_1}
    \forall i \in \left\{ 1, \dots, n \right\}. \quad \iota \neq \iota_i  .
  \end{equation}
  Furthermore let $\beta$ be compatible with $\beta_i$ for all $i = 1, \dots,
  n$.
  Then
  \begin{equation*}
    S \Rrightarrow^{R^{\iota, \beta}} S_{R^{\iota, \beta}} \Rrightarrow^{\bar{R}} S'.
  \end{equation*}
\end{lemma}

\begin{figure}
  \centering
  \begin{tikzpicture}
    \node[circle,fill,minimum size=0.2cm,inner sep=0pt] (S) at (0,0) {};
    \node[below left] at (S) {$S$};
    \node[circle,fill,minimum size=0.2cm,inner sep=0pt] (Sk) at (0,5) {};
    \node[below left] at (Sk) {$S_{\bar{R}}$};
    \node[circle,fill,minimum size=0.2cm,inner sep=0pt] (T) at (0,6) {};
    \node[above left, red] at (T) {$S'$};
    \node[circle,fill,minimum size=0.2cm,inner sep=0pt] (Tmod) at (1,6) {};
    \node[above right, red] at (Tmod) {$S'$};
    \node[circle,fill,minimum size=0.2cm,inner sep=0pt] (Siotabeta) at (1,1) {};
    \node[below right] at (Siotabeta) {$S_{R^{\iota,\beta}}$};
    \node[circle,fill,minimum size=0.2cm,inner sep=0pt] (S1) at (0,1) {};
    \node[above left] at (S1) {$S_{1}$};
    \node[circle,fill,minimum size=0.2cm,inner sep=0pt] (S1iotabeta) at (1,2) {};
    \node[above right] at (S1iotabeta) {$S_{1,R^{\iota,\beta}}$};
    \draw[-{Latex[length=0.2cm]}] (S1) -- node[left] {$\bar{R}_{\geq 2}$} (Sk);
    \draw[-{Latex[length=0.2cm]},blue] (Sk) -- node[left] {$R^{\iota, \beta}$}
    (T);
    \draw[-{Latex[length=0.2cm]},blue] (S)-- node[below right] {$R^{\iota, \beta}$}
    (Siotabeta);
    \draw[-{Latex[length=0.2cm]}] (Siotabeta) -- node[right] {$\bar{R}_{\geq 2}$} (Tmod);
    \draw[-{Latex[length=0.2cm]}] (S) -- node[left] {$R_1^{\iota_1, \beta_1}$} (S1);
    \draw[-{Latex[length=0.2cm]}] (Siotabeta) -- node[right] {$R_1^{\iota_1,
    \beta_1}$} (S1iotabeta);
    \draw[-{Latex[length=0.2cm]},blue] (S1) -- node[above=3pt] {$R^{\iota,
    \beta}$} (S1iotabeta);
    %\draw[-{Latex[length=0.3cm]}] (c1) -- node[above] {$f_1$} (c3);
    \draw[decoration={brace,raise=40pt,amplitude=10pt},decorate,grey] (S) --
    node[left=50pt]
    {$\bar{R}$} (Sk);
    \draw[decoration={brace,raise=40pt,amplitude=10pt,mirror},decorate,grey] (Siotabeta) --
    node[right=50pt]
    {$\bar{R}$} (Tmod);
  \end{tikzpicture}
  \caption{$R^{\iota, \beta}$ is applicable at $S_1$. By induction hypothesis
  $S_1 \Rrightarrow^{R^{\iota,\beta}} S_{1, R^{\iota, \beta}}
  \Rrightarrow^{\bar{R}_{\geq 2}} S'$. By Lemma~\ref{lem:lemma1} $S
  \Rrightarrow^{R^{\iota, \beta}, R_1^{\iota_1, \beta_1}} S_{1, R^{\iota,
  \beta}}$. Thus $S \Rrightarrow^{R^{\iota,\beta}, \bar{R}} S'$.}
  \label{fig:lemma2_pf}
\end{figure}

% TODO make figure explaining the proof
\begin{proof}
  By induction on $n$, the length of $\bar{R}$.
  \begin{description}
    \item[Case $n = 1$:] Follows immediately from Lemma~\ref{lem:lemma1}.
    \item[Case $n = i+1, i \geq 1$:] This case can be visualized with 
      Figure~\ref{fig:lemma2_pf}. Let
      \begin{equation*}
        \bar{R}_{\geq 2} = R_2^{\iota_2, \beta_2}, \dots, R_n^{\iota_n, \beta_n}.
      \end{equation*}
      Since $\bar{R}$ is applicable to $S$ we have 
      \begin{equation}
        S \Rrightarrow^{R_1^{\iota_1, \beta_1}} S_1,
      \end{equation}
      for some state $S_1$. Clearly
      \begin{equation}
        S_1 \Rrightarrow^{\bar{R}_{\geq 2}} S_{\bar{R}}.
      \end{equation}
      
      By \eqref{eq:lemma2_1}, thread $\iota$ never executes in the sequence
      specified by $\bar{R}$. Combining this with $R^{\iota, \beta}$ being
      applicable at $S$ and the compatibility of $\beta$, it is not hard to see
      that $R^{\iota, \beta}$ is applicable at $S_1$, i.e.,
      \begin{equation}
        S_1 \Rrightarrow^{R^{\iota, \beta}} S_{1,R^{\iota, \beta}}
      \end{equation}
      for some state $S_{1,R^{\iota, \beta}}$. Then, by induction hypothesis
      \begin{equation} \label{eq:lemma2_2}
        S_1 \Rrightarrow^{R^{\iota, \beta}} S_{1,R^{\iota, \beta}} \Rrightarrow^{\bar{R}_{\geq 2}} S'.
      \end{equation}
      Furthermore, $R_1^{\iota_1, \beta_1}$ and $R^{\iota, \beta}$ fulfills
      preconditions of Lemma~\ref{lem:lemma1}. Using this lemma we get
      \begin{equation} \label{eq:lemma2_3}
        S \Rrightarrow^{R^{\iota, \beta}} S_{R^{\iota, \beta}}
        \Rrightarrow^{R_1^{\iota_1, \beta_1}} S_{1, R^{\iota, \beta}}.
      \end{equation}
      But then combining \eqref{eq:lemma2_2} and \eqref{eq:lemma2_3} we have
      \begin{equation}
        S \Rrightarrow^{R^{\iota, \beta}} S_{R^{\iota, \beta}} \Rrightarrow^{\bar{R}} S'.
      \end{equation}
  \end{description}
\end{proof}


\begin{proposition}
  Let $S_1 = H_1, P_1 \simeq H_2, P_2 = S_2$ be well typed states and let $g, h$
  be the bijections such that
  \begin{equation} \label{eq:propnewgh1}
    H_2 = \pi(H_1, g, h) \andalso P_2 = \rho(P_1, g, h).
  \end{equation}
  Let $o_1, o_2 \in \OIDs$ be fresh at $S_1, S_2$ respectively.
  Let $\iota_1, \iota_2 \in \TIDs$ be fresh at $S_1, S_2$ respectively.
  Then there are bijections
  \begin{equation*}
    g' \in \OIDs \hookrightarrow \OIDs \andalso h' \in \TIDs \hookrightarrow \TIDs
  \end{equation*}
  such that 
  \begin{equation} \label{eq:propnewgh2}
    g'(o_1) = o_2 \andalso h'(\iota_1) = \iota_2
  \end{equation} 
  and
  \begin{equation} \label{eq:propnewgh3}
    H_2 = \pi(H_1, g', h') \andalso P_2 = \rho(P_1, g', h').
  \end{equation}
\end{proposition}

\begin{proof}
  The role of $g$ and $h$ is merely to define a renaming of identifiers in
  $\OIDs(S_1)$ and $\TIDs(S_1)$ to identifiers in $\OIDs(S_2)$ and $\TIDs(S_2)$.
  Thus it should not be hard to see that
  \begin{equation*}
    \begin{gathered}
      \forall o \in \OIDs(S_1). \quad g(o) \in \OIDs(S_2) \\
      \forall \iota \in \TIDs(S_1). \quad h(\iota) \in \TIDs(S_2)
    \end{gathered}
  \end{equation*}
  Since $g$ and $h$ are bijections,
  \begin{equation*}
    \begin{gathered}
      g(o_1) \not\in \OIDs(S_2) \andalso h(\iota_1) \not\in \TIDs(S_2) \\
      g^{-1}(o_2) \not\in \OIDs(S_1) \andalso h^{-1}(\iota_2) \not\in \TIDs(S_1).
    \end{gathered}
  \end{equation*}
  Because of the above, it should be clear that the following functions are well
  defined, are bijections and satisfy both \eqref{eq:propnewgh2} and
  \eqref{eq:propnewgh3}.
  \begin{equation*}
    \begin{gathered}
      g'(o) =
      \begin{cases}
        o_2 & \text{ if } o = o_1 \\
        g(o_1) &\text{ if } g(o) = o_2 \\
        g(o) & \text{ otherwise.} 
      \end{cases} \\
      h'(\iota) =
      \begin{cases}
        \iota_2 & \text{ if } \iota = \iota_1 \\
        h(\iota_1) & \text{ if } h(\iota) = \iota_2 \\
        h(\iota) & \text{ otherwise.} 
      \end{cases}
    \end{gathered}
  \end{equation*}
\end{proof}

Because of this proposition, the following is well defined.

\begin{definition} \label{def:bijectionmod}
  Let $S_1 = H_1, P_1 \simeq H_2, P_2 = S_2$ be well typed states and let $g, h$
  be the bijections such that
  \begin{equation} 
    H_2 = \pi(H_1, g, h) \andalso P_2 = \rho(P_1, g, h).
  \end{equation}
  Let $o_1, o_2 \in \OIDs$ be fresh at $S_1, S_2$ respectively.
  Let $\iota_1, \iota_2 \in \TIDs$ be fresh at $S_1, S_2$ respectively.
  Then define
  \begin{equation}
    g[o_1 \mapsto o_2] \in \OIDs \hookrightarrow \OIDs 
    \andalso h[\iota_1 \mapsto \iota_2] \in \TIDs \hookrightarrow \TIDs
  \end{equation}
  as any bijections satisfying \eqref{eq:propnewgh2} and \eqref{eq:propnewgh3}
  with $g' = g[o_1 \mapsto o_2], h' = h[\iota_1 \mapsto \iota_2]$.
\end{definition}

\begin{lemma} \label{lem:lemma3}
  Let $S, S', T$ be well typed states and $\bar{R}$ a transition sequence such that
  \begin{equation*}
    S \Rrightarrow^{\bar{R}} T \andalso S \simeq S'.
  \end{equation*}
  Then there is a state $T'$ and a transition sequence $\bar{R'}$ of the same
  length as $\bar{R}$ such that
  \begin{equation*}
    S' \Rrightarrow^{\bar{R'}} T' \andalso T \simeq T'.
  \end{equation*}
\end{lemma}


\begin{proof}
  By induction on the length $n$ of $\bar{R}$.
  \begin{description}
    \item[Case $n = 0$:] $\bar{R}$ is the empty sequence and thus $S = T$. Take
      $T' = S'$ and $\bar{R'} = \bar{R}$ and we are done.
    \item[Case $n = i + 1, i \geq 0$:] Let
      \begin{equation*}
        \begin{gathered}
          \bar{R} = R_1^{\iota_1, \beta_1}, \dots, R_{i+1}^{\iota_{i+1},
          \beta_{i+1}} \\
          \bar{R}_{\leq i} = R_1^{\iota_1, \beta_1}, \dots, R_{i}^{\iota_{i},
          \beta_{i}}
        \end{gathered}
      \end{equation*}
      By assumption
      \begin{equation*}
        S \Rrightarrow^{\bar{R}_{\leq i}} S_i \Rrightarrow^{R_{i+1}^{\iota_{i+1},
        \beta_{i+1}}} T
      \end{equation*}
      for some state $S_i$. By induction hypothesis there is $\bar{R'}_{\leq i}, S'_i$
      such that
      \begin{equation}
        S' \Rrightarrow^{\bar{R'}_{\leq i}} S'_i \andalso S_i \simeq S'_i.
      \end{equation}
      Since $S_i$ can take another step
      \begin{equation*}
        S_i = H, P \andalso S'_i = H', P'.
      \end{equation*}
      
      Now consider $R_{i+1}^{\iota_{i+1}, \beta_{i+1}}$. Since $S_i \simeq
      S'_i$ there are
      \begin{equation*}
        g \in \OIDs \hookrightarrow \OIDs \andalso h \in \TIDs \hookrightarrow \TIDs
      \end{equation*}
      such that
      \begin{equation*}
        H' = \pi(H, g, h)  \andalso P' = \eta(P, g, h).
      \end{equation*}
      Now let 
      \begin{equation*}
        \begin{gathered}
          R = R_{i+1} \andalso \iota = h(\iota_{i+1}) \\
          \beta = 
          \begin{cases}
            \smiley         & \text{ if } \beta_{i+1} = \smiley \\
            o'_{\text{new}} & 
            \begin{aligned}[t]
              \text{ if }&\beta_{i+1} = o_{\text{new}} \in \OIDs \\
              &o_{\text{new}} \text{ is fresh at } S_i \\
              &o'_{\text{new}} \text{ is fresh at } S'_i 
            \end{aligned} \\
            \iota'_{\text{new}} & \begin{aligned}[t]
              \text{ if }&\beta_{i+1} = \iota_{\text{new}} \in \TIDs \\
              &\iota_{\text{new}} \text{ is fresh at } S_i \\
              &\iota'_{\text{new}} \text{ is fresh at } S'_i
            \end{aligned}
          \end{cases} \\
          g_{\text{new}} = 
          \begin{cases}
            g & \text{ if } \beta_{i+1} = \smiley \text{ or } \beta_{i+1} = 
            \iota_{\text{new}} \\
            g[o_{\text{new}} \mapsto o'_{\text{new}}] & \text{ if } \beta_{i+1}
            = o_{\text{new}}
          \end{cases} \\
          h_{\text{new}} = 
          \begin{cases}
            h & \text{ if } \beta_{i+1} = \smiley \text{ or } \beta_{i+1} = 
            o_{\text{new}} \\
            h[\iota_{\text{new}} \mapsto \iota'_{\text{new}}] & \text{ if } \beta_{i+1}
            = \iota_{\text{new}}
          \end{cases}
        \end{gathered}
      \end{equation*}

      Since
      \begin{equation}
        S_i \Rrightarrow^{R_{i+1}^{\iota_{i+1}, \beta_{i+1}}} T,
      \end{equation}
      by rule inspection it is easy to verify that
      \begin{equation}
        S'_i \Rrightarrow^{R^{\iota, \beta}} T'
      \end{equation}
      for some $T'$. If $T = \Error$ then by rule inspection we must have $T' =
      \Error$ and trivially that $T \simeq T'$. Furthermore, if $T = H_{T},
      P_{T}$ then by rule inspection $T' = H_{T'}, P_{T'}$. By considering
      definition~\ref{def:bijectionmod} it is not hard to verify that
      we have
      \begin{equation*}
        H_{T'} = \pi(H_T, g_{\text{new}}, h_{\text{new}}) \andalso P_{T'} =
        \eta(P_T, g_{\text{new}}, h_{\text{new}}).
      \end{equation*}
      Thus $T \simeq T'$.
  \end{description}
\end{proof}

\begin{lemma} \label{lem:lemma4}
  Let 
  \begin{equation*}
    \bar{R} = R_1^{\iota_1, \beta_1}, \dots, R_n^{\iota_n, \beta_n}
  \end{equation*}
  be a transition sequence and $S, T$ be well typed states such that.
  \begin{equation*}
    S \Rrightarrow^{\bar{R}} T.
  \end{equation*}
  Then there is a state $T'$ and a series of mutually compatible 
  \begin{equation*}
    \beta'_i, i = 1, \dots, n
  \end{equation*}
  such that
  \begin{equation*}
    \begin{gathered}
      \beta'_i \not\in \OIDs(S), \beta'_i \not\in \TIDs(S) \andalso i =
      1, \dots, n, \\
      \bar{R'} = R_1^{\iota_1, \beta'_1}, \dots, R_n^{\iota_1, \beta'_n} \\
      S \Rrightarrow^{\bar{R'}} T' \andalso T \simeq T'.
    \end{gathered}
  \end{equation*}
\end{lemma}

\begin{proof}{(Sketch)}
  This is obvious from these two facts:
  \begin{itemize}
    \item There are infinitely many elements in both $\OIDs$ and $\TIDs$ and
      only finitely many in $\OIDs(S)$ and $\TIDs(S)$. This follows from
      the definition of $\OIDs(S)$ and $\TIDs(S)$.
    \item All choices of fresh identifiers results in equivalent states. I.e.
      for any state $S$, and transition identifiers $R^{\iota, \beta},
      R^{\iota, \beta'}$ such that 
      \begin{equation*}
        S \Rrightarrow^{R^{\iota, \beta}} S_1 \andalso S
        \Rrightarrow^{R^{\iota, \beta'}} S'_1,
      \end{equation*}
      we have
      \begin{equation*}
        S_1 \simeq S'_1.
      \end{equation*}
      This is not hard to prove.
  \end{itemize}
\end{proof}

Finally we restate and prove Theorem~\ref{thm:qd}.
\begin{theorem*}
  Let $S, S', T, T'$ be well typed states not equal to $\Error$ such that
  \begin{equation*}
    \begin{gathered}
      S \simeq S' \\
      S \Rrightarrow^{\bar{R}} T \andalso S' \Rrightarrow^{\bar{R'}} T'.
    \end{gathered}
  \end{equation*}
  Let $n$ and $n'$ be the length of $\bar{R}$ and $\bar{R'}$ respectively.
  Furthermore we assume that neither $T$ or $T'$ can make a step.  Then
  \begin{equation*}
    n = n' \quad \text{and} \quad T \simeq T'.
  \end{equation*}
\end{theorem*}

\begin{proof}
  If $S \neq S'$ by Lemma~\ref{lem:lemma3} there is a state $T''$ and a
  transition sequence $\bar{R''}$ such that
  \begin{equation}
    S \Rrightarrow^{\bar{R''}} T'' \andalso T' \simeq T''.
  \end{equation}
  Because of this and transistiveness of $\simeq$ WLOG we can assume that $S =
  S'$.

  We proceed by induction on $n$ and $n'$.
  \begin{description}
    \item[Case $n \leq 0, n' \leq 0$:] Clearly, $n = n' = 0$ and we are done
      since $T = S = S' = T'$. 
      
     % Clearly $n = n'$. Otherwise, when considering
     % Lemma~\ref{lem:lemma3} we would have a contradiction with the assumption
     % that neither $T$ or $T'$ can take another step .
     % If $n = n' = 0$ we are done since $T = S \simeq S' = T'$. Thus we assume
     % that $n = n' = 1$. We let
     % \begin{equation*}
     %   \bar{R} = R^{\iota, \beta} \andalso \bar{R'} = R'^{\iota', \beta'}.
     % \end{equation*}
     % We must have 
     % \begin{equation*} 
     %   \iota = \iota'
     % \end{equation*} 
     % since otherwise $T$ and $T'$ can take another step. Since $R$, $R'$ is
     % completely decided by the state $S$ and $\iota, \iota'$ we have
     % \begin{equation}\label{eq:qd_1}
     %   R = R'.
     % \end{equation}
     % If $\beta = \beta'$ then we are done since this implies $T = T'$.
     % Otherwise let $g, h$ be the bijections that witnesses the relation $S
     % \simeq S$. Because of \eqref{eq:qd_1} we must have that either $\beta,
     % \beta' \in \OIDs$ or $\beta, \beta' \in \TIDs$. In the first case, the
     % bijections
     % \begin{equation*}
     %   g[\beta \mapsto \beta'] \andalso h
     % \end{equation*}
     % witnesses the relation $T \simeq T'$. In the second case it is witnessed
     % by
     % \begin{equation*}
     %   g \andalso h[\beta \mapsto \beta'].
     % \end{equation*}
     % Thus $T \simeq T'$.
    \item[Case $n \leq i+1, n' \leq i+1$ and $i \geq 0$:] First of all, the case
      where $n = 0$ ($n' = 0$) and $n' > 0$ ($n > 0$) is impossible since this
      immediately contradicts that $T$ ($T'$) cannot take another step. Thus we
      have that either $n = n' = 0$ (which falls under the case above) or $n
      \geq 1$ and $n' \geq 1$. Thus we assume the latter. In the
      continuation, this case can be visualized as in
      Figure~\ref{fig:thm_qd_pf}. We let 

      \begin{figure}
        \centering
        \begin{tikzpicture}
          \node[circle,fill,minimum size=0.2cm,inner sep=0pt] (S) at (0,0) {};
          \node[below left] at (S) {$S$};
          \node[circle,fill,minimum size=0.2cm,inner sep=0pt] (Spk1) at (0,3) {};
          \node[below left] at (Spk1) {$S'_{k-1}$};
          \node[circle,fill,minimum size=0.2cm,inner sep=0pt] (Spk2) at (0,4) {};
          \node[above left, red] at (Spk2) {$S'_{k}$};
          \node[circle,fill,minimum size=0.2cm,inner sep=0pt] (Tp) at (0,7) {};
          \node[above] at (Tp) {$T'$};
          \node[circle,fill,minimum size=0.2cm,inner sep=0pt] (Spmodk) at (1,3) {};
          \node[right, red] at (Spmodk) {$S'_{k}$};
          \node[circle,fill,minimum size=0.2cm,inner sep=0pt] (Tpmod) at (1,6) {};
          \node[above] at (Tpmod) {$T'$};
          \node[circle,fill,minimum size=0.2cm,inner sep=0pt] (S1) at (1,0) {};
          \node[below right] at (S1) {$S_{1}$};
          \node[circle,fill,minimum size=0.2cm,inner sep=0pt] (T) at (7,0) {};
          \node[right] at (T) {$T$};
          \draw[-{Latex[length=0.2cm]}] (S) -- node[left] {$\bar{R'}_{<k}$}
          (Spk1);
          \draw[-{Latex[length=0.2cm]},blue] (Spk1) -- node[left] {$R^{\iota, \beta}$}
          (Spk2);
          \draw[-{Latex[length=0.2cm]}] (Spk2)-- node[left] {$\bar{R'}_{>k}$} (Tp);
          \draw[-{Latex[length=0.2cm]},blue] (S)-- node[below,blue] {$R^{\iota, \beta}$}
          (S1);
          \draw[-{Latex[length=0.2cm]}] (S1)-- node[below] {$\bar{R}_{\geq 2}$} (T);
          \draw[-{Latex[length=0.2cm]}] (S1) -- node[right] {$\bar{R'}_{<k}$} (Spmodk);
          \draw[-{Latex[length=0.2cm]}] (Spmodk)-- node[right] {$\bar{R'}_{>k}$} (Tpmod);
          \draw[decoration={brace,raise=30pt,amplitude=10pt},decorate,grey] (S) --
          node[left=40pt]
          {$\bar{R'}$} (Tp);
          \draw[decoration={brace,raise=30pt,amplitude=10pt,mirror},decorate,grey] (S) --
          node[below=40pt]
          {$\bar{R}$} (T);
          %\draw[-{Latex[length=0.3cm]}] (c1) -- node[above] {$f_1$} (c3);
        \end{tikzpicture}
        \caption{$R^{\iota, \beta}$ is the first transition identifier of $\bar{R}$.
        It must also occur somewhere in the transition sequence $\bar{R'}$. We
        transform $\bar{R'}$ by moving $R^{\iota, \beta}$ to the beginning of the
        sequence. The halting state is the same because of Lemma~\ref{lem:lemma2}. The
        new sequence $R^{\iota,\beta}, \bar{R'}_{<k}, \bar{R'}_{>k}$ begins with the
        same step as $\bar{R}$ and therefore steps to $S_1$ in the first step. Thus we
        can examine the two shorter sequences $\bar{R'}_{<k}, \bar{R'}_{>k}$ and
        $\bar{R}_{\geq 2}$ and conclude that $T \simeq T'$ by the induction
        hypothesis.}
        \label{fig:thm_qd_pf}
      \end{figure}

      \begin{equation*}
        \begin{gathered}
          \bar{R} = R_1^{\iota_1, \beta_1}, \dots, R_n^{\iota_n, \beta_n} \\
          \bar{R'} = {R'_1}^{\iota'_1, \beta'_1}, \dots, {R'_{n'}}^{\iota'_{n'},
          \beta'_{n'}}.
        \end{gathered}
      \end{equation*}
      Furthermore let
      \begin{equation*}
        \bar{R}_{\geq 2} = R_2^{\iota_2, \beta_2}, \dots, R_n^{\iota_n, \beta_n} \\
      \end{equation*}
      Consider the first transition identifier of $\bar{R}$: $R^{\iota, \beta} = R_1^{\iota_1,
      \beta_1}$. We let
      $k$ be such
      that ${R'_k}^{\iota'_k, \beta'_k}$ is the first transition identifier in
      $\bar{R'}$ for which 
      \begin{equation} \label{eq:qd1}
        \iota'_k = \iota.
      \end{equation} 
      This $k$ must exist since otherwise $T'$ can take another step. This is
      due to that thread $\iota$ can execute at $S$ and if it never executes
      before $T'$, it will still be able to execute, thus generating another
      step.  By the remark following definition~\ref{def:trans_id} we have 
      \begin{equation} \label{eq:qd2}
        R'_k = R.
      \end{equation} 
      We let
      \begin{equation*}
        \begin{gathered}
          \bar{R'}_{<k} = {R'_1}^{\iota'_1, \beta'_1}, \dots,
          {R'_{k-1}}^{\iota'_{k-1}, \beta'_{k-1}}  \\
          \bar{R'}_{>k} = {R'_{k+1}}^{\iota'_{k+1}, \beta'_{k+1}}, \dots,
          {R'_{n'}}^{\iota'_{n'}, \beta'_{n'}}.
        \end{gathered}
      \end{equation*}
      We can thus write
      \begin{equation} \label{eq:qd4}
        S \Rrightarrow^{\bar{R'}_{<k}} S'_{k-1} \Rrightarrow^{R^{\iota,
        \beta'_k}} S'_k \Rrightarrow^{\bar{R'}_{>k}} T'.
      \end{equation}

      WLOG we can assume that
      \begin{equation} \label{eq:qd3}
        \beta'_k = \beta
      \end{equation}
      and that $\beta$ is compatible with $\beta'_i$ for all $i = 1, \dots,
      k-1$.  Otherwise we could just reassign $\beta_1, \dots, \beta_n,
      \beta'_1, \dots, \beta'_n$, in accordance with Lemma~\ref{lem:lemma4}.

%      By lemma~\ref{lem:lemma3} we can (using similar reasoning as before) WLOG
%      assume that $\beta'_k$ is compatible with $\beta'_i$ for all $i = 1,
%      \dots, k-1$.
%
%      Now either $\beta'_k = \beta_1 = \smiley$, $\beta'_k, \beta_1 \in \OIDs$
%      or $\beta'_k, \beta_1 \in \TIDs$. In the first case it is immediately
%      clear that ${R'_k}^{\iota'_k, \beta'_k}$ is applicable at $S$. In the
%      other two cases WLOG we can assume that $\beta'_k$ is fresh at $S$ due to
%      lemma~\ref{lem:lemma3}. This makes $R^{\iota, \beta'_k}$
%      applicable at $S$.

      By the choice of $k$,
      \begin{equation*}
        \forall i \text{ s.t. } 1 \leq i < k. \quad \iota \neq \iota'_i.
      \end{equation*}
      Then by \eqref{eq:qd4} and Lemma~\ref{lem:lemma2} we have
      \begin{equation*}
        S \Rrightarrow^{{R}^{\iota, \beta}} S_1 \Rrightarrow^{\bar{R'}_{<k}}
        S'_k,
      \end{equation*}
      which combined with \eqref{eq:qd4} yields
      \begin{equation*}
        S \Rrightarrow^{{R}^{\iota, \beta}} S_1 \Rrightarrow^{\bar{R'}_{<k}}
        S'_k \Rrightarrow^{\bar{R'}_{>k}} T'.
      \end{equation*}
      Furthermore, by $\bar{R}$ being applicable at $S$ we have
      \begin{equation*}
        S \Rrightarrow^{R^{\iota, \beta}} S_1 \Rrightarrow^{\bar{R}_{\geq 2}}
        T.
      \end{equation*}
      The transition sequences $\bar{R}_{\geq 2}$ and $\bar{R'}_{<k},
      \bar{R'}_{>k}$ are of lengths $n - 1$ and $n' - 1$ respectively, both less
      than or equal to $i$.  Thus, by the induction hypothesis $T \simeq T'$ and
      $n - 1 = n' - 1$. Thus,
      \begin{equation*}
        n = n' \andalso T \simeq T'.
      \end{equation*}
  \end{description}
\end{proof}


% TODO add figures to the lemmas and theorem proofs





\end{document}
