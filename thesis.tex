\documentclass{kththesis}

\usepackage{csquotes} % Recommended by biblatex
\usepackage{biblatex}
\addbibresource{references.bib} % The file containing our references, in BibTeX format

\usepackage{mathtools, graphicx, hyperref, amssymb, algorithm2e}
\usepackage[utf8]{inputenc}
\usepackage[T1]{fontenc}
\usepackage{listings}
\usepackage{bcprules}
\usepackage{multicol}
\usepackage{tikz}
\usepackage{mathrsfs}
\usepackage{amsthm}
\usepackage{scrextend}
\usepackage{stmaryrd}
\usepackage{wasysym}
\usepackage{float}
\usepackage[labelfont=bf]{caption}
\usepackage{subcaption}
\usepackage{enumitem}

\usepackage{inconsolata}

\setlist[description]{listparindent=\parindent}

% tikz packages
\usetikzlibrary{arrows.meta,decorations.pathreplacing}

% Configuration of font and stuff for listings
\input{listings_settings.tex}

% Definitions of commands
%% This file is for definitions of commands

\newcommand{\image}{\text{Im }}
\newcommand{\reals}{\mathbb{R}}
\newcommand{\integers}{\mathbb{Z}}
\newcommand{\ratio}{\mathbb{Q}}
\newcommand{\ilc}[1]{\lstinline[keepspaces=true]$#1$}
\newcommand{\ilclang}[2]{\lstinline[keepspaces=true,language=#1]$#2$}
\newcommand{\lacasa}{\textsc{LaCasa}}
\newcommand{\CLCone}{\textsc{CLC}$^1$}
\newcommand{\scrule}[3]{\infrule[\textsc{#1}]{#2}{#3}}
\newcommand{\stof}{<:}
\newcommand{\ocap}{\mbox{\texttt{ocap}}}

\newcommand{\LatType}{\mbox{$\mathcal{L}$}}
\newcommand{\AnyRefType}{\mbox{$\mathsf{AnyRef}$}}
\newcommand{\CellType}{\mbox{$\mathsf{Cell}$}}
\newcommand{\NullType}{\mbox{$\mathsf{Null}$}}

\newcommand{\LatVals}{\mbox{$\mathscr{L}$}}

% Commands for writing language syntax

\newcommand{\ClassDef}[4]{\mbox{\texttt{class}~#1~\texttt{extends}~#2~\texttt{\{}}#3~#4\texttt{\}}}
\newcommand{\VarDecl}[2]{\mbox{\texttt{var}~#1~\texttt{:}~#2}}
\newcommand{\MethodDef}[5]{\mbox{\texttt{def}~#1\texttt{(}#2~\texttt{:}~#3\texttt{)}~\texttt{:}~#4~\texttt{=}~#5}}
\newcommand{\Let}[3]{\mbox{\texttt{let}~#1~\texttt{=}~#2~\texttt{in}~#3}}
\newcommand{\NullVal}{\mbox{\texttt{null}}}
\newcommand{\FSel}[2]{\mbox{#1\texttt{.}#2}}
\newcommand{\FAss}[3]{\mbox{#1\texttt{.}#2~\texttt{=}~#3}}
\newcommand{\New}[1]{\mbox{\texttt{new}~#1}}
\newcommand{\NewCell}{\mbox{\texttt{new}~{Cell}}}
\newcommand{\Call}[3]{\mbox{#1\texttt{.}#2\texttt{(}#3\texttt{)}}}
\newcommand{\Put}[2]{\mbox{#1~\texttt{put}~#2}}
\newcommand{\When}[3]{\mbox{\texttt{when}~#1~\texttt{pass}~#2~\texttt{then}~#3}}
\newcommand{\Capt}[2]{\mbox{#1~\texttt{=}~#2}}
\newcommand{\CB}[3]{\mbox{(#1, #2~$\Rightarrow$~#3)}}


% Declarations of theorems and alike
\theoremstyle{definition}
\newtheorem{definition}{Definition}[chapter]

\theoremstyle{theorem}
\newtheorem{theorem}{Theorem}
\newtheorem*{theorem*}{Theorem}
\newtheorem{proposition}[definition]{Proposition}%[chapter]
\newtheorem{lemma}[definition]{Lemma}
\newtheorem{corollary}[definition]{Corollary}
\newtheorem*{claim}{Claim}

\theoremstyle{remark}
\newtheorem*{remark}{Remark}
\newtheorem*{note}{Note}
\newtheorem*{notation}{Notation}



\title{Concurrent Determinism Using Lattices and the Object Capability Model}
\alttitle{Determinism i parallelliserade program med hjälp av gitterstrukturer och objektsförmågor}
\author{Ellen Arvidsson}
\email{magarv@kth.se}
\supervisor{Philipp Haller}
\examiner{Mads Dam}
\programme{Master in Computer Science}
\school{School of Electrical Engineering and Computer Science}
\date{\today}

\widowpenalty10000
\clubpenalty10000

\begin{document}

% Frontmatter includes the titlepage, abstracts and table-of-contents
\frontmatter

\titlepage

\begin{abstract}
  Parallelization is an important part of modern data systems. However, the
  non-determinism of thread scheduling introduces the difficult problem of
  considering all different execution orders when constructing an algorithm.
  Therefore deterministic-by-design concurrent systems are attractive.  A new
  approach called LVars consists of using data which is part of a lattice, with
  a predefined join operation. Updates to shared data are carried out using the
  join operation and thus the updates commute. Together with limiting the reads
  of shared data, this guarantees a deterministic result. The Reactive Async
  framework follows a similar approach but has several aspects which can cause a
  non-deterministic result. The goal with this thesis is to explore how we can
  ammend Reactive Async in order to guarantee a deterministic result. First an
  exploration into the subtleties of lattice based data combined with
  concurrency is made.  Then a formal model based on a simple object-oriented
  language is constructed.  The constructed small-step operational semantics and
  type system are shown to guarantee a form of determinism. This shows that
  LVars-similar system can be implemented in an object-oriented setting.
  Furthermore the work can act as a basis for future revisions of Reactive Async
  and similar frameworks.
\end{abstract}


\begin{otherlanguage}{swedish}
  \begin{abstract}
    Parallellisering är en viktig del i moderna datasystem. Flertrådade
    applikationer innebär dock en svårighet i och med att programmerare måste ta
    alla exekveringsordningar i beaktning. Därför är beräkningsmodeller vars
    resultat är garanterat deterministiskt en attraktiv utväg. En ny modell,
    kallad LVars, använder gitterstrukturer tillsammans med en
    supremum-operation för att garantera att uppdateringar av delad data
    kommuterar. Detta tillsammans med begränsningar av läsning av datan
    garanterar ett deterministiskt resultat. Reactive Async är ett
    programmeringsramverk som följer en liknande strategi. Det finns dock flera
    delar i dess konstruktion som i en oförsiktig programmerares händer kan
    orsaka att ett programs resultat blir icke-deterministiskt. Målet med detta
    examensarbete är att utforska vilka modifikationer som skulle kunna göras av
    Reactive Async för att garantera determinism. Först görs en undersökning av de mer
    svårförståeliga delarna i kombinationen av gitterbaserad data med
    flertrådad exekvering. Sedan kostrueras en formell beräkningsmodell baserad på
    ett enkelt objekt-orienterat språk. Konstruktionens småstegade operationella
    semantik tillsammans med dess typsystem visas kunna garantera en form av
    determinism. Detta visar att ett system liknande LVars kan implementeras i
    ett objekt-orienterat språk. Därmed skulle detta arbete kunna ligga till
    grund för framtida versioner av Reactive Async.
  \end{abstract}
\end{otherlanguage}


\tableofcontents


% Mainmatter is where the actual contents of the thesis goes
\mainmatter


%We use the \emph{biblatex} package to handle our references.  We therefore use the
%command \texttt{parencite} to get a reference in parenthesis, like this
%\parencite{heisenberg2015}.  It is also possible to include the author
%as part of the sentence using \texttt{textcite}, like talking about
%the work of \textcite{einstein2016}.

\chapter{Introduction}
\label{cha:introduction}

The main approach to achieving concurrency in general applications has long been
to use locks and similar constructs in order to locally synchronize data
accesses. The major drawback with this is that it demands a lot of effort from
the programmer in order to achieve good parallelization, even of simple tasks.
Furthermore the likelihood of concurrency related errors such as deadlocks, data
races and livelocks naturally becomes higher. 

Concurrent deterministic programs are concurrent programs which always produce
the same results for the same inputs. This is attractive since it provides
reproducible computations.  Deterministic-by-design concurrent programming
models are systems which guarantees this property at compile time and thus
significantly simplifies concurrent programming. This greatly eases effort
from the programmer. 

A novel deterministic-by-design model call LVish ~\parencite{kuper2013lvars,
kuper2014freeze} has been introduced.  It uses data types that fulfill the
condition of being part of a lattice, or more explicitly, having some partial
order defined on its values. By defining a join operation on the values and only
allowing writes to the data in form of this operation, you can achieve concurrent
determinism with very high parallelism using event based computation.  This is
also the basic approach of the Reactive Async
project~\parencite{conf/scala/HallerGES16}, which has shown potential to speed up
e.g.\ static analysis of source code. The main problem with Reactive Async as
of now, is that it does not include any construct to ensure that data races
between event handlers do not occur. Futhermore, there are also more fundamental
problems with callback spawning, which could also lead to non-determinism.

This thesis aims to construct a formal model which could later be adopted by
future revisions of Reactive Async. Constructing a model similar to that
of LVish but in an object-oriented setting, the final goal
will be to prove a form of determinism for the system. The work will use
the object capability model from computer security, also utilized in previous
work by \textcite{conf/oopsla/HallerL16}.

\section{Contributions}%
\label{sec:contributions}

With this work, the following technical contributions are made.
\begin{itemize}
  \item We explore and, using examples, describe problems with both Reactive
    Async and LVish.
  \item We present a core object-oriented language, together with a formalization
    in the form of small-step operational semantics and a type system. 
  \item We present a proof of soundness for our semantics and type system.
  \item We present a proof of quasi-determinism, a form of determinism, for our
    formalized system.
\end{itemize}


\section{Report Structure}%
\label{sec:report_structure}

In Chapter~\ref{cha:background} some technical background is described. This
includes some mathematical definitions, an introduction to programming language
formalization and an informal description of the object capability
model. Chapter~\ref{cha:related_work} describes some selected related work, upon
which our model will build. In Chapter~\ref{cha:challenges} we describe and
exemplify some problems with existing deterministic-by-design concurrent
systems. In Chapter~\ref{cha:core_language} we introduce our core language and
formalization. Chapter~\ref{cha:properties_of_racl} describes the theorems and
proofs of soundness and quasi-determinism. Finally, in
Chapter~\ref{cha:discussion_and_conclusion} we discuss extensions and
implementation of our formal system, as well as conclude the report.


\chapter{Background} \label{cha:background}

This chapter is a brief introduction to language and type system theory. In
section~\ref{sec:language_syntax} some basics of language syntax will be
explained. Section~\ref{sec:language_semantics} gives a brief introduction to
programming language semantics. Finally section~\ref{sec:type_systems} explains
some basics of programming language type systems.

As a basis the well known While language will be used as an example language.

% TODO add a section about basic math, like lattices join operations, partial
% maps

\section{Language Syntax} \label{sec:language_syntax}

% Describe the foundations of syntaxes for languages: BNF, ANF
A language syntax is a formal way to describe the structure of programs written
in a programming language. It gives us a basis on which to build the language
semantics and type systems as will be clear later. In this section the
Backus-Naur-form (BNF) will be introduced. 

\subsection{The Backus-Naur-form} \label{sub:the_backus_naur_form} 

Backus-Naur-form (BNF) is a very common way to describe the syntax of
programming languages. BNF describes a syntax using rules of the form
\begin{equation*}
  s ::= \text{<expression>}.
\end{equation*}
The the left side symbol $s$ is refered to as a non-terminal. <expression> could
also be a list of the
possible forms of $s$, separated by $|$, i.e. 
\begin{equation*}
  s ::= \text{<expression1>} \: | \dots | \: \text{<expressionN>}
\end{equation*}
The expressions can themselves contain non-terminals and terminals.
Terminals are symbols which does not occur on the left side of any rule.

The While language can be written in BNF as described
in~\parencite{nielson2007semantics} For reference we give a similar description
here with the addition of a very simple type system.

Furthermore we let $\WhVar$ be the set of all variable names and $\WhVal$ the
set of all values which for the While language will be the integers $\integers$,
and the boolean values $\WhTrue$ and $\WhFalse$. 
\begin{equation*}
  \WhVal = \integers \cup \left\{ \WhTrue, \WhFalse \right\}
\end{equation*}
Note that in~\parencite{nielson2007semantics} 
the conversion between the syntactical (numerals\slash boolean literals) and semantic
(integers\slash boolean values) versions of values are explicit whereas here 
this conversion is implicit.

\begin{figure}[h]
  \centering
  $\begin{array}{lll@{\hspace{4mm}}l}
    c &::= &  & \mbox{Program code} \\
    &  & \varepsilon & \mbox{Empty code} \\
    &| & s; c' & \mbox{Statement concatenation} \\
    s &::= & & \mbox{Program statement} \\
    &  & \WhDecl{x}{\tau} & \mbox{Variable declaration} \\
    &| & \WhAssign{x}{e} & \mbox{Assignment} \\
    &| & \WhSkip & \mbox{Skip} \\
    &| & \WhIf{e}{c_1}{c_2} & \mbox{If branch} \\
    &| & \WhWhile{e}{c} & \mbox{While loop} \\
    &&&\\
    e & ::= & & \mbox{Expressions} \\
    & & x & \mbox{Variable} \\
    &| & v & \mbox{Value literal} \\
    &| & \WhAdd{e_1}{e_2} & \mbox{Addition} \\
    &| & \WhTimes{e_1}{e_2} & \mbox{Multiplication} \\
    &| & \WhMinus{e_1}{e_2} & \mbox{Subtraction} \\
    &| & \WhEq{e_1}{e_2} & \mbox{Equality comparison} \\
    &| & \WhLeq{e_1}{e_2} & \mbox{$\leq$ comparison} \\
    &| & \WhLnot{e} & \mbox{Logical not} \\
    &| & \WhLand{e_1}{e_2} & \mbox{Logical and} \\
    &&&\\
    \tau & ::= & & \mbox{Types} \\
    & & \WhBool & \mbox{Boolean type} \\
    &| & \WhInt & \mbox{Integer type} \\
    &&&\\
    v & ::= & & \mbox{Values} \\
    &  & n & \mbox{Integer} \\
    &| & \WhTrue & \mbox{Boolean true} \\
    &| & \WhFalse & \mbox{Boolean false} \\
  \end{array}$
  \caption{Grammar of While}
  \label{fig:while_grammar}
\end{figure}

\section{Small-step Operational Semantics} \label{sec:language_semantics}

% Describe small step operational semantics and give example with while lang
In order to prove properties of a program you must be able to formally describe
the "meaning" of the program, i.e. to describe how programs statements modify
state and what execution steps can be taken. A common approach to this is
small-step operational semantics, also called structural operational semantics
(SOS). Other approaches includes big-step operational semantics or natural
semantics, and denotational semantics. Since SOS models concurrency well this is
the only one described here.

Small-step operational semantics defines rules which can be used to derive
single execution steps (or transitions) between program states. We exemplify
this with the While language defined in the above section. In order to define
rules for state transitions a state must be defined. Therefore we make the
following definition for the While language.

\begin{definition}
  A \emph{state} $S$ for the While language is a value of the form
  \begin{equation*}
    \WhState{V, c}  
  \end{equation*}
  where $c$ is a statement and $V$ is a partial map
  \begin{equation*}
    V: \WhVar \rightharpoonup \WhVal
  \end{equation*}
\end{definition}

We can now define rules for state transitions (see Figure~\ref{fig:while_sos}).
All rules are of the form
\infrule[Rule name]
{\text{precondition}}
{S \rightarrow S'}
Which basically says "if precondition holds then we can step from $S$
to $S'$".  If there are no preconditions we just write
\infax[Rule name]
{S \rightarrow S'}

\begin{figure}[h]
  \centering
  \infax[WhE-Decl1]
  {\WhState{V, \WhDecl{x}{\WhInt};c} \: \rightarrow \: \WhState{V[ x \mapsto 0],
  c}}
  
  \RuleSpace

  \infax[WhE-Decl2]
  {\WhState{V, \WhDecl{x}{\WhBool};c} \: \rightarrow \: \WhState{V[ x \mapsto
  \WhFalse], c}}
  
  \RuleSpace

  \infax[WhE-Assign]
  {\WhState{V, \WhAssign{x}{e};c} \: \rightarrow \: \WhState{V[ x \mapsto
  \WhEval{V}(e)], c}}

  \RuleSpace

  \infax[WhE-Skip]
  {\WhState{V, \WhSkip; c} \: \rightarrow \: \WhState{V, c}}

  \RuleSpace

  \infrule[WhE-IfTrue]
  {\WhEval{V}(e) = \WhTrue}
  {\WhState{V, \WhIf{e}{c_1}{c_2};c} \: \rightarrow \: \WhState{V, c_1;c}}

  \RuleSpace 

  \infrule[WhE-IfFalse]
  {\WhEval{V}(e) = \WhTrue}
  {\WhState{V, \WhIf{e}{c_1}{c_2}} \: \rightarrow \: \WhState{V, c_2;c}}
  
  \RuleSpace

  \infax[WhE-While]
  {\WhState{V, \WhWhile{e}{c'};c} \: \rightarrow \: \WhState{V,
  \WhIf{e}{(\WhConc{c'}{\WhWhile{e}{c'}})}{\WhSkip};c}}

  \caption{Small-step operational semantics for While}
  \label{fig:while_sos}
\end{figure}

Most of the rules are very intuitive. For example, rules {\sc WhE-Decl1} and {\sc
WhE-Decl2} extends the state map $V$ with $x$ and a default value corresponding
to the declared type. Rule {\sc WhE-Assign} evaluates the expression $e$ and
maps $x$ to the result. Rules {\sc WhE-IfTrue} and {\sc WhE-IfFalse} evaluates
its expression $e$ and if the result is $\WhTrue$ or $\WhFalse$ they prepend the
code of the corresponding branch to the rest of the program. We can also see
that {\sc WhE-While} is just an expansion of a while loop into an if statement.
Finally we note that we can not step from $\WhState{V, \varepsilon}$ which means
that this is a halting state.

The SOS rules of While heavily relies on the expression evaluation function
$\WhEval{\cdot}(\cdot)$. This is a function which takes a state map $V$ and an
expression $e$ and returns a value in $\WhVal$. We define this recursively as follows
\begin{equation*}
  \begin{array}{lll}
    \WhEval{V}(v) &=& v \\
    \WhEval{V}(x) &=& V(x) \\
    \WhEval{V}(e_1 + e_2) &=& \WhEval{V}(e_1) + \WhEval{V}(e_2) \\
    \WhEval{V}(e_1 * e_2) &=& \WhEval{V}(e_1) * \WhEval{V}(e_2) \\
    \WhEval{V}(e_1 - e_2) &=& \WhEval{V}(e_1) - \WhEval{V}(e_2) \\
    \WhEval{V}(e_1 = e_2) &=& \WhEval{V}(e_1) = \WhEval{V}(e_2) \\
    \WhEval{V}(e_1 \leq e_2) &=& \WhEval{V}(e_1) \leq \WhEval{V}(e_2) \\
    \WhEval{V}(e_1 \land e_2) &=& \WhEval{V}(e_1) \land \WhEval{V}(e_2) \\
    \WhEval{V}(\lnot e_1) &=& \lnot \WhEval{V}(e_1) \\
  \end{array}
\end{equation*}
Here we implicitly rely on that the operations $+, -, *, =, \leq$ are only
defined for integer values, and that $\lnot, \land$ are only defined for boolean
values.  If we encounter a state map $V$ and an expression $e$ where the
corresponding integer or boolean operation above is undefined, the value for
$\WhEval{V}(e)$ is undefined. Since the definition of $\WhEval{\cdot}(\cdot)$ is
recursive, the occurrence of an undefined value should also propagate upwards.
For example, if $\WhEval{V}(e_1)$ is undefined then $\WhEval{V}(e_1 + e_2)$ is undefined
aswell. This is implicit in our definition.

The evaluation function $\WhEval{\cdot}(\cdot)$ together with the derivation
rules in Figure~\ref{fig:while_sos} gives us a complete description of what
steps are allowed. It is implicit in the description that if the evaluation
function is undefined at some point in the derivation of a step, execution
cannot progress and we get stuck.

\section{Type Systems} \label{sec:type_systems}

A type system is a mathematical construct that consists of elements called types
and a set of rules that assign types to parts of a programming language, e.g.,
statements, variables and expressions. Type systems are most commonly used to
prevent programming errors such as feeding a data structure to a function for
which it was not made, and this can be enforced both using static checking at
compile time or dynamic checking at runtime. Type systems are found in many
modern programming languages such as Java, Scala, Haskell, Python and C++.

The focus here will be on static type systems. We will explain the basics using
our previous example of the While language. Then some uses and extensions will be
explained.

\subsection{A Type System for While}
\label{sub:a_type_system_for_while}

For the While language we have the following types:
\begin{equation*}
  \WhInt \andalso \WhBool \andalso \WhVoid
\end{equation*}
We call the set of all types 
\begin{equation*}
  \WhTpe = \left\{ \WhInt, \WhBool, \WhVoid \right\}.
\end{equation*}
Our typing rules will define a relation of the form
\begin{equation} \label{eq:tpe_sys1}
  \Gamma \vdash r : \tau.
\end{equation}
Here $\Gamma$ is a \emph{typing environment}, i.e. a partial map
\begin{equation*}
  \Gamma: \WhVar \rightharpoonup \WhTpe,
\end{equation*}
$r$ is either code or an expression and $\tau \in \WhTpe$.
Equation \eqref{eq:tpe_sys1} basically reads "$r$ is typed as $\tau$ under the
typing environment $\Gamma$." A program $c$ is typeable if
\begin{equation*}
  \emptyset \vdash c : \WhVoid
\end{equation*}
where $\emptyset$ is the empty typing environment, i.e. $\emptyset(x)$ is
undefined for all $x \in \WhVar$.

\begin{notation}
  Sometimes you would like to remap a key of, or extend a partial map $M$. One notation
  for this commonly used is 
  \begin{equation*}
    M[k \mapsto v],
  \end{equation*}
  which means that $M(l) = M[k \mapsto v](l)$ for all $l \neq k$ and
  $M(k) = v$.
  For typing environments we will also use the notation
  \begin{equation*}
    \Gamma, x: \tau
  \end{equation*}
  which is equivalent to $\Gamma[x \mapsto \tau]$.
\end{notation}

\begin{figure}[h]
  \infax[WhT-Empty]
  {\Gamma \vdash \varepsilon : \WhVoid}

  \RuleSpace

  \infrule[WhT-Decl]
  {\Gamma(x) \text{ undefined} \andalso \Gamma, x: \tau \vdash c : \WhVoid}
  {\Gamma \vdash \WhDecl{x}{\tau}; c : \WhVoid}

  \RuleSpace 

  \infrule[WhT-Assign]
  {(x: \tau) \in \Gamma \andalso \Gamma \vdash e: \tau \andalso \Gamma \vdash c
  : \WhVoid}
  {\Gamma \vdash \WhAssign{x}{e}; c : \WhVoid}

  \RuleSpace

  \infrule[WhT-Skip]
  {\Gamma \vdash c : \WhVoid}
  {\Gamma \vdash \WhSkip; c : \WhVoid}

  \RuleSpace

  \infrule[WhT-If]
  {\Gamma \vdash e : \WhBool \andalso \Gamma \vdash c_1: \WhVoid \\
  \Gamma \vdash c_2 : \WhVoid \andalso \Gamma \vdash c: \WhVoid}
  {\Gamma \vdash \WhIf{e}{c_1}{c_2}; c : \WhVoid}

  \RuleSpace

  \infrule[WhT-While]
  {\Gamma \vdash e: \WhBool \andalso \Gamma \vdash c' : \WhVoid \andalso \Gamma
  \vdash c : \WhVoid}
  {\Gamma \vdash \WhWhile{e}{c'}; c : \WhVoid}

  \caption{While program code typing rules.}
  \label{fig:while_code_tpe}
\end{figure}

Typing rules for program code are defined in Figure~\ref{fig:while_code_tpe}.
Generally While program code can only be typed as $\WhVoid$. Intuitively this is
because the execution of program code does not have a resulting value. Instead
it can only modify the state map $V$. The rules for typing program code should
not be hard to understand. For example, {\sc WhT-Assign} states that for code that
starts with an assignment the expression in the assignment must be of the same
type as the variable. {\sc WhT-If} requires that the expression is typeable as
$\WhBool$ and that both branches should be typeable as $\WhVoid$.

\begin{figure}[h]
  \begin{multicols}{3}
  \infax[WhT-Num]
  {\Gamma \vdash n : \WhInt}

  \infax[WhT-True]
  {\Gamma \vdash \WhTrue: \WhBool}

  \infax[WhT-False]
  {\Gamma \vdash \WhFalse: \WhBool}
  \end{multicols}

  \RuleSpace

  \infrule[WhT-Var]
  {(x: \tau) \in \Gamma}
  {\Gamma \vdash x: \tau}

  \RuleSpace

  \infrule[WhT-Add]
  {\Gamma \vdash e_1: \WhInt \andalso \Gamma \vdash e_2: \WhInt}
  {\Gamma \vdash e_1 + e_2: \WhInt}

  \RuleSpace

  \infrule[WhT-Times]
  {\Gamma \vdash e_1: \WhInt \andalso \Gamma \vdash e_2: \WhInt}
  {\Gamma \vdash e_1 * e_2: \WhInt}

  \RuleSpace

  \infrule[WhT-Minus]
  {\Gamma \vdash e_1: \WhInt \andalso \Gamma \vdash e_2: \WhInt}
  {\Gamma \vdash e_1 - e_2: \WhInt}

  \RuleSpace 

  \infrule[WhT-Eq]
  {\Gamma \vdash e_1: \WhInt \andalso \Gamma \vdash e_2: \WhInt}
  {\Gamma \vdash e_1 = e_2: \WhBool}

  \RuleSpace 

  \infrule[WhT-Leq]
  {\Gamma \vdash e_1: \WhInt \andalso \Gamma \vdash e_2: \WhInt}
  {\Gamma \vdash e_1 \leq e_2: \WhBool}

  \RuleSpace

  \infrule[WhT-Not]
  {\Gamma \vdash e : \WhBool}
  {\Gamma \vdash \lnot e: \WhBool}

  \RuleSpace 

  \infrule[WhT-And]
  {\Gamma \vdash e_1: \WhBool \andalso \Gamma \vdash e_2: \WhBool}
  {\Gamma \vdash e_1 \land e_2: \WhBool}

  \caption{While expression typing rules.}
  \label{fig:while_expr_tpe}
\end{figure}

Finally the rules for typing expressions are found in
Figure~\ref{fig:while_expr_tpe}. The possible types for an expression are
$\WhInt$ and $\WhBool$. The rules themselves should not be difficult to
understand.


\subsection{Uses \& Extensions}
\label{sub:uses_and_extensions}

So what are type systems used for? As mentioned earlier, commonly they are put
in place to prevent errors such as uncompatible data structures being fed as arguments
to a function which cannot handle such properly. For example, in Java you cannot feed a
List to a method which is denoted to take an Integer as an argument. This will
result in a compiler error.

\subsection{Preservation \& Progress}
\label{sub:preservation_&_progress}

A formal type system can furthermore be used to prove properties like
\emph{preservation} and \emph{progress} for programs which are properly type
checked. These are important properties since they can beforehand ensure that
programs terminate properly, or at least that, e.g., if it terminates erroneously it
must have been the result of a null-pointer exception. Progress and preservation
together is often refered to as \emph{soundness} of a type system and to prove this
is a standard approach. For a more in depth explanation
see~\parencite{pierce2002types}.

As an example we can state preservation and progress properties of While as follows. Let
$\Gamma(V)$ be defined as 
\[
  \Gamma(V)(x) = \begin{cases}
    \WhInt & \text{ if } V(x) \in \integers \\
    \WhBool & \text{ if } V(x) \in \left\{\WhTrue, \WhFalse \right\} \\
    \text{undefined} & \text{ otherwise }
  \end{cases}
\]
\begin{proposition}{(Preservation of While)} 
  Let
  \begin{equation*}
    \WhState{V, c} \: \rightarrow \: \WhState{V', c'} \andalso \Gamma(V) \vdash
    c: \WhVoid.
  \end{equation*}
  Then 
  \begin{equation}
    \Gamma(V') \vdash c': \WhVoid
  \end{equation}
\end{proposition}
\begin{proposition}{(Progress of While)}
  Let $\WhState{V, c}$ be a state and let
  \begin{equation*}
    \Gamma(V) \vdash c: \WhVoid.
  \end{equation*}
  Then either $c = \varepsilon$ or there is a state $\WhState{V', c'}$ such that
  \begin{equation*}
    \WhState{V, c} \: \rightarrow \: \WhState{V', c'}.
  \end{equation*}
\end{proposition}


\subsection{Extensions}
\label{sub:extensions}

The example type system for While is of course very simple. This is mostly due
to the simplicity of the language itself, more complicated languages have more
complicated type systems and a type system is oftenmost designed in together 
with the language itself. 

One common construct in type systems is \emph{subtyping}. This introduces a a
(semi)-lattice with based on a partial order relation denoted $\stof$ between
types and it oftenmost is used to say that "if $\sigma \stof \tau$ then values
of type $\sigma$ can be used in the same way as values of type $\tau$". This is
the case for Java where a class $C$ can extend a class $D$. This for simply
stated means that the fields and methods of $D$ are also availible in $C$. We
will see an example of how subtyping can be used in
chapter~\ref{cha:core_language}. 

In chapter~\ref{cha:core_language} the typing relation will also include an
effect $a$. In short this means that the typing rules are more or less strict
depending on the value of $a$.



\chapter{Related Work}\label{cha:related_work}

In this chapter four major pieces of major work is going to be described.
Section~\ref{sec:lvars} describes the LVars system and LVish, the extension
of LVars which introduces the concepts of quiescence and freezing. 
Section~\ref{sec:reactive_async} describes Reactive Async, a programming model
inspired by the extended LVars system. In section~\ref{sec:lacasa} the LaCasa
system is introduced, and finally the concept of spores is briefly introduced in
section~\ref{sec:spores}.

\section{LVars}\label{sec:lvars}

LVars~\parencite{kuper2013lvars} is a programming model that was introduced as a
solution to the problem of guaranteed deterministic concurrent programs. It
generalizes the concept of write-once data
structures~\parencite{nikhil1989structures}, also called IVars, with the ability
to write more than once but limiting update operations to being monotonic. I.e.
the values taken by LVars are part of a programmer specified lattice and all
updates are done through a join operation of the old and new values. This
ensures that writes commute~\parencite{kuper2013lvars}.

\subsection{Stores \& Lattice}%
\label{sub:stores_and_lattice}

At the foundation of the LVars system lays lattices. The values resulting from a
computation is going to be elements from a lattice $\LVarsLat$, specified by the
programmer. These lattice values are stored in a \emph{store}.  This is a set of
pairs consisting of a location and a lattice element. For a location there is
exactly one value. Letting $\LVarsLoc$ be the set of locations, a store $S$ can
be represented using a partial map
\begin{equation*}
  S: \LVarsLoc \rightharpoonup \LVarsLat.
\end{equation*}

\subsection{LVars Operations}%
\label{sub:lvars_operations}

The LVars model supports three main operations
\begin{itemize}
  \item Extending the store with a new location. This takes a fresh location and
    sets its value to $\bot_{\LVarsLat}$, the bottom element of $\LVarsLat$.
  \item Updating a store location, also called a \emph{put} operation. This
    operation takes a store location $r$ and a lattice value $l$. Given a store
    $S$ this updates location $r$ with $l \sqcup S(r)$. To ensure determinism,
    any put that takes a store location to $\top_{\LVarsLat}$ results in an
    error.
  \item A read operation also referred to as \emph{get}. This operation is
    further specified with a threshold set, i.e. a set of lattice values, and a
    store location. The operation blocks until the store location has passed one
    of the values in the threshold set, upon which it returns this value. In
    order to ensure determinism the elements in the threshold set are required
    to be mutually \emph{incompatible}. Two elements $l_1, l_2$ are incompatible if
    \begin{equation*}
      l_1 \sqcup l_2 = \top_{\LVarsLat}
    \end{equation*}
    where $\top$ is the top element of $\LVarsLat$.
\end{itemize}

% TODO brief introduction to lvars:
% Programming model to ensure deterministic concurrent execution
% Quasi determinism (maybe later)
% stores and lattice variables
% put and get operations

\subsection{LVish: Extending LVars}%
\label{sub:lvish_extending_lvars}

% TODO describe lvish, the lvars extension:
% freezing
% quasi determinism
% quiescence
% indicate that there are problems

\begin{equation}
\end{equation}

% Should describe the basic ideas of LVars and also mention that there are
% problems with the proof and hint of bigger problems.

\section{Reactive Async}\label{sec:reactive_async}

% Should describe reactive async. There are multiple issues with this system,
% e.g. that great care has to be taken in order to make the system
% deterministic, e.g. make sure that only certain types of operations are
% allowed in the callbacks.

\section{LaCasa}\label{sec:lacasa}

% Describes the basic ideas of lacasa and the idea of using OCAP constraints to 
% assure determinism

\section{Spores}\label{sec:spores}

% Describe the basic ideas of spores: (dis)allow certain types of captures to
% enforce certain properties





\chapter{Challenges of Deterministic Concurrency}
\label{cha:challenges}

% Describe the problem of arbitrary reading data from an object that is
% concurrently being changed. Describe the idea of threshold reads and how it
% can be utilized to get determinism.

% TODO describe problems inherent with concurrent determinism
% TODO problems inherent in reactive async
% TODO problem inherent with freeze in LVish

% problems of reactive async includes:
% access to shared state is permitted within callbacks
% even then semantics of callback thread spawning is flawed

% Should we do these in the challenges? Prolly yes.
% TODO explain the problem of callback spawning
% TODO explain the problem of shared state

As noted in the introduction, achieving a high level of parallelization and
ensuring determinism is a difficult task. The inherent non-determinism of
concurrent operations leaves the programmer with the hard task of considering
all paths of execution and making sure that they all lead to the same result.

Systems like Reactive Async and LVish both have the ambition of moving this
burden off the programmer and onto the programming model and type system. Both
however have problems. In this chapter we describe these problems in an informal
way. As a running example we will use the lattice of integers $(\integers,
\leq)$ with a bottom value of $-\infty$.


\section{Syntax Explanation}%
\label{sec:operations}

In order to construct examples we will use a language akin to Reactive Async and
Scala. This will have a syntax similar to Scala function
syntax~\parencite{scalabasics}. In this syntax we write a callback function as in
Figure~\ref{fig:callback_syntax}. This takes a lattice value \ilc{l} as
argument. In our example callback code we will use common programming language
constructs such as conditionals (\ilc{if}-statements) and field accesses.
Furthermore we will use two operations that will be explained below.

\begin{figure}
  \centering
  \begin{minipage}{0.763\textwidth}
    \begin{lstlisting}[]
(l) => {
  /* callback code here */
}
    \end{lstlisting}
  \end{minipage}
  \caption{Example of callback syntax.}
  \label{fig:callback_syntax}
\end{figure}

\subsection{Cells}%
\label{sub:cells}

An integral part of both LVish and Reactive Async is a data type that holds 
the resulting value of a computation. Callbacks also need to be
registered somehow. Although the implementations differ, we will
use a unified representation for our examples below. All examples should be
reconstructable in their respective setting of Reactive Async or LVish.

We call our data type a \emph{cell}, in alignment with Reactive Async. A cell
holds two values: A lattice element (also called cell value below) and a freeze bit.
This is similar to an LVar in LVish. Registration and spawning of callbacks will
be explained for each example separately.

\subsection{The \ilc{put} and \ilc{freeze} Statements}%
\label{sub:the_put_statement}

There are two main operations used in our examples below. The freeze
statement \ilc{frz(...)} takes one argument of cell type and freezes the cell
specified, i.e., sets its associated freeze bit to true. Furthermore it returns
the associated cell value. The put statement \ilc{put(...)} takes two arguments,
one of cell type and one lattice value. This updates the cell value with the
specified one, using the join operation, similar to how an LVar is
updated~\parencite{kuper2013lvars}. If the freeze bit is true and the join
result is not the same as the current value, the put operation results in a step
to $\Error$. The return value of \ilc{put} is the cell reference itself. We can
see an example of both statements in Figure~\ref{fig:ex_cell_op}. We note that
the behaviour of both operations mimic their LVish counterpart closely.


% TODO change update to change above in callback spawning semantics.
\begin{figure}
  \centering
  \begin{subfigure}[t]{0.4\textwidth}
    \begin{lstlisting}[numbers=none,mathescape=true]
x = put($c$, l)
    \end{lstlisting}
    \caption{Example put operation. We put the lattice value assigned to \ilc{l} into
    cell $c$. This may result in an error.}
  \end{subfigure}
  \quad
  \begin{subfigure}[t]{0.4\textwidth}
    \begin{lstlisting}[numbers=none,mathescape=true]
x = frz($c$)
    \end{lstlisting}
    \caption{Example freeze operation. We freeze the cell $c$ and assign its cell
    value to variable \ilc{x}.}
  \end{subfigure}
  \caption{Example cell operations.}
  \label{fig:ex_cell_op}
\end{figure}

\section{Problems of Reactive Async}%
\label{sec:problems_of_reactive_async}

For Reactive Async, many put operations are implicit in code due to the
programming framework structure~\parencite{conf/scala/HallerGES16}. However, in
order to reduce complexity, all put operations in the examples below will be
explicit.

Reactive Async has two main problems. To highlight these we use a similar
dependency graph, similar to the one in Figure~\ref{fig:ra_example}.
Figure~\ref{fig:ra_example2} describes a dependency graph which is a little more
complex. Here we have three callback functions $f_1, f_2$ and $f_3$ registered
to cells $c_1, c_2$ and $c_3$ respectively. The arrows annotated with the above
callbacks indicate to what cell the corresponding function will make a
put operation.

\subsection{Callback Spawning Semantics}%
\label{sub:callback_spawning_semantics_ra}

We now informally explain how callbacks are spawned in Reactive Async.  Let a
callback $f$ be registered to a cell $c$, and say the cell value of $c$ changes
to $l$. Then a new callback thread running $f(l)$, i.e., function $f$ with
parameter $l$, will be spawned.

\begin{figure}
  \centering
  \begin{tikzpicture}
    \node[circle,draw] (c1) at (0, 1) {$c_1$};
    \node[circle,draw] (c2) at (0,-1) {$c_2$};
    \node[circle,draw] (c3) at (2, 0) {$c_3$};
    \node[circle,draw] (c4) at (4, 0) {$c_4$};
    \draw[-{Latex[length=0.3cm]}] (c1) -- node[above] {$f_1$} (c3);
    \draw[-{Latex[length=0.3cm]}] (c2) -- node[above] {$f_2$} (c3);
    \draw[-{Latex[length=0.3cm]}] (c3) -- node[above] {$f_3$} (c4);
  \end{tikzpicture}
  \caption{Reactive Async dependency graph.}
  \label{fig:ra_example2}
\end{figure}

\subsection{Problematic State Sharing}%
\label{sub:sharing_state}

The first problem of Reactive Async is that it is not prohibited for callbacks
to share mutable state. To exemplify this let $f_1$ and $f_2$ from above be as
in Figure~\ref{fig:ra_fun_shared_state}. We ignore $f_3$ and $c_4$ for now. 
Say both $f_1$ and $f_2$ are scheduled to run and that the initial value of
\ilc{Global.someInt} is $0$.  The lattice is $(\integers, \leq)$ and we let the
initial cell value of all cells be $0$.  It should be clear that depending on
the order in which $f_1$ and $f_2$ are scheduled to execute, $c_3$ will have 
different final cell values. If $f_1$ runs first $c_3$ will have a final cell
value of $1$. If $f_2$ runs first, $c_3$ will have a final value of $0$. This is
due to that the callbacks share mutable state through the \ilc{Global} object
and that put operations updates a cell using the join operation.

This problem can be remedied using the OCAP model in a similar manner to LaCasa.
By ensuring that callback threads are unconnected in the reference graph, we ensure
that they do not share any state. We do this by ensuring that for each pair of
threads, the intersection of their connected components will only contain cell
objects. In chapter~\ref{cha:core_language}, the type system is used to enforce
this property.

\begin{figure}
  \begin{subfigure}[b]{0.5\textwidth}
    \begin{lstlisting}
(l) => {
  Global.someInt = 1
  put($c_3$,0)
}
    \end{lstlisting}
    \caption{Code of $f_1$.}
  \end{subfigure}
  ~
  \begin{subfigure}[b]{0.5\textwidth}
    \begin{lstlisting}
(l) => {
  if (Global.someInt == 1)
    put($c_3$, 1)
  else put($c_3$,0)
}
    \end{lstlisting}
    \caption{Code of $f_2$.}
  \end{subfigure}
  \caption{Two callback functions sharing state.}
  \label{fig:ra_fun_shared_state}
\end{figure}


\subsection{Problematic Callback Spawning Semantics}%
\label{sub:problematic_callback_spawning_semantics}

Remedying the above problem does not however yield a deterministic system.
There is a flaw inherent in how Reactive Async decides to spawn a callback
thread. A thread is only spawned when a cell value actually changes, which means
that all put operations does not necessarily spawn a callback. To see that this
is a problem, let $f_1$, $f_2$ and $f_3$ from Figure~\ref{fig:ra_example2} be as in
Figure~\ref{fig:ra_fun_callback_spawn}. As before, let the lattice be
$(\integers, \leq)$ and all cells have an initial cell value of $0$. Furthermore
let both $f_1$ and $f_2$ be scheduled to run.

Say $f_1$ is scheduled to run before $f_2$. The result will be a put of $2$ to
cell $c_3$.  Since $2 \sqcup 0 = 2$, the cell value of $c_3$ will be changed to
$2$. Thus it will spawn the callback $f_3(2)$. This callback will result in a
put of $1$ to $c_4$, i.e., $c_4$ will have its cell value updated to $1$ by $1
\sqcup 0 = 1$.  Running $f_2$ now results in the value of $c_3$ being updated to
$3$ which will spawn the callback $f_3(3)$. The resulting put of $0$ to $c_4$
will not update $c_4$ since $0 \leq 1$. The final cell value of $c_4$ in this
case will be $1$.

Now let $f_2$ execute before $f_3$. This results in a put of $3$ to $c_3$, which
results in $c_3$ being updated to $3$. The callback $f_3(3)$ will be spawned and
result in a put of $0$ to $c_4$. Since $0 \sqcup 0 = 0$ this will not change the
value of $c_4$. Now, if $f_1$ executes, it will result in a put of $2$ to $c_3$.
Since the cell value of $c_3$ is already $3$ and $2 \sqcup 3 = 3$, this will not
result in an update of $c_3$. Thus the callback $f_3(2)$ is never spawned. The
final cell value of $c_4$ for this execution order is $0$, clearly different
from before.
\begin{figure}
  \begin{minipage}{0.5\textwidth}
    \begin{subfigure}[b]{\linewidth}
      \begin{lstlisting}
(l) => put($c_3$,2)
      \end{lstlisting}
      \caption{Code of $f_1$.}
    \end{subfigure}
    
    \begin{subfigure}[b]{\linewidth}
      \begin{lstlisting}
(l) => put($c_3$,3)
      \end{lstlisting}
      \caption{Code of $f_2$.}
    \end{subfigure}
  \end{minipage}
  ~
  \begin{minipage}{0.5\textwidth}
    \begin{subfigure}[b]{\linewidth}
      \begin{lstlisting}
(l) => {
  if (l == 2)
    put($c_4$,1)
  else put($c_4$,0)
}
      \end{lstlisting}
      \caption{Code of $f_3$.}
    \end{subfigure}
  \end{minipage}
  \caption{Simple callback functions breaking determinism using callback
  spawning semantics of Reactive Async.}
  \label{fig:ra_fun_callback_spawn}
\end{figure}

The problem of callback spawning can be remedied with the use of the threshold
sets of LVish. Instead of only spawning a callback thead when a cell value
changes, a callback will be spawned for each lattice value passed in the
specified threshold set.

% INTRO
% As noted in the introduction, a 
% there are problems inherent in both reactive async and LVish. 

% TODO describe the reactive async problems first

\section{A Problem with LVish}%
\label{sec:a_problem_of_lvish}

As hinted in the end of Section~\ref{sec:lvars}, the proof of quasi-determinism
of LVish is flawed. While this leaves the quasi-determinism of LVish an open
problem this is not really interesting. This is due to the fact that by adding a
simple if-statement to the language, we can easily construct a counterexample.

\subsection{Callback Spawning Semantics}%
\label{sub:callback_spawning_semantics_lvish}

The callback thread spawning semantics of the below example are as follows. If
the value of a cell $c$ with a registerered callback $f$ passes $l \in T$, $T$
being the threshold set associated with $f$, the callback $f(l)$ will be spawned
eventually. This is equivalent to the semantics of LVish.

%The core language of LVish is very minimal with a very implicit way to implement
%threads. We will therefore resort to writing the counterexample in a language
%more akin to the system of Reactive Async, with the modification of having all
%put operations being explicit in code and requiring all callbacks to have an
%associated threshold set, similar to LVish. It can easily be translated into a
%language and semantics of LVish (with the extension of an if-statement).

\subsection{LVish Freezing Problem}%
\label{sub:lvish_freezing_problem}

Say we have two cells $c_1$ and $c_2$, and two callbacks $f_1$ and $f_2$. Let the
lattice be $(\integers, \leq)$, let the initial cell values of both cells be 
$0$ and let both associated freeze bits be $\LVarsFalse$. The
callbacks $f_1$ and $f_2$ are registered to some auxilliary cell $c$ according to
Table~\ref{tab:cellreg}.

\begin{table}
  \centering
  \begin{tabular}{c|c|c}
    Callback & Cell & Threshold Set \\
    \hline
    $f_1$ & $c$ & $\left\{ 0 \right\}$ \\
    $f_2$ & $c$ & $\left\{ 0 \right\}$ \\
  \end{tabular}
  \caption{Cell registration and threshold sets for $f_1$ and $f_2$.}
  \label{tab:cellreg}
\end{table}

Now let $f_1$ and $f_2$ be as in Figure~\ref{fig:lvish_fun_breaking}.
Furthermore let both $f_1$ and $f_2$ be scheduled (due to some put of integer
value greater than $0$ to auxilliary cell $c$).

If $f_1$ executes before $f_2$ we end up with an execution as described by
Table~\ref{stab:f_1exec}. It is clear that we never step to $\Error$.
If $f_2$ executes before $f_1$ we can describe the execution with
Table~\ref{stab:f_2exec}. Clearly the results differ and are non-erroneous, a
non-quasi-deterministic result.

\begin{figure}
  \centering
  \begin{subfigure}[t]{0.4\textwidth}
    \begin{lstlisting}[mathescape=true]
(l) => {
  x = frz($c_1$)
  if ( x == 0 )
    put($c_2$, 1)
}
    \end{lstlisting}
    \caption{Callback code for $f_1$.}
  \end{subfigure}
  \quad
  \begin{subfigure}[t]{0.4\textwidth}
    \begin{lstlisting}[mathescape=true]
(l) => {
  x = frz($c_2$)
  if ( x == 0 )
    put($c_1$, 1)
}
    \end{lstlisting}
    \caption{Callback code for $f_2$.}
  \end{subfigure}
  \caption{Callback code to break determinism in LVish.}
  \label{fig:lvish_fun_breaking}
\end{figure}

\begin{table}
  \centering
  \begin{subtable}[t]{\textwidth}
    \centering
    \begin{tabular}{l|c|c|c|c}
      Execution point & $c_1$ cell val & $c_1$ frz bit & $c_2$ cell val & $c_2$
      frz bit \\
      \hline
      $f_1$ start & $0$ & \LVarsFalse & $0$ & \LVarsFalse \\
      $f_1$ termination & $0$ & \LVarsTrue & $1$ & \LVarsFalse \\
      $f_2$ start & $0$ & \LVarsTrue & $1$ & \LVarsFalse \\
      $f_2$ termination & $0$ & \LVarsTrue & $1$ & \LVarsTrue \\
    \end{tabular}
    \caption{$f_1$ executes before $f_2$.}
    \label{stab:f_1exec}
  \end{subtable}

  \vspace{0.5em}

  \begin{subtable}[t]{\textwidth}
    \centering
    \begin{tabular}{l|c|c|c|c}
      Execution point & $c_1$ cell val & $c_1$ frz bit & $c_2$ cell val & $c_2$
      frz bit \\
      \hline
      $f_2$ start & $0$ & \LVarsFalse & $0$ & \LVarsFalse \\
      $f_2$ termination & $1$ & \LVarsFalse & $0$ & \LVarsTrue \\
      $f_1$ start & $1$ & \LVarsFalse & $0$ & \LVarsTrue \\
      $f_1$ termination & $1$ & \LVarsTrue & $0$ & \LVarsTrue \\
    \end{tabular}
    \caption{$f_2$ executes before $f_1$.}
    \label{stab:f_2exec}
  \end{subtable}
  \caption{Two different execution paths contradicting LVish quasi-determinism.}
\end{table}

This shows that as soon as we introduce the if-statement, a basic part of any
programming language, we can easily break quasi-determinism. The problem is inherent
in the freeze statement. By freezing a variable we could attain information
about the value and freeze bit of another cell. In the example above, by
freezing cell $c_1$ ($c_2$) and inspecting its value we can attain information
about whether $c_2$ ($c_1$) has been frozen or not. By conditioning on this
information, we can skip a put operation that would otherwise lead to an error.

A simple solution to this problem is to simply remove the capability of freezing
a cell and this is the approach of the system presented in
chapter~\ref{cha:core_language}. It should however be possible to add other
forms of freezing. This will be discussed in
chapter~\ref{cha:discussion_and_conclusion}.



\chapter{Core Language}%
\label{cha:core_language}

In this chapter a basic core language is introduced. It will build a lot upon
the core language of LaCasa and incorporate features of LVish. In the end we will
have an object oriented language incorporating many of the features of LVish with
a type system enforcing OCAP properties, much like the system of LaCasa.

\section{Syntax}
\label{sec:syntax}

The language which we shall call Reactive Async Core Language (RACL) has a big
similarity with the core language from LaCasa. Many expressions are the same
except for a few removals and additions. The language grammar is defined in
Figure~\ref{fig:racl_grammar}. It is a simple object oriented language which is
parameterized on the lattice $(\LatVals, \sqsubseteq)$.

\begin{figure}
  \centering
  $\begin{array}{lll@{\hspace{4mm}}l}
    p &::= &\overline{cd}~\overline{vd}~t  & \mbox{Program} \\
    cd &::= &\texttt{class}~C~\texttt{extends}~D~\{\overline{fd}~\overline{md}\}
    & \mbox{Class} \\
    vd,fd &::= &\texttt{var}~f~:~\tau & \mbox{Variable/Field} \\
    md &::= &\texttt{def}~m(x: \sigma):~\tau = t & \mbox{Method}\\
    &&&\\
    \sigma,\tau & ::= & & \mbox{Types} \\
    & & C, D & \mbox{Class types} \\ 
    &| & \CellType & \mbox{Cell type} \\
    %&| & \NullType & \mbox{Null type} \\
    &| & \LatType & \mbox{Lattice value type} \\
    &&&\\
    t &::=& & \mbox{Terms} \\
    & & x & \mbox{Variable} \\
    &|& \texttt{let}~x = e~\texttt{in}~t &\mbox{Let binding} \\
    &&&\\
    e &::=& & \mbox{Expression} \\
    & & l & \mbox{Lattice value} \\
    &|& \texttt{null} & \mbox{Null reference} \\
    &|& x &\mbox{Variable} \\
    &|& x.f &\mbox{Field selection} \\
    &|& x.f = y & \mbox{Field assignment} \\
    &|& \texttt{new}~C & \mbox{Class instance creation} \\
    &|& \texttt{new}~\texttt{Cell} & \mbox{Cell instance creation} \\
    &|& x.m(y) & \mbox{Method invocation} \\
    &|& x~\texttt{put}~y & \mbox{Cell value update} \\
    &|& \texttt{when}~x~\texttt{pass}~y~\texttt{then}~(\overline{cap}, z
    \Rightarrow t) & \mbox{Dependency creation} \\
    &&&\\
    cap & ::= & x = y & \mbox{Variable capture} \\
  \end{array}$
  \caption{Grammar of RACL}
  \label{fig:racl_grammar}
\end{figure}

We can see that a program $p$ consists of a sequence of class definitions
$\overline{cd}$, a sequence of variable declarations $\overline{vd}$ and a term
$t$. A class definition $cd$ consists of a name specifier $C$, inheritance
specifier $D$, field declarations $\overline{fd}$ and method definitions
$\overline{md}$. Variable and field declarations both have the same form
consisting of a name specifier $f$ and a type $\tau$. Method definitions are
also standard, with one thing to note that all methods takes exactly one input.
More complicated inputs can be constructed using a container class.

The type specifiers $\sigma$ and $\tau$ can take on the values as specified.
Note that we have the special \CellType{} and \LatType{} types which are meant
to represent the cells of reactive async and values from the used lattice
respectively. Types will be discussed more in Section~\ref{sec:type_system}.

As in LaCasa, the terms of RACL are written in A-normal form (i.e.\ every
subexpression is named). Most of the expressions should be self-explanatory.
However, note for example that we have a separate instance creation expression
for cells, an expression for updating the value of a cell aswell as an
expression for creating dependencies between cells. The dependency creation
expression is probably the most interesting since it mimics the syntax of
spores~\parencite{conf/ecoop/MillerHO14}. All captured variables are clearly
stated in the sequence of captures $\overline{cap}$. 

\section{Semantics}%
\label{sec:semantics}

In this section a small-step operational semantic of \RACL{} will be
introduced. First a brief overview is made and then a few of the more
interesting or non-standard execution rules will be explained.

In short, the following definitions says that the state of the execution of a
\RACL{} program consists of a \emph{heap} and a \emph{thread set}. The heap is
represented by a partial map from object identifiers $\OIDs$ to objects
$\Objects$. The thread set is a set of threads, each of which consisting of call
frame stacks. Each call frame holds a local variable environment and a term to
be executed. Steps between states can be made accoring to rules on either frame,
frame stack or thread set level. All steps taken on lower levels are propagated
to thread set level using auxilliary rules. After the following definitions we
describe the execution rules in more detail.

\subsection{Semantical Definitions}%
\label{sub:semantical_definitions}

\begin{definition}[Sets] We define the following sets.
  \begin{itemize}
    \item We let $\VarNames$ be the set of all allowed variable names.
    \item We let $\FieldNames$ be the set of all allowed field names.
    \item We let $\LatVals$ be the set of all lattice elements.
    \item We let $\OIDs$ be the set of all object identifiers.
    \item We let $\TIDs$ be the set of all thread identifiers.
    \item We let $\NullVal$ be the special null value.
    \item We let $\Values$ be the set of all possible runtime values
      \begin{equation*}
        \Values = \LatVals \cup \OIDs \cup \left\{ \NullVal \right\}.
      \end{equation*}
    \item We let $\ocapstats$ be the set of OCAP statuses
      \begin{equation*}
        \ocapstats = \left\{ \ocap, \nocap \right\}.
      \end{equation*}
  \end{itemize}
\end{definition}

\begin{definition}[Heap Objects]\label{def:heap_obj}
  We let $\Objects$ be the set of all \emph{heap objects}, i.e. objects of the
  form
  \begin{equation*}
    \Obj{C, \FM} \text{ or } \Cell{\DEP, l}.
  \end{equation*}
  $\Obj{C, \FM}$ is a \emph{class object} where $C$ is a class name. The \emph{field
  map} $\FM$ is a partial map
  \begin{equation*}
    \FM \in \FieldNames \rightharpoonup \Values.
  \end{equation*}
  In $\Cell{\DEP, l}$ is a \emph{cell object} where $l \in \LatVals$. The
  \emph{dependency set} $\DEP$ is a set of elements of the form
  \begin{equation*}
    (l' , (L_{\text{env}}, z \Rightarrow t))^\iota
  \end{equation*}
  where $l' \in \LatVals$, $L_{\text{env}}$ is a local environment, $z \in
  \VarNames$, $t$ is a term and $\iota \in \TIDs$ is a unique thread identifier. The
  uniqueness of $\iota$ is global to the state of a program.
\end{definition}

\begin{definition}
  A \emph{heap} $H$ is a partial map
  \begin{equation*}
    H \in \OIDs \rightharpoonup \Objects.
  \end{equation*}
  A heap must always contain the global object as specified in
  definition~\ref{def:state_zero}.
\end{definition}

\begin{definition}
  A \emph{local environment} $L$ is a partial map
  \begin{equation*}
    L \in \VarNames \rightharpoonup \Values.
  \end{equation*}
\end{definition}

\begin{definition}\label{def:thread_sets}
  A \emph{frame} $F$ is an object of the form
  \begin{equation*}
    \sframe{L, t}
  \end{equation*}
  where $L$ is a local environment, $t$ is a term and $s \in \VarNames$ is a
  return tag.

  A \emph{frame stack} $\FS$ is a finitely large stack of frames. We can write
  all of these recursively as either the empty stack $\FS = \varepsilon$, or as
  $\FS = F \circ \GS$ where $\GS$ is also a frame stack.

  A \emph{thread set} $P$ is a finite set 
  \begin{equation*}
    P = \left\{ \FS_i|_{a_i}^{\iota_i} \right\}_{i = 1}^{n}
  \end{equation*}
  of non-empty frame stacks $\FS_i$ tagged with a unique thread id $\iota_i \in \TIDs$ and an
  OCAP status $a_i \in \ocapstats$. Note that $P$ can be the empty set. We call
  a thread tagged with $\ocap$ an \emph{ocap thread}
\end{definition}

\begin{definition}
  We define a \emph{state} to be either the error state $\Error$ or a pair $H,
  P$, where $H$ is a heap and $P$ is a thread set.
\end{definition}

\begin{definition}
  The set of all valid types $\ValidTypes$ for a given program $p$ consists of
  $\AnyRefType$, $\NullType$, $\CellType$, $\LatType$ and all defined
  class types in $p$, $\ClassTypes$. I.e.
  \begin{equation*}
    \ValidTypes = \ClassTypes \cup \left\{ \AnyRefType, \NullType, \CellType,
    \LatType \right\}
  \end{equation*}
\end{definition}

\begin{definition}
  The \emph{default value function} $\default \in \ValidTypes \to \Values$ is
  defined as follows.
  \begin{equation*}
    \default(\tau) =
    \begin{cases}
      \bot_{\LatVals} & \text{ if } \tau = \LatType \\
      \NullVal        & \text{ otherwise } 
    \end{cases}
  \end{equation*}
\end{definition}

\begin{definition} \label{def:state_zero}
  Execution of a program starts from state $S_0 = H_0, P_0$. $H_0$ and
  $P_0$ are defined as follows for a program $p =
  \overline{cd}~\overline{vd}~t$.
  \begin{equation*}
    \begin{gathered}
      H_0 = [o_g \mapsto \Obj{C_g, \FM_g}] \\
      \FM_g = [f \mapsto \default(\sigma) \mid (\VarDecl{f}{\sigma}) \in \overline{vd}]
    \end{gathered}
  \end{equation*}
  where $o_g$ is a reserved object identifier for the \emph{global object} $\Obj{C_g,
  \FM_g}$. $C_g$ is the \emph{global class name}.  
  \begin{equation*}
    \begin{gathered}
      P_0 = \left\{ \xframe{L_0, t} \circ \varepsilon
      |_{\nocap}^{\iota_{\text{main}}} \right\} \andalso
      L_0 = [ \Global \mapsto o_g ].
    \end{gathered}
  \end{equation*}
  where $\iota_{\text{main}}$ is a reserved thread identifier.
\end{definition}

In our reduction rules below, we refer to fresh identifiers. The following is
one way to define this. 
\begin{definition}
  For each state $S = H, P$ there are two related sets of currently used object
  identifiers $\mathcal{O}(S)$ and thread identifiers $\mathcal{D}(S)$.
  Respectively these contain all object and thread identifiers occuring in $H$
  and $P$. 
  
  We say that an object (thread) identifier $o$ ($d$) is \emph{fresh} at a state $S$ if
  $o \not\in \OIDs(S)$ ($d \not\in \TIDs(S)$).
\end{definition}

% TODO define \Gamma_0

\subsection{Reduction Rules}%
\label{sub:reduction_rules}

An execution step from state $S$ to $S'$ in RACL is expressed using the relation
\begin{equation*}
  S \Rrightarrow S'.
\end{equation*}
RACL reduction rules are expressed at three different levels:
Thread set, frame stack and frame level. 
Therefore we will also see the relations 
\begin{equation*}
  H, \FS \;\FSRedTo\; H', \FS' \quad \text{ and } \quad H, F\; \FRedTo\; H', F'
\end{equation*}
This is to allow expression of e.g. single frame execution,
thread creation and method calls as will be explained below. 
A reduction on frame or frame stack level is propagated up to thread set level
with the rules \EFProp{} and \EFSProp{} which are defined in
figures~\ref{fig:fs_red_rules} and~\ref{fig:threads_red_rules}.

In Figure~\ref{fig:frame_red_rules}, reduction rules for single frames are
defined. These are rules that advance the state of a single frame $F$ and
possibly changes the heap $H$. We have very simple ones like the rule {\sc
E-LVal}, which assigns the specified lattice value to a local variable, and the
rule {\sc E-Var} which assigns the value of one local variable to another.
The rules \ESelect{} and \EAssign{} respectively fetches and sets the value of a
field $f$ of an object on the heap. Rules \ENew{} and \ENewCell{} creates a new
object on the heap of either class or cell type and assigns the corresponding
new object identifier to a local variable. All fields of a new class object are
initiated with a default value according to the $\default$ function. The rule
\EPut{} updates the cell value of a cell object on the heap through a join
operation.  Finally the rule \EWhen{} is responsible for adding a new dependency
callback. Looking at this rule more closely we see that it captures the
variables specified by $\overline{cap}$ and creates a local environment
$L_{\text{env}}$. This is then stored, together with the closure $w \Rightarrow
t'$, a threshold value $l'$ and a fresh thread identifier $\iota$, in the new
dependency set $\DEP'$. \EWhen{} mimics the capture semantics of
spores~\parencite{conf/ecoop/MillerHO14}.  Note that all rules in
Figure~\ref{fig:frame_red_rules} assigns something to the variable $x$.

\begin{figure}[]
  \scax{E-Null}
  {H, \sFrame{L}{\Let{x}{\NullVal}{t}} \; \FRedTo \; H, \sFrame{L[x \mapsto
  \NullVal]}{t}}

  \RuleSpace{}

  \scax{E-LVal}
  {H, \sFrame{L}{\Let{x}{l}{t}} \; \FRedTo \; H, \sFrame{L[x \mapsto
  l]}{t}}

  \RuleSpace{}

  \scax{E-Var}
  {H, \sFrame{L}{\Let{x}{y}{t}} \; \FRedTo \; H, \sFrame{L[x \mapsto
  L(y)]}{t}}

  \RuleSpace{}

  \scrule{E-Select}
  {L(y) = o \andalso H(o) = \Obj{C, \FM} \andalso f \in \dom(\FM)}
  {H, \sFrame{L}{\Let{x}{ \FSel{y}{f} }{t}} \; \FRedTo  \\ 
  H, \sFrame{L[x \mapsto \FM(f)]}{t}}

  \RuleSpace{}

  \scrule{E-Assign}
  {L(y) = o \andalso H(o) = \Obj{C, \FM} \andalso f \in \dom(\FM) \\
  \FM' = \FM[f \mapsto L(z)] \andalso H' = H[o \mapsto \Obj{C, \FM'}]}
  {H, \sFrame{L}{\Let{x}{\FAss{y}{f}{z}}{t}} \; \FRedTo \\
   H', \sFrame{L[x \mapsto L(z)]}{t}}

  \RuleSpace{}

  \scrule{E-New}
  {o\text{ fresh object identifier } \\
  \FM = [f \mapsto \default(\sigma): (\VarDecl{f}{\sigma}) \in \fdecls(C)] \\
  H' = H[o \mapsto \Obj{C, \FM}]}
  {H, \sFrame{L}{ \Let{x}{\New{C}}{t} } \; \FRedTo \\
  H', \sFrame{L[x \mapsto o]}{t}}
  
  \RuleSpace{}

  \scrule{E-NewCell}
  {o\text{ fresh object identifier } \andalso
  H' = H[o \mapsto \Cell{\emptyset, \bot_{\LatVals{}}}]}
  {H, \sFrame{L}{ \Let{x}{\NewCell}{t} } \; \FRedTo \\
  H', \sFrame{L[x \mapsto o]}{t}}

  \RuleSpace{}
  
  \scrule{E-Put}
  {L(y) = o \andalso H(o) = \Cell{\DEP, l} \\
  L(z) = l' \andalso c' = \Cell{\DEP, l \sqcup l'} \\
  H' = H[o \mapsto c']}
  {H, \sFrame{L}{\Let{x}{\Put{y}{z}}{t}} \; \FRedTo \\
  H', \sFrame{L[x \mapsto L(y)]}{t}}

  \RuleSpace{}

  \scrule{E-When}
  {L(y) = o \andalso H(o) = \Cell{\DEP, l} \andalso L(z) = l' \\
  L_{\text{env}} = [u \mapsto L(u') \mid (\Capt{u}{u'}) \in \overline{cap}]
  \andalso cb = (L_{\text{env}}, w \Rightarrow t') \\
  \iota\text{ fresh thread identifier } \andalso \DEP' = \DEP \cup (l', cb)^\iota \\
  H' = H[o \mapsto \Cell{\DEP', l}] }
  { H, \sFrame{L}{ \Let{x}{ \When{y}{z}{ (\overline{cap}, w \Rightarrow t') }}{t} } \\ \FRedTo \;
  H', \sFrame{L[x \mapsto L(y)]}{t} }
  \caption{\RACL{} single frame reduction rules.}
  \label{fig:frame_red_rules}
\end{figure}


In Figure~\ref{fig:error_red_rules}, rules that lead to $\Error$ are defined. These
are all what would be called null-pointer exceptions in a language like Java.
That is, that we try to access an object through the $\NullVal$ identifier.
Errors are propagated to frame stack and then thread set level using rules
\EErrorFS{} and \EErrorP{} from figures~\ref{fig:fs_red_rules} and
\ref{fig:threads_red_rules}.

\begin{figure}
  \scrule{E-NullSelect}
  {L(y) = \NullVal}
  {H, \sFrame{L, \Let{x}{\FSel{y}{f}}{t} } \; \FRedTo \; \Error}

  \RuleSpace{}

  \scrule{E-NullAssign}
  {L(y) = \NullVal}
  {H, \sFrame{L, \Let{x}{\FAss{y}{f}{z}}{t} } \; \FRedTo \; \Error}

  \RuleSpace{}

  \scrule{E-NullCall}
  {L(y) = \NullVal}
  {H, \sFrame{L, \Let{x}{\Call{y}{m}{z}}{t} } \; \FRedTo \; \Error}

  \RuleSpace{}

  \scrule{E-NullPut}
  {L(y) = \NullVal}
  {H, \sFrame{L, \Let{x}{\Put{y}{z}}{t} } \; \FRedTo \; \Error}

  \RuleSpace{}

  \scrule{E-NullWhen}
  {L(y) = \NullVal}
  {H, \sFrame{L}{\Let{x}{  \When{y}{z}{ (\overline{cap}, w \Rightarrow t')}}{t}
  }  \\ \FRedTo \; \Error}
  \caption{\RACL{} error spawning rules.}
  \label{fig:error_red_rules}
\end{figure}

In Figure~\ref{fig:fs_red_rules}, rules for frame stack reductions are defined.
Here we find the aforementioned \EFProp{} and \EErrorFS{} together with rules
for method calls and returns. \ECall{} handles method calls by creating a new
frame with the corresponding term and local environment. Note that the new local
environment differs depending on the OCAP status of the working thread. If the
thread is tagged with $a = \nocap$, the new frame also has access to the global
environment, while if $a = \ocap$ it will not. Note also that the new frame
stack is tagged with the variable name $x$, the variable to which the method
result will be assigned to with rule \ERet{}. \ERet{} corresponds to method
return and utilizes the aforementioned variable name tag. Note that none of
these rules changes the thread identifier or OCAP status of a thread.

\begin{figure}
  \scrule{E-Call}
  {
    L(y) = o \andalso H(o) = \Obj{C, \FM} \\
    \mbody(m, C) = w \to t' \\
    L_{\text{base}} =
    \begin{cases}
      \emptyset & \text{if } a = \ocap \\
      L_0 & \text{if } a = \nocap
    \end{cases} \\
    L' = L_{\text{base}}[\This \mapsto L(y), w \mapsto L(z)] 
  }
  {H, \sFrame{L}{ \Let{x}{ \Call{y}{m}{z} }{t} } \circ \FS |_a^\iota \; \FSRedTo \\
  H, \Frame{L'}{t'}{x} \circ \sFrame{L}{t} \circ \FS |_a^\iota}

  \RuleSpace{}

  \scax{E-Ret}
  {H, \Frame{L'}{x}{y} \circ \sframe{L, t} \circ \FS |_a^\iota \; \FSRedTo \\
  H, \sFrame{L[y \mapsto L'(x)]}{t} \circ \FS |_a^\iota }

  \RuleSpace{}

  \scrule{E-FProp}
  {H, F \; \FRedTo \; H', F'}
  {H, F \circ \FS |_a^\iota \; \FSRedTo \; H', F' \circ \FS |_a^\iota }

  \RuleSpace{}
  
  \scrule{E-ErrorFS}
  {H, F \; \FRedTo \; \Error }
  {H, F \circ \FS |_a^\iota \; \FSRedTo \; \Error}

  \caption{\RACL{} frame stack reduction rules.}
  \label{fig:fs_red_rules}
\end{figure}

Finally, in Figure~\ref{fig:threads_red_rules}, the rules for thread set
reductions are defined. Here we find \EFSProp{} and \EErrorP{} which were
mentioned above. Furthermore we have rule \ESpawn{}, which spawns a new callback
thread for the threshold value $l$ in the dependency set of the  cell specified
by $o$. Note that this uses the thread identifier with which the dependency is
tagged. \ETerm{} is a rule to remove any thread stack finished with its
execution.

\begin{figure}
  \scrule{E-Spawn}
  {
    o \in \dom(H) \andalso H(o) = \Cell{\DEP, l} \\ 
    l' \sqsubseteq l \andalso (l', cb)^\iota \in \DEP \andalso cb = (L_{\text{env}}, z
    \Rightarrow t) \\
    L = L_{\text{env}}[z \mapsto l'] \\
    H' = H[o \mapsto \Cell{\DEP - (l', cb)^\iota, l}]
  }
  {
    H,P \Rrightarrow H', P \cup \left\{ \Frame{L}{t}{-} \circ \varepsilon
    |_{\ocap}^{\iota} \right\}
  }

  \RuleSpace{}

  \scrule{E-Term}
  {P = P' \cup_D \left\{ \sFrame{L}{x} \circ \varepsilon |_a^\iota \right\} }
  {H,P \Rrightarrow H, P'}

  \RuleSpace{}

  \scrule{E-FSProp}
  {H, \FS \twoheadrightarrow H', \FS'}
  {H, P \cup_D \left\{ \FS \right\} \Rrightarrow H', P \cup \left\{ \FS' \right\} }

  \RuleSpace{}

  \scrule{E-ErrorP}
  {H, \FS \twoheadrightarrow \Error}
  {H, P \cup_D \left\{ \FS \right\} \Rrightarrow \Error }
  \caption{\RACL{} thread set reduction rules.}
  \label{fig:threads_red_rules}
\end{figure}


\section{Type System}
\label{sec:type_system}

The type system will now be introduced starting with the types themselves.  The
types and type lattice of RACL is summarized in Figure~\ref{fig:racl_typelat}.
The lattice structure reflects the subtyping relation defined in
Figure~\ref{fig:def_stof}.  Except for standard types like the top, bottom, and
class types, we see that the \CellType{} type is a subtype of \AnyRefType{} like
the class types. Intuitively this is due to that cell objects are stored
on the heap. We also have a separate lattice value type \LatType{}. 

\begin{figure}[]
  \centering
  \begin{tikzpicture} 
    \node (top) at (1,3) {$\RaclTop$};
    \node (anyref) at (2,2) {\AnyRefType};
    \node (cell) at (1,1) {\CellType};
    \node (classes) at (3,1) {$C,D$};
    \node (null) at (2,0) {\NullType};
    \node (lat) at (-1, 1) {\LatType};
    \node (bot) at (1, -1) {$\RaclBot$};
    \draw (top) -- (lat) -- (bot) -- (null) -- (classes) -- (anyref) -- (top);
    \draw (anyref) -- (cell) -- (null);
  \end{tikzpicture}
  \caption{Type lattice of RACL}
  \label{fig:racl_typelat}
\end{figure}

\begin{figure}
  \begin{multicols}{2}

    \scax{ST-Top}
    {\tau \stof \RaclTop}

    \RuleSpace

    \scax{ST-Bot}
    {\RaclBot \stof \tau}

    \RuleSpace

    \scax{ST-Cell-AnyRef}
    {\CellType \stof \AnyRefType}

    \RuleSpace

    \scax{ST-C-AnyRef}
    {C \stof \AnyRefType}

    \RuleSpace

    \scax{ST-Null-Cell}
    {\NullType \stof \CellType}
    
    \RuleSpace

    \scax{ST-Null-C}
    {\NullType \stof C}
  \end{multicols}

  \RuleSpace

  \scrule{ST-C-D}
  {p \vdash \ClassDef{C}{D}{...}{...}}
  {C \stof D}

  \caption{Subtyping relation of RACL}
  \label{fig:def_stof}
\end{figure}

\begin{remark}
  Note that the subtyping judgement $\sigma \stof \tau$ is implicitly dependent
  on $p$ due to the inclusion of $p$ in the precondition of rule {\sc ST-C-D}.
  In many rules for both execution and typing this holds true aswell. E.g. the
  use of $\mbody$, a shorthand for the method body, in rule \ECall{} also
  implicitly levarages the program $p$. We will keep this dependency implicit in
  order to simplify presentation.
\end{remark}

\subsection{Terms \& Expressions}%
\label{sub:terms_and_expressions}

The basic building blocks of our type system are the typing of expressions and
terms. Our typing relation is written
\begin{equation}
  \TypeRel{\Gamma}{a}{t}{\tau} \quad \text{ or } \quad
  \TypeRel{\Gamma}{a}{e}{\tau}. \notag
\end{equation}
Apart from the usual components, i.e., typing environment $\Gamma$, term $t$ or
expression $e$ and type $\tau$, we note that it includes a designator $a$ which
can take on the values \nocap{} and \ocap{}. The latter indicates that the term
or expression is typed under OCAP constraints. This is equivalent to the OCAP
typing of \LaCasa{}~\parencite{conf/oopsla/HallerL16}. 

All typing rules for terms and expressions can be found in
Figure~\ref{fig:expr_typing}. Most of the type system rules are standard. For
example, rule {\sc T-Let} types a let-term if the subexpression $e$ is typeable
as $\tau$ under $\Gamma; a$, and the subterm $t$ is typeable under the extended
environment $\Gamma, x: \tau$ and $a$. Rule {\sc T-Var} types a variable under
$\Gamma$ provided $x \in \dom(\Gamma)$. {\sc T-New} types $\New{C}$ under effect
$a = \ocap$ only if the class $C$ is typeable as $\ocap$. The rules for typing a
class as $\ocap$ is defined in Figure~\ref{fig:ocap_typing}. Typing rule {\sc
T-Call} states that a method call $\Call{x}{m}{y}$, is only typeable if the type
of $y$ is a subtype of the method parameter type $\sigma$. {\sc T-Put} types the
expression $\Put{x}{y}$ if $x$ is typeable as \CellType{} and $y$ is of the lattice
type \LatType.

As a final example, the rule {\sc T-When} describes typing of the dependency
creation expression. It says that in order to register a callback for lattice
value $y$ in cell $x$, first $x$ and $y$ must be typeable as $\CellType$ and
$\LatType$ respectively. Furthermore all captured variables in $\overline{cap}$
must be typeable as $\CellType$. This is to ensure that all objects shared
between threads are of cell type, which is necessary to prove preservation of
thread isolation in the reference graph later. The restricting of capturable
types is similar to capturing-constraints in
spores~\parencite{conf/ecoop/MillerHO14}.  Finally, the term $t$ from callback
closure $z \Rightarrow t$ must be typeable in an environment containing the
captured variables and $z: \LatType$.


\begin{figure}[]
  \centering
  %\begin{multicols}{2}
    \scrule{T-Let}
    {\TypeRel{\Gamma}{a}{e}{\tau} \andalso \TypeRel{\Gamma,x:
    \tau}{a}{t}{\sigma}}
    {\TypeRel{\Gamma}{a}{ \Let{x}{e}{t} }{\sigma}}

    \vspace{0.5em}

    \scax{T-Null}{\TypeRel{\Gamma}{a}{\NullVal}{\NullType}}

    \vspace{0.5em}

    \scax{T-LVal}{\TypeRel{\Gamma}{a}{l}{\LatType}}
    
    \vspace{0.5em}

    \scrule{T-Var}{x \in \dom(\Gamma)}{\TypeRel{\Gamma}{a}{x}{\Gamma(x)}}

    \vspace{0.5em}

    \scrule{T-Select}
    {\TypeRel{\Gamma}{a}{x}{C} \andalso \ftype(f, C) = \tau }
    {\TypeRel{\Gamma}{a}{\FSel{x}{f}}{\tau}}
    
    \vspace{0.5em}

    \scrule{T-Assign}
    {\TypeRel{\Gamma}{a}{x}{C} \andalso \ftype(f, C) = \tau \\
    \TypeRel{\Gamma}{a}{y}{\tau'} \andalso \tau' \stof \tau }
    {\TypeRel{\Gamma}{a}{\FAss{x}{f}{y}}{\tau}}

    \vspace{0.5em}

    \scrule{T-New}
    {a = \ocap \Longrightarrow \ocap(C)}
    {\TypeRel{\Gamma}{a}{\New{C}}{C}}

    \vspace{0.5em}

    \scax{T-NewCell}{\TypeRel{\Gamma}{a}{\NewCell}{\CellType}}
    
    \vspace{0.5em}

    \scrule{T-Call}
    {\TypeRel{\Gamma}{a}{x}{C} \andalso \mtype(C,m) = \sigma \to \tau \\
    \TypeRel{\Gamma}{a}{y}{\sigma'} \andalso \sigma' \stof \sigma }
    {\TypeRel{\Gamma}{a}{\Call{x}{m}{y}}{\tau}}
    
    \vspace{0.5em}

    \scrule{T-Put}
    {\TypeRel{\Gamma}{a}{x}{\CellType} \andalso \TypeRel{\Gamma}{a}{y}{\LatType}}
    {\TypeRel{\Gamma}{a}{\Put{x}{y}}{\CellType}}
    
    \vspace{0.5em}

    \scrule{T-When}
    {\TypeRel{\Gamma}{a}{x}{\CellType} \andalso \TypeRel{\Gamma}{a}{y}{\LatType} \\
    \forall \Capt{u}{u'} \in \overline{cap}. \; \TypeRel{\Gamma}{a}{u'}{\CellType}\\
    \Gamma_{\text{cells}} = [u \mapsto \CellType \mid (\Capt{u}{u'}) \in \overline{cap}]\\
    \TypeRel{\Gamma_{\text{cells}}, z : \LatType}{\ocap}{t}{\sigma}}
    { \TypeRel{\Gamma}{a}{\When{x}{y}{ \CB{\overline{cap}}{z}{t}}}{\CellType} }

  %\end{multicols}
  \caption{\RACL{} typing rules for expressions and terms.}
  \label{fig:expr_typing}
\end{figure}

\subsection{Well Formed Programs}%
\label{sub:well_formed_programs}

Intuitively, a well formed program is a program which obeys our type system. We
define well-formedness in Figure~\ref{fig:wf_typing}. We also need the
definition of the global typing environment $\Gamma_0$. 

\begin{figure}[]
  \scrule{WF-Prog}
  {p \vdash \overline{cd} \andalso p \vdash \overline{vd} \andalso
  \TypeRel{\Gamma_0}{\nocap}{t}{\tau}}
  {p \vdash \overline{cd} \: \overline{vd} \: t}

  \RuleSpace{}

  \scrule{WF-Global}
  {\sigma = \LatType \lor \sigma = \CellType ~ \lor  \\
  (\sigma = C \land p \vdash \ClassDef{C}{...}{...}{...})}
  {p \vdash \VarDecl{f}{\sigma}}
  
  \RuleSpace{}

  \scrule{WF-Class}
  {C \vdash \overline{md} \\ D = \AnyRefType{} \lor
  p \vdash \ClassDef{D}{...}{...}{...} \\
  \forall (\MethodDef{m}{...}{...}{...}{...}) \in \overline{md} . \: \override(m,
  C, D) \\
  \forall (\VarDecl{f}{\tau}) \in \overline{vd} . \: f \notin \fields(D) }
  {p \vdash \ClassDef{C}{D}{\overline{vd}}{\overline{md}}}

  \RuleSpace{}

  \scrule{WF-Override}
  {\mtype(m, D)\text{ not def. } \lor \mtype(m, C) = \mtype(m, D)}
  {\override(m, C, D)}
  
  \RuleSpace{}

  \scrule{WF-Method}
  { \TypeRel{\Gamma_0, \This:C, x : \sigma}{\nocap}{t}{\tau'} \\
  \tau' \stof \tau}
  {C \vdash \MethodDef{m}{x}{\sigma}{\tau}{t}}
  \caption{\RACL{} rules for well formedness of programs.}
  \label{fig:wf_typing}
\end{figure}

\begin{definition}
  Let the \emph{global typing environment} $\Gamma_0$ be defined as 
  \begin{equation*}
    \Gamma_0 = \Global : C_g
  \end{equation*}
\end{definition}

{\sc WF-Prog} says that in order for a program $p =
\overline{cd}~\overline{vd}~t$ to be well formed, all classes $\overline{cd}$,
global variables $\overline{vd}$ and the program term $t$ must be well formed.
Rule {\sc WF-Global} says that in order for a global variable declaration to be
well formed, the denoted type must either be $\LatType$, $\CellType$ or be of
class type $C$ where the class definition of $C$ is well formed. In order for a
class definition of $C$ to be well formed, rule {\sc WF-Class} declares that all
methods must be well formed under $C$, the extended class $D$ must either be
$\AnyRefType$ or must also be well formed. Furthermore, all methods must obey the
rules of overriding and $C$ cannot redeclare any fields that has been declared
in an extended class $D$. {\sc WF-Override} declares that for a method to be
correctly overriden it must either not be declared in any extended class or it
must have the same declared type as in the extended class. Finally {\sc
WF-Method} states that in order for a method to be well formed, its term must be
typable as a type $\tau'$ under an environment consisting of the global
environment, the class itself $\This : C$ and the method parameter $x: \sigma$.
Futhermore $\tau' \stof \tau$, where $\tau$ is the declared return type of the
method.


\subsection{OCAP Typing}%
\label{sub:ocap_typing}

In Figure~\ref{fig:ocap_typing} we find the rules for classifying a type as $\ocap$.
Immediately we see that unconditionally, both $\AnyRefType$ and $\CellType$ are
$\ocap$. To type a class $C$ as $\ocap$, {\sc OCAP-Class} says that apart from being
well formed, the superclass $D$ must be $\ocap$. Furthermore, all methods must
be typable under the special judgement $\vdash_{\ocap}$ and all fields must be
$\ocap$.  For a method to be typed under $\vdash_{\ocap}$, rule {\sc
OCAP-Method} declares that the method term must be typable under effect $\ocap$
without access to the global environment $\Gamma_0$. Being typable under effect
$\ocap$ means that the expression $\New{C}$ is only allowed if $C$ is $\ocap$, as
stated in rule {\sc T-New}. In short, an ocap term can only instansiate ocap
classes. This part of the type system is similar to the corresponding part of
LaCasa~\parencite{conf/oopsla/HallerL16}.

\begin{figure}
  \scax{OCAP-AnyRef}
  {\ocap{(\AnyRefType{})}}

  \RuleSpace{}

  \scax{OCAP-Cell}
  {\ocap(\CellType)}

  \RuleSpace{}

  \scrule{OCAP-Class}
  {\ocap{(D)} \andalso C \vdash_{\ocap} \overline{md} \\
  \forall (\VarDecl{f}{\sigma}) \in \overline{vd}. \: ocap(\sigma)}
  {\ocap{(C)}}

  \RuleSpace{}

  \scrule{OCAP-Method}
  {\TypeRel{\This : C, x : \sigma}{\ocap}{t}{\tau'} \\
  \tau' \stof \tau}
  {C \vdash_{\ocap} \MethodDef{m}{x}{\sigma}{\tau}{t}}
  \caption{\RACL{} OCAP rules.}
  \label{fig:ocap_typing}
\end{figure}


\section{State Properties}
\label{sec:properties}

In order to state and prove things such as progress and preservation we need a few more
definitions. Many of them are just auxilliary properties of things like states
and types and build up to the final definition of a well typed state.

\subsection{Well Typed Heap}%
\label{sub:well_typed_heap}

This first definition is a straightforward partial function defining the dynamic
type of a value.
\begin{definition}
  The partial function $\typeOf$ is defined as follows:
  \begin{equation}
    \typeOf{(k, H)} =
    \begin{cases}
      \LatType, &\text{ if }k \in \LatVals \\
      \NullType, &\text{ if }k = \NullVal \\
      \CellType, &\text{ if }k \in \dom{(H)}\text{ and } H(k) = \Cell{...} \\
      C, &\text{ if }k \in \dom{(H)}\text{ and } H(k) = \Obj{C, ...} \\
    \end{cases}
  \end{equation}
\end{definition}
In order to simplify the definition of a well typed heap we define the
following.
\begin{definition}
  The typing environment $\Gamma_{\CellType}(L)$ is defined as follows:
  \begin{equation}
    \Gamma_{\CellType}(L) = \left[(x: \CellType) \mid x \in \dom(L)\right]
  \end{equation}
\end{definition}
Next comes the definition of a well typed heap. To say that a heap is
well typed intuitively means that for all class objects, all field values are of
a subtype of the declared type. Furthermore, our definition includes a statement
about cell objects. We say that for all callbacks stored in dependency sets, all
captured values must be of cell type and the callback term must be typeable
under an environment containing the captured variables and closure parameter
$z$.
\begin{definition}[Well Typed Heap]
  A heap $H$ is well typed, written $\vdash{H}$, if
  $\forall o \in \dom{(H)}$:

  If $H(o) = \Obj{C, \FM}$ then
  \begin{equation} \label{eq:defwth1}
    \begin{aligned}
      \forall f &\in \fields{(C)}. \\ 
      &f \in \dom{(\FM)} \: \land \\ 
      &\typeOf{(\FM(f), H)} \stof \ftype{(f, C)}
    \end{aligned}
  \end{equation}
  and if $H(o) = \Cell{\DEP, l}$ then
  \begin{equation} \label{eq:defwth2}
    \begin{aligned}
      \forall (l'&, (L_{\text{env}}, z \Rightarrow t))^\iota \in \DEP. \\
      &\forall (x \mapsto k) \in L_{\text{env}}.\: \typeOf{(k, H)} \stof
      \CellType \: \land \\
      &\TypeRel{\Gamma_{\CellType{}}(L_{\text{env}}), z:
      \LatType}{\ocap}{t}{\gamma} 
    \end{aligned}
  \end{equation}
\end{definition}

\subsection{Well Typed Threads}%
\label{sub:well_typed_threads}

A non erroneous state $S = H, P$ has two parts. In order to define and
prove preservation properties, we furthermore need to put restrictions on the
thread set $P$. We do this with the relation $H \vdash P$, saying $P$ is well
typed under heap $H$. This relation is defined in Figure~\ref{fig:ts_typing}.
{\sc T-Procs} and {\sc T-Empty} basically says that in order for a thread set
$P$ to be well typed under heap $H$, all threads need to be well typed under
$H$ and its OCAP status $a$.  For a thread $\GS|_a^\iota$ we write this as $H; a
\vdash \GS$. This relation is defined in Figure~\ref{fig:fs_typing}. 

\begin{figure}
  \scax{T-Empty}
  {H \vdash \emptyset}

  \RuleSpace{}

  \scrule{T-Procs}
  { H; a \vdash \FS \andalso H \vdash P }
  { H \vdash P \cup \left\{\FS |_a^\iota \right\} }

  \caption{Rules for typing thread sets under some heap $H$.}
  \label{fig:ts_typing}
\end{figure}

\begin{figure}
  \scax{T-FSEmpty1}
  {H; a \vdash \varepsilon}

  \RuleSpace{}

  \scax{T-FSEmpty2}
  {H; a \vdash^{x : \sigma} \varepsilon}

  \RuleSpace{}

  \scrule{T-FS1}
  { F = \Frame{L}{t}{x} \andalso H \vdash \Gamma; L \\
  \TypeRel{\Gamma}{a}{t}{\sigma'} \andalso \sigma' \stof \sigma \andalso H; a \vdash^{x: \sigma} \FS }
  { H; a \vdash F \circ \FS }
  
  \RuleSpace{}

  \scrule{T-FS2}
  { F = \Frame{L}{t}{y} \andalso H \vdash \Gamma; L \\
  \TypeRel{\Gamma, x: \tau}{a}{t}{\sigma'} \andalso \sigma' \stof \sigma 
  \andalso H; a \vdash^{y: \sigma} \FS }
  { H; a \vdash^{x: \tau} F \circ \FS }

  \caption{Rules for typing frame stacks under some heap $H$ and effect $a$.}
  \label{fig:fs_typing}
\end{figure}

\begin{figure}
  \scrule{WF-EnvVar}
  {\typeOf{(L(x), H) \stof \Gamma{(x)}}}
  { H \vdash \Gamma; L; x }

  \RuleSpace{}

  \scrule{WF-Env}
  {\dom{(\Gamma)} \subseteq \dom{(L)} \\
    \forall x \in \dom{(\Gamma)}. \: H \vdash \Gamma; L; x 
  }
  { H \vdash \Gamma; L }

  \caption{Rules for classifying local environments $L$ as well typed.}
  \label{fig:local_typing}
\end{figure}

For $H; a \vdash \GS$ to hold, either $\FS = \varepsilon$ as in rule {\sc
T-FSEmpty1} (this is actually impossible since frame stacks of a state cannot be
empty, see definition~\ref{def:thread_sets}), or the term $t$ of top frame $F$
is typeable with some environment $\Gamma$ which is conformant with its local
variable map $L$ and $H$. This conformancy is expressed through the relation $H
\vdash \Gamma; L$ which is defined in Figure~\ref{fig:local_typing}. Simply
stated, it says that all types specified in $\Gamma$ aligns with the dynamic
types of the values in $L$. Furthermore {\sc T-FS1} declares that the rest of the frame
stack $\FS$ must be typable under the judgement $\vdash^{x: \sigma}$. This
judgement is defined by rules {\sc T-FSEmpty2} and {\sc T-FS2}. The latter
reflects that the top frame returns some value to its underlying frame stack. It
is closely connected with the well-formedness of methods and call\slash return
execution semantics, as defined by rules {\sc WF-Method} and \ECall{}\slash\ERet{}
respectively.

\subsection{Isolation}%
\label{sub:isolation}

The next two definitions are used when defining isolation of threads in the
reference graph of the heap. The ruling $\reach(H, o, o')$ is defined in
Figure~\ref{fig:def_reach}.

\begin{figure}
  \scrule{Reach1}
  {o \in \dom{(H)}}
  {\reach{(H, o, o)}}

  \RuleSpace{}

  \scrule{Reach2}
  {
    o \in \dom{(H)} \andalso H(o) = \Obj{C, \FM} \\
    o'' \in \image{(\FM)} \andalso \reach{(H, o'', o')}
  }
  { \reach{(H, o, o')} }
  \caption{Definition of heap reachability.}
  \label{fig:def_reach}
\end{figure}

\begin{definition}[Class Object Separation]
  For any heap $H$ and heap identifiers $o, o'$ we have class object separation,
  $\csep{(H, o, o')}$ iff
  \begin{align}
    \label{eq:csep_def}
    \forall q, q' &\in \dom{(H)}. \notag\\
    & \reach{(H, o, q)} \: \land \: \reach{(H, o', q')} \implies \notag\\ 
    &q \neq q' \: \lor \typeOf{(q, H)} = \CellType
  \end{align}
\end{definition}

\begin{definition}[Accessible Roots]
  For a heap $H$ and a frame stack $\FS$ we define $\accRoots{(\FS, H)}$ as
  \begin{equation}
    \accRoots{(\FS, H)} = \left\{ o \in \dom{(H)} \:\: \middle| \:\:
    \begin{aligned}
      \exists \sframe{L, t} &\in \FS, x \in \VarNames  \\
         \text{s.t. } & (x \mapsto o) \in L
    \end{aligned}
    \right\}
  \end{equation}
\end{definition}

Isolation of threads is defined in Figure~\ref{fig:def_isolation}. Rule {\sc
ISO-FS} states that for two threads to be isolated with regards to heap $H$, all
accessible roots of the two threads must have class object separation, i.e. that
all objects that are reachable from both threads must be of type $\CellType$.
Rule {\sc ISO-Procs} states that we have isolation for a thread set $P$ under
heap $H$, if there is isolation between all pairs of threads such that at least
one is $\ocap$.

\begin{figure}
  \scrule{ISO-FS}
  { 
    \forall o \in \accRoots(\FS,H), o' \in \accRoots(\GS,H) . \: \csep{(H, o, o')}
  }
  {
    \isolated{(H, \FS, \GS)}
  }
  
  \RuleSpace{}

  \scrule{ISO-Procs}
  {
    \forall \FS|_a^\iota, \GS_b^{\iota'} \in P \text{ where } \FS|_a^\iota \neq
    \GS|_b^{\iota'} . \\
    a = \ocap \: \lor \: b = \ocap \implies \isolated{(H, \FS, \GS)}
  }
  {
    \isolation{(H, P)}
  }
  \caption{Definition of isolation}
  \label{fig:def_isolation}
\end{figure}

\subsection{OCAP Reachability}%
\label{sub:ocap_reachability}

In order to prove preservation of isolation we also need to ensure that all
objects reachable from an $\ocap$-annotated thread can be typed as an $\ocap$
type. The next definition together with Figure~\ref{fig:def_ocapreach} defines
this property formally.
\begin{definition}[OCAP Reachability]
  For a heap $H$ and frame stack $\FS$ we have $\ocrloc{(\FS,H)}$ iff
  \begin{align}
    \label{eq:ocr_def}
    \forall o \in \: &\accRoots(\FS,H), o' \in \dom(H). \notag\\
    &\reach{(H, o, o')} \implies \ocap{(\typeOf{(o', H)})}
  \end{align}
\end{definition}

\begin{figure}
  \scrule{OCR-FS}
  { a = \ocap \implies \ocrloc(\FS, H) }
  { H; a \vdash \FS \tsep \ocr }

  \RuleSpace{}

  \scrule{OCR-P}
  {\forall \FS|_a^\iota \in P. \: H; a \vdash \FS \tsep \ocr}
  {H \vdash P \tsep \ocr}

  \caption{Definition of OCAP reachability}
  \label{fig:def_ocapreach}
\end{figure}

\subsection{Global Object Separation}%
\label{sub:global_object_separation}

Another thing needed to prove preservation of isolation is global object
separation. Simply stated this means that no $\ocap$ thread can reach the global
object $o_g$ through heap references. This is formally defined in
Figure~\ref{fig:def_gsep}.

\begin{figure}
  \scrule{GSep-Threads}
  {\forall \FS|_a^\iota \in P. \: a = \ocap \implies \forall o \in \accRoots{(\FS, H)}. \: \csep{(H, o, o_g)} }
  {H \vdash P \tsep \gsep}

  \caption{Definition of global separation}
  \label{fig:def_gsep}
\end{figure}

\subsection{No Thread Spawning}%
\label{sub:no_thread_spawning}

The next definition is needed to state and prove progress. Intuitively
$\noSpawn(H)$ means there are no threads waiting to spawn due to rule
\ESpawn{}.
\begin{definition}[No Spawn]
  For any heap $H$ we have $\noSpawn(H)$ if and only if
  \begin{equation}
    \begin{aligned}
      \forall o &\in \dom(H). \\
        & H(o) = \Cell{\DEP, l} \implies 
        \forall (l', cb)^\iota \in \DEP. \: \lnot (l' \sqsubseteq l).
    \end{aligned}
  \end{equation}
\end{definition}

\subsection{Unique Main Thread}%
\label{sub:unique_main_thread}

In order to prove determinism we must be sure that there is at most one
non-$\ocap$ thread running. Otherwise these could interfere since there are no
constraints on whether these can share data. Therefore we define the following
property.
\begin{definition}[Unique Main Thread]
  For a thread $\FS|_a^\iota$ let
  \begin{equation*}
    \chi_{\nocap}(a) =
    \begin{cases}
      1 & \text{ if } a = \nocap \\
      0 & \text{ o.w. }
    \end{cases}
  \end{equation*}
  For a thread set $P$ let
  \begin{equation*}
    \chi(P) = \sum_{\FS|_a^\iota \in P} \chi_{\nocap}(a).
  \end{equation*}
  This is the number of non OCAP protected threads in $P$. Finally we define
  \begin{equation*}
    \uniqMain(P) \iff \chi(P) \leq 1
  \end{equation*}
\end{definition}


\subsection{Well Typed States}%
\label{sub:well_typed_states}

Finally we can define the notion of a well typed state. This combines many of
the properties already defined into one.
\begin{definition}[Well Typed States]
  For a state $S$ we say that it is \emph{well typed} with regards to a well typed
  program $p$, written
  \begin{equation}
    \vdash S \tsep \stateok
  \end{equation}
  if $S = \Error$ or $S = H, P$ and
  \begin{equation*}
    \begin{gathered}
      \vdash H \andalso H \vdash P \andalso H \vdash P \tsep \ocr \\
      \isolation{(H, P)} \andalso H \vdash P \tsep \gsep \andalso \uniqMain(P)
    \end{gathered}
  \end{equation*}
\end{definition}

\begin{remark}
  We will henceforth refer to this only as being well typed. The well typed
  program $p$ will therefore in most cases be implicit.
\end{remark}

%\begin{figure}
%  \scrule{AR-F}
%  { (x \mapsto o) \in L }
%  { \accRoot{(o, \sFrame{L}{t})} }
%
%  \RuleSpace{}
%
%  \scrule{AR-FS}
%  { \accRoot{(o, F)} \: \lor \: \accRoot{(o, FS)} }
%  { \accRoot{(o, F \circ FS)} }
%  \caption{Definition of accRoot}
%  \label{fig:def_accroot}
%\end{figure}

% TODO include the \Gamma_{\CellType} definition somewhere


% Introduce the core language with syntax, type system and state properties like
% WT heap, isolation, well typed state




\chapter{Preservation and Progress}
\label{cha:preservation_and_progress}

% Write about the basic ideas behind the proof of preservation and progress

\begin{theorem}{(Preservation)}
  \label{thm:preservation}
  Let $S, S'$ be states such that $\vdash S \tsep \stateok$ and $S \Rrightarrow
  S'$. Then $\vdash S' \tsep \stateok$.
\end{theorem}

\begin{theorem}{(Progress)}
  \label{thm:progress}
  Let $S$ be a state such that $\vdash S \tsep \stateok$. Then either 
  \begin{enumerate}
    \item $\exists S'$ s.t. $S \Rrightarrow S'$, 
    \item $S = H, \emptyset$ for some heap $H$ s.t. $\noSpawn{(H)}$ or
    \item $S = \Error$.
  \end{enumerate}
\end{theorem}



\chapter{Quasi Determinism}
\label{cha:determinism}

% Describe:
% Theorem of determinism
% Lemmas leading up to the proof of this

\begin{definition}
  For each state $S = H, P$ there are two related sets of currently used object
  identifiers $\mathcal{O}(S)$ and thread identifiers $\mathcal{D}(S)$. These contain
  all object and thread identifiers occuring in $H$ and $P$. We call an object
  identifier $o$ (reference identifier $d$) \emph{fresh} if $o \not\in \mathcal{O}(S)$ ($d \not\in
  \mathcal{D}(S)$).
\end{definition}

\begin{definition} \label{def:eqrel}
  Let $\simeq$ be a binary relation on the set of states $\States$.
  We let $S \simeq S'$ if
  \begin{equation*}
    S = S' = \Error
  \end{equation*}
  or if
  \begin{equation}
    S = H, P \andalso S' = H', P'
  \end{equation}
  and there exists bijections $g: \OIDs(S) \to \OIDs(S'), h: \TIDs(S) \to \TIDs(S')$
  such that
  \begin{equation}
    H' = \pi(H, g) \andalso P' = \rho(P, g, h)
  \end{equation}
  
  The functions $\pi$ and $\rho$ are defined in definition~\ref{def:pirho}.
\end{definition}

% TODO define \pi and \rho


\begin{proposition} \label{prop:eqrel}
  $\simeq$ is an equivalence relation.
\end{proposition}

\begin{proposition} \label{prop:eqrel_stateok}
  For any $S, S' \in \States$ such that $S \simeq S'$
  \begin{equation}
    \vdash S \tsep \stateok \iff \vdash S' \tsep \stateok
  \end{equation}
\end{proposition}


\begin{theorem}
  % state final QD theorem here
\end{theorem}



\printbibliography[heading=bibintoc] % Print the bibliography (and make it appear in the table of contents)

\appendix

\chapter{Proof of Preservation and Progress}
\label{cha:proof_of_pnp}

\section{Preliminaries}%
\label{sec:preliminaries}

\begin{definition}[Heap Induced Graph]
  The induced graph of heap $H$, written $\Graph{(H)}$ is the directed graph
  $(V, E)$ such that $V = \dom{(H)}$ and 
  \begin{equation}
    E = \left\{ (o, o')_f \in \dom{(H)}^2 \times \FieldNames \mid
      H(o) = \Obj{C, \FM} \text{ and } \FM(f) = o' \right\}
  \end{equation}
\end{definition}

\begin{definition}[Graph Reachability]
  We say an object $o'$ is reachable from $o$ in graph $G = (V,
  E)$, written $\reach{(G, o, o')}$ if there is a finite sequence $o_0, \dots,
  o_n \in V$ such that $o = o_0, o_n = o'$ and
  \begin{equation}
    \forall i = 1,\dots, n-1. \: \exists f_i \in \FieldNames \text{ s.t. } (o_i,
    o_{i+1})_{f_i} \in E.
  \end{equation}
  The sequence $o_0, \dots, o_n$ is called a \emph{path}.
\end{definition}

\begin{definition}
  Given a graph $G = (V, E)$ and a set $O \subseteq V$, we define
  \begin{equation}
    \reachable{(O, G)} = \left\{ q \in V: \exists o\in O.\: \reach{(G, o, q)}
    \right\} \notag
  \end{equation}
\end{definition}

\begin{proposition}[Reachability equivalence]
  \label{prop:reacheq}
  For a heap $H$ 
  \begin{equation}
    \reach{(H, o, o')} \iff \reach{(\Graph{(H)}, o, o')}
  \end{equation}
\end{proposition}

\begin{proof}
  Both directions of implication are simple to prove.
  \begin{description}
    \item[Case $\implies$:] By induction on the shape of derivation tree.
    \item[Case $\impliedby$:] By induction on the path length in graph
      $\Graph{(H)}$.
  \end{description}
\end{proof}

\begin{proposition} For any heap $H$ and frame stack $\FS$ we have
  \begin{flalign*}
    &\ocrloc{(\FS, H)} &\\
    &\iff &\\
    &\begin{aligned}
    \forall q &\in \reachable{{(\accRoots{(\FS,H)}, \Graph{(H)})}}. \notag\\
    & \ocap{(\typeOf{(q, H)})}
    \end{aligned}&
  \end{flalign*}
\end{proposition}

\begin{proof}
  Follows from definition of $\ocrloc$, $\accRoots$ and $\reachable$.
\end{proof}

\begin{proposition} \label{prop:csep_eq}
  For any heap $H$ and object references $o, o'$ we have
  \begin{flalign*}
    &\csep{(H, o, o')} \iff &\\
    & \begin{aligned}
        \forall q \in \: &\reachable{(\{o\}, \Graph{(H)})} \cap \reachable{(\{o'\},
        \Graph{(H)})}. \\
        & \typeOf{(q, H)} \stof \CellType
    \end{aligned}&
  \end{flalign*}
\end{proposition}

\begin{proof}
  First of all we let $R = \reachable{(\{o\}, \Graph{(H)})}, R' =
  \reachable{(\{o'\}, \Graph{(H)})}$.
  We prove each direction of implication separately.
  \begin{description}
    \item[Case $\implies$:] Take any $q \in  R\cap R'$. By definition of
      $\reachable$, reachability equivalence (prop.~\ref{prop:reacheq}) and 
      definition of $\csep{(H, o, o')}$ we have $\typeOf{(q, H)} \stof \CellType$.
    \item[Case $\impliedby$:] Take any $q, q'\in \dom{(H)}$ such that
      $\reach{(H, o, q)}$ and $\reach{(H, o', q')}$. If $q \neq q$ we are done.
      Otherwise $q = q'$ and by definition of $\reachable$, $q \in R \cap R'$.
      By assumption $\typeOf{(q, H)} \stof \CellType$.
  \end{description}
\end{proof}

\begin{proposition} \label{prop:2.6}
  For any heap $H$ and frame stacks $\FS, \HS$ we have
  \begin{flalign*}
    &\isolated{(H, \FS, \HS)}  \iff &\\
    &\begin{aligned}
        \forall q \in \:&\reachable{(\accRoots{(\FS, H)}, \Graph{(H)})} \cap \\
        & \reachable{(\accRoots{(\HS, H)}, \Graph{(H)})}. \\
        & \typeOf{(q, H)} \stof \CellType
    \end{aligned}&
  \end{flalign*}
\end{proposition}

\begin{proof}
  We let 
  \begin{equation*}
    \begin{gathered}
      R = \reachable(\accRoots(\FS, H), \Graph(H))  \\
      R' = \reachable(\accRoots(\HS, H), \Graph(H))  
    \end{gathered}
  \end{equation*}
  We prove each direction of implication separately.
  \begin{description}
    \item[Case $\implies$:] Take $q \in R \cap R'$. Then 
      \begin{equation*}
        \exists o \in \accRoots(\FS, H), o' \in \accRoots(\HS, H)
      \end{equation*}
      such that
      \begin{equation*}
        q \in \reachable(\{o\}, \Graph(H)) \cap \reachable(\{o'\}, \Graph(H)).
      \end{equation*}
      Thus by $\isolated(H, \FS, \HS)$ we have $\csep(H, o, o')$. 
      Proposition~\ref{prop:csep_eq} yields $\typeOf(q, H) \stof \CellType$.
    \item[Case $\impliedby$:] Take any $o, o'$ such that
      \begin{equation*}
        o \in \accRoots(\FS, H) \andalso o' \in \accRoots(\HS, H).
      \end{equation*}
      By definition
      \begin{gather*}
        \reachable(\{o\}, \Graph(H)) \subseteq R \\
        \reachable(\{o'\}, \Graph(H)) \subseteq R'.
      \end{gather*}
      Thus by assumption 
      \begin{align*}
        \forall q \in &\reachable(\{o\}, \Graph(H)) \cap \reachable(\{o'\},
        \Graph(H)). \\ 
        &\typeOf(q, H) \stof \CellType.
      \end{align*}
      Finally by Proposition~\ref{prop:csep_eq}, we have $\csep(H, o, o')$. Since
      $o, o'$ arbitrary we are done.
  \end{description}
\end{proof}

\begin{proposition} \label{prop:2.11}
  Let $H, H'$ be heaps and $P, P'$ thread sets such that
  \begin{enumerate}
    \item $P = Q \cup_D \left\{ \FS|_a^\iota \right\}$, $\isolation{(H, P)}$, $H
      \vdash P \tsep \ocr$ and $P' = Q \cup_D \left\{ \FS'|_a^\iota \right\}$
    \item $\Graph{(H)} = \Graph{(H')}$
    \item $\forall \HS|_b^{\iota'} \in Q.$ \\ 
      $\reachable{(\accRoots{(\HS, H')}, \Graph{(H')})} \subseteq$ \\
      $\reachable{(\accRoots{(\HS, H)}, \Graph{(H)})}$
    \item $\reachable{(\accRoots{(\FS', H')}, \Graph{(H')})} \subseteq$ \\
      $\reachable{(\accRoots{(\FS, H)}, \Graph{(H)})}$
    \item $\forall o \in \dom{(H)}. \: \typeOf{(o, H)} = \typeOf{(o, H')}$
  \end{enumerate}
  Then $\isolation{(H', P')}$ and $H' \vdash P' \tsep \ocr$.
\end{proposition}

\begin{remark}
  Note that $\Graph{(H)} = \Graph{(H')}$ implies $\dom{(H)} = \dom{(H')}$.
  Otherwise many of the preconditions above would not make sense.
\end{remark}

\begin{proof}
  We prove that 
  \begin{equation*}
    \forall \text{ distinct } \HS|_b^{\iota'}, \GS|_c^{\iota''} \in P'.\; b = \ocap \lor c = \ocap \implies
    \isolated(H', \HS, \GS).
  \end{equation*}
  This implies $\isolation(H', P')$ by rule {\sc ISO-Procs}.
  Thus take any distinct $\HS|_b^{\iota'}, \GS|_c^{\iota''} \in P'$. Then by assumption 3, 4 and basic set
  properties
  \begin{equation} \label{eq:reachinclusion1}
    \begin{gathered}
      \reachable(\accRoots(\HS, H'), \Graph(H')) \cap \reachable(\accRoots(\GS, H'),
      \Graph(H')) \\
      \subseteq \\
      \reachable(\accRoots(\HS, H), \Graph(H)) \cap \reachable(\accRoots(\GS, H),
      \Graph(H)).
    \end{gathered}
  \end{equation}
  By this, Proposition~\ref{prop:2.6}, $\isolation(H, \HS, \GS)$ and assumption 5, we have
  $\isolation(H', \HS, \GS)$. By \eqref{eq:reachinclusion1} and assumption 3,4 and 5
  we have $H' \vdash P' \tsep \ocr$.
\end{proof}

\begin{corollary} \label{cor:2.11}
  Let $H, H'$ be heaps and $P, P'$ thread sets such that
  \begin{enumerate}
    \item $P = Q \cup_D \left\{ \FS|_a^\iota \right\}$, $\isolation{(H, P)}$, $H
      \vdash P \tsep \ocr$ and $P' = Q \cup_D \left\{ \FS'|_a^\iota \right\}$
    \item $\Graph{(H)} = \Graph{(H')}$
    \item $\forall \HS|_b^{\iota'} \in Q. \: \accRoots{(\HS, H')} \subseteq \accRoots{(\HS, H)}$
    \item $\accRoots{(\FS', H')} \subseteq \accRoots{(\FS, H)}$
    \item $\forall o \in \dom{(H)}. \: \typeOf{(o, H)} = \typeOf{(o, H')}$
  \end{enumerate}
  Then $\isolation{(H', P')}$ and $H' \vdash P' \tsep \ocr$.
\end{corollary}

\begin{proof}
  Assumption 2,3 and 4 implies assumption 2,3 and 4 of
  Proposition~\ref{prop:2.11}
\end{proof}

\begin{definition} \label{def:ptilde}
  Let $\FS_g$ be any frame stack where 
  \begin{equation*}
    \accRoots(\FS_g, H) = \left\{ o_g \right\}
  \end{equation*}
  Furthermore let $H; \nocap \vdash \FS_g$.
  $\FS_g$ thus is any well typed frame stack that only has the global object as
  an accessible root.  Now, for any thread set $P$, let $\tilde{P}$ be the
  thread set
  \begin{equation*}
    \tilde{P} = P \cup \left\{ \FS_g|_{\nocap}^\iota \right\}
  \end{equation*}
  for some fresh $\iota$.
\end{definition}

\begin{proposition} \label{prop:ocrtilde_eq}
  For any $H, P$ we have
  \begin{equation*}
    H \vdash P \tsep \ocr \iff H \vdash \tilde{P} \tsep \ocr
  \end{equation*}
  and
  \begin{equation*}
    H \vdash P  \iff H \vdash \tilde{P}. 
  \end{equation*}
\end{proposition}
\begin{proof}
  Follows almost immediately from Proposition~\ref{prop:2.13} and the
  definitions of OCAP reachability and $\tilde{P}$.
\end{proof}

\begin{proposition} \label{prop:2.8}
  For any heap $H$ and thread set $P$
  \begin{equation*}
    \isolation{(H, \tilde{P})} \iff \isolation{(H, P)} \text{ and } H \vdash P
    \tsep \gsep 
  \end{equation*}
\end{proposition}

\begin{proof}
  This follows almost immediately by the definitions of isolation and global
  object separation, and the fact that $\accRoots(\FS_g) = \left\{
    o_g \right\}$ 
\end{proof}

We state the following without proof.
\begin{proposition} \label{prop:tilde_trans}
  Let $H, P \Rrightarrow^{R^{\iota, \beta}} H', P'$. Then $H, \tilde{P}
  \Rrightarrow^{R^{\iota, \beta}} H', \tilde{P'}$
\end{proposition}

\begin{proposition} \label{prop:2.9}
  For any $H, H', P, P'$ such that $\vdash H, H \vdash P$, $H \vdash P \tsep
  \ocr$ and $H, P \Rrightarrow^{R^{\iota, \beta}} H', P'$, if
  \begin{equation*}
      \isolation(H, P) 
      \implies 
      \isolation(H', P')
  \end{equation*}
  then
  \begin{equation*}
    \isolation(H, \tilde{P}) \implies \isolation(H', \tilde{P'})
  \end{equation*}
\end{proposition}

\begin{proof}
  By Proposition~\ref{prop:tilde_trans}, $H, \tilde{P} \Rrightarrow^{R^{\iota,
  \beta}} H', \tilde{P'}$. It is easy to see that $\vdash H$, $H \vdash
  \tilde{P}$ and $H \vdash \tilde{P} \tsep \ocr$.
  Since the assumption of the proposition refers to any $H, H', P, P'$ for which
  these three properties hold the statement also holds for $H, H',
  \tilde{P}, \tilde{P'}$.
\end{proof}

\begin{remark}
  This proposition may seem a bit circular, but what it does is allow us to get
  the global separation property for free in any case of the proof of
  Theorem~\ref{thm:preservation} where we do not rely on the global separation
  property in order to prove preservation of isolation. This ends up to be a
  substantial number of cases. We state this in the following corollary and it
  can be proven by combining Proposition~\ref{prop:2.9} with
  Proposition~\ref{prop:2.8}. 
\end{remark}

\begin{corollary} \label{cor:2.9}
  For any $H, H', P, P'$ such that $\vdash H, H \vdash P$, $H \vdash P \tsep
  \ocr$ and $H, P \Rrightarrow^{R^{\alpha, \beta}} H', P'$, if
  \begin{equation*}
      \isolation(H, P) 
      \implies 
      \isolation(H', P')
  \end{equation*}
  then
  \begin{equation*}
    \begin{gathered}
      \isolation(H, P) \text{ and } H \vdash P \tsep \gsep \\
      \implies \\
      \isolation(H', P') \text{ and } H \vdash P' \tsep \gsep
    \end{gathered}
  \end{equation*}
\end{corollary}
\begin{remark}
  In particular we can use this corollary in all cases where we can also use
  Proposition~\ref{prop:2.11} or its corollary \ref{cor:2.11}.
\end{remark}

% TODO define \Values set
\begin{proposition} \label{prop:2.12}
  Let $H, H'$ be heaps. If $\dom{(H)} = \dom{(H')}$ and
  \begin{equation*}
    \forall o \in \dom{(H)}. \: \typeOf{(o, H)} = \typeOf{(o, H')}
  \end{equation*}
  then 
  \begin{equation*}
    \forall k \in \Values. \: \typeOf{(k, H)} = \typeOf{(k, H')}
  \end{equation*}
\end{proposition}

\begin{proof}
  It is easy to see since the only values in $\Values$ where its type depends on
  $H$ is the object references $o$.
\end{proof}

\begin{proposition} \label{prop:2.19}
  If
  \begin{equation*}
    \forall o \in \dom(H) \subseteq \dom(H'). \: \typeOf(o, H) = \typeOf(o, H')
  \end{equation*}
  and $H \vdash \Gamma; L$, then $H' \vdash \Gamma; L$.
\end{proposition}

\begin{proof}
  It is obvious from the structure of the rules {\sc WF-EnvVar} and {\sc
  WF-Env} and the assumptions of the proposition.
\end{proof}

\begin{proposition} \label{prop:2.13}
  For any $H, P$ we have 
  \begin{equation}
    H \vdash P \iff \forall \HS|_b^{\iota'} \in P.\: H;b \vdash \HS
  \end{equation}
\end{proposition}

\begin{proof} (Sketch) Done in each direction separately.
  \begin{description}
    \item[Case $\implies$:] By induction on the shape of the derivation tree.
    \item[Case $\impliedby$:] By induction on size of $P$.
  \end{description}
\end{proof}

\begin{proposition} \label{prop:2.14}
  Let $H, H'$ be heaps such that $\dom{(H)} \subseteq \dom{(H')}$ and 
  \begin{equation*}
    \forall o \in \dom{(H)}. \: \typeOf{(o, H)} = \typeOf{(o, H')}.
  \end{equation*}
  Then for any frame stack $\FS$
  \begin{equation}\label{eq:fs_impl_typing1}
    H; a \vdash \FS \implies H'; a \vdash \FS
  \end{equation}
  and
  \begin{equation} \label{eq:fs_impl_typing2}
    H; a \vdash^{x :\sigma} \FS \implies H'; a \vdash^{x: \sigma} \FS 
  \end{equation}
\end{proposition}

\begin{proof}
  We first prove \eqref{eq:fs_impl_typing2} by induction on the lenght $n$ of
  $\FS$. To prove the implication we assume that $H; a \vdash^{x: \tau} \FS$.
  
  If $n = 0$ we have $\FS = \varepsilon$. By {\sc T-FSEmpty2} we are done.
  
  If $n = i + 1$ we have $\FS = F \circ \GS$ where length of $\GS$ is $i$. By  
  $H; a \vdash^{x: \sigma} \FS$ and {\sc T-FS2} we have 
  \begin{equation*}
    F = \xframe{L, t}^{y} \andalso H \vdash \Gamma; L \andalso \TypeRel{\Gamma, x:
    \sigma}{a}{t}{\tau} \andalso H; a \vdash^{y: \tau} \GS.
  \end{equation*}
  By induction hypothesis we then have $H'; a \vdash^{y: \tau} \GS$. We get $H'
  \vdash \Gamma; L$ from Proposition \ref{prop:2.19}. Applying rule {\sc T-FS2}
  yields $H'; a \vdash^{x: \sigma} \FS$.

  Now we prove \eqref{eq:fs_impl_typing1}. Similarly assuming $H; a \vdash \FS$
  we have two cases. If $\FS = \varepsilon$ we are done by {\sc T-FSEmpty1}. If
  $\FS = F \circ \GS$ we by {\sc T-FS1} have
  \begin{equation*}
    F = \xframe{L, t}^{x} \andalso H \vdash \Gamma; L \andalso
    \TypeRel{\Gamma}{a}{t}{\tau} \andalso H; a \vdash^{x: \sigma} \GS.
  \end{equation*}
  By \eqref{eq:fs_impl_typing2}, $H'; a \vdash^{x: \sigma} \GS$. 
  Proposition~\ref{prop:2.19} then yields $H' \vdash \Gamma; L$. Applying rule {\sc
  T-FS1} gives us $H'; a \vdash \FS$.
\end{proof}

\begin{proposition} \label{prop:ocrloc_eq}
  For any $H, \HS$ we have
  \begin{equation*}
    \begin{gathered}
      \ocrloc(\HS, H) \\
      \iff  \\
      \begin{aligned}
        \forall q &\in \reachable(\accRoots(\HS, H), \Graph(H)). \\
        & \ocap(\typeOf(q, H))
      \end{aligned}
    \end{gathered}
  \end{equation*}
\end{proposition}
\begin{proof}
  Follows immediately from reachability equivalence and the definition of
  $\ocrloc$.
\end{proof}


\section{Proof of preservation}
\label{sec:proof_of_preservation}

\begin{theorem*}[Preservation]
  Let $S, S'$ be states such that $\vdash S \tsep \stateok$ and $S \Rrightarrow
  S'$. Then $\vdash S' \tsep \stateok$.
\end{theorem*}

\begin{proof} 
  First of all we have that $S \neq \Error$ since no step can be made from this
  state. Thus $S = H, P$ and
  \begin{equation*}
    \begin{gathered}
      \vdash H \andalso H \vdash P \andalso H \vdash P \tsep \ocr \\
      \isolated{(H, P)} \andalso H \vdash P \tsep \gsep \andalso \uniqMain(P).
    \end{gathered}
  \end{equation*}
  We also assume we have $S' = H', P'$ since otherwise $S' = \Error$ and the
  theorem holds trivially. What remains is to prove
  \begin{equation*}
    \begin{gathered}
      \vdash H' \andalso H' \vdash P' \andalso H' \vdash P' \tsep \ocr \\
      \isolated{(H', P')} \andalso H' \vdash P' \tsep \gsep \andalso
      \uniqMain(P).
    \end{gathered}
  \end{equation*}
  We easily see that $\uniqMain(P')$ must hold since no reduction rule changes
  the OCAP status of threads, and the only rule which spawns new threads is {\sc
  E-Spawn} which spawns an OCAP thread.

  To prove everything else we proceed by cases.
  \begin{description}
    \item[Case {\sc E-FSProp}:] The \EFSProp{} rule definition says $P =
      Q \cup_D \left\{ \FS|_a^\iota \right\}$ and $P' = Q \cup \left\{
        \FS'|_a^\iota \right\}$. We proceed by cases.
      \begin{description}
        \item[Case {\sc E-FProp}:] By \EFProp{} rule definition, $\FS = F \circ
          \GS$ and $\FS' = F' \circ \GS$. Furthermore we can assume
          that $F = \sframe{L, t}$ and $F' = \sframe{L', t'}$.
          We proceed by cases.
          \begin{description}
            \item[Case {\sc E-Null}:] By this rule we have $t =
              \Let{x}{\NullVal}{t'}$, $H = H'$ and $L' = L[x \mapsto \NullVal]$.
              Immediatelly we have $\vdash H'$ by $H = H'$.

              By {\sc T-Procs} and Proposition~\ref{prop:2.13} we have 
              \begin{equation} \label{eq:enull1}
               H \vdash Q  \andalso H; a \vdash \FS 
              \end{equation}
              Trivially $H' \vdash Q$. 
              By~\eqref{eq:enull1} and rule {\sc T-FS1} we have
              \begin{equation} \label{eq:enull2}
                \begin{gathered}
                  H \vdash \Gamma; L \andalso \TypeRel{\Gamma}{a}{t}{\sigma'} \\
                  \sigma' \stof \sigma \andalso H; a \vdash^{s: \sigma} \GS.
                \end{gathered}
              \end{equation}
              Let $\Gamma' = \Gamma, x: \NullType$. By {\sc T-Let}, {\sc
              T-Null} and \eqref{eq:enull2} we have
              \begin{equation} \label{eq:enull3}
                \TypeRel{\Gamma'}{a}{t'}{\sigma'}.
              \end{equation}
              Since $\typeOf(L(x), H') \stof \NullType$, by inspection of rules
              {\sc WF-EnvVar} and {\sc WF-Env} we can see that
              \begin{equation} \label{eq:enull4}
                H' \vdash \Gamma';L'
              \end{equation}
              By {\sc T-FS1}, \eqref{eq:enull2}, \eqref{eq:enull3} and \eqref{eq:enull4}
              \begin{equation}\label{eq:enull5}
                H';a \vdash \FS'
              \end{equation}
              By \eqref{eq:enull5}, {\sc T-Procs} and $H = H'$ we get  
              \begin{equation}
                H' \vdash P'
              \end{equation}
              $H' \vdash P' \tsep \ocr$ and $\isolation{(H', P')}$ follows from
              the Corollary~\ref{cor:2.11}. Then $H' \vdash P'
              \tsep \gsep$ follows from Corollary~\ref{cor:2.9} (see remarks
              for Proposition~\ref{prop:2.9} and Corollary~\ref{cor:2.9}).

            \item[Case {\sc E-LVal}:] Similarly.

            \item[Case {\sc E-Var}:] First, by rule {\sc E-Var} we have
              \begin{equation} 
                \begin{gathered}
                  F = \sframe{L, t} \andalso t = \Let{x}{y}{t'} \\ 
                  F' = \sframe{L', t'} \andalso L' = L[x \mapsto L(y)].
                \end{gathered}
              \end{equation}
              Also
              \begin{equation} \label{eq:evar1}
                H = H'
              \end{equation}
              so $\vdash H'$.
              By $H \vdash P$, {\sc T-FS1} and Proposition \ref{prop:2.13}
              \begin{equation} \label{eq:evar3}
                \begin{gathered}
                  H \vdash \Gamma; L \andalso \TypeRel{\Gamma}{a}{t}{\sigma'} \\
                  \sigma' \stof \sigma \andalso H; a \vdash^{s: \sigma} \GS
                \end{gathered}
              \end{equation}
              By {\sc T-Let, T-Var} and $\TypeRel{\Gamma}{a}{t}{\sigma'}$
              \begin{equation} \label{eq:evar4}
                \TypeRel{\Gamma, x: \gamma}{a}{t'}{\sigma'} \andalso \Gamma(y) =
                \gamma
              \end{equation}
              Let $\Gamma' = \Gamma, x: \gamma$.
              Then by \eqref{eq:evar4}
              \begin{equation} \label{eq:evar5}
                \TypeRel{\Gamma'}{a}{t'}{\sigma'}.
              \end{equation}
              By $H \vdash \Gamma; L$, \eqref{eq:evar1}, \eqref{eq:evar5}, and
              rules {\sc WF-EnvVar}, {\sc WF-Env}
              \begin{equation} \label{eq:evar8}
                \typeOf(L'(x), H') = \typeOf(L(y), H) \stof \Gamma(y) =
                \Gamma'(x).
              \end{equation}
              From definitions of $\Gamma', L'$ and $H \vdash \Gamma;L$ we also have 
              \begin{equation} \label{eq:evar9}
                \begin{aligned}
                  \forall z \in &\dom(\Gamma') \text{ s.t. } z \neq x. \\
                  & \typeOf(L'(z), H') \leq \Gamma'(z).
                \end{aligned}
              \end{equation}
              Thus by {\sc WF-EnvVar}, {\sc WF-Env}, \eqref{eq:evar8} and
              \eqref{eq:evar9} 
              \begin{equation} \label{eq:evar6}
                H' \vdash \Gamma'; L'
              \end{equation}
              By {\sc T-FS1}, \eqref{eq:evar3}, \eqref{eq:evar5} and
              \eqref{eq:evar6} we have
              \begin{equation} \label{eq:evar7}
                H'; a \vdash \FS'.
              \end{equation}
              By Proposition \ref{prop:2.13} and $H \vdash P$ we get
              \begin{equation*} 
                \forall \HS|_b^{\iota'} \in Q. \: H; b \vdash \HS.
              \end{equation*}
              By this, \eqref{eq:evar1} and Proposition \ref{prop:2.13}
              \begin{equation*}
                H' \vdash Q
              \end{equation*}
              Combining this with \eqref{eq:evar7} and {\sc T-Procs} finally
              yields
              \begin{equation*}
                H' \vdash P'
              \end{equation*}

              We now move on to OCAP reachability, isolation and global object
              separation. We note that all preconditions of
              Corollary~\ref{cor:2.11} holds. Thus we immediately have 
              \begin{equation*}
                H' \vdash P' \tsep \ocr \andalso \isolation(H', P').
              \end{equation*}
              $H' \vdash P' \tsep \gsep$ immediately follows by
              Corollary~\ref{cor:2.9}.

            \item[Case {\sc E-Select}:] By rule {\sc E-Select}
              \begin{equation} \label{eq:eselect1}
                \begin{gathered}
                  H = H' \andalso F = \sframe{L, t} \andalso t =
                  \Let{x}{\FSel{y}{f}}{t'} \\
                  L(y) = o \andalso H(o) = \Obj{C, \FM} \andalso f \in \dom(\FM)
                  \\
                  F' = \sframe{L', t'} \andalso L' = L[x \mapsto \FM(f)]
                \end{gathered}
              \end{equation}
              We immediately see $\vdash H'$. By $\vdash H$ we have
              \begin{equation} \label{eq:eselect2}
                \typeOf(\FM(f), H) \stof \ftype(f, C).
              \end{equation}
              $H \vdash P$ and propositions \ref{prop:2.13} and \ref{prop:2.14} yields
              $H \vdash Q$ and thus $H' \vdash Q$.
              $H \vdash P$ and Proposition \ref{prop:2.13} gives 
              \begin{equation} \label{eq:eselect3}
                H; a \vdash \FS
              \end{equation}
              which under inspection of rule {\sc T-FS1} immediately gives
              \begin{equation} \label{eq:eselect4}
                \begin{gathered}
                  H \vdash \Gamma; L \andalso \TypeRel{\Gamma}{a}{t}{\sigma'} \\
                  \sigma' \stof \sigma \andalso H; a \vdash^{s: \sigma} \GS.
                \end{gathered}
              \end{equation}
              $\TypeRel{\Gamma}{a}{t}{\sigma'}$ together with rule {\sc T-Let}
              gives
              \begin{equation} \label{eq:eselect5}
                \TypeRel{\Gamma}{a}{\FSel{y}{f}}{\gamma} \andalso
                \TypeRel{\Gamma, x: \gamma}{a}{t'}{\sigma'}.
              \end{equation}
              which combined with {\sc T-Select} and {\sc T-Var} gives us
              \begin{equation} \label{eq:eselect6}
                (y: C) \in \Gamma \andalso \ftype(f, C) = \gamma
              \end{equation}
              Let $\Gamma' = \Gamma, x: \gamma$. Similarly to case {\sc E-Var}
              we see that
              \begin{equation} \label{eq:eselect7}
                \forall z \in L \text{ s.t. } z \neq x . \: \typeOf(L(z), H')
                \stof \Gamma'(z).
              \end{equation}
              Furthermore, because of \eqref{eq:eselect2} and $H = H'$
              \begin{equation} \label{eq:eselect8}
                \begin{aligned}
                  \typeOf(L'(x), H') &= \typeOf(\FM(f), H') \\
                                     &\stof \ftype(f, C) \\
                                     &= \Gamma'(x)
                \end{aligned}
              \end{equation}
              Using {\sc WF-Env, WF-EnvVar}, \eqref{eq:eselect7} and
              \eqref{eq:eselect8} we have
              \begin{equation} \label{eq:eselect9}
                H' \vdash \Gamma';L'.
              \end{equation}
              Using {\sc T-FS1} with $H = H'$, \eqref{eq:eselect4}, \eqref{eq:eselect5}
              and \eqref{eq:eselect9} we finally get
              \begin{equation}
                H';a \vdash \FS'
              \end{equation}
              which with the help of {\sc T-Procs} and $H' \vdash Q$ gives us 
              \begin{equation*}
                H'\vdash P'.
              \end{equation*}
              
              Moving on to OCAP reachability, isolation and global object
              separation we can easily see that the preconditions of Proposition
              \ref{prop:2.11} holds and thus we immediately get
              \begin{equation*}
                H' \vdash P' \tsep \ocr \andalso \isolation(H', P')
              \end{equation*}
              Again using Corollary~\ref{cor:2.9} similarly to previous cases, we get
              \begin{equation*}
                H' \vdash P' \tsep \gsep.
              \end{equation*}


            \item[Case {\sc E-Assign}:] The {\sc E-Assign} rule gives us that
              \begin{equation} \label{eq:eassign1}
                \begin{gathered}
                  F = \sframe{L, t} \andalso t = \Let{x}{\FAss{y}{f}{z}}{t'} \\
                  F' = \sframe{L', t'} \andalso L' = L[x \mapsto L(z)] \\
                  L(y) = o_y \andalso H(o) = \Obj{C, \FM} \andalso f \in \dom(\FM)
                  \\
                  \FM' = \FM[f \mapsto L(z)] \andalso H' = H[o \mapsto \Obj{C, \FM'}]
                \end{gathered}
              \end{equation}
              We begin by proving $\vdash H'$. First we note
              that only change from $H$ to $H'$ is in what object $o_y$ maps to.
              Clearly from \eqref{eq:eassign1} we have
              \begin{equation} \label{eq:eassign2}
                \forall o \in \dom(H) = \dom(H'). \: \typeOf(o, H) = \typeOf(o,
                H')
              \end{equation}
              Thus by Proposition \ref{prop:2.12} 
              \begin{equation} \label{eq:eassign3}
                \forall k \in \Values. \: \typeOf(k, H) = \typeOf(k,
                H')
              \end{equation}
              \begin{addmargin}[1em]{1em}
                \begin{remark}
                  This means that immediately it is clear that the conditions
                  for $\vdash H'$ (see equations \eqref{eq:defwth1} and
                  \eqref{eq:defwth2}) holds for all $o' \in \dom(H')$ s.t. $o'
                  \neq o_y$.
                \end{remark}
              \end{addmargin}

              \vspace{2em}
              By $\vdash H$, definition of $H'$ and \eqref{eq:eassign3}
              \begin{equation} \label{eq:eassign4}
                \begin{aligned}
                  \forall f' &\in \fields(C), f' \neq f. \\ 
                                 &\typeOf(\FM'(f'), H') \stof \ftype(f, C)
                \end{aligned}
              \end{equation}
              By $H \vdash P$ and Proposition \ref{prop:2.13}
              \begin{equation} \label{eq:eassign5}
                H;a \vdash \FS.
              \end{equation}
              This and {\sc T-FS1} yields
              \begin{equation}\label{eq:eassign6}
                \begin{gathered}
                  H \vdash \Gamma; L \andalso \TypeRel{\Gamma}{a}{t}{\sigma'} \\
                  \sigma' \stof \sigma \andalso H; a \vdash^{s: \sigma} \GS
                \end{gathered}
              \end{equation}
              By $H;a \vdash^{s: \sigma} \GS$, \eqref{eq:eassign2} and prop.
              \ref{prop:2.14} we have 
              \begin{equation} \label{eq:eassign7} 
                H'; a \vdash^{s: \sigma} \GS
              \end{equation}
              $\TypeRel{\Gamma}{a}{t}{\sigma'}$ and {\sc T-Let} gives
              \begin{equation} \label{eq:eassign8}
                \TypeRel{\Gamma}{a}{\FAss{y}{f}{z}}{\gamma} \andalso
                \TypeRel{\Gamma, x: \gamma}{a}{t'}{\sigma'}
              \end{equation}
              which whith {\sc T-Assign} gives us
              \begin{equation} \label{eq:eassign9}
                \begin{gathered}
                  \TypeRel{\Gamma}{a}{y}{C'} \andalso \ftype(f, C') = \alpha \\
                  \TypeRel{\Gamma}{a}{z}{\alpha'} \andalso \alpha' \stof \alpha
                \end{gathered}
              \end{equation}
              By {\sc T-Var} and \TypeRel{\Gamma}{a}{y}{C'}
              \begin{equation}\label{eq:eassign10}
                (y: C') \in \Gamma 
              \end{equation}
              or equivalently $\Gamma(y) = C'$.
              Similarly
              \begin{equation} \label{eq:eassign11}
                (z: \alpha') \in \Gamma.
              \end{equation}
              By $H \vdash \Gamma; L$ we have $H \vdash \Gamma; L; z$ and thus
              \begin{equation} \label{eq:eassign12}
                \begin{aligned}
                  \typeOf(L(z), H) &= \alpha'' \\ 
                                   &\stof \alpha' \\ 
                                   &\stof \alpha \\ 
                                   &= \ftype(f, C') \\ 
                                   &= \ftype(f, C)
                \end{aligned}
              \end{equation}
              where the last equality comes from the fact that the classes are
              well formed. By \eqref{eq:eassign1} we have $\FM'(f) = L(z)$ and
              combining this with \eqref{eq:eassign12} we get
              \begin{equation} \label{eq:eassign13}
                \typeOf(\FM'(f), H) \stof \ftype(f, C).
              \end{equation}
              Combined with \eqref{eq:eassign3}, \eqref{eq:eassign4} and the
              remark above we get
              \begin{equation*} 
                \vdash H'
              \end{equation*}

              $H' \vdash P'$ follows similarly to {\sc E-Var} case.

              We now move on to OCAP reachability, isolation and global object
              separation.  Note that the only insteresting case is if $L(z)$ is
              an object refence, since otherwise the preconditions to Corollary
              \ref{cor:2.11} holds and we are done similarly to previous cases.
              Therefore we assume $L(z) \in \dom(H)$.  To help our efforts we
              state the following lemma.

              \begin{lemma} \label{lem:2.15}
                We let everything be as in case \EAssign{} above. Furthermore we
                let $L(z) \in \dom(H)$. Then
                \begin{equation} \label{eq:lem2.15a}
                  \begin{aligned}
                    &\reachable(\accRoots(\FS', H'), \Graph(H')) \subseteq \\
                    &\reachable(\accRoots(\FS, H), \Graph(H))
                  \end{aligned}
                \end{equation}
                If $\HS|_b^{\iota'} \in Q$
                \begin{equation}\label{eq:lem2.15b}
                  \begin{aligned}
                    &\reachable(\accRoots(\HS, H'), \Graph(H')) \subseteq \\
                    &\reachable(\accRoots(\HS, H), \Graph(H)) \cup  \\
                    &\reachable(\accRoots(\FS, H), \Graph(H)).
                  \end{aligned}
                \end{equation}
                Furthermore, if $a = \ocap$ or $b = \ocap$ then
                \begin{equation}\label{eq:lem2.15c}
                  \begin{aligned}
                    &\reachable(\accRoots(\HS, H'), \Graph(H')) \subseteq \\
                    &\reachable(\accRoots(\HS, H), \Graph(H))
                  \end{aligned}
                \end{equation}
              \end{lemma}

              \begin{proof}
                Let $\Graph(H) = (V_H, E_H)$. Then clearly
                \begin{equation}
                  \begin{gathered}
                    \Graph(H') = (V_H, E_{H'}) \\
                    E_{H'} = (E_H \setminus \{(o_y, \FM(f))_f \}) \cup \{(o_y,
                    o_z)_f \}
                  \end{gathered}
                \end{equation}
                We prove each part separately.  We begin with
                \eqref{eq:lem2.15a}.  Let
                \begin{equation*}
                  q \in \reachable(\accRoots(\FS', H'), \Graph(H')).
                \end{equation*}
                Because of reachability equivalence (Proposition
                \ref{prop:reacheq}) this means there is an $o \in \accRoots(\FS',
                H')$ such that
                \begin{equation}
                  \exists \text{ path } o_0, \dots, o_n \in \Graph(H')
                  \text{ s.t. } o_0 = o, o_n = q
                \end{equation}
                Let $f_i$ be the associated field names from which the path is
                formed, i.e. $(o_i, o_{i+1})_{f_i} \in \Graph(H')$.
                By \eqref{eq:eassign1}
                \begin{equation}
                  \accRoots(\FS', H') \subseteq \accRoots(\FS, H).
                \end{equation}
                Clearly $o \in \accRoots(\FS, H)$.

                There are two cases:
                \begin{enumerate}
                  \item For some $i \in \left\{ 0, \dots, n-1 \right\}$ $o_i =
                    o_y, o_{i+1} = o_z$ and $f_i = f$, i.e. the edge $(o_y,
                    o_z)_{f}$ is part of the path.
                  \item There is no such $i$.
                \end{enumerate}
                We prove each case separately. For case 1 let $i$ be the last
                such index. Examine the path 
                \begin{equation}
                  o_{i+1}, \dots, o_n
                \end{equation}
                Clearly this path is in $\Graph(H)$ since we chose $i$ to be the
                last index that fulfills case 1.  Since $o_{i+1} = o_z$ and
                $o_z$ obviously is in $\accRoots(\FS, H)$
                \begin{equation}
                  q \in \reachable(\accRoots(\FS, H), \Graph(H)).
                \end{equation}

                For case 2 the new edge $(o_y, o_z)_f$ is not part of the path.
                Thus all edges $(o_i, o_{i+1})_{f_i}$ are in $\Graph(H)$ and and
                thus the path $o_0, \dots, o_n$ is a path in $\Graph(H)$ aswell.
                This means $q \in \reachable(\accRoots(\FS, H), \Graph(H))$ which
                concludes the proof of the first part of the lemma.

                To prove \eqref{eq:lem2.15b} take any
                \begin{equation}
                  q \in \reachable(\accRoots(\HS, H'), \Graph(H')).
                \end{equation}
                Similarly to before there is an $o \in \accRoots(\HS,H')$ such
                that
                \begin{equation}
                  \exists \text{ path } o_0, \dots, o_n \in \Graph(H')
                  \text{ s.t. } o_0 = o, o_n = q
                \end{equation}
                We let $f_i$ be the corresponding field names as before.
                We have exactly the same cases. For case 1 we let $i$ be the last
                such index. Then similarly to before we get that $o_{i+1} \in
                \accRoots(\FS, H)$ and $q \in \reachable(\accRoots(\FS, H),
                \Graph(H))$.
                
                Case 2 is analogous and reaches the conclusion that $q \in
                \reachable(\accRoots(\HS, H), \Graph(H))$.

                Lastly we prove that if $a = \ocap$ or $b = \ocap$ then
                \eqref{eq:lem2.15c} holds. To prove this implication we assume
                that either $a = \ocap$ or $b = \ocap$. As before we take any
                \begin{equation}
                  q \in \reachable(\accRoots(\HS, H'), \Graph(H')).
                \end{equation}
                Similarly we get that there there must be an $o \in
                \accRoots(\HS, H')$ such that
                \begin{equation}
                  \exists \text{ path } o_0, \dots, o_n \in \Graph(H') \text{ s.t. } o_0 = o,
                  o_n = q
                \end{equation}
                with related field names $f_i$ as before. We get the exact same
                cases as above. For case 1 we instead let $i$ be the first index with
                the properties described. As noted this means we have the edge
                $(o_y, o_z)_f$ as part of our path. Specifically this means
                \begin{equation}\label{eq:lemma2.15-p3-1}
                  o_y \in \reachable(\accRoots(\HS, H), \Graph(H))
                \end{equation}
                since $o_0, \dots, o_i$ is a path in $\Graph(H)$ from $o$ to
                $o_y$, due to the way $i$ was chosen.  Moreover
                \begin{equation}\label{eq:lemma2.15-p3-2}
                  o_y \in \reachable(\accRoots(\FS, H), \Graph(H))
                \end{equation}
                since $(y \mapsto o_y) \in L$. From \eqref{eq:eassign1} it is clear that
                \begin{equation}\label{eq:lemma2.15-p3-3}
                  \typeOf(o_y, H) \not\stof \CellType.
                \end{equation}
                Now \eqref{eq:lemma2.15-p3-1}, \eqref{eq:lemma2.15-p3-2} and
                \eqref{eq:lemma2.15-p3-3} contradicts $\isolation(H, P)$ since $b =
                \ocap$. Since case 1 leads to a contradiction, it is impossible.
                
                For case 2 it is clear that $o_0, \dots, o_n$ is also a
                path from $o$ to $q$ in $\Graph(H)$ which immediately implies 
                \begin{equation}
                  q \in \reachable(\accRoots(\HS, H), \Graph(H)).
                \end{equation}
              \end{proof}

              Continuing with the proof of case {\sc E-Assign} first off we use
              the lemma to prove OCAP reachability.  We first take any thread
              $\HS|_b^{\iota'} \in Q$. Studying the $\ocr$ rules it is clear that we can
              prove that $H'; b \vdash \HS \tsep \ocr$ holds trivially for any
              thread $\HS|_b^{\iota'}$ for which $b = \nocap$. Therefore we from now on
              assume that $b = \ocap$.  By $H \vdash P \tsep \ocr$, $H; b \vdash
              \FS \tsep \ocr$. By Proposition~\ref{prop:ocrloc_eq} and $b =
              \ocap$,
              \begin{equation}\label{eq:eassign14}
                \begin{aligned}
                  \forall o &\in \reachable(\accRoots(\HS, H), H). \\
                            &\ocap(\typeOf(o, H)).
                \end{aligned}
              \end{equation}
              Lemma~\ref{lem:2.15} (equation \eqref{eq:lem2.15c}) and equations
              \eqref{eq:eassign2} and \eqref{eq:eassign14} means 
              \begin{equation}
                \begin{aligned}
                  \forall o &\in \reachable(\accRoots(\HS, H'), \Graph(H')). \\
                            &\ocap(\typeOf(o, H')).
                \end{aligned}
              \end{equation}
              But this is equivalent to $H';b \vdash \HS \tsep \ocr$ by
              Proposition~\ref{prop:ocrloc_eq}. 

              In order to prove $H' \vdash P' \tsep \ocr$ the only thing that
              remains is to show $H';a \vdash \FS' \tsep \ocr$. This is anologous
              to the above reasoning about $H';b \vdash \HS~\ocr$ using
              \eqref{eq:lem2.15a} instead. Thus we have
              \begin{equation*}
                H' \vdash P' \tsep \ocr.
              \end{equation*}

              $\isolation(H', P')$ follows immediately from $\isolation(H, P)$,
              Proposition~\ref{prop:2.6}, Lemma~\ref{lem:2.15}, equation
              \eqref{eq:eassign2} and applying rule {\sc ISO-Procs}. We also
              note that nowhere have we relied on the fact that
              $H~\vdash~P~\tsep~\gsep$ so using Corollary~\ref{cor:2.9} we also
              have $H'~\vdash~P'~\tsep~\gsep$.

            % TODO add definition of \default, maybe \fdecls also
            \item[Case {\sc E-New}:] According to rule {\sc E-New}
              \begin{equation} \label{eq:enew1}
                \begin{gathered} 
                  F = \sframe{L, t} \andalso t = \Let{x}{\New{C}}{t'} \\
                  F' = \sframe{L', t'} \andalso L' = L[x \mapsto o_x] \\
                  o_x \text{ fresh object reference } \\
                  \FM = [f \mapsto \default(\sigma) \mid (\VarDecl{f}{\sigma}) \in
                  \fdecls(C)] \\
                  H' = H[o_x \mapsto \Obj{C, \FM}]
                \end{gathered}
              \end{equation}
              If we let $\Graph(H) = (V_H, E_H)$ it is easy to see that
              \begin{equation}
                \Graph(H') = ( V_H \cup \left\{ o_x \right\}, E_H )
              \end{equation}
              By definition of $\default$
              \begin{equation} \label{eq:enew-h1}
                \begin{aligned}
                  \forall f &\in \fields(C).\\
                  &f \in \dom(\FM) \land \typeOf(\FM(f), H') \stof \ftype(f, C).
                \end{aligned}
              \end{equation}
              Also
              \begin{equation}
                \forall o \in \dom(H). \: H(o) = H'(o) \label{eq:enew-h2},
              \end{equation}
              and thus
              \begin{equation}
                \forall o \in \dom(H). \: \typeOf(o, H) = \typeOf(o, H').
                \label{eq:enew-h3}
              \end{equation}
              Using the definition of a well typed heap together with
              \eqref{eq:enew-h1}, \eqref{eq:enew-h2} and \eqref{eq:enew-h3} we
              can easily show that $\vdash H'$.

              By Proposition~\ref{prop:2.13}
              \begin{equation} 
                H \vdash Q \andalso H; a \vdash \FS.
              \end{equation}
              Propositions \ref{prop:2.13} and \ref{prop:2.14} implies
              \begin{equation} \label{eq:enew-typing1}
                H' \vdash Q.
              \end{equation}
              $H;a \vdash \FS$ and {\sc T-FS1} yields
              \begin{equation}
                \begin{gathered}
                  H \vdash \Gamma; L \andalso \TypeRel{\Gamma}{a}{t}{\sigma'}
                  \\
                  \sigma' \stof \sigma \andalso H;a \vdash^{s: \sigma} \GS.
                \end{gathered} 
              \end{equation}
              Applying {\sc T-Let} and {\sc T-New} we get
              \begin{equation} \label{eq:enew-typing5}
                \TypeRel{\Gamma, x: C}{a}{t'}{\sigma'} \andalso a = \ocap
                \implies \ocap(C).
              \end{equation}
              Letting $\Gamma' = \Gamma, x: C$ we have
              \begin{equation} \label{eq:enew-typing2}
                \TypeRel{\Gamma'}{a}{t'}{\sigma'}.
              \end{equation}
              We have that $H' \vdash \Gamma';L';x$ since $\typeOf(L'(x), H')
              \stof \Gamma'(x)$. Thus we have
              \begin{equation} \label{eq:enew-typing3}
                H' \vdash \Gamma'; L'.
              \end{equation}
              By Proposition \ref{prop:2.14}
              \begin{equation} \label{eq:enew-typing4}
                H';a \vdash^{s: \sigma} \GS.
              \end{equation}
              Using {\sc T-FS1} together with $\sigma' \stof \sigma$, \eqref{eq:enew-typing2},
              \eqref{eq:enew-typing3} and \eqref{eq:enew-typing4} we get
              \begin{equation}
                H';a \vdash \FS',
              \end{equation}
              and combined with \eqref{eq:enew-typing1} and {\sc T-Procs} we finally
              get
              \begin{equation}
                H' \vdash P'.
              \end{equation}

              Moving on to OCAP reachability. For any $\HS|_b^{\iota'} \in Q$,
              we see that
              \begin{equation} \label{eq:enew-ocr1}
                \begin{aligned}
                  &\reachable(\accRoots(\HS, H'), \Graph(H')) = \\
                  &\reachable(\accRoots(\HS, H), \Graph(H)) 
                \end{aligned}
              \end{equation}
              and furthermore that
              \begin{equation} \label{eq:enew-ocr2}
                \begin{aligned}
                  &\reachable(\accRoots(\FS', H'), \Graph(H')) = \\
                  &\reachable(\accRoots(\FS, H), \Graph(H)) \cup \left\{ o_x
                  \right\}.
                \end{aligned}
              \end{equation}
              Clearly $\typeOf(o_x, H') = C$. Combining this with
              \eqref{eq:enew-h3}, \eqref{eq:enew-typing5}, \eqref{eq:enew-ocr1},
              \eqref{eq:enew-ocr2}, {\sc OCR-P}, $H \vdash P \tsep \ocr$ and
              Proposition~\ref{prop:ocrloc_eq} we see that
              \begin{equation}
                H' \vdash P' \tsep \ocr.
              \end{equation}

              Similarly it is easy to show $\isolation(H', P')$ using equations
              \eqref{eq:enew-h3}, \eqref{eq:enew-ocr1} and \eqref{eq:enew-ocr2}
              together with Proposition~\ref{prop:2.6} and {\sc ISO-Procs}.
              Since the isolation proof never relies on $H \vdash P \tsep \gsep$
              we can apply Corollary~\ref{cor:2.9}.
              \begin{equation}
                \isolation(H', P') \andalso H' \vdash P' \tsep \gsep
              \end{equation}

            \item[Case {\sc E-NewCell}:] We have
              \begin{equation}
                H' = H[o \mapsto \Cell{\DEP, \bot_{\LatVals}}] \andalso \DEP = \emptyset
              \end{equation}
              for some fresh object reference $o$. 
              Since $\DEP$ is empty \eqref{eq:defwth2} holds vacuously and since
              all other heap elements are the same as in $H$ we have
              \begin{equation}
                \vdash H'.
              \end{equation}
              Proving the rest is similar to case {\sc E-New}.

            \item[Case {\sc E-Put}:] According to rule {\sc E-Put}
              \begin{equation}
                \begin{gathered}
                  F = \sframe{L, t} \andalso t = \Let{x}{\Put{y}{z}}{t'} \\
                  F' = \sframe{L', t'} \andalso L' = L[x \mapsto L(y)] \\
                  L(y) = o_y \andalso H(o_y) = \Cell{\DEP, l} \\
                  L(z) = l' \andalso H' = H[o_y \mapsto \Cell{\DEP, l \sqcup
                  l'}].
                \end{gathered}
              \end{equation}
              Since the only change to the heap is $l$ being updated to $l
              \sqcup l'$, by inspecting the definition for a well typed heap and
              using $\vdash H$ it is easy to conclude that
              \begin{equation}
                \vdash H'.
              \end{equation}
              The proof of $H' \vdash P'$ follows similar to case {\sc E-Var}.
              The preconditions of Corollary~\ref{cor:2.11} clearly hold. Thus 
              \begin{equation}
                H' \vdash P' \tsep \ocr \andalso \isolation(H', P').
              \end{equation}
              We apply Corollary~\ref{cor:2.9} to get
              \begin{equation}
                H' \vdash P' \tsep \gsep
              \end{equation}

            \item[Case {\sc E-When}:] From rule {\sc E-When} we know
              \begin{equation}
                \begin{gathered}
                  F = \sframe{L, t} \\ 
                  t = \Let{x}{\When{y}{z}{\CB{\overline{cap}}{w}{t''}}}{t'} \\
                  F' = \sframe{L', t'} \andalso L' = L[x \mapsto L(y)] \\
                  L(y) = o_y \andalso H(o_y) = \Cell{\DEP, l} \andalso L(z) = l' \\
                  L_{\text{env}} = [u \mapsto L(u') \mid (\Capt{u}{u'}) \in
                  \overline{cap}] \andalso cb = (L_{\text{env}}, w \Rightarrow
                  t'') \\
                  \iota \text{ fresh thread id} \andalso \DEP' = \DEP \cup (l',
                  cb)^\iota
                  \\
                  H' = H[o_y \mapsto \Cell{\DEP', l}].
                \end{gathered}
              \end{equation}
              We note that $\dom(H) = \dom(H')$ and that
              \begin{equation} \label{eq:ewhen-h5}
                \begin{aligned}
                  \forall o &\in \dom(H).  \\
                    &\typeOf(o, H) = \typeOf(o, H'),
                \end{aligned}
              \end{equation}
              and thus
              \begin{equation}
                \begin{aligned}
                  \forall k \in \Values . \\
                    &\typeOf(k, H) = \typeOf(k, H').
                \end{aligned}
              \end{equation}
              Combining this with the fact that the only object on heap changed
              is $H(o_y)$ we have shown $\vdash H'$ if we can show that
              \begin{equation}
                \begin{aligned}
                  \forall (l^{\text{cb}}&, (L^{\text{cb}}_{\text{env}},
                  z^{\text{cb}} \Rightarrow t^{\text{cb}}))^{\iota^{\text{cb}}} \in
                  \DEP'. \\
                  &\forall (x \mapsto k) \in L^{\text{cb}}_{\text{env}}.\: \typeOf{(k, H')} \stof
                  \CellType \: \land \\
                  &\TypeRel{\Gamma_{\CellType{}}(L^{\text{cb}}_{\text{env}}), z^{\text{cb}}:
                  \LatType}{\ocap}{t^{\text{cb}}}{\gamma} .
                \end{aligned}
              \end{equation}
              It is simple to see that, because of $\vdash H$ and that the only
              difference between $\DEP$ and $\DEP'$ is the addition of $(l',
              (L_{\text{env}}, w \Rightarrow t''))$, we have proved $\vdash H'$
              if we can prove
              \begin{equation} \label{eq:ewhen-h1}
                \forall (x \mapsto k) \in L_{\text{env}}.\: \typeOf{(k, H')} \stof
                \CellType
              \end{equation}
              and
              \begin{equation} \label{eq:ewhen-h2}
                \TypeRel{\Gamma_{\CellType{}}(L_{\text{env}}), w:
                \LatType}{\ocap}{t''}{\gamma}.
              \end{equation}
              Similarly to earlier cases, using $H \vdash P$ together with
              typing rules such as {\sc T-Let} and {\sc T-When} we can easily
              get
              \begin{equation}\label{eq:ewhen-h3}
                H \vdash \Gamma; L \andalso \forall (\Capt{u}{u'}) \in
                \overline{cap}. \: \TypeRel{\Gamma}{a}{u'}{\CellType} 
              \end{equation}
              and
              \begin{equation} \label{eq:ewhen-h4}
                \begin{gathered}
                  \Gamma_{\text{cells}} = [u \mapsto \CellType \mid (\Capt{u}{u'}) \in
                  \overline{cap}] \\
                  \TypeRel{\Gamma_{\text{cells}}, w:
                  \LatType}{\ocap}{t''}{\gamma'}.
                \end{gathered}
              \end{equation}
              \eqref{eq:ewhen-h4} is just a different formulation of
              \eqref{eq:ewhen-h2} since the actual value of $\gamma$ does not
              matter. Moreover, using equations \eqref{eq:ewhen-h5} and
              \eqref{eq:ewhen-h3} it is simple to prove \eqref{eq:ewhen-h1} with
              the help of rules {\sc WF-EnvVar}, {\sc WF-Env} and {\sc T-Var}.
              Thus we are done and have
              \begin{equation}
                \vdash H'.
              \end{equation}

              Proof of $H' \vdash P'$ is similar to case {\sc E-Var} and
              similarly to other cases we can apply corollaries \ref{cor:2.11}
              and \ref{cor:2.9} to get
              \begin{equation}
                H' \vdash P' \tsep \ocr \andalso \isolation(H', P') \andalso H'
                \vdash P' \tsep \gsep.
              \end{equation}

              {\bf This concludes the case} {\sc E-FProp}.
          \end{description}

        \item[Case {\sc E-Call}:] From rule {\sc E-Call}
          \begin{equation} \label{eq:ecall1}
            \begin{gathered}
              \FS = \sframe{L, t} \circ \GS \andalso t = \Let{x}{\Call{y}{m}{z}}{t''}\\
              \FS' = \xframe{L', t'}^x \circ \sframe{L, t} \circ \GS \\
              L(y) = o_y \andalso H(o_y) = \Obj{C, \FM} \\
              \mbody(m, C) = u \to t' \\
              L_{\text{base}} =
              \begin{cases}
                \emptyset & \text{ if } a = \ocap \\
                L_0       & \text{ if } a = \nocap
              \end{cases} \\
              L' = L_{\text{base}}[\This \mapsto L(y), u \mapsto L(z)] \\
              H = H'.
            \end{gathered}
          \end{equation}
          Since $H = H'$ 
          \begin{equation*}
            \vdash H'.
          \end{equation*}
          By $H = H'$, $H \vdash P$ and propositions~\ref{prop:2.13},
          and \ref{prop:2.14},  we have
          \begin{equation}
            \begin{gathered}
              H' \vdash Q \andalso H;a \vdash \FS.
            \end{gathered}
          \end{equation}
          Using the second of these and rule {\sc T-FS1} we have
          \begin{equation} \label{eq:ecall-hp0}
            \begin{gathered}
              H \vdash \Gamma; L \andalso \TypeRel{\Gamma}{a}{t}{\sigma'} \\
              \sigma' \stof \sigma \andalso H; a \vdash^{s:\sigma} \GS.
            \end{gathered}
          \end{equation}
          Since $H = H'$
          \begin{equation} \label{eq:ecall-hp5}
            H' \vdash \Gamma;L \andalso H'; a \vdash^{s: \sigma} \GS.
          \end{equation}
          Using $\TypeRel{\Gamma}{a}{t}{\sigma}$ with rules {\sc T-Let, T-Call} and
          {\sc T-Var} we have that
          \begin{equation} \label{eq:ecall-hp1}
            \begin{gathered}
              \TypeRel{\Gamma}{a}{\Call{y}{m}{z}}{\tau} \andalso \TypeRel{\Gamma,
              x: \tau}{a}{t''}{\sigma'} \\
              \Gamma(y) = C' \andalso \mtype(m, C') = \gamma \mapsto \tau \\
              \Gamma(z) = \gamma' \andalso \gamma' \leq \gamma.
            \end{gathered}
          \end{equation}
          Using {\sc T-FS2} with \eqref{eq:ecall-hp0}, \eqref{eq:ecall-hp5} and
          \eqref{eq:ecall-hp1} we get 
          \begin{equation} \label{eq:ecall-hp6}
            H';a \vdash^{x: \tau} \FS.
          \end{equation}
          By $H' \vdash \Gamma;L$ we have
          \begin{equation} \label{eq:ecall-hp2}
            \typeOf(L'(u), H') = \typeOf(L(z), H') \stof \gamma' \stof \gamma,
          \end{equation}
          and similarly
          \begin{equation} \label{eq:ecall-hp3}
            \typeOf(L'(\This), H') = \typeOf(L(y), H') \stof C \stof C'.
          \end{equation}

          Now, $a$ can be either $\nocap$ or $\ocap$. We first assume that $a =
          \nocap$.
          By our program and thus by extension our classes being well formed, together
          with $C \stof C'$, we must have a class $C''$ s.t. $C \stof C'' \stof
          C'$ such that the the class definition of $C''$ includes the method
          definition
          \begin{equation*}
            \MethodDef{m}{u}{\gamma}{\tau}{t'}.
          \end{equation*}
          By $C''$ being well formed
          \begin{equation}
            C'' \vdash \MethodDef{m}{u}{\gamma}{\tau}{t'}.
          \end{equation}
          By {\sc WF-Method} this means that
          \begin{equation} \label{eq:ecall-hp4}
            \TypeRel{\Gamma_0,\This: C'', u: \gamma}{\nocap}{t'}{\tau'}
            \andalso \tau' \stof \tau.
          \end{equation}
          We let
          \begin{equation}
            \Gamma_{\nocap}' = \Gamma_0, \This: C'', u: \gamma.
          \end{equation}
          It is not hard to prove that 
          \begin{equation} \label{eq:ecall-hp7}
            H' \vdash \Gamma_{\nocap}'; L'.
          \end{equation}
          By \eqref{eq:ecall-hp4} we clearly have
          \begin{equation} \label{eq:ecall-hp8}
            \TypeRel{\Gamma_{\nocap}'}{a}{t'}{\tau'} \andalso \tau' \stof \tau.
          \end{equation}
          Using  \eqref{eq:ecall-hp6}, \eqref{eq:ecall-hp7},
          \eqref{eq:ecall-hp8} and rule {\sc T-FS1} we have
          \begin{equation}
            H';a \vdash \FS'.
          \end{equation}
          Combining this with $H' \vdash Q$ and {\sc T-Procs} we get
          \begin{equation}
            H' \vdash P'.
          \end{equation}

          Now assume that $a = \ocap$. By $H \vdash P \tsep \ocr$ we have
          \begin{equation}
            \ocrloc(\FS, H)
          \end{equation}
          and thus by Proposition~\ref{prop:ocrloc_eq}
          \begin{equation}
            \begin{aligned}
              \forall q &\in \reachable(\accRoots(\FS, H), \Graph(H)) \\
              & \ocap(\typeOf(q, H)).
            \end{aligned}
          \end{equation}
          
          Of course this means that $\ocap(\typeOf(L(y), H))$ or equivalently
          $\ocap(C)$. Proceeding similarly to the case $a = \nocap$ we
          additionally have $\ocap(C'')$ due to rule {\sc OCAP-Class}, which
          demands that parent classes to $C$ should be ocap aswell. Again using
          {\sc OCAP-Class}, we get that 
          \begin{equation}
            C'' \vdash_{\ocap} \MethodDef{m}{u}{\gamma}{\tau}{t'}.
          \end{equation}
          By {\sc OCAP-Method} we get that 
          \begin{equation} \label{eq:ecall-hp9}
            \TypeRel{\This: C'', u: \gamma}{\nocap}{t'}{\tau'}
            \andalso \tau' \stof \tau.
          \end{equation}
          The rest is similar to the case where $a = \nocap$,
          replacing $\Gamma_{\nocap}'$ with 
          \begin{equation}
            \Gamma'_{\ocap} = \This: C'', u: \gamma.
          \end{equation}

          We now prove isolation, OCAP reachability and global object
          separation. Instead of working with $P$ directly we instead use
          $\tilde{P}$. It is clear from Propositions~\ref{prop:2.8} and
          \ref{prop:ocrtilde_eq} that in order to show 
          \begin{equation}
            H' \vdash P' \tsep \ocr \andalso \isolation(H', P') \andalso H'
            \vdash P' \tsep \gsep
          \end{equation}
          we can instead show
          \begin{equation}
            H' \vdash \tilde{P'} \tsep \ocr \andalso \isolation(H', \tilde{P'}).
          \end{equation}

          To show this we start by using propositions~\ref{prop:2.8} and
          \ref{prop:ocrtilde_eq} to get
          \begin{equation}
            H \vdash \tilde{P} \tsep \ocr \andalso \isolation(H, \tilde{P}).
          \end{equation}
          We also note that 
          \begin{equation}
            \tilde{P} = \tilde{Q} \cup_D \left\{ \FS|_a^\iota \right\} \text{ and }
            \tilde{P'} = \tilde{Q} \cup_D \left\{ \FS'|_a^\iota \right\}.
          \end{equation}

          We first assume that $a = \ocap$. It is easy to see that
          \begin{equation}
            \accRoots(\FS', H') \subseteq \accRoots(\FS, H)
          \end{equation}
          and that
          \begin{equation}
            \begin{aligned}
              \forall \HS|_b^{\iota'} &\in \tilde{Q}. \\
              &\accRoots(\HS, H') = \accRoots(\HS, H).
            \end{aligned}
          \end{equation}
          Clearly then, the preconditions to Corollary~\ref{cor:2.11} holds and
          we have $\isolation(H', \tilde{P'})$ and $H' \vdash \tilde{P'} \tsep
          \ocr$. Thus we are done.

          Now instead assume $a = \nocap$. Since $\isolation(H, \tilde{P})$ we have
          $\isolated(H, \HS_1, \HS_2)$ for any two distinct $\HS_1|_b^{\iota'},
          \HS_2|_c^{\iota''}
          \in \tilde{Q}$ such that $b = \ocap$ or $c = \ocap$. Therefore, since $H = H'$
          \begin{equation}
            \begin{aligned}
              \forall &\text{ distinct } \HS_1|_b^{\iota'}, \HS_2|_c^{\iota''} \in \tilde{Q}. \\ 
              & b = \ocap \lor c = \ocap \implies \isolated(H', \HS_1, \HS_2).
            \end{aligned}
          \end{equation}
          Thus all we need for $\isolation(H', \tilde{P'})$ is to show that
          \begin{equation}
            \begin{aligned}
              \forall \HS|_b^{\iota'} &\in \tilde{Q} . \\
              &b = \ocap \implies \isolated(H', \FS', \HS)
            \end{aligned}
          \end{equation}
          since $a = \nocap$. To prove this implication we take any
          $\HS|_b^{\iota'} \in
          \tilde{Q}$ such that $b = \ocap$. Note that this implies $\HS \neq
          \FS_g$. Because of $a = \nocap$ and the definition of $L'$ in
          \eqref{eq:ecall1}
          \begin{equation} \label{eq:ecall-iso1}
            \accRoots(\FS', H') = \accRoots(\FS, H) \cup \left\{ o_g \right\}.
          \end{equation}
          By $\isolation(H, \tilde{P})$  and $b = \ocap$,
          \begin{equation} \label{eq:ecall-iso2}
            \isolated(H, \HS, \FS_g)
          \end{equation}
          and
          \begin{equation} \label{eq:ecall-iso4}
            \isolated(H, \FS, \HS).
          \end{equation}
          By definition
          \begin{equation} \label{eq:ecall-iso3}
            \accRoots(\FS_g, H) = \left\{o_g \right\}.
          \end{equation}
          By inspecting \eqref{eq:ecall-iso1}, \eqref{eq:ecall-iso2},
          \eqref{eq:ecall-iso3} and \eqref{eq:ecall-iso4} and considering
          Proposition~\ref{prop:2.6} it is fairly easy to show
          \begin{equation}
            \isolated(H, \FS', \HS).
          \end{equation}
          Specifically, to show this, assume for a contradiction that we do not
          have $\isolated(H, \FS', \HS)$. This would entail that
          \begin{equation}
            \begin{aligned}
              \exists q &\in \dom(H), o \in \accRoots(\HS, H), o' \in
              \accRoots(\FS', H) . \\
              &\typeOf(q, H) \not\stof \CellType \land \reach(H, o, q) \land
              \reach(H, o', q).
            \end{aligned}
          \end{equation}
          If $o' = o_g$ this immediately contradicts $\isolated(H, \HS, \FS_g)$ by
          Proposition~\ref{prop:2.6}. If we instead have $o' \neq o_g$ then by
          \eqref{eq:ecall-iso1} we must have
          \begin{equation}
            o' \in \accRoots(\FS, H).
          \end{equation}
          Similarly this would contradict \eqref{eq:ecall-iso4}.  Since
          $\HS|_b^{\iota'}$ was arbitrary and $H = H'$ we have 
          \begin{equation}
            \isolation(H', \tilde{P'}).
          \end{equation}

          By $H \vdash \tilde{P} \tsep \ocr$, $H = H'$ and
          Proposition~\ref{prop:ocrtilde_eq} we get
          \begin{equation} \label{eq:ecall-ocr1}
            H \vdash \tilde{Q} \tsep \ocr.
          \end{equation}
          Since $a = \nocap$ we have that 
          \begin{equation} \label{eq:ecall-ocr2}
            H'; a \vdash \FS' \tsep \ocr 
          \end{equation} 
          follows vacuously. Using rule {\sc OCR-P} together with
          \eqref{eq:ecall-ocr1} and \eqref{eq:ecall-ocr2} we get
          \begin{equation}
            H' \vdash \tilde{P'} \tsep \ocr.
          \end{equation}  
          Thus we are done with case $a = \nocap$.
          
        \item[Case {\sc E-Ret}:] Clearly from rule {\sc E-Ret} we have
          \begin{equation}
            \begin{gathered}
              \FS = \xframe{L, z}^x \circ \sframe{L', t'} \circ \GS \\
              \FS' = \sframe{L'', t'} \circ \GS \\
              L'' = L'[x \mapsto L(z)] \\
              H = H'.
            \end{gathered}
          \end{equation}
          By $H \vdash P$ and Proposition~\ref{prop:2.13} we have
          \begin{equation} \label{eq:eret-hp0}
            H \vdash Q \andalso H; a \vdash \FS.
          \end{equation}
          $H = H'$ immediately yields
          \begin{equation}
            H' \vdash Q.
          \end{equation}
          $H; a \vdash \FS$ together with rule {\sc T-FS1} gives us that
          \begin{equation}
            \begin{gathered}
              H \vdash \Gamma; L \andalso \TypeRel{\Gamma}{a}{z}{\tau'} \\
              \tau' \stof \tau \andalso H; a \vdash^{x: \tau} \sframe{L', t'} \circ \GS,
            \end{gathered}
          \end{equation}
          the last of which together with {\sc T-FS2} gives us
          \begin{equation} \label{eq:eret-hp1}
            \begin{gathered}
              H \vdash \Gamma'; L' \andalso \TypeRel{\Gamma', x:
              \tau}{a}{t'}{\sigma'} \\
              \sigma' \stof \sigma \andalso H; a \vdash^{s: \sigma} \GS
            \end{gathered}
          \end{equation}
          We let $\Gamma'' = \Gamma', x : \tau$. By $H = H'$, $H \vdash \Gamma;
          L$ and $\tau' \stof \tau$, it is obvious that
          \begin{equation} \label{eq:eret-hp3}
            \typeOf(L''(x), H') = \typeOf(L(z), H') \stof \tau.
          \end{equation}
          Using $H \vdash \Gamma'; L'$, \eqref{eq:eret-hp3} and rules {\sc WF-EnvVar}, {\sc
          WF-Env} we get
          \begin{equation} \label{eq:eret-hp4}
            H' \vdash \Gamma'', L''.
          \end{equation}
          By definition of $\Gamma''$, \eqref{eq:eret-hp1}, \eqref{eq:eret-hp4}
          and rule {\sc T-FS1} we have that
          \begin{equation} \label{eq:eret-hp2}
            H';a \vdash \FS'
          \end{equation}
          Applying {\sc T-Procs} together with \eqref{eq:eret-hp0} and
          \eqref{eq:eret-hp2} we finally get
          \begin{equation}
            H' \vdash P'
          \end{equation}

          Similarly to many previous cases we can apply corollarys
          \ref{cor:2.11} and \ref{cor:2.9} to get
          \begin{equation*}
            H' \vdash P' \tsep \ocr \andalso \isolation(H', P') \andalso H'
            \vdash P' \tsep \gsep
          \end{equation*}
      \end{description}
      {\bf This concludes the {\sc E-FSProp} case.}
      
    \item[Case {\sc E-Spawn}:] By rule {\sc E-Spawn} we have
      \begin{equation}
        \begin{gathered} \label{eq:espawn1}
          o \in \dom(H) \andalso H(o) = \Cell{\DEP, l} \\
          l' \sqsubseteq l \andalso (l', cb)^\iota \in \DEP \andalso cb =
          (L_{\text{env}}, z \Rightarrow t') \\
          L' = L_{\text{env}}[z \mapsto l'] \\
          H' = H[o \mapsto \Cell{\DEP - (l', cb)^\iota, l}] \\
          P' = P \cup \left\{ \FS'|_{\ocap}^\iota \right\} \\
          \FS' = \xframe{L', t'}^- \circ \varepsilon.
        \end{gathered}
      \end{equation}
      
      By definition of $H'$, $\vdash H$ and equation \eqref{eq:defwth2} from the
      definition of the well typed heap it should be fairly obvious that
      \begin{equation}
        \vdash H'
      \end{equation}
      since the only heap change is a removal of an element from $\DEP$. 


      We note that
      \begin{equation} \label{eq:espawn-hp0}
        \begin{aligned}
          \forall o &\in \dom(H) = \dom(H'). \\
            &\typeOf(o, H) = \typeOf(o, H').
        \end{aligned}
      \end{equation}
      By propositions \ref{prop:2.13} and \ref{prop:2.14} we have
      \begin{equation} \label{eq:espawn-hp4}
        H' \vdash P
      \end{equation}
      From $\vdash H$ we can see that
      \begin{equation} \label{eq:espawn-hp1}
        \begin{aligned}
          \forall (x \mapsto k) &\in L_{\text{env}}. \\
            &\typeOf(k, H) \stof \CellType
        \end{aligned}
      \end{equation}
      and
      \begin{equation} \label{eq:espawn-hp2}
        \TypeRel{\Gamma_{\CellType}(L_{\text{env}}), z:
        \LatType}{\ocap}{t'}{\gamma}.
      \end{equation}
      Letting
      \begin{equation}
        \Gamma' = \Gamma_{\CellType}(L_{\text{env}}), z: \LatType
      \end{equation}
      we can use \eqref{eq:espawn-hp0},\eqref{eq:espawn-hp1} and rules {\sc
      WF-EnvVar, WF-Env} to get
      \begin{equation} \label{eq:espawn-hp3}
        H' \vdash \Gamma', L'
      \end{equation}
      By rules {\sc T-FSEmpty2, T-FS1} and \eqref{eq:espawn-hp2},
      \eqref{eq:espawn-hp3} we get
      \begin{equation}
        H'; \ocap \vdash \FS'
      \end{equation}
      which combined with \eqref{eq:espawn-hp4} and {\sc T-Procs} yields
      \begin{equation}
        H' \vdash P'.
      \end{equation}
      
      Next we prove OCAP reachability. It is easy to see that
      \begin{equation} \label{eq:espawn-ocr0}
        \Graph(H') = \Graph(H).
      \end{equation}
      Because of this, $H \vdash P \tsep \ocr$, \eqref{eq:espawn-hp0} and
      Proposition~\ref{prop:ocrloc_eq},
      \begin{equation}
        H' \vdash P \tsep \ocr.
      \end{equation}
      This leaves only to prove
      \begin{equation}
        H'; \ocap \vdash \FS' \tsep \ocr.
      \end{equation}
      in order to get $H' \vdash P' \tsep \ocr$.
      Inspecting {\sc OCR-FS} we note that we are done if we can prove
      \begin{equation} \label{eq:espawn-ocr1}
        \ocrloc(\FS', H').
      \end{equation}
      But this is easy. By definition of $\FS'$
      \begin{equation}
        \forall o \in \accRoots(\FS', H'). \: \typeOf(o, H') \stof \CellType
      \end{equation}
      which implies
      \begin{equation} \label{eq:espawn-ocr2}
        \begin{aligned}
          \forall o &\in \reachable(\accRoots(\FS', H'), \Graph(H')). \\
            &\typeOf(o, H') \stof \CellType
        \end{aligned}
      \end{equation}
      The only objects of type \CellType{} on the heap are cell objects 
      whose exact type are \CellType{}, which is $\ocap$ by {\sc OCAP-Cell}.
      Combined with \eqref{eq:espawn-ocr2} this yields
      \begin{equation}
        \begin{aligned}
          \forall o &\in \reachable(\accRoots(\FS', H'), \Graph(H')). \\
            &\ocap(\typeOf(o, H')).
        \end{aligned}
      \end{equation}
      By Proposition \ref{prop:ocrloc_eq} we have shown \eqref{eq:espawn-ocr1}.

      Continuing by showing isolation we note that by \eqref{eq:espawn-hp0},
      \eqref{eq:espawn-ocr0} and \ref{prop:2.6} we have that
      \begin{equation}
        \isolation(H', P).
      \end{equation}
      Thus by Proposition \ref{prop:2.6} we are done if we can prove 
      \begin{equation}
        \HS|_b^{\iota'} \in P. \: \isolated(H', \FS', \HS).
      \end{equation}
      Again using Proposition~\ref{prop:2.6} we are done if we show
      \begin{equation}
        \begin{aligned}
          \forall q \in \: &\reachable(\accRoots(\FS', H'), \Graph(H')) \cap \\
            &\reachable(\accRoots(\HS, H'), \Graph(H')) . \\
            &\typeOf(q, H') \stof \CellType.
        \end{aligned}
      \end{equation}
      But this follows immediately from \eqref{eq:espawn-ocr2} and 
      by the above we have
      \begin{equation}
        \isolation(H', P').
      \end{equation} 
      Furthermore we never relied on the fact that $H \vdash P \tsep \gsep$ to
      prove this and thus get
      \begin{equation}
        H' \vdash P' \tsep \gsep 
      \end{equation}
      for free using Corollary~\ref{cor:2.9}.

    \item[Case {\sc E-Term}:] Trivial.
  \end{description}
\end{proof}


\section{Proof of Progress}
\label{sec:proof_of_progress}

\begin{theorem*}[Progress]
  Let $S$ be a state such that $\vdash S \tsep \stateok$. Then either 
  \begin{enumerate}
    \item $\exists S'$ s.t. $S \Rrightarrow S'$, 
    \item $S = H, \emptyset$ for some heap $H$ s.t. $\noSpawn{(H)}$ or
    \item $S = \Error$.
  \end{enumerate}
\end{theorem*}

\begin{proof}
  Assume for a contradiction that $\vdash S \tsep \stateok$ but that neither 1,
  2 or 3 from the theorem holds. We must then have that the following three
  statements hold
  \begin{enumerate}
    \item $\not\exists S'$ s.t. $S \Rrightarrow S'$.
    \item $S = H, \emptyset$ but $\noSpawn(H)$ does not hold, or $S = H, P$
      where $P \neq \emptyset$.
    \item $S \neq \Error$.
  \end{enumerate}
  This gives us two cases: Either $S = H, \emptyset$ and $\noSpawn(H)$ does not hold,
  or $S = H, P$ and $P \neq \emptyset$.

  We begin with the case where $S = H, \emptyset$ and $\noSpawn(H)$ does not
  hold. Looking at the definition of $\noSpawn$ this means that there is at
  least one $o \in \dom(H)$ such that $H(o) = \Cell{\DEP, l}$ and
  \begin{equation*}
    \exists (l', cb)^\iota \in \DEP. \: l' \sqsubseteq l.
  \end{equation*}
  This clearly satisfies the preconditions of execution rule {\sc E-Spawn} and
  thus there is another state $S' = H', P'$ such that $S \Rrightarrow S'$. But
  this contradicts statement 1 above.

  We proceed with the case where $S = H, P$ and $P \neq \emptyset$. This means
  that
  \begin{equation*}
    P = Q \cup_D \left\{ \FS|_a^\iota \right\} \andalso \FS = \sframe{L, t} \circ \GS
  \end{equation*}
  By $\vdash S \tsep \stateok$ we must have
  \begin{equation*}
    H \vdash P.
  \end{equation*}
  By Proposition \ref{prop:2.13} this means that 
  \begin{equation*}
    H;a \vdash \FS.
  \end{equation*}
  Inspecting rule {\sc T-FS1} we see that
  \begin{equation} \label{eq:prog_tbase}
    \begin{gathered}
      H \vdash \Gamma;L \andalso \TypeRel{\Gamma}{a}{t}{\sigma'} \\
      \sigma' \stof \sigma \andalso H;a \vdash^{s: \sigma} \GS.
    \end{gathered}
  \end{equation}
  We proceed by cases on the form of $t$. 
  
  \begin{note}
    Note that we only state which \emph{base rule} is used for each step found below.
    This means that if, e.g., the rule used is a frame reduction rule (one of the
    rules defined in Figure~\ref{fig:frame_red_rules}) the application of {\sc
    E-FProp} and {\sc E-FSProp} is implied. We will use the symbol \lightning to
    indicate a contradiction.
  \end{note}

  \begin{description}
    \item[Case $t = x$:] 
      If $\GS = \varepsilon$ then we can make a step to $S' = H, Q$ by applying
      rule {\sc E-Term}. \contradiction

      Otherwise we have that $\GS = \xframe{L', t'}^{s'} \circ \HS$. This means we
      can step to $S' = H, P'$ where
      \begin{equation*}
        \begin{gathered}
          P' = Q \cup_D \left\{ \FS'|_a^\iota \right\} \andalso \FS' = \xframe{L'',
          t'}^{s'} \circ \HS \\ 
          L'' = L'[s \mapsto L(x)],
        \end{gathered} \end{equation*}
      by base rule {\sc E-Ret}. $L(x)$ is defined by $H \vdash \Gamma; L$. \contradiction
    
    \item[Case $t = \Let{x}{e}{t'}$:]
      We proceed on cases of $e$.
      \begin{description}
        \item[Case $e = \NullVal$:] 
          We can step to $S' = H, P'$ where
          \begin{equation*}
            \begin{gathered}
              P' = Q \cup \left\{\FS'|_a^\iota \right\} \andalso \FS' = \sframe{L',
              t'} \\
              L' = L[x \mapsto \NullVal]
            \end{gathered}
          \end{equation*}
          by base rule {\sc E-Null}. \contradiction

        \item[Case $e = l$:]
          We can step to $S' = H, P'$ where 
          \begin{equation*}
            \begin{gathered}
              P' = Q \cup \left\{\FS'|_a^\iota \right\} \andalso \FS' = \sframe{L',
              t'} \\
              L' = L[x \mapsto l]
            \end{gathered}
          \end{equation*}
          by rule {\sc E-LVal}. \contradiction

        \item[Case $e = y$:]
          We can step to $S' = H, P'$ where
          \begin{equation*}
            \begin{gathered}
              P' = Q \cup \left\{\FS'|_a^\iota \right\} \andalso \FS' = \sframe{L',
              t'} \\
              L' = L[x \mapsto L(y)]
            \end{gathered}
          \end{equation*}
          by rule {\sc E-Var}. $L(y)$ is defined by $H \vdash \Gamma;L$. \contradiction

        \item[Case $e = \FSel{y}{f}$:]
          By \eqref{eq:prog_tbase}, {\sc T-Let}, {\sc T-Select} and {\sc T-Var}
          \begin{equation*}
            \Gamma(y) = C \andalso \ftype(f, C) = \tau.
          \end{equation*}
          By $H \vdash \Gamma;L$
          \begin{equation*}
            \typeOf(L(y), H) \stof C
          \end{equation*}
          which means that either $\typeOf(L(y), H) = C' \stof C$ or \\
          $\typeOf(L(y), H) = \NullType$. 
          
          If $\typeOf(L(y), H) = \NullType$ by definition of $\typeOf$
          \begin{equation*}
            L(y) = \NullVal
          \end{equation*}
          Then we can step to $S' = \Error$ by {\sc E-NullSelect}.
          \contradiction

          If $\typeOf(L(y), H) = C'$ then
          \begin{equation*}
            L(y) = o_y \andalso H(o_y) = \Obj{C', \FM}
          \end{equation*}
          By $C' \stof C$, $\ftype(f, C) = \tau$ and well-formedness of
          classes,
          \begin{equation*}
            f \in \fields(C') \andalso \ftype(f, C') = \tau.
          \end{equation*}
          Then $\vdash H$ implies
          \begin{equation*}
            f \in \dom(\FM).
          \end{equation*}
          But then by rule {\sc E-Select} we can step to $S' = H, P'$ where
          \begin{equation*}
            \begin{gathered}
              P' = Q \cup \left\{\FS'|_a^\iota \right\} \andalso \FS' = \sframe{L',
              t'} \\
              L' = L[x \mapsto \FM(f)].
            \end{gathered}
          \end{equation*}
          \contradiction

        \item[Case $e = \FAss{y}{f}{z}$:]
          By \eqref{eq:prog_tbase}, {\sc T-Let}, {\sc T-Assign} and {\sc T-Var}
          \begin{equation*}
            \begin{gathered}
              \Gamma(y) = C \andalso \ftype(f,C) = \tau  \\
              \Gamma(z) = \tau' \andalso \tau' \stof \tau.
            \end{gathered}
          \end{equation*}
          By $H \vdash \Gamma;L$
          \begin{equation*}
            \typeOf(L(y), H) \stof C,
          \end{equation*}
          which means that either $\typeOf(L(y), H) = C' \stof C$ or \\
          $\typeOf(L(y), H) = \NullType$.

          If $\typeOf(L(y), H) = \NullType$ we get
          \begin{equation*}
            L(y) = \NullVal.
          \end{equation*}
          We can then step to $\Error$ by rule {\sc E-NullAssign}.
          \contradiction

          If $\typeOf(L(y), H) = C'$ then
          \begin{equation*}
            L(y) = o_y \andalso H(o_y) = \Obj{C', \FM}.
          \end{equation*}
          By $C' \stof C$, $\ftype(f, C) = \tau$ and well-formedness of classes
          \begin{equation*}
            f \in \fields(C') \andalso \ftype(f, C') = \tau.
          \end{equation*}
          Then by $\vdash H$ we have 
          \begin{equation*}
            f \in \dom(\FM).
          \end{equation*}
          We can then by {\sc E-Assign} step to $S' = H', P'$ where
          \begin{equation*}
            \begin{gathered}
              P' = Q \cup \left\{\FS'|_a^\iota \right\} \andalso \FS' = \sframe{L',
              t'} \\
              L' = L[x \mapsto L(z)] \andalso \FM' = \FM[f \mapsto L(z)] \\
              H' = H[o_y \mapsto \Obj{C', \FM'}]
            \end{gathered}
          \end{equation*}
          $L(z)$ is defined by $H \vdash \Gamma; L$. \contradiction

        \item[Case $e = \New{C}$:]
          By \eqref{eq:prog_tbase}, {\sc T-Let} and {\sc T-New}
          \begin{equation*}
            \TypeRel{\Gamma}{a}{\New{C}}{C}.
          \end{equation*}
          Since the program is well formed the class $C$ exists. 
          \begin{equation*}
            \FM = [f \mapsto \default(\tau): (\VarDecl{f}{\tau}) \in \fdecls(C)]
          \end{equation*}
          is therefore well defined. By {\sc E-New} we can therefore step to $S'
          = H', P'$ where
          \begin{equation*}
            \begin{gathered}
              P' = Q \cup \left\{\FS'|_a^\iota \right\} \andalso \FS' = \sframe{L',
              t'} \\
              L' = L[x \mapsto o] \andalso o \text{ fresh object reference } \\
              H' = H[o \mapsto \Obj{C, \FM}].
            \end{gathered}
          \end{equation*}
          \contradiction

        \item[Case $e = \NewCell$:]
          We can immediately step to $S' = H', P'$ where
          \begin{equation*}
            \begin{gathered}
              P' = Q \cup \left\{\FS'|_a^\iota \right\} \andalso \FS' = \sframe{L',
              t'} \\
              L' = L[x \mapsto o] \andalso o \text{ fresh object reference } \\
              H' = H[o \mapsto \Cell{\emptyset, \bot_{\LatVals}}]
            \end{gathered}
          \end{equation*}
          by rule {\sc E-NewCell}. \contradiction
      
        \item[Case $e = \Call{y}{m}{z}$:]
          By \eqref{eq:prog_tbase}, {\sc T-Let}, {\sc T-Call} and {\sc T-Var}
          \begin{equation*}
            \begin{gathered}
              \Gamma(y) = C \andalso \mtype(m, C) = \gamma \to \tau \\
              \Gamma(z) = \gamma' \andalso \gamma' \stof \gamma.
            \end{gathered}
          \end{equation*}
          By $H \vdash \Gamma;L$ 
          \begin{equation*}
            \typeOf(L(y), H) \stof C.
          \end{equation*}
          Then either $\typeOf(L(y), H) = C' \stof C$ or $\typeOf(L(y)) =
          \NullType$.

          If $\typeOf(L(y), H) = \NullType$ then
          \begin{equation*}
            L(y) = \NullVal
          \end{equation*}
          and we can apply rule {\sc E-NullCall} to step to \Error.
          \contradiction

          If $\typeOf(L(y), H) = C'$
          \begin{equation*}
            L(y) = o_y \andalso H(o_y) = \Obj{C', \FM}.
          \end{equation*}
          $C' \stof C$ and classes being well typed means that if $\mtype(m, C)$
          is defined then $\mbody(m, C')$ is defined aswell. Thus let
          \begin{equation*}
            \mbody(m, C) = w \to t''.
          \end{equation*}
          By rule {\sc E-Call} we can then step to $S' = H, P'$ where
          \begin{equation*}
            \begin{gathered}
              P' = Q \cup \left\{\FS'|_a^\iota \right\} \andalso \FS' = \xframe{L'',
              t''}^x \circ \sframe{L, t'} \\
              L_{\text{base}} =
              \begin{cases}
                \emptyset & \text{ if } a = \ocap \\
                L_0       & \text{ if } a = \nocap
              \end{cases} \\
              L'' = L_{\text{base}}[\This \mapsto L(y), w \mapsto L(z)].
            \end{gathered}
          \end{equation*}
          $L(z)$ is defined by $H \vdash \Gamma; L$. \contradiction

        \item[Case $e = \Put{y}{z}$:]
          By \eqref{eq:prog_tbase}, {\sc T-Let}, {\sc T-Put} and {\sc T-Var}
          \begin{equation*}
            \Gamma(y) = \CellType \andalso \Gamma(z) = \LatType.
          \end{equation*}
          Thus by $H \vdash \Gamma; L$
          \begin{equation*}
            \typeOf(L(y), H) \stof \CellType \andalso \typeOf(L(z), H) \stof
            \LatType.
          \end{equation*}
          By definition of $\typeOf$ and the structure of the type lattice this
          means that $\typeOf(L(z), H) = \LatType$ and that either
          $\typeOf(L(y), H) = \CellType$ or $\typeOf(L(y), H) = \NullType$. \\
          $\typeOf(L(z), H) = \LatType$ implies
          \begin{equation*}
            L(z) = l' \andalso l' \in \LatVals.
          \end{equation*}

          If $\typeOf(L(y), H) = \NullType$
          \begin{equation*}
            L(y) = \NullVal
          \end{equation*}
          and we can apply {\sc E-NullPut} to step to $\Error$. \contradiction

          If $\typeOf(L(y), H) = \CellType$ then 
          \begin{equation*}
            L(y) = o_y \andalso H(o_y) = \Cell{\DEP, l}
          \end{equation*}
          Then we can use rule {\sc E-Put} to step to $S' = H', P'$ where 
          \begin{equation*}
            \begin{gathered}
              P' = Q \cup \left\{\FS'|_a^\iota \right\} \andalso \FS' = \sframe{L',
              t'} \\
              L' = L[x \mapsto L(y)] \andalso c' = \Cell{\DEP, l \sqcup l'} \\
              H' = H[o_y \mapsto c']
            \end{gathered}
          \end{equation*}
          \contradiction

        \item[Case $e = \When{y}{z}{(\overline{cap}, w \Rightarrow t'')}$:]
          From \eqref{eq:prog_tbase}, {\sc T-Let}, {\sc T-When} and {\sc T-Var}
          \begin{equation*}
            \begin{gathered}
              \Gamma(y) = \CellType \andalso \Gamma(z) = \LatType \\
              \forall(\Capt{u}{u'}) \in \overline{cap}. \: \Gamma(u') = \CellType \\
              \Gamma_{\text{cells}} = [(u: \CellType) \mid (\Capt{u}{u'}) \in
              \overline{cap}] \\
              \TypeRel{\Gamma_{\text{cells}}, w: \LatType}{\ocap}{t}{\gamma}.
            \end{gathered}
          \end{equation*}
          Combinig the first two with $H \vdash \Gamma;L$
          \begin{gather*}
            \typeOf(L(y), H) \stof \CellType \\
            \typeOf(L(z), H) \stof \LatType.
          \end{gather*}
          Similarly to the previous case we have
          \begin{equation*}
            L(z) = l' \andalso l' \in \LatVals.
          \end{equation*}
          Also, either
          \begin{equation*}
            \typeOf(L(y), H) = \CellType
          \end{equation*}
          or
          \begin{equation*}
            \typeOf(L(z), H) = \NullType.
          \end{equation*}

          If $\typeOf(L(y), H) = \NullType$ then
          \begin{equation*}
            L(y) = \NullVal.
          \end{equation*}
          By {\sc E-NullWhen} we can step to $\Error$. \contradiction

          If $\typeOf(L(y), H) = \CellType$
          \begin{equation*}
            L(y) = o_y  \andalso H(o_y) = \Cell{\DEP, l}
          \end{equation*}
          By {\sc E-When} we can step to $S' = H', P'$ where
          \begin{equation*}
            \begin{gathered}
              P' = Q \cup \left\{\FS'|_a^\iota \right\} \andalso \FS' = \sframe{L',
              t'} \\
              L' = L[x \mapsto L(y)] \andalso L_{\text{env}} = [u \mapsto L(u')
              \mid
              (\Capt{u}{u'}) \in \overline{cap}] \\
              cb = ( L_{\text{env}}, w \Rightarrow t'' ) \andalso \DEP' = \DEP
              \cup (l', cb)^\iota \\
              \iota \text{ fresh thread id } \andalso H' = H[o_y \mapsto \Cell{\DEP',
              l}].
            \end{gathered}
          \end{equation*}
          $L(u')$ is defined for all $u'$s by $H \vdash \Gamma; L$. \contradiction
      \end{description}
      {\bf This concludes the case $t = \Let{x}{e}{t'}$.}
  \end{description}
  Clearly all cases for $t$ leads to a contradiction and therefore we are done.
\end{proof}




\chapter{Proof of Quasi Determinism}

\begin{definition} \label{def:pi}
  Let $S = H, P$ and let 
  \begin{equation*}
    g: \OIDs(S) \to \OIDs' \andalso h: \TIDs(S) \to \TIDs' 
  \end{equation*}
  be bijections where $\OIDs'$ ($\TIDs'$)is some subset of $\OIDs$ ($\TIDs$).
  Then 
  \begin{equation}
    \pi(H, g, h) = H^*
  \end{equation}
  where
  \begin{equation}
    \begin{gathered}
      H^*(o) =
      \begin{cases}
        \Obj{C, FM^*}   & \text{ if } H(g^{-1}(o)) = \Obj{C, FM} \\
        \Cell{DEP^*, l} & \text{ if } H(g^{-1}(o)) = \Cell{DEP, l}
      \end{cases} \\
      FM^* = \delta(FM, g) \andalso DEP^* = \eta(DEP, g, h)
    \end{gathered}
  \end{equation}
\end{definition}

\begin{notation}
  % TODO define notation thm env
  % Put the notation of partial map here
\end{notation}

\begin{definition} \label{def:rho}
  Let 
\end{definition}

\begin{proof}{(Proposition~\ref{prop:eqrel})} \\
  This is a proof
\end{proof}


\end{document}
