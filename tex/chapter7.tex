\chapter{Discussion \& Conclusion}
\label{cha:discussion_and_conclusion}

In this chapter, RACL, the formal system of this thesis is discussed from two
angles: extensions and implementation. Finally the report is summarized with
conclusions.
% EXTENSIONS:
% TODO discuss quiescence: allows us to use the result and freeze values
% TODO discuss allowing gets
% TODO 

% Could this be in conclusions?
% GENERAL:
% TODO great care has to be taken when designing and proving things about
% deterministic-by-design systems

% IMPLEMENTATION
% Threshold set is not a viable approach and only an abstraction. In reality you
% would have to do something similar to the implementation of LVish: Atomic
% lattices. Have some
% function that generates callbacks: I.e. for the new value, spawn some new
% callbackthreads
\section{A Remark About Determinism}%
\label{sec:a_remark_about_determinism}

In this report, the quasi-determinism of RACL is proved. However, it
should be possible to prove complete determinism, meaning that you get the same
halting result (also considering $\Error$ as a result) in all executions which
start in the same state. This is due to that, from the point of view of a
thread, all return values from cell operations are agnostic to the actual
cell value.  This implies that callback threads, with the current semantics are
in practice completely isolated and cannot share any information. This in turn
means that the execution of a thread is deterministic. By this, all callback
threads spawned in one execution will spawn in all possible executions, meaning
that if there is an error spawning thread in one execution path, there will be
in all other executions aswell.

\section{Possible Extensions}%
\label{sec:extensions}

Currently, RACL does not have any way to actually use the result of a
computation. That is, there is no way to access the final cell values. The
freeze operation of LVish~\parencite{kuper2014freeze} is flawed and allows us to
write non-deterministic programs, as shown in the end of
Section~\ref{sec:a_problem_of_lvish}. However, as hinted in the end of the same
section, there should be a way to freeze values and access them without breaking
determinism.  This uses a form of of quiescence, the concept introduced
by~\textcite{kuper2014freeze}. The difference is that this new form of freezing
needs to be done globally, and at a time where changes to cell values are no
longer going to take place. Quiescence allows us to do this. For example, if no
threads are running or waiting to spawn, we can be sure that the cell values
will not change. Therefore, with this definition of quiescence, introducing a
program statement which blocks until quiescence and then freezes all cells,
should allow us to access the result without risking non-determinism. In order
to ensure this does not cause deadlocks, this \emph{on-quiescence-freeze}
operation should only be allowed in the sole main (non-$\ocap$) thread of RACL.

As mentioned in Section~\ref{sec:lvars}, LVars has an additional \emph{get}
operation, not modeled in RACL. Adopting the semantics of LVars, it should be
possible to extend RACL with this operation aswell.

Apart from cell objects, it should also be possible to share immutable data
between threads. This would require objects being deeply immutable. Apart from
requiring the references passed to be immutable, an immutable object should only be able
to contain immutable references. This property must also be transitive,
requiring that all references held by referenced objects be immutable. A similar way
to define immutability for Scala was examined by~\textcite{HallerA17}.

\section{Implementation}%
\label{sec:implementation}

In order to implement RACL effectively, some changes have to be made in the way
callbacks are registered and spawned. This is mainly due to that the dependency
set sematics of our formal system are hard to implement effectively, since it
requires scanning all the dependency sets for threshold values that have been
passed. The time required for this operation is linear in the combined size of
the dependency sets.  

The LVish implementation solved this inefficiency problem by optimizing their
library for so called atomic lattices, i.e., lattices where all values can be
written as a join of so called \emph{atoms}. An atom $a$ is an element of the
lattice $\LatVals$ where $l \sqsubseteq a \implies l \in \{\bot, a\}$. For
example, the lattice of integer sets $(2^{\integers}, \subseteq)$ has the atoms
\begin{equation*}
  \{ i \} \andalso i \in \integers.
\end{equation*}
Working with atomic lattices allows for expressing an LVar update by a delta
value, i.e., a difference between the old and new value. Instead of using the
threshold set semantics, LVish registers callback generators, which for each
update to an LVar generates callbacks to be run. Levaraging the delta
representation, these generators can be made efficient in many
cases~\parencite{kuper2014freeze}. In a RACL implementation, these generators
would of course also have to be isolated similar to regular callbacks, in
order to guarantee determinism.


\section{Conclusion}%
\label{sec:conclusion}

With this thesis, the aim was to construct a similar formal system to LVish, but
with object oriented characteristics, mostly levaraging the work
of~\textcite{conf/oopsla/HallerL16}. Furthermore the aim was to prove
quasi-determinism similar to LVish. This goal has been met.  However, the
discovery of an error in the proof of~\textcite{kuper2014freezeTR} led to the
elimination of the freeze operation in its current form. This indicates two
things: Firstly, when formal systems get larger, proofs get longer and great
care has to be taken in order to ensure correctness. Secondly, the goal of
determinisic-by-design concurrent systems is difficult to understand, and
seemingly safe operations such as freezing could have unforseen consequences.

% TODO discussion: bring up fact of running callbacks twice does not
% affect result

% Discuss the addition of quiescence in order to actually use the result of the
% computation. Also discuss the efficiency of our system. In order to make it
% usable we need to replace the threshold checks with some callback that spawns
% the required callbacks. This of course is also required to not depend on any
% shared mutable state.
% Discuss threshold reads: Could introduce a get stmt. Draw some
% conclusions of the problematic nature of deterministic concurrency and that
% great care has to be taken when trying to prove determinism for so called
% "deterministic-by-design" systems.
% SHould also



