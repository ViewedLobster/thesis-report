\chapter{Quasi Determinism}
\label{cha:determinism}

% Describe:
% Theorem of determinism
% Lemmas leading up to the proof of this

\begin{definition}
  For each state $S = H, P$ there are two related sets of currently used object
  identifiers $\mathcal{O}(S)$ and thread identifiers $\mathcal{D}(S)$.
  Respectively these contain all object and thread identifiers occuring in $H$
  and $P$. 
  %We call an object
  %identifier $o$ (reference identifier $d$) \emph{fresh} if $o \not\in \mathcal{O}(S)$ ($d \not\in
  %\mathcal{D}(S)$).
\end{definition}

\begin{definition}
  We say that an object (thread) identifier $o$ ($d$) is \emph{fresh} at a state $S$ if
  $o \not\in \OIDs(S)$ ($d \not\in \TIDs(S)$).
\end{definition}

\begin{definition} \label{def:eqrel}
  Let $\simeq$ be a binary relation on the set of states $\States$.
  We let $S \simeq S'$ if
  \begin{equation*}
    S = S' = \Error
  \end{equation*}
  or if
  \begin{equation}
    S = H, P \andalso S' = H', P'
  \end{equation}
  and there exists bijections $g: \OIDs \to \OIDs, h: \TIDs \to \TIDs$
  such that
  \begin{equation}
    H' = \pi(H, g) \andalso P' = \rho(P, g, h)
  \end{equation}
  
  The functions $\pi$ and $\rho$ are defined in definition~\ref{def:pirho}.
\end{definition}

% TODO define \pi and \rho


\begin{proposition} \label{prop:eqrel}
  $\simeq$ is an equivalence relation.
\end{proposition}

\begin{proposition} \label{prop:eqrel_stateok}
  For any $S, S' \in \States$ such that $S \simeq S'$
  \begin{equation}
    \vdash S \tsep \stateok \iff \vdash S' \tsep \stateok
  \end{equation}
\end{proposition}


\begin{theorem}{(Quasi Determinism)}
  Let $S_0, S'_0, S_1, S'_1$ be states not equal to $\Error$ such that
  \begin{equation*}
    \begin{gathered}
      S_0 \simeq S'_0 \\
      S_0 \Rrightarrow^{\bar{R}} S_1 \andalso S'_0 \Rrightarrow^{\bar{R'}} S'_1.
    \end{gathered}
  \end{equation*}
  Furthermore neither $S_1$ or $S'_1$ can make a step.
  Then
  \begin{equation*}
    S_1 \simeq S'_1.
  \end{equation*}
\end{theorem}

