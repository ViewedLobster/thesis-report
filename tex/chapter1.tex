\chapter{Introduction \& Motivation}
\label{cha:introduction}

The main approach to achieving concurrency in general applications has long been
to use locks and similar constructs in order to locally synchronize data
accesses. The major drawback with this is that it demands a lot of effort from
the programmer in order to achieve good parallelization, even of simple tasks.
Furthermore the likelihood of concurrency related errors such as deadlocks, data
races and livelocks naturally becomes higher. A simple example of a data race
is illustrated in Figure~\ref{fig:datarace} where \ilc{thread1} and
\ilc{thread2} are two theads with access to a common variable \ilc{x}.
Depending on the execution order, \ilc{x} will have a different final value.

% TODO give some example of a dangerous situation using pseudo code
\begin{figure}
  \centering
  \begin{subfigure}[t]{0.4\textwidth}
    \begin{lstlisting}[numbers=none,mathescape=true]
var x: Int = 0

thread1 = () => {
  /* doing cool stuff */
  ...

  x = 1
  
  /* doing more cool stuff */
  ...
}

thread2 = () => {
  /* doing wicked stuff */
  ...

  x = 2
  
  /* doing more wicked stuff */
  ...
}
    \end{lstlisting}
    \caption{Example of a data race.}
    \label{fig:datarace}
  \end{subfigure}
  \quad\quad\quad\quad
  \begin{subfigure}[t]{0.4\textwidth}
    \begin{lstlisting}[numbers=none,mathescape=true]
var x: Int = 0

thread1 = () => {
  /* doing cool stuff */
  ...

  maxWrite(x,1)
  
  /* doing more cool stuff */
  ...
}

thread2 = () => {
  /* doing wicked stuff */
  ...

  maxWrite(x,2)
  
  /* doing more wicked stuff */
  ...
}
    \end{lstlisting}
    \caption{The deterministic program.}
    \label{fig:maxwrite}
  \end{subfigure}
  \caption{Program examples.}
\end{figure}


Concurrent deterministic programs are concurrent programs which always produce
the same results for the same inputs. This is attractive since it provides
reproducible computations.  Deterministic-by-design concurrent programming
models are systems which guarantees this property at compile time and thus
significantly simplifies concurrent programming. This greatly eases effort
from the programmer. As an example, if in a similar setting to the program in
Figure~\ref{fig:datarace}, say you where only interested in the maximum value of
\ilc{x}. You could make the program deterministic by introducing a special
\ilc{maxWrite(x,v)} operation. \ilc{maxWrite} takes a variable \ilc{x} and a
value \ilc{v} and sets the new value of \ilc{x} to the maximum of its current
value and \ilc{v}. The modified program can be seen in
Figure~\ref{fig:maxwrite}. It is not hard to see that this is deterministic.
Now, by somehow statically enforcing that all writes to shared variables should be
through this special write operation, we could at compile time guaranee that a
program has deterministic results.

\begin{figure}
  \centering
\end{figure}
% TODO give an example using some reactive async-like pseudo code

A novel deterministic-by-design model call LVish~\parencite{kuper2013lvars,
kuper2014freeze} has been introduced. This generalizes the \ilc{maxWrite} from
above. It uses data types that fulfill the condition of being part of a lattice,
More explicitly this means the shared data has some partial order defined on its
values, and a so called join operation (a generalized maximum operation) defined
for all value pairs. By only allowing writes to the data in form of this join
operation, you can achieve concurrent determinism with very high parallelism
using event based computation. This is also the approach of the Reactive Async
project~\parencite{conf/scala/HallerGES16}, a Scala based framework inspired by
LVish. Both LVish and Reactive Async has shown potential to speed up
computations for algorithms concerning graphs and static code
analysis~\parencite{kuper2014freeze,conf/scala/HallerGES16}.

The main problem with Reactive Async as of now, is that it does not include any
construct to ensure that data races between event handlers do not occur.
There are also more fundamental problems with callback spawning,
which could also lead to non-determinism. Furthermore, there are problems with
LVish and its proof of determinism. Therefore, the main concern of this thesis
is to analyze and find a solution to these problems.

% TODO expand upon the applications of lvars and reactive async

\section{Research Question \& Goal}%
\label{sec:goal}


This thesis aims to construct a formal model which could later be adopted by
future revisions of Reactive Async. Constructing a model similar to that
of LVish but in an object-oriented setting, the final goal
will be to mathematically prove a form of determinism for the system.

Doing this will hopefully give us an answer to the research question:

\begin{quotation}
  How can we enforce a deterministic concurrent model similar to LVish in an
  object-oriented setting?
\end{quotation}

\section{Methodology}%
\label{sec:methodology}

We use the theory of programming language semantics and type systems to build a
simple object-oriented language. The type system incorporates theory that
relates to the object capability model from computer security, also utilized in
previous work by \textcite{conf/oopsla/HallerL16}.  After defining the property
of a well-typed program state, we prove that this is preserved during execution.
Furthermore, we prove that an execution of a well-typed program will not get
stuck unless there is a null-pointer exception. Finally, we prove a form of
determinism for the system.

\section{Contributions}%
\label{sec:contributions}

With this work, the following technical contributions are made:
\begin{itemize}
  \item Identification and exemplification of  problems with previous work on
    deterministic concurrency.
  \item Formalization of a core object-oriented language for deterministic
    concurrency.
  \item Design and formalization of an accompanying type system.
  \item A proof of soundness for the core language and type system.
  \item A proof of quasi-determinism, a form of determinism, for our
    formalized system.
\end{itemize}

\section{Report Structure}%
\label{sec:report_structure}

In Chapter~\ref{cha:background} some technical background is described. This
includes some mathematical definitions, an introduction to programming language
formalization and an informal description of the object capability
model. Chapter~\ref{cha:related_work} describes some selected related work, upon
which our model will build. In Chapter~\ref{cha:challenges} we describe and
exemplify some problems with some existing deterministic-by-design concurrent
systems. In Chapter~\ref{cha:core_language} we introduce our core language and
formalization. Chapter~\ref{cha:properties_of_racl} describes the theorems and
proofs of soundness and quasi-determinism. Finally, in
Chapter~\ref{cha:discussion_and_conclusion} we discuss extensions and
implementation of our formal system as well as ethical aspects and
sustainability. We finally include a small conclusion.



