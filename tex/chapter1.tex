\chapter{Introduction}
\label{cha:introduction}

The main approach to achieving concurrency in general applications has long been
to use locks and similar constructs in order to locally synchronize data
accesses. The major drawback with this is that it demands a lot of effort from
the programmer in order to achieve good parallelization, even of simple tasks.
Furthermore the likelihood of concurrency related errors such as deadlocks, data
races and livelocks naturally becomes higher. 

Concurrent deterministic programs are concurrent programs which always produce
the same results for the same inputs. This is attractive since it provides
reproducible computations.  Deterministic-by-design concurrent programming
models are systems which guarantees this property at compile time and thus
significantly simplifies concurrent programming. This greatly eases effort
from the programmer. 

A novel deterministic-by-design model~\parencite{kuper2013lvars,
kuper2014freeze} has been brought forward.  They use data types that fulfill the
condition of being part of a lattice, or more explicitly having some partial
order defined on its values. By defining a join operation on the values and only
allowing writes to the data in form of this operation you can achieve concurrent
determinism with very high parallelism using event based computation.  This is
also the basic approach of the Reactive Async
project~\parencite{conf/scala/HallerGES16} which has shown potential to speed up
e.g.\ static analysis of source code. The main problem with this as of now is
that it does not include any construct to ensure that data races between event
handlers do not occur. Futhermore there are also more fundamental problems with
callback spawning which could also lead to non-determinism.

This thesis aims to construct a formal model which could later be adopted by
future revisions of Reactive Async. Constructing a model similar to that
of~\parencite{kuper2014freeze}, using concepts from computer security also
utilized in previous work by~\parencite{conf/oopsla/HallerL16}, the final goal
will be to prove a form of determinism for the system.
