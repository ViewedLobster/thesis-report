\chapter{Proof of Quasi Determinism}

\begin{definition} \label{def:pi}
  Let $S = H, P$ and let 
  \begin{equation*}
    g: \OIDs(S) \to \OIDs' \andalso h: \TIDs(S) \to \TIDs' 
  \end{equation*}
  be bijections where $\OIDs'$ ($\TIDs'$)is some subset of $\OIDs$ ($\TIDs$).
  Then 
  \begin{equation}
    \pi(H, g, h) = H^*
  \end{equation}
  where
  \begin{equation}
    \begin{gathered}
      H^*(o) =
      \begin{cases}
        \Obj{C, FM^*}   & \text{ if } H(g^{-1}(o)) = \Obj{C, FM} \\
        \Cell{DEP^*, l} & \text{ if } H(g^{-1}(o)) = \Cell{DEP, l}
      \end{cases} \\
      FM^* = \delta(FM, g) \andalso DEP^* = \eta(DEP, g, h)
    \end{gathered}
  \end{equation}
\end{definition}

\begin{notation}
  % TODO define notation thm env
  % Put the notation of partial map here
\end{notation}

\begin{definition} \label{def:rho}
  Let 
\end{definition}

\begin{proof}{(Proposition~\ref{prop:eqrel})} \\
  This is a proof
\end{proof}
