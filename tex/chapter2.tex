\chapter{Background \& Preliminaries} \label{cha:background}

This chapter is an introduction to some mathematical concepts aswell as some
language and type system theory.  In Section~\ref{sec:mathematical_definitions}
we will initially give some mathematical definitions. In
Section~\ref{sec:language_syntax} some basics of language syntax will be
explained. Section~\ref{sec:language_semantics} gives a brief introduction to
programming language semantics. Section~\ref{sec:type_systems} explains
some basics of programming language type systems. Finally
Section~\ref{sec:the_object_capability_model} explains the object capability
security model.

\section{Mathematical Preliminaries}
\label{sec:mathematical_definitions}

\begin{definition}
  Let $(V, \sqsubseteq)$ be a partial order. Then the \emph{least upper bound}
  of two elements $v, v' \in V$ is a an element $u \in V$ such that
  \begin{equation*}
    v \sqsubseteq u \andalso v' \sqsubseteq u
  \end{equation*}
  and for all $v'' \in V$ such that $v \sqsubseteq v'', v' \sqsubseteq v''$
  \begin{equation*}
    u \sqsubseteq v''
  \end{equation*}
\end{definition}
It is easy to show that if a least upper bound exists, it is unique.
\begin{definition}
  A \emph{join-semilattice} is a partially ordered set $(\LatVals, \sqsubseteq)$ such
  that for any two $l, l' \in \LatVals$ there is a least upper bound $u \in
  \LatVals$. $u$ is also called the \emph{join} of $l$ and $l'$ and we can write
  this as
  \begin{equation*}
    u = l \sqcup l'
  \end{equation*}
\end{definition}
From now on we will refer to a join-semilattice simply as a \emph{lattice}.
\begin{definition}
  A \emph{partial function} or \emph{partial map} $f$ from $A$ to $B$, written
  \begin{equation*}
    f\in A \rightharpoonup B
  \end{equation*}
  is a function defined on some subset of $A$, i.e.
  \begin{equation*}
    f \in A' \to B \andalso A' \subseteq A.
  \end{equation*}
\end{definition}

\begin{notation}
  If a map $g\in A \to B$ is a bijection we write
  \begin{equation*}
    g \in A \hookrightarrow B.
  \end{equation*}
\end{notation}


\section{Language Syntax} \label{sec:language_syntax}

% Describe the foundations of syntaxes for languages: BNF, ANF
A language syntax or grammar is a formal way to describe the syntactical
structure a programming language. It gives us a basis on which to build the
language semantics and type systems as will be clear later. This section will
give a brief introduction to the Backus-Naur-form (BNF) and exemplify this with
a simple programming language.

\subsection{The Backus-Naur-form} \label{sub:the_backus_naur_form} 

Backus-Naur-form (BNF) is a very common way to describe the grammar of
programming languages. BNF describes a grammar using rules of the form
\begin{equation*}
  s ::= \text{<expression>}.
\end{equation*}
The left side symbol $s$ is refered to as a non-terminal. <expression> could
also be a list of the
possible forms of $s$, separated by $|$, i.e. 
\begin{equation*}
  s ::= \text{<expression1>} \: | \dots | \: \text{<expressionN>}
\end{equation*}
The expressions can themselves contain non-terminals and terminals.
Terminals are symbols which does not occur on the left side of any rule.

As an example, the While language can be written in BNF as described
by \textcite{nielson2007semantics}. For reference we give a similar description
here, with the addition of a very simple type system which will be explained in
Section~\ref{sec:type_systems}. The grammar is presented in
Figure~\ref{fig:while_grammar}.

\begin{figure}[]
  \centering
  $\begin{array}{lll@{\hspace{4mm}}l}
    c &::= &  & \mbox{Program code} \\
    &  & \varepsilon & \mbox{Empty code} \\
    &| & s; c & \mbox{Statement concatenation} \\
    s &::= & & \mbox{Program statement} \\
    &  & \WhDecl{x}{\tau} & \mbox{Variable declaration} \\
    &| & \WhAssign{x}{e} & \mbox{Assignment} \\
    &| & \WhSkip & \mbox{Skip} \\
    &| & \WhIf{e}{c}{c} & \mbox{If branch} \\
    &| & \WhWhile{e}{c} & \mbox{While loop} \\
    &&&\\
    e & ::= & & \mbox{Expressions} \\
    & & x & \mbox{Variable} \\
    &| & v & \mbox{Value literal} \\
    &| & \WhAdd{e}{e} & \mbox{Addition} \\
    &| & \WhTimes{e}{e} & \mbox{Multiplication} \\
    &| & \WhMinus{e}{e} & \mbox{Subtraction} \\
    &| & \WhEq{e}{e} & \mbox{Equality comparison} \\
    &| & \WhLeq{e}{e} & \mbox{$\leq$ comparison} \\
    &| & \WhLnot{e} & \mbox{Logical not} \\
    &| & \WhLand{e}{e} & \mbox{Logical and} \\
    &&&\\
    \tau & ::= & & \mbox{Types} \\
    & & \WhBool & \mbox{Boolean type} \\
    &| & \WhInt & \mbox{Integer type} \\
    &&&\\
    v & ::= & & \mbox{Values} \\
    &  & n & \mbox{Integer} \\
    &| & \WhTrue & \mbox{Boolean true} \\
    &| & \WhFalse & \mbox{Boolean false} \\
  \end{array}$
  \caption{Grammar of While}
  \label{fig:while_grammar}
\end{figure}

\section{Small-step Operational Semantics} \label{sec:language_semantics}

% Describe small step operational semantics and give example with while lang
In order to prove properties of a program you must be able to formally describe
the "meaning" of the program, i.e., to describe how programs statements modify
state and what execution steps can be taken. A common approach to this is
small-step operational semantics, also called structural operational semantics
(SOS). Other approaches includes big-step operational semantics or natural
semantics, and denotational semantics. The presentation here will focus on SOS
since this can model concurrency well. 

Small-step operational semantics defines rules which can be used to derive
single execution steps (or transitions) between program states. We exemplify
this with the While language defined in the above section. 
We let $\WhVar$ be the set of all variable names and $\WhVal$ the
set of all values, which for the While language will be the integers $\integers$,
and the boolean values $\WhTrue$ and $\WhFalse$,
\begin{equation*}
  \WhVal = \integers \cup \left\{ \WhTrue, \WhFalse \right\}.
\end{equation*}
\begin{remark}
  Note that in~\parencite{nielson2007semantics} the conversion between the
  syntactical (numerals\slash boolean literals) and semantic (integers\slash
  boolean values) versions of values are explicit whereas here this conversion
  is implicit.
\end{remark}

In order to define rules for state transitions a state must be defined.
Therefore we make the following definition for the While language.

\begin{definition}
  A \emph{state} $S$ for the While language is a value of the form
  \begin{equation*}
    \WhState{V, c}  
  \end{equation*}
  where $c$ is program code and $V$ is a partial map
  \begin{equation*}
    V: \WhVar \rightharpoonup \WhVal
  \end{equation*}
\end{definition}

We can now define rules for state transitions.
All rules are of the form
\infrule[Rule name]
{\text{preconditions}}
{S \rightarrow S'}
This basically says {\it if the preconditions holds then we can step from $S$ to
$S'$}. If there are no preconditions we just write
\infax[Rule name]
{S \rightarrow S'}

All rules for state transistions are defined in Figure~\ref{fig:while_sos} and
most of them should be intuitive. For example, rules {\sc WhE-Decl1} and {\sc
WhE-Decl2} extends the state map $V$ with $x$ and a default value corresponding
to the declared type. Rule {\sc WhE-Assign} evaluates the expression $e$ and
maps $x$ to the result. Rules {\sc WhE-IfTrue} and {\sc WhE-IfFalse} evaluates
its expression $e$ and if the result is $\WhTrue$ or $\WhFalse$, the code of the
corresponding branch is prepended to the rest of the program. We can also see
that {\sc WhE-While} is just an expansion of a while loop into an if statement.
Finally we note that we can not step from $\WhState{V, \varepsilon}$ which means
that this is a halting state.

\begin{figure}[]
  \centering
  \infax[WhE-Decl1]
  {\WhState{V, \WhDecl{x}{\WhInt};c} \: \rightarrow \: \WhState{V[ x \mapsto 0],
  c}}
  
  \RuleSpace

  \infax[WhE-Decl2]
  {\WhState{V, \WhDecl{x}{\WhBool};c} \: \rightarrow \: \WhState{V[ x \mapsto
  \WhFalse], c}}
  
  \RuleSpace

  \infax[WhE-Assign]
  {\WhState{V, \WhAssign{x}{e};c} \: \rightarrow \: \WhState{V[ x \mapsto
  \WhEval{V}(e)], c}}

  \RuleSpace

  \infax[WhE-Skip]
  {\WhState{V, \WhSkip; c} \: \rightarrow \: \WhState{V, c}}

  \RuleSpace

  \infrule[WhE-IfTrue]
  {\WhEval{V}(e) = \WhTrue}
  {\WhState{V, \WhIf{e}{c_1}{c_2};c} \: \rightarrow \: \WhState{V, c_1;c}}

  \RuleSpace 

  \infrule[WhE-IfFalse]
  {\WhEval{V}(e) = \WhTrue}
  {\WhState{V, \WhIf{e}{c_1}{c_2}} \: \rightarrow \: \WhState{V, c_2;c}}
  
  \RuleSpace

  \infax[WhE-While]
  {\WhState{V, \WhWhile{e}{c'};c} \: \rightarrow \: \WhState{V,
  \WhIf{e}{(\WhConc{c'}{\WhWhile{e}{c'}})}{\WhSkip};c}}

  \caption{Small-step operational semantics for While}
  \label{fig:while_sos}
\end{figure}

The SOS rules of While heavily relies on the expression \emph{evaluation
function} $\WhEval{\cdot}(\cdot)$. This is a function which takes a state map
$V$ and an expression $e$ and returns a value in $\WhVal$. We define this
recursively as follows
\begin{equation*}
  \begin{array}{lll}
    \WhEval{V}(v) &=& v \\
    \WhEval{V}(x) &=& V(x) \\
    \WhEval{V}(e_1 + e_2) &=& \WhEval{V}(e_1) + \WhEval{V}(e_2) \\
    \WhEval{V}(e_1 * e_2) &=& \WhEval{V}(e_1) * \WhEval{V}(e_2) \\
    \WhEval{V}(e_1 - e_2) &=& \WhEval{V}(e_1) - \WhEval{V}(e_2) \\
    \WhEval{V}(e_1 = e_2) &=& \WhEval{V}(e_1) = \WhEval{V}(e_2) \\
    \WhEval{V}(e_1 \leq e_2) &=& \WhEval{V}(e_1) \leq \WhEval{V}(e_2) \\
    \WhEval{V}(e_1 \land e_2) &=& \WhEval{V}(e_1) \land \WhEval{V}(e_2) \\
    \WhEval{V}(\lnot e_1) &=& \lnot \WhEval{V}(e_1) \\
  \end{array}
\end{equation*}
Here we implicitly rely on that the operations $+, -, *, =, \leq$ are only
defined for integer values, and that $\lnot, \land$ are only defined for boolean
values.  If we encounter a state map $V$ and an expression $e$ where the
corresponding integer or boolean operation above is undefined, the value for
$\WhEval{V}(e)$ is undefined. Since the definition of $\WhEval{\cdot}(\cdot)$ is
recursive, the occurrence of an undefined value should also propagate upwards.
For example, if $\WhEval{V}(e_1)$ is undefined then $\WhEval{V}(e_1 + e_2)$ is undefined
aswell. This is implicit in our definition.

The evaluation function $\WhEval{\cdot}(\cdot)$ together with the derivation
rules in Figure~\ref{fig:while_sos} gives us a complete description of what
steps are allowed. It is implicit in the description that if the evaluation
function is undefined at some point in the derivation of a step, execution
cannot progress and we get stuck.

\section{Type Systems} \label{sec:type_systems}

A type system is a mathematical construct that consists of elements called types
and a set of rules that assign types to parts of a programming language, e.g.,
statements, variables and expressions. Type systems are most commonly used to
prevent programming errors such as undefined data operations, e.g., feeding a
data structure to a function for which it was not made. This can be enforced
both using static checking at compile time or dynamic checking at runtime. Type
systems are found in many modern programming languages such as Java, Scala,
Haskell, Python and C++.

The focus here will be on static type systems. We will explain the basics using
our previous example of the While language. Then some uses and extensions will be
discussed.

\subsection{A Type System for While}
\label{sub:a_type_system_for_while}

For the While language we have the following types:
\begin{equation*}
  \WhInt \andalso \WhBool \andalso \WhVoid
\end{equation*}
We call the set of all types 
\begin{equation*}
  \WhTpe = \left\{ \WhInt, \WhBool, \WhVoid \right\}.
\end{equation*}
Our typing rules will define a relation of the form
\begin{equation} \label{eq:tpe_sys1}
  \Gamma \vdash r : \tau.
\end{equation}
Here $\Gamma$ is a \emph{typing environment}, i.e. a partial map
\begin{equation*}
  \Gamma: \WhVar \rightharpoonup \WhTpe.
\end{equation*}
$r$ is either code or an expression and $\tau \in \WhTpe$.
Equation \eqref{eq:tpe_sys1} basically reads: {\it $r$ is typeable as $\tau$ under the
typing environment $\Gamma$}. A program $c$ is \emph{well formed} if
\begin{equation*}
  \emptyset \vdash c : \WhVoid
\end{equation*}
where $\emptyset$ is the empty typing environment, i.e. $\emptyset(x)$ is
undefined for all $x \in \WhVar$.

\begin{notation}
  Sometimes you would like to remap a key of, or extend a partial map $M$. One
  common notation
  for this is 
  \begin{equation*}
    M[k \mapsto v],
  \end{equation*}
  which means that $M(l) = M[k \mapsto v](l)$ for all $l \neq k$ and
  $M(k) = v$.
  For typing environments we will also use the notation
  \begin{equation*}
    \Gamma, x: \tau
  \end{equation*}
  which is equivalent to $\Gamma[x \mapsto \tau]$. We will also see the notation
  \begin{equation*}
    (x: \tau) \in \Gamma,
  \end{equation*}
  which is equivalent to $\Gamma(x) = \tau$.
\end{notation}

Typing rules for program code are defined in Figure~\ref{fig:while_code_tpe}.
Generally While program code can only be typed as $\WhVoid$. Intuitively this is
because the execution of program code does not have a resulting value. Instead
it can only modify the state map $V$, according to the semantics of the previous
section. The rules for typing program code should not be hard to understand. For
example, {\sc WhT-Assign} states that for code that starts with an assignment
the expression in the assignment must be of the same type as the variable. {\sc
WhT-If} requires that the expression is typeable as $\WhBool$ and that both
branches should be typeable as $\WhVoid$.

\begin{figure}[]
  \infax[WhT-Empty]
  {\Gamma \vdash \varepsilon : \WhVoid}

  \RuleSpace

  \infrule[WhT-Decl]
  {\Gamma(x) \text{ undefined} \andalso \Gamma, x: \tau \vdash c : \WhVoid}
  {\Gamma \vdash \WhDecl{x}{\tau}; c : \WhVoid}

  \RuleSpace 

  \infrule[WhT-Assign]
  {(x: \tau) \in \Gamma \andalso \Gamma \vdash e: \tau \andalso \Gamma \vdash c
  : \WhVoid}
  {\Gamma \vdash \WhAssign{x}{e}; c : \WhVoid}

  \RuleSpace

  \infrule[WhT-Skip]
  {\Gamma \vdash c : \WhVoid}
  {\Gamma \vdash \WhSkip; c : \WhVoid}

  \RuleSpace

  \infrule[WhT-If]
  {\Gamma \vdash e : \WhBool \andalso \Gamma \vdash c_1: \WhVoid \\
  \Gamma \vdash c_2 : \WhVoid \andalso \Gamma \vdash c: \WhVoid}
  {\Gamma \vdash \WhIf{e}{c_1}{c_2}; c : \WhVoid}

  \RuleSpace

  \infrule[WhT-While]
  {\Gamma \vdash e: \WhBool \andalso \Gamma \vdash c' : \WhVoid \andalso \Gamma
  \vdash c : \WhVoid}
  {\Gamma \vdash \WhWhile{e}{c'}; c : \WhVoid}

  \caption{While program code typing rules.}
  \label{fig:while_code_tpe}
\end{figure}

Finally the rules for typing expressions are found in
Figure~\ref{fig:while_expr_tpe}. The possible types for an expression are
$\WhInt$ and $\WhBool$. The rules themselves should be quite self-explanatory.

\begin{figure}[]
  \begin{multicols}{3}
  \infax[WhT-Num]
  {\Gamma \vdash n : \WhInt}

  \infax[WhT-True]
  {\Gamma \vdash \WhTrue: \WhBool}

  \infax[WhT-False]
  {\Gamma \vdash \WhFalse: \WhBool}
  \end{multicols}

  \RuleSpace

  \infrule[WhT-Var]
  {(x: \tau) \in \Gamma}
  {\Gamma \vdash x: \tau}

  \RuleSpace

  \infrule[WhT-Add]
  {\Gamma \vdash e_1: \WhInt \andalso \Gamma \vdash e_2: \WhInt}
  {\Gamma \vdash e_1 + e_2: \WhInt}

  \RuleSpace

  \infrule[WhT-Times]
  {\Gamma \vdash e_1: \WhInt \andalso \Gamma \vdash e_2: \WhInt}
  {\Gamma \vdash e_1 * e_2: \WhInt}

  \RuleSpace

  \infrule[WhT-Minus]
  {\Gamma \vdash e_1: \WhInt \andalso \Gamma \vdash e_2: \WhInt}
  {\Gamma \vdash e_1 - e_2: \WhInt}

  \RuleSpace 

  \infrule[WhT-Eq]
  {\Gamma \vdash e_1: \WhInt \andalso \Gamma \vdash e_2: \WhInt}
  {\Gamma \vdash e_1 = e_2: \WhBool}

  \RuleSpace 

  \infrule[WhT-Leq]
  {\Gamma \vdash e_1: \WhInt \andalso \Gamma \vdash e_2: \WhInt}
  {\Gamma \vdash e_1 \leq e_2: \WhBool}

  \RuleSpace

  \infrule[WhT-Not]
  {\Gamma \vdash e : \WhBool}
  {\Gamma \vdash \lnot e: \WhBool}

  \RuleSpace 

  \infrule[WhT-And]
  {\Gamma \vdash e_1: \WhBool \andalso \Gamma \vdash e_2: \WhBool}
  {\Gamma \vdash e_1 \land e_2: \WhBool}

  \caption{While expression typing rules.}
  \label{fig:while_expr_tpe}
\end{figure}


\subsection{Uses \& Extensions}
\label{sub:uses_and_extensions}

So what are type systems used for? As mentioned earlier, they are commonly put
in place to prevent errors such as undefined data operations. For example, in
Java you cannot feed a List to a method which is denoted to take an Integer as
an argument. This will result in a compile time error.

\subsection{Preservation \& Progress}
\label{sub:preservation_&_progress}

A formal type system can furthermore be used to prove properties like
\emph{preservation} and \emph{progress} for programs which are properly type
checked. These are important properties since they can beforehand ensure that
programs terminate properly, or at least that, e.g., if it terminates erroneously it
must have been the result of a null-pointer exception. Progress and preservation
together is often refered to as \emph{soundness} of a type system and to prove this
is a standard approach. For a more in depth explanation
see~\parencite{pierce2002types}.

As an example we can state preservation and progress properties of While as follows. Let
$\Gamma(V)$ be defined as 
\[
  \Gamma(V)(x) = \begin{cases}
    \WhInt & \text{ if } V(x) \in \integers \\
    \WhBool & \text{ if } V(x) \in \left\{\WhTrue, \WhFalse \right\} \\
    \text{undefined} & \text{ otherwise }
  \end{cases}
\]
\begin{proposition}[Preservation of While] 
  Let
  \begin{equation*}
    \WhState{V, c} \: \rightarrow \: \WhState{V', c'} \andalso \Gamma(V) \vdash
    c: \WhVoid.
  \end{equation*}
  Then 
  \begin{equation*}
    \Gamma(V') \vdash c': \WhVoid
  \end{equation*}
\end{proposition}
\begin{proposition}[Progress of While]
  Let $\WhState{V, c}$ be a state and let
  \begin{equation*}
    \Gamma(V) \vdash c: \WhVoid.
  \end{equation*}
  Then either $c = \varepsilon$ or there is a state $\WhState{V', c'}$ such that
  \begin{equation*}
    \WhState{V, c} \: \rightarrow \: \WhState{V', c'}.
  \end{equation*}
\end{proposition}


\subsection{Extensions}
\label{sub:extensions}

The example type system for While is of course very simple. This is mostly due
to the simplicity of the language itself. More complicated languages have more
complicated type systems and a type system is oftenmost designed together 
with the language itself. 

One common construct in type systems is \emph{subtyping}. This introduces a a
lattice based on a partial order relation between types called the subtyping
relation, denoted $\stof$.  It is oftenmost used to say that \emph{if $\sigma \stof
\tau$ then values of type $\sigma$ can be used in the same way as values of type
$\tau$}. This is the case for Java where a class $C$ can extend a class $D$.
This, simply stated means that the fields and methods of $D$ are also availible
in $C$. We will see an example of how subtyping can be used in
chapter~\ref{cha:core_language}. 

In chapter~\ref{cha:core_language} the typing relation will also include an
effect $a$. In short this means that the typing rules are more or less strict
depending on the value of $a$.

\section{The Object Capability Model}%
\label{sec:the_object_capability_model}

The object capability (OCAP) model is a computer security model that was
introduced in 1966 by \textcite{Dennis-VanHorn66}. In this model, the access
rights to an object are equivalent to a reference to the object itself. What
this means is that if we somehow accuire a reference to the object, we are free
to use it as we see fit. While this might seem like a severe loosening of
security, implementing access restrictions similar to other security models can
be done using proxy-like objects~\parencite{Miller06b}.
% TODO find better phrasing of loosening of security

In the OCAP model we formalize the concept of an object. This could be of
two types: Either a primitive object or an instance. A primitive object is
for example data such as a character string or number. In a setting where
communication with some outside world is modeled this could also be for example
a device. An instance contains state and code. For example, a class
object in an object oriented language could be modeled as an
instance.

Instances can only interact by sending messages over references. This implies
that for an instance to interact with another one, it needs a reference to the
other. Continuing the example with the class object above, a message corresponds
to a method call. In essence, we send a command (specify the method), together
with some information (the method parameters). The method can only be called if
we have a reference to the corresponding class object beforehand.

We can model references and objects using a directed graph. If object $A$ has
has a reference to object $B$, then edge $(A, B)$ is part of the graph.
In the OCAP model, references can only be obtained in four ways:
\begin{enumerate}
  \item Initial connectivity: The system must have some initial references in
    the connectivity graph.
  \item Parenthood: If some instance A creates an object B, at creation it has
    a reference to B.
  \item Endowment: If some instance A has access to B and creates C with a
    reference to B, then C has a reference to B.
  \item Introduction: If instance A sends a message to (calls a method on)
    instance B, B gets all references A sends in the message (gives as
    parameters).
\end{enumerate}
 The above four rules imply that two disjoint components in a reference graph
 can never become connected~\parencite{Miller06b}.

A language implementing the OCAP model is called an OCAP language. Most common
object oriented languages are not OCAP. This is partly because they allow global
references, i.e., objects referable from all instances.
