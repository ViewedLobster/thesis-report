\chapter{Background} \label{cha:background}

This chapter is a brief introduction to language and type system theory. In
section~\ref{sec:language_syntax} some basics of language syntax will be
explained. Section~\ref{sec:language_semantics} gives a brief introduction to
programming language semantics. Finally section~\ref{sec:type_systems} explains
some basics of programming language type systems.

% TODO give a reference
As a basis the well known While language will be used as an example language.

\section{Language Syntax} \label{sec:language_syntax}

% Describe the foundations of syntaxes for languages: BNF, ANF
A language syntax is a formal way to describe the structure of programs written
in a programming language. It gives us a basis on which to build the language
semantics and type systems as will be clear later. In this section the
Backus-Naur-form (BNF) will be introduced. 

\subsection{The Backus-Naur-form} \label{sub:the_backus_naur_form} 

Backus-Naur-form (BNF) is a very common way to describe the syntax of
programming languages. BNF describes a syntax using rules of the form
\begin{equation*}
  s ::= \text{<expression>}.
\end{equation*}
The the left side symbol $s$ is refered to as a non-terminal. <expression> could
also be a list of the
possible forms of $s$, separated by $|$, i.e. 
\begin{equation*}
  s ::= \text{<expression1>} \: | \dots | \: \text{<expressionN>}
\end{equation*}
The expressions can themselves contain non-terminals and terminals.
Terminals are symbols which does not occur on the left side of any rule.

% TODO add reference to Semantics with Applications An Appetizer
The While language can be written in BNF as described in~\cite{} % FIXME reference
For reference we give a similar description here with the addition of a very
simple type system.

% TODO add reference to Semantics with Applications An Appetizer
Furthermore we let $\WhVar$ be the set of all variable names and $\WhVal$ the
set of all values which for the While language will be the integers $\integers$,
and the boolean values $\WhTrue$ and $\WhFalse$. 
\begin{equation*}
  \WhVal = \integers \cup \left\{ \WhTrue, \WhFalse \right\}
\end{equation*}
Note that in~\cite{} % FIXME reference
the conversion between the syntactical (numerals\slash boolean literals) and semantic
(integers\slash boolean values) versions of values are explicit whereas here we will
make this conversion implicit.

\begin{figure}[h]
  \centering
  $\begin{array}{lll@{\hspace{4mm}}l}
    s &::= & & \mbox{Program statement} \\
    &  & \WhDecl{x}{\tau} & \mbox{Variable declaration} \\
    &| & \WhAssign{x}{e} & \mbox{Assignment} \\
    &| & \WhSkip & \mbox{Skip} \\
    &| & \WhConc{s_1}{s_2} & \mbox{Concatenation} \\
    &| & \WhIf{e}{s_1}{s_2} & \mbox{If branch} \\
    &| & \WhWhile{e}{s} & \mbox{While loop} \\
    &&&\\
    e & ::= & & \mbox{Expressions} \\
    & & x & \mbox{Variable} \\
    &| & v & \mbox{Value literal} \\
    &| & \WhAdd{e_1}{e_2} & \mbox{Addition} \\
    &| & \WhTimes{e_1}{e_2} & \mbox{Multiplication} \\
    &| & \WhMinus{e_1}{e_2} & \mbox{Subtraction} \\
    &| & \WhEq{e_1}{e_2} & \mbox{Equality comparison} \\
    &| & \WhLeq{e_1}{e_2} & \mbox{$\leq$ comparison} \\
    &| & \WhLnot{e} & \mbox{Logical not} \\
    &| & \WhLand{e_1}{e_2} & \mbox{Logical and} \\
    &&&\\
    \tau & ::= & & \mbox{Types} \\
    & & \WhBool & \mbox{Boolean type} \\
    &| & \WhInt & \mbox{Integer type} \\
    &&&\\
    v & ::= & & \mbox{Values} \\
    &  & n & \mbox{Integer} \\
    &| & \WhTrue & \mbox{Boolean true} \\
    &| & \WhFalse & \mbox{Boolean false} \\
  \end{array}$
  \caption{Grammar of While}
  \label{fig:while_grammar}
\end{figure}

\section{Small-step Operational Semantics} \label{sec:language_semantics}

% Describe small step operational semantics and give example with while lang
In order to prove properties of a program you must be able to formally describe
the "meaning" of the program, i.e. to describe how programs statements modify
state and what execution steps can be taken. A common approach to this is
small-step operational semantics, also called structural operational semantics
(SOS). Other approaches includes big-step operational semantics or natural
semantics, and denotational semantics. Since SOS models concurrency well this is
the only one described here.

A small-step operational semantics defines rules which can be used to derive
single execution steps (or transitions) between program states. We exemplify
this with the While language defined in the above section. In order to define
rules for state transitions a state must be defined. Therefore we make the
following definition for the While language.

\begin{definition}
  A \emph{state} $S$ for the While language is a value of the form
  \begin{equation*}
    \WhState{V, s} \text{ or } V
  \end{equation*}
  where $s$ is a statement and $V$ is a partial map
  \begin{equation*}
    V: \WhVar \rightharpoonup \WhVal
  \end{equation*}
\end{definition}

We can now define rules for state transitions (see Figure~\ref{fig:while_sos}).
All rules are of the form
\infrule[Rule name]
{\text{precondition}}
{\WhState{V, s} \rightarrow S}
Which basically says "if precondition holds then we can step from \WhState{V, s}
to $S$".  If there are no preconditions we just write
\infax[Rule name]
{\WhState{V, s} \rightarrow S}

\begin{figure}
  \centering
  \infax[WhE-Decl1]
  {\WhState{V, \WhDecl{x}{\WhInt}} \: \rightarrow \: V[ x \mapsto 0]}
  
  \RuleSpace

  \infax[WhE-Decl2]
  {\WhState{V, \WhDecl{x}{\WhBool}} \: \rightarrow \: V[ x \mapsto \WhFalse]}
  
  \RuleSpace

  \infax[WhE-Assign]
  {\WhState{V, \WhAssign{x}{e}} \: \rightarrow \: V[ x \mapsto \WhEval{V}(e)]}

  \RuleSpace

  \infax[WhE-Skip]
  {\WhState{V, \WhSkip} \: \rightarrow \: V}

  \RuleSpace

  \infrule[WhE-Comp1]
  {\WhState{V, s_1} \: \rightarrow \: \WhState{V', s'_1}}
  {\WhState{V, \WhConc{s_1}{s_2}} \: \rightarrow \: \WhState{V',
  \WhConc{s'_1}{s_2}}}
  
  \RuleSpace

  \infrule[WhE-Comp2]
  {\WhState{V, s_1} \: \rightarrow \: V'}
  {\WhState{V, \WhConc{s_1}{s_2}} \: \rightarrow \: \WhState{V',
  s_2}}

  \RuleSpace

  \infrule[WhE-IfTrue]
  {\WhEval{V}(e) = \WhTrue}
  {\WhState{V, \WhIf{e}{s_1}{s_2}} \: \rightarrow \: \WhState{V, s_1}}

  \RuleSpace 

  \infrule[WhE-IfFalse]
  {\WhEval{V}(e) = \WhTrue}
  {\WhState{V, \WhIf{e}{s_1}{s_2}} \: \rightarrow \: \WhState{V, s_2}}
  
  \RuleSpace

  \infax[WhE-While]
  {\WhState{V, \WhWhile{e}{s}} \: \rightarrow \: \WhState{V,
  \WhIf{e}{(\WhConc{s}{\WhWhile{e}{s}})}{\WhSkip}}}

  \caption{Small-step operational semantics for While}
  \label{fig:while_sos}
\end{figure}

The SOS rules of While heavily relies on the expression evaluation function
$\WhEval{\cdot}(\cdot)$. This is a function which takes a state map $V$ and an
expression $e$ and returns a value in $\WhVal$. We define this recursively as follows
\begin{equation*}
  \begin{array}{lll}
    \WhEval{V}(v) &=& v \\
    \WhEval{V}(x) &=& V(x) \\
    \WhEval{V}(e_1 + e_2) &=& \WhEval{V}(e_1) + \WhEval{V}(e_2) \\
    \WhEval{V}(e_1 * e_2) &=& \WhEval{V}(e_1) * \WhEval{V}(e_2) \\
    \WhEval{V}(e_1 - e_2) &=& \WhEval{V}(e_1) - \WhEval{V}(e_2) \\
    \WhEval{V}(e_1 = e_2) &=& \WhEval{V}(e_1) = \WhEval{V}(e_2) \\
    \WhEval{V}(e_1 \leq e_2) &=& \WhEval{V}(e_1) \leq \WhEval{V}(e_2) \\
    \WhEval{V}(e_1 \land e_2) &=& \WhEval{V}(e_1) \land \WhEval{V}(e_2) \\
    \WhEval{V}(\lnot e_1) &=& \lnot \WhEval{V}(e_1) \\
  \end{array}
\end{equation*}
Here we implicitly rely on that the operations $+, -, *, =, \leq$ are only
defined for integer values, and that $\lnot, \land$ are only defined for boolean
values.  If we encounter values where an operation is undefined the value for
$\WhEval{V}(e)$ is undefined for the corresponding state map $V$ and expression
$e$. Since the definition is recursive, the occurrence of an undefined value
should also propagate upwards, e.g. if $\WhEval{V}(e_1)$ is undefined
then $\WhEval{V}(e_1 + e_2)$ is undefined aswell. This is implicit in our
definition.

% TODO explain some of the rules in more detail.

The evaluation function $\WhEval{\cdot}(\cdot)$ together with the derivation
rules in Figure~\ref{fig:while_sos} gives us a complete description of what
steps are allowed. It is implicit in the description that if the evaluation
function is undefined at some point in the derivation of a step, execution
cannot progress and we get stuck.

\section{Type Systems} \label{sec:type_systems}

% Describe basic type systems and properties that should hold e.g. 
% TODO Preservation and Progress

A type system is a mathematical construct that consists of elements called types
and a set of rules that assign types to parts of a programming language, e.g.
statements, variables and expressions. Type systems are most commonly used to
prevent programming errors such as feeding a data structure to a function for
which it was not made, and this can be enforced both using static checking at
compile time or dynamic checking at runtime. Type systems are found in many
modern programming languages such as Java, Scala, Haskell, Python and C++.

The focus here will be on static type systems. We will explain the basics using
our previous 



