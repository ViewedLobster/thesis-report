\chapter{Related Work}\label{cha:related_work}

In this chapter four major pieces of major work is going to be described.
Section~\ref{sec:lvars} describes the LVars system and LVish, the extension
of LVars which introduces the concepts of quiescence and freezing. 
Section~\ref{sec:reactive_async} describes Reactive Async, a programming model
inspired by the extended LVars system. In section~\ref{sec:lacasa} the LaCasa
system is introduced, and finally the concept of spores is briefly introduced in
section~\ref{sec:spores}.

\section{LVars}\label{sec:lvars}

LVars~\parencite{kuper2013lvars} is a programming model that was introduced as a
solution to the problem of guaranteed deterministic concurrent programs. It
generalizes the concept of write-once data
structures~\parencite{nikhil1989structures}, also called IVars, with the ability
to write more than once but limiting update operations to being monotonic. I.e.
the values taken by LVars are part of a programmer specified lattice and all
updates are done through a join operation of the old and new values. This
ensures that writes commute~\parencite{kuper2013lvars}.

\subsection{Stores \& Lattice}%
\label{sub:stores_and_lattice}

At the foundation of the LVars system lays lattices. The values resulting from a
computation is going to be elements from a lattice $\LVarsLat$, specified by the
programmer. These lattice values are stored in a \emph{store}.  This is a set of
pairs consisting of a location and a lattice element. For a location there is
exactly one value. Letting $\LVarsLoc$ be the set of locations, a store $S$ can
be represented using a partial map
\begin{equation*}
  S: \LVarsLoc \rightharpoonup \LVarsLat.
\end{equation*}

\subsection{LVars Operations}%
\label{sub:lvars_operations}

The LVars model supports three main operations
\begin{itemize}
  \item Extending the store with a new location. This takes a fresh location and
    sets its value to $\bot_{\LVarsLat}$, the bottom element of $\LVarsLat$.
  \item Updating a store location, also called a \emph{put} operation. This
    operation takes a store location $r$ and a lattice value $l$. Given a store
    $S$ this updates location $r$ with $l \sqcup S(r)$. To ensure determinism,
    any put that takes a store location to $\top_{\LVarsLat}$ results in an
    error.
  \item A read operation also referred to as \emph{get}. This operation is
    further specified with a threshold set, i.e. a set of lattice values, and a
    store location. The operation blocks until the store location has passed one
    of the values in the threshold set, upon which it returns this value. In
    order to ensure determinism the elements in the threshold set are required
    to be mutually \emph{incompatible}. Two elements $l_1, l_2$ are incompatible if
    \begin{equation*}
      l_1 \sqcup l_2 = \top_{\LVarsLat}
    \end{equation*}
    where $\top$ is the top element of $\LVarsLat$.
\end{itemize}

% TODO brief introduction to lvars:
% Programming model to ensure deterministic concurrent execution
% Quasi determinism (maybe later)
% stores and lattice variables
% put and get operations

\subsection{LVish: Extending LVars}%
\label{sub:lvish_extending_lvars}

% TODO describe lvish, the lvars extension:
% freezing
% quasi determinism
% quiescence
% indicate that there are problems

\begin{equation}
\end{equation}

% Should describe the basic ideas of LVars and also mention that there are
% problems with the proof and hint of bigger problems.

\section{Reactive Async}\label{sec:reactive_async}

% Should describe reactive async. There are multiple issues with this system,
% e.g. that great care has to be taken in order to make the system
% deterministic, e.g. make sure that only certain types of operations are
% allowed in the callbacks.

\section{LaCasa}\label{sec:lacasa}

% Describes the basic ideas of lacasa and the idea of using OCAP constraints to 
% assure determinism

\section{Spores}\label{sec:spores}

% Describe the basic ideas of spores: (dis)allow certain types of captures to
% enforce certain properties



