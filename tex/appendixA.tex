\chapter{Proof of Preservation and Progress}
\label{proof_of_pnp}

\begin{definition}{(Heap Induced Graph)}
  The induced graph of heap $H$, written $\Graph{(H)}$ is the directed graph
  $(V, E)$ such that $V = \dom{(H)}$ and 
  \begin{equation}
    E = \left\{ (o, o')_f \in \dom{(H)}^2 \times \FieldNames: 
      H(o) = \Obj{C, FM} \text{ and } FM(f) = o' \right\}
  \end{equation}
\end{definition}

\begin{definition}{(Graph Reachability)}
  We say an object $o'$ is reachable from $o$ in graph $G = (V,
  E)$, $\reach{(G, o, o')}$ if there is a finite sequence $o_0, \dots,
  o_n \in V$ such that $o = o_0, o_n = o'$ and
  \begin{equation}
    \forall i = 1,\dots, n-1. \: \exists f \in \FieldNames \text{ s.t. } (o_i,
    o_{i+1})_f \in E.
  \end{equation}
  The sequence $o_0, \dots, o_n$ is called a \emph{path}.
\end{definition}

\begin{definition}
  Given a graph $G = (V, E)$ and a set $O \subseteq V$, we define
  \begin{equation}
    \reachable{(O, G)} = \left\{ q \in V: \exists o\in O.\: \reach{(G, o, q)}
    \right\} \notag
  \end{equation}
\end{definition}

\begin{proposition}{(Reachability equivalence)}
  \label{prop:reacheq}
  For a heap $H$ 
  \begin{equation}
    \reach{(H, o, o')} \iff \reach{(\Graph{(H)}, o, o')}
  \end{equation}
\end{proposition}

\begin{proof}
  Both directions of implication are simple to prove.
  \begin{description}
    \item[Case $\implies$:] By induction on the shape of derivation tree.
    \item[Case $\impliedby$:] By induction on the path length in graph
      $\Graph{(H)}$.
  \end{description}
\end{proof}

% TODO change definition of ocrloc
\begin{proposition} For any heap $H$ and frame stack $FS$ we have
  \begin{flalign*}
    &\ocrloc{(FS, H)} &\\
    &\iff &\\
    &\begin{aligned}
    \forall o, q &\in \dom{(H)}. \\
    & \accRoot{(o, FS, H)} \land \reach{(H, o, q)} \implies \\
    & \ocap{(\typeOf{(o, H)})} 
    \end{aligned}&\\
    &\iff &\\
    &\begin{aligned}
    \forall q &\in \reachable{{(\accRoots{(FS,H)}, \Graph{(H)})}}. \notag\\
    & \ocap{(\typeOf{(q, H)})}
    \end{aligned}&
  \end{flalign*}
\end{proposition}

% TODO should I elaborate?
\begin{proof}
  Follows from definition of $\accRoot$, $\accRoots$ and $\reachable$.
\end{proof}

\begin{proposition} \label{prop:csep_eq}
  For any heap $H$ and object references $o, o'$ we have
  \begin{flalign*}
    &\csep{(H, o, o')} \iff &\\
    & \begin{aligned}
        \forall q \in \: &\reachable{(\{o\}, \Graph{(H)})} \cap \reachable{(\{o'\},
        \Graph{(H)})}. \\
        & \typeOf{(q, H)} \stof \CellType
    \end{aligned}&
  \end{flalign*}
\end{proposition}

\begin{proof}
  First of all we let $R = \reachable{(\{o\}, \Graph{(H)})}, R' =
  \reachable{(\{o'\}, \Graph{(H)})}$.
  We prove each direction of implication separately.
  \begin{description}
    \item[Case $\implies$:] Take any $q \in  R\cap R'$. By definition of
      $\reachable$, reachability equivalence (prop.~\ref{prop:reacheq}) and 
      definition of $\csep{(H, o, o')}$ we have $\typeOf{(q, H)} \stof \CellType$.
    \item[Case $\impliedby$:] Take any $q, q'\in \dom{(H)}$ such that
      $\reach{(H, o, q)}$ and $\reach{(H, o', q')}$. If $q \neq q$ we are done.
      Otherwise $q = q'$ and by definition of $\reachable$ $q \in R \cap R'$.
      Finally $\typeOf{(q, H)} \stof \CellType$ by assumption.
  \end{description}
\end{proof}

\begin{proposition} \label{prop:2.6}
  For any heap $H$ and frame stacks $FS, HS$ we have
  \begin{flalign*}
    &\isolated{(H, FS, HS)}  \iff &\\
    &\begin{aligned}
        \forall q \in \:&\reachable{(\accRoots{(FS, H)}, \Graph{(H)})} \cap \\
        & \reachable{(\accRoots{(HS, H)}, \Graph{(H)})}. \\
        & \typeOf{(q, H)} \stof \CellType
    \end{aligned}&
  \end{flalign*}
\end{proposition}

\begin{proof}
  We let 
  \begin{equation*}
    \begin{gathered}
      R = \reachable(\accRoots(FS, H), \Graph(H)) 
      R' = \reachable(\accRoots(HS, H), \Graph(H))  
    \end{gathered}
  \end{equation*}
  We prove each direction of implication separately.
  \begin{description}
    \item[Case $\implies$:] Take $q \in R \cap R'$. Then 
      \begin{equation*}
        \exists o \in \accRoots(FS, H), o' \in \accRoots(HS, H)
      \end{equation*}
      such that
      \begin{equation*}
        q \in \reachable(\{o\}, \Graph(H)) \cap \reachable(\{o'\}, \Graph(H))
      \end{equation*}
      By definition of $\accRoots$ we have
      \begin{equation*}
        \accRoot(o, FS, H) \land \accRoot(o', HS, H)
      \end{equation*}
      and thus by $\isolated(H, FS, HS)$ we have $\csep(H, o, o')$. 
      Proposition~\ref{prop:csep_eq} yields $\typeOf(q, H) \stof \CellType$.
    \item[Case $\impliedby$:] Take any $o, o' \in \dom(H)$ such that
      \begin{equation*}
        \accRoot(o, FS, H) \land \accRoot(o', HS).
      \end{equation*}
      Clearly then $o \in \accRoots(FS, H)$ and $o \in \accRoots(HS, H)$.
      Moreover
      \begin{gather*}
        \reachable(\{o\}, \Graph(H)) \subseteq R \\
        \reachable(\{o'\}, \Graph(H)) \subseteq R'
      \end{gather*}
      By assumption then
      \begin{align*}
        \forall q \in &\reachable(\{o\}, \Graph(H)) \cap \reachable(\{o'\},
        \Graph(H)). \\ 
        &\typeOf(q, H) \stof \CellType
      \end{align*}
      and by proposition~\ref{prop:csep_eq} we have $\csep(H, o, o')$. Since
      $o, o'$ arbitrary we are done.
  \end{description}
\end{proof}

\begin{proposition} \label{prop:2.11}
  Let $H, H'$ be heaps and $P, P'$ thread sets such that
  \begin{enumerate}
    \item $P = Q \cup_D \left\{ FS|_a^\iota \right\}$, $\isolation{(H, P)}$, $H
      \vdash P \tsep \ocr$ and $P' = Q \cup_D \left\{ FS'|_a^\iota \right\}$
    \item $\Graph{(H)} = \Graph{(H')}$
    \item $\forall HS|_b^{\iota'} \in Q.$ \\ 
      $\reachable{(\accRoots{(HS, H')}, \Graph{(H')})} \subseteq$ \\
      $\reachable{(\accRoots{(HS, H)}, \Graph{(H)})}$
    \item $\reachable{(\accRoots{(FS', H')}, \Graph{(H')})} \subseteq$ \\
      $\reachable{(\accRoots{(FS, H)}, \Graph{(H)})}$
    \item $\forall o \in \dom{(H)}. \: \typeOf{(o, H)} = \typeOf{(o, H')}$
  \end{enumerate}
  Then $\isolation{(H', P')}$ and $H' \vdash P' \tsep \ocr$.
\end{proposition}

\begin{remark}
  Note that $\Graph{(H)} = \Graph{(H')}$ implies $\dom{(H)} = \dom{(H')}$.
  Otherwise many of the preconditions above would not make sense.
\end{remark}

\begin{proof}
  We prove that 
  \begin{equation*}
    \forall \text{ distinct } HS|_a^\iota, GS|_b^{\iota'} \in P'.\; a = \ocap \lor b = \ocap \implies
    \isolated(H', HS, GS).
  \end{equation*}
  This implies $\isolation(H', P')$ by rule {\sc ISO-Procs}.
  Take any distinct $HS|_b^{\iota'}, GS|_c^{\iota''} \in P$. Then by assumption 3, 4 and basic set
  properties
  \begin{equation} \label{eq:reachinclusion1}
    \begin{gathered}
      \reachable(\accRoots(HS, H'), \Graph(H')) \cap \reachable(\accRoots(GS, H'),
      \Graph(H')) \\
      \subseteq \\
      \reachable(\accRoots(HS, H), \Graph(H)) \cap \reachable(\accRoots(GS, H),
      \Graph(H)).
    \end{gathered}
  \end{equation}
  By this, proposition~\ref{prop:2.6} and $\isolation(H, HS, GS)$ we have
  $\isolation(H', HS, GS)$. By \eqref{eq:reachinclusion1} and assumption 5
  we have $H' \vdash P' \tsep \ocr$.
\end{proof}

\begin{corollary} \label{cor:2.11}
  Let $H, H'$ be heaps and $P, P'$ thread sets such that
  \begin{enumerate}
    \item $P = Q \cup_D \left\{ FS|_a^\iota \right\}$, $\isolation{(H, P)}$, $H
      \vdash P \tsep \ocr$ and $P' = Q \cup_D \left\{ FS'|_a^\iota \right\}$
    \item $\Graph{(H)} = \Graph{(H')}$
    \item $\forall HS|_b^{\iota'} \in Q. \: \accRoots{(HS, H')} \subseteq \accRoots{(HS, H)}$
    \item $\accRoots{(FS', H')} \subseteq \accRoots{(FS, H)}$
    \item $\forall o \in \dom{(H)}. \: \typeOf{(o, H)} = \typeOf{(o, H')}$
  \end{enumerate}
  Then $\isolation{(H', P')}$ and $H' \vdash P' \tsep \ocr$.
\end{corollary}

\begin{proof}
  Assumption 2 and 3 implies assumption 2 and 3 of
  proposition~\ref{prop:2.11}
\end{proof}

\begin{definition} \label{def:ptilde}
  Let $FS_g$ be any framestack where 
  \begin{equation*}
    \accRoots(FS_g, H) = \left\{ o_g \right\}.
  \end{equation*}
  I.e., $FS_g$ only has the global object as an accessible root.  Then for
  any thread set $P$, let $\tilde{P}$ be the thread set
  \begin{equation*}
    \tilde{P} = P \cup \left\{ FS_g|_{\nocap}^d \right\}
  \end{equation*}
  for some valid $d$.
\end{definition}

\begin{proposition} \label{prop:2.8}
  For any heap $H$ and thread set $P$
  \begin{equation*}
    \isolation{(H, \tilde{P})} \iff \isolation{(H, P)} \text{ and } H \vdash P
    \tsep \gsep 
  \end{equation*}
\end{proposition}

\begin{proof}
  This follows almost immediately by the definitions of isolation and global
  object separation, and the fact that $\accRoots(FS_g) = \left\{
    o_g \right\}$ 
\end{proof}

We state the following two propositions without proof, but they should be quite
obvious.
% TODO define transition identifier
\begin{proposition} \label{prop:uniq_trans}
  Let $H, P \Rrightarrow H', P'$. Then there is a unique transition identifier
  $R^{\alpha, \beta}$ such that $H, P \Rrightarrow^{R^{\alpha, \beta}} H', P'$.
\end{proposition}

\begin{proposition} \label{prop:tilde_trans}
  Let $H, P \Rrightarrow^{R^{\alpha, \beta}} H', P'$. Then $H, \tilde{P}
  \Rrightarrow^{R^{\alpha, \beta}} H', \tilde{P'}$
\end{proposition}

%TODO check formulation of prop
\begin{proposition} \label{prop:2.9}
  For any $H, H', P, P'$ such that $\vdash H, H \vdash P$, $H \vdash P \tsep
  \ocr$ and $H, P \Rrightarrow^{R^{\alpha, \beta}} H', P'$, if
  \begin{equation*}
      \isolation(H, P) 
      \implies 
      \isolation(H', P')
  \end{equation*}
  then
  \begin{equation*}
    \isolation(H, \tilde{P}) \implies \isolation(H', \tilde{P'})
  \end{equation*}
\end{proposition}

\begin{proof}
  By proposition~\ref{prop:tilde_trans} $H, \tilde{P} \Rrightarrow^{R^{\alpha,
  \beta}} H', \tilde{P'}$. It is easy to see that $\vdash H$ and $H \vdash
  \tilde{P}$.
  Since the assumption of the proposition refers to any $H, H', P, P'$ for which
  these three properties hold the statement also holds for $H, H',
  \tilde{P}, \tilde{P'}$.
\end{proof}

\begin{remark}
  This proposition may seem a bit circular, but what it does is allow us to get
  the global separation property for free in most cases in the proof of
  preservation by combining it with proposition~\ref{prop:2.8}. We state
  this in the following corollary. In particular we can for example use this in
  all cases where we can also use proposition~\ref{prop:2.11} or its corollary
  \ref{cor:2.11}.
\end{remark}

\begin{corollary} \label{cor:2.9}
  For any $H, H', P, P'$ such that $\vdash H, H \vdash P$, $H \vdash P \tsep
  \ocr$ and $H, P \Rrightarrow^{R^{\alpha, \beta}} H', P'$, if
  \begin{equation*}
      \isolation(H, P) 
      \implies 
      \isolation(H', P')
  \end{equation*}
  then
  \begin{equation*}
    \begin{gathered}
      \isolation(H, P) \text{ and } H \vdash P \tsep \gsep \\
      \implies \\
      \isolation(H', P') \text{ and } H \vdash P' \tsep \gsep
    \end{gathered}
  \end{equation*}
\end{corollary}

% TODO define \Values set
\begin{proposition} \label{prop:2.12}
  Let $H, H'$ be heaps. If $\dom{(H)} = \dom{(H')}$ and
  \begin{equation*}
    \forall o \in \dom{(H)}. \: \typeOf{(o, H)} = \typeOf{(o, H')}
  \end{equation*}
  then 
  \begin{equation*}
    \forall k \in \Values. \: \typeOf{(k, H)} = \typeOf{(k, H')}
  \end{equation*}
\end{proposition}

\begin{proof}
  It is easy to see since the only values in $\Values$ where its type depends on
  $H$ is the object references $o$.
\end{proof}

\begin{proposition} \label{prop:2.19}
  If
  \begin{equation*}
    \forall o \in \dom(H) \subseteq \dom(H'). \: \typeOf(o, H) = \typeOf(o, H')
  \end{equation*}
  and $H \vdash \Gamma; L$, then $H' \vdash \Gamma; L$.
\end{proposition}

\begin{proof}
  It is obvious from the structure of the rules {\sc WF-EnvVar} and {\sc
  WF-Env} and the assumptions of the proposition.
\end{proof}

\begin{proposition} \label{prop:2.13}
  For any $H, P$ we have 
  \begin{equation}
    H \vdash P \iff \forall HS|_b^{\iota'} \in P.\: H;b \vdash HS
  \end{equation}
\end{proposition}

\begin{proof} (Sketch) Done in each direction separately.
  \begin{description}
    \item[Case $\implies$:] By induction on the shape of the derivation tree.
    \item[Case $\impliedby$:] By induction on size of $P$.
  \end{description}
\end{proof}

% TODO define heap
\begin{proposition} \label{prop:2.14}
  Let $H, H'$ be heaps such that $\dom{(H)} \subseteq \dom{(H')}$ and 
  \begin{equation*}
    \forall o \in \dom{(H)}. \: \typeOf{(o, H)} = \typeOf{(o, H')}.
  \end{equation*}
  Then for any frame stack $FS$
  \begin{equation}\label{eq:fs_impl_typing1}
    H; a \vdash FS \implies H'; a \vdash FS
  \end{equation}
  and
  \begin{equation} \label{eq:fs_impl_typing2}
    H; a \vdash^{x :\sigma} FS \implies H'; a \vdash^{x: \sigma} FS 
  \end{equation}
\end{proposition}

\begin{proof}
  We first prove \eqref{eq:fs_impl_typing2} by induction on the lenght $n$ of
  $FS$. To prove the implication we assume that $H; a \vdash^{x: \tau} FS$.
  
  If $n = 0$ we have $FS = \varepsilon$. By {\sc T-FSEmpty2} we are done.
  
  If $n = i + 1$ we have $FS = F \circ GS$ where length of $GS$ is $i$. By  
  $H; a \vdash^{x: \sigma} FS$ and {\sc T-FS2} we have 
  \begin{equation*}
    F = \xframe{L, t}^{y} \andalso H \vdash \Gamma; L \andalso \TypeRel{\Gamma, x:
    \sigma}{a}{t}{\tau} \andalso H; a \vdash^{y: \tau} GS.
  \end{equation*}
  By induction hypothesis we then have $H'; a \vdash^{y: \tau} GS$. We get $H'
  \vdash \Gamma; L$ from proposition \ref{prop:2.19}. Applying rule {\sc T-FS2}
  yields $H'; a \vdash^{x: \sigma} FS$.

  Now we prove \eqref{eq:fs_impl_typing1}. Similarly assuming $H; a \vdash^{x:
  \tau} FS$ we have two cases. If $FS = \varepsilon$ we are done by {\sc
  T-FSEmpty1}. If $FS = F \circ GS$ we by {\sc T-FS1} have
  \begin{equation*}
    F = \xframe{L, t}^{x} \andalso H \vdash \Gamma; L \andalso
    \TypeRel{\Gamma}{a}{t}{\tau} \andalso H; a \vdash^{x: \sigma} GS.
  \end{equation*}
  By \eqref{eq:fs_impl_typing1} $H'; a \vdash^{x: \sigma} GS$ and
  proposition~\ref{prop:2.19} yields $H' \vdash \Gamma; L$. Applying rule {\sc
  T-FS1} gives us $H'; a \vdash FS$.
\end{proof}

The following propositions are stated without proof, but they should not be hard
to show.
\begin{proposition} \label{prop:ocr_eq}
  For any $H, P$ we have
  \begin{equation*}
    \begin{gathered}
      H \vdash P \tsep \ocr \\
      \iff \\
      \forall HS|_b^{\iota'} \in P. \quad
         b = \ocap \implies \ocrloc(HS, H)
    \end{gathered}
  \end{equation*}
\end{proposition}

\begin{proposition} \label{prop:ocrloc_eq}
  For any $H, HS$ we have
  \begin{equation*}
    \begin{gathered}
      \ocrloc(HS, H) \\
      \iff  \\
      \begin{aligned}
        \forall q &\in \reachable(\accRoots(HS, H), \Graph(H)). \\
        & \ocap(\typeOf(q, H))
      \end{aligned}
    \end{gathered}
  \end{equation*}
\end{proposition}

\begin{proposition} \label{prop:ocrtilde_eq}
  For any $H, P$ we have
  \begin{equation*}
    H \vdash P \tsep \ocr \iff H \vdash \tilde{P} \tsep \ocr
  \end{equation*}
\end{proposition}
\begin{proof}
  Follows immediately from proposition~\ref{prop:ocr_eq} and the definition of
  $\tilde{P}$.
\end{proof}


\section{Proof of preservation}
\label{sec:proof_of_preservation}

\begin{theorem*}[Preservation]
  Let $S, S'$ be states such that $\vdash S \tsep \stateok$ and $S \Rrightarrow
  S'$. Then $\vdash S' \tsep \stateok$.
\end{theorem*}

\begin{proof} 
  First of all we have that $S \neq \Error$ since no step can be made from this
  state. Thus $S = H, P$ and
  \begin{equation*}
    \begin{gathered}
      \vdash H \andalso H \vdash P \andalso H \vdash P \tsep \ocr \\
      \isolated{(H, P)} \andalso H \vdash P \tsep \gsep \andalso \uniqMain(P).
    \end{gathered}
  \end{equation*}
  We also assume we have $S' = H', P'$ since otherwise $S' = \Error$ and the
  theorem holds trivially. What remains is to prove
  \begin{equation*}
    \begin{gathered}
      \vdash H' \andalso H' \vdash P' \andalso H' \vdash P' \tsep \ocr \\
      \isolated{(H', P')} \andalso H' \vdash P' \tsep \gsep \andalso
      \uniqMain(P).
    \end{gathered}
  \end{equation*}
  We easily see that $\uniqMain(P')$ must hold since no rule changes the OCAP
  status of threads, and the only rule which spawns new threads is {\sc E-Spawn}
  which spawns an OCAP thread.

  To prove everything else we proceed by cases.
  \begin{description}
    \item[Case {\sc E-FSProp}:] According to this rule $P = Q \cup_D \left\{
        FS|_a^\iota \right\}$ and $P' = Q \cup \left\{ FS'|_a^\iota \right\}$. We proceed by cases.
      \begin{description}
        \item[Case {\sc E-FProp}:] By the rule definition $FS|_a^\iota = F \circ
          GS|_a^\iota$ and $FS'|_a^\iota = F' \circ GS|_a^\iota$. Furthermore we can assume
          that $F = \sframe{L, t}$ and $F' = \sframe{L', t'}$.
          We proceed by cases.
          \begin{description}
            \item[Case {\sc E-Null}:] By this rule we have $t =
              \Let{x}{\NullVal}{t'}$, $H = H'$ and $L' = L[x \mapsto \NullVal]$.
              Immediatelly we have $\vdash H'$.

              By {\sc T-Procs} and proposition~\ref{prop:2.13} we have 
              \begin{equation} \label{eq:enull1}
               H \vdash Q  \andalso H; a \vdash FS 
              \end{equation}
              Trivially $H' \vdash Q$. 
              By~\eqref{eq:enull1} and rule {\sc T-FS1} we have
              \begin{equation} \label{eq:enull2}
                \begin{gathered}
                  H \vdash \Gamma; L \andalso \TypeRel{\Gamma}{a}{t}{\sigma'} \\
                  \sigma' \stof \sigma \andalso H; a \vdash^{s: \sigma} GS.
                \end{gathered}
              \end{equation}
              Let $\Gamma' = \Gamma, x: \NullType$. By {\sc T-Let}, {\sc
              T-Null} and \eqref{eq:enull2} we have
              \begin{equation} \label{eq:enull3}
                \TypeRel{\Gamma'}{a}{t'}{\sigma'}.
              \end{equation}
              Since $\typeOf(L(x), H') \stof \NullType$, by inspection of rules
              {\sc WF-EnvVar} and {\sc WF-Env} we can see that
              \begin{equation} \label{eq:enull4}
                H' \vdash \Gamma';L'
              \end{equation}
              By {\sc T-FS1}, \eqref{eq:enull2}, \eqref{eq:enull3} and \eqref{eq:enull4}
              \begin{equation}\label{eq:enull5}
                H;a \vdash FS'
              \end{equation}
              By \eqref{eq:enull5}, {\sc T-Procs} and $H = H'$ we get  
              \begin{equation}
                H' \vdash P'
              \end{equation}
              $H' \vdash P' \tsep \ocr$ and $\isolation{(H', P')}$ follows from
              the corollary of proposition~\ref{prop:2.11}. Then $H' \vdash P'
              \tsep \gsep$ follows from propositions~\ref{prop:2.8}
              and~\ref{prop:2.9} since we never use the $\gsep$ property to
              prove isolation.

            \item[Case {\sc E-LVal}:] Similarly.

            \item[Case {\sc E-Var}:] First, by rule {\sc E-Var} we have
              \begin{equation*} 
                \begin{gathered}
                  F = \sframe{L, t} \andalso t = \Let{x}{y}{t'} \\ 
                  F' = \sframe{L', t'} \andalso L' = L[x \mapsto L(y)].
                \end{gathered}
              \end{equation*}
              Also
              \begin{equation} \label{eq:evar1}
                H = H'
              \end{equation}
              so $\vdash H'$.
              By $H \vdash P$, {\sc T-FS1} and proposition \ref{prop:2.13}
              \begin{equation} \label{eq:evar3}
                \begin{gathered}
                  H \vdash \Gamma; L \andalso \TypeRel{\Gamma}{a}{t}{\sigma'} \\
                  \sigma' \stof \sigma \andalso H; a \vdash^{s: \sigma} GS
                \end{gathered}
              \end{equation}
              By {\sc T-Let, T-Var} and $\TypeRel{\Gamma}{a}{t}{\sigma'}$
              \begin{equation} \label{eq:evar4}
                \TypeRel{\Gamma, x: \gamma}{a}{t'}{\sigma'} \andalso \Gamma(y) =
                \gamma
              \end{equation}
              Let $\Gamma' = \Gamma, x: \gamma$.
              Then by \eqref{eq:evar4}
              \begin{equation} \label{eq:evar5}
                \TypeRel{\Gamma'}{a}{t'}{\sigma'}.
              \end{equation}
              By $H \vdash \Gamma; L$, \eqref{eq:evar1}, \eqref{eq:evar5}, and
              rules {\sc WF-EnvVar}, {\sc WF-Env}
              \begin{equation*}
                \typeOf(L'(x), H') = \typeOf(L(y), H) \stof \Gamma(y) =
                \Gamma'(x).
              \end{equation*}
              From definitions of $\Gamma', L'$ and $H \vdash \Gamma;L$ we also have 
              \begin{equation*}
                \begin{aligned}
                  \forall z \in &\dom(\Gamma') \text{ s.t. } z \neq x. \\
                  & \typeOf(L'(z), H') \leq \Gamma'(z).
                \end{aligned}
              \end{equation*}
              Thus
              \begin{equation} \label{eq:evar6}
                H' \vdash \Gamma'; L'
              \end{equation}
              by utilizing rules {\sc WF-EnvVar} and {\sc WF-Env}.
              By {\sc T-FS1}, \eqref{eq:evar3}, \eqref{eq:evar5},
              \eqref{eq:evar6} we have
              \begin{equation} \label{eq:evar7}
                H'; a \vdash FS'.
              \end{equation}
              By proposition \ref{prop:2.13} and $H \vdash P$ we get
              \begin{equation*} 
                \forall HS|_b^{\iota'} \in Q. \: H; b \vdash HS.
              \end{equation*}
              By this, \eqref{eq:evar1} and proposition \ref{prop:2.13}
              \begin{equation*}
                H' \vdash Q
              \end{equation*}
              Combining this with \eqref{eq:evar7} and {\sc T-Procs} finally
              yields
              \begin{equation*}
                H' \vdash P'
              \end{equation*}

              We now move on to OCAP reachability, isolation and global object
              separation. We note that all preconditions of
              corollary~\ref{cor:2.11} holds. Thus we immediately have 
              \begin{equation*}
                H' \vdash P' \tsep \ocr \andalso \isolation(H', P').
              \end{equation*}
              $H' \vdash P' \tsep \gsep$ immediately follows by corollary to
              proposition \ref{prop:2.9}.

            \item[Case {\sc E-Select}:] By rule {\sc E-Select}
              \begin{equation} \label{eq:eselect1}
                \begin{gathered}
                  H = H' \andalso F = \sframe{L, t} \andalso t =
                  \Let{x}{\FSel{y}{f}}{t'} \\
                  L(y) = o \andalso H(o) = \Obj{C, FM} \andalso f \in \dom(FM)
                  \\
                  F' = \sframe{L', t'} \andalso L' = L[x \mapsto FM(f)]
                \end{gathered}
              \end{equation}
              We immediately see $\vdash H'$. By $\vdash H$ we have
              \begin{equation} \label{eq:eselect2}
                \typeOf(FM(f), H) \stof \ftype(f, C).
              \end{equation}
              $H \vdash P$, propositions \ref{prop:2.13}, \ref{prop:2.14} yields
              $H \vdash Q$ and thus $H' \vdash Q$.
              $H \vdash P$ and proposition \ref{prop:2.13} gives 
              \begin{equation} \label{eq:eselect3}
                H; a \vdash FS
              \end{equation}
              which under inspection of rule {\sc T-FS1} immediately gives
              \begin{equation} \label{eq:eselect4}
                \begin{gathered}
                  H \vdash \Gamma; L \andalso \TypeRel{\Gamma}{a}{t}{\sigma'} \\
                  \sigma' \stof \sigma \andalso H; a \vdash^{s: \sigma} GS.
                \end{gathered}
              \end{equation}
              $\TypeRel{\Gamma}{a}{t}{\sigma}$ together with rule {\sc T-Let}
              gives
              \begin{equation} \label{eq:eselect5}
                \TypeRel{\Gamma}{a}{\FSel{y}{f}}{\gamma} \andalso
                \TypeRel{\Gamma, x: \gamma}{a}{t'}{\sigma'}.
              \end{equation}
              which combined with {\sc T-Select} and {\sc T-Var} gives us
              \begin{equation} \label{eq:eselect6}
                (y: C) \in \Gamma \andalso \ftype(f, C) = \gamma
              \end{equation}
              Let $\Gamma' = \Gamma, x: \gamma$. Similarly to case {\sc E-Var}
              we see that
              \begin{equation} \label{eq:eselect7}
                \forall z \in L \text{ s.t. } z \neq x . \: \typeOf(L(z), H')
                \stof \Gamma'(z).
              \end{equation}
              Furthermore, because of \eqref{eq:eselect2} and $H = H'$
              \begin{equation} \label{eq:eselect8}
                \begin{aligned}
                  \typeOf(L'(x), H') &= \typeOf(FM(f), H') \\
                                     &\stof \ftype(f, C) \\
                                     &= \Gamma'(x)
                \end{aligned}
              \end{equation}
              Using {\sc WF-Env, WF-EnvVar}, \eqref{eq:eselect7} and
              \eqref{eq:eselect8} we have
              \begin{equation} \label{eq:eselect9}
                H' \vdash \Gamma';L'.
              \end{equation}
              Using {\sc T-FS1} with $H = H'$, \eqref{eq:eselect4}, \eqref{eq:eselect5}
              and \eqref{eq:eselect9} we finally get
              \begin{equation}
                H';a \vdash FS'
              \end{equation}
              which with the help of {\sc T-Procs} and $H' \vdash Q$ gives us 
              \begin{equation*}
                H'\vdash P'.
              \end{equation*}
              
              Moving on to OCAP reachability, isolation and global object
              separation we can easily see that the preconditions of proposition
              \ref{prop:2.11} holds and thus we immediately get
              \begin{equation*}
                H' \vdash P' \tsep \ocr \andalso \isolation(H', P')
              \end{equation*}
              Again using corollary \ref{cor:2.9} we get
              \begin{equation*}
                H' \vdash P' \tsep \gsep.
              \end{equation*}


            \item[Case {\sc E-Assign}:] The {\sc E-Assign} rule gives us that
              \begin{equation} \label{eq:eassign1}
                \begin{gathered}
                  F = \sframe{L, t} \andalso t = \Let{x}{\FAss{y}{f}{z}}{t'} \\
                  F' = \sframe{L', t'} \andalso L' = L[x \mapsto L(z)] \\
                  L(y) = o \andalso H(o) = \Obj{C, FM} \andalso f \in \dom(FM)
                  \\
                  FM' = FM[f \mapsto L(z)] \andalso H' = H[o \mapsto \Obj{C, FM'}]
                \end{gathered}
              \end{equation}
              We begin by proving $\vdash H'$. First we note
              that only change from $H$ to $H'$ is in what object $o$ maps to.
              Clearly from \eqref{eq:eassign1} we have
              \begin{equation} \label{eq:eassign2}
                \forall o \in \dom(H) = \dom(H'). \: \typeOf(o, H) = \typeOf(o,
                H')
              \end{equation}
              Thus by proposition \ref{prop:2.12} 
              \begin{equation} \label{eq:eassign3}
                \forall k \in \Values. \: \typeOf(k, H) = \typeOf(k,
                H')
              \end{equation}

              \begin{addmargin}[1em]{1em}
                \begin{remark}
                  This means that immediately it is clear that the conditions for
                  $\vdash H'$ holds for all $o' \in \dom(H')$ s.t. $o' \neq o$.
                  See \eqref{eq:defwth1} and \eqref{eq:defwth2} in the
                  definition of a well typed heap.
                \end{remark}
              \end{addmargin}

              By $\vdash H$, definition of $H'$ and \eqref{eq:eassign3}
              \begin{equation} \label{eq:eassign4}
                \begin{aligned}
                  \forall f' &\in \fields(C), f' \neq f. \\ 
                                 &\typeOf(FM'(f'), H') \stof \ftype(f, C)
                \end{aligned}
              \end{equation}
              By $H \vdash P$ and prop. \ref{prop:2.13}
              \begin{equation} \label{eq:eassign5}
                H;a \vdash FS.
              \end{equation}
              This and {\sc T-FS1} yields
              \begin{equation}\label{eq:eassign6}
                \begin{gathered}
                  H \vdash \Gamma; L \andalso \TypeRel{\Gamma}{a}{t}{\sigma'} \\
                  \sigma' \stof \sigma \andalso H; a \vdash^{s: \sigma} GS
                \end{gathered}
              \end{equation}
              By $H;a \vdash^{s: \sigma} GS$, \eqref{eq:eassign2} and prop.
              \ref{prop:2.14} we have 
              \begin{equation} \label{eq:eassign7} 
                H'; a \vdash^{s: \sigma} GS
              \end{equation}
              $\TypeRel{\Gamma}{a}{t}{\sigma'}$ and {\sc T-Let} gives
              \begin{equation} \label{eq:eassign8}
                \TypeRel{\Gamma}{a}{\FAss{y}{f}{z}}{\gamma} \andalso
                \TypeRel{\Gamma, x: \gamma}{a}{t'}{\sigma'}
              \end{equation}
              which whith {\sc T-Assign} gives us
              \begin{equation} \label{eq:eassign9}
                \begin{gathered}
                  \TypeRel{\Gamma}{a}{y}{C'} \andalso \ftype(f, C') = \alpha \\
                  \TypeRel{\Gamma}{a}{z}{\alpha'} \andalso \alpha' \stof \alpha
                \end{gathered}
              \end{equation}
              By {\sc T-Var} and \TypeRel{\Gamma}{a}{y}{C'}
              \begin{equation}\label{eq:eassign10}
                (y: C') \in \Gamma 
              \end{equation}
              or equivalently $\Gamma(y) = C'$.
              Similarly
              \begin{equation} \label{eq:eassign11}
                (z: \alpha') \in \Gamma.
              \end{equation}
              By $H \vdash \Gamma; L$ we have $H \vdash \Gamma; L; z$ and thus
              \begin{equation} \label{eq:eassign12}
                \begin{aligned}
                  \typeOf(L(z), H) &= \alpha'' \\ 
                                   &\stof \alpha' \\ 
                                   &\stof \alpha \\ 
                                   &= \ftype(f, C') \\ 
                                   &= \ftype(f, C)
                \end{aligned}
              \end{equation}
              where the last equality comes from the fact that the classes are
              well formed. By \eqref{eq:eassign1} we have $FM'(f) = L(z)$ and
              combining this with \eqref{eq:eassign12} we get
              \begin{equation} \label{eq:eassign13}
                \typeOf(FM'(f), H) \stof \ftype(f, C).
              \end{equation}
              Combined with \eqref{eq:eassign3}, \eqref{eq:eassign4} and the
              remark above we get
              \begin{equation*} 
                \vdash H'
              \end{equation*}

              $H' \vdash P'$ follows similarly to {\sc E-Var} case.

              We now move on to OCAP reachability, isolation and global object
              separation.  Note that the only insteresting case is if $L(z)$ is
              an object refence, since otherwise the preconditions to corollary
              \ref{cor:2.11} holds and we are done similarly to previous cases.
              Therefore we assume $L(z) \in \dom(H)$.  To help our efforts we
              state the following lemma.

              \begin{lemma} \label{lem:2.15}
                Let $HS|_b^{\iota'} \in Q$ and $L(z) \in \dom(H)$. Then
                \begin{equation} \label{eq:lem2.15a}
                  \begin{aligned}
                    &\reachable(\accRoots(FS', H'), \Graph(H')) \subseteq \\
                    &\reachable(\accRoots(FS, H), \Graph(H))
                  \end{aligned}
                \end{equation}
                and
                \begin{equation}\label{eq:lem2.15b}
                  \begin{aligned}
                    &\reachable(\accRoots(HS, H'), \Graph(H')) \subseteq \\
                    &\reachable(\accRoots(HS, H), \Graph(H)) \cup  \\
                    &\reachable(\accRoots(FS, H), \Graph(H)).
                  \end{aligned}
                \end{equation}
                Furthermore, if $a = \ocap$ or $b = \ocap$ then
                \begin{equation}\label{eq:lem2.15c}
                  \begin{aligned}
                    &\reachable(\accRoots(HS, H'), \Graph(H')) \subseteq \\
                    &\reachable(\accRoots(HS, H), \Graph(H))
                  \end{aligned}
                \end{equation}
              \end{lemma}

              \begin{proof}
                Let $\Graph(H) = (V_H, E_H)$. Then clearly
                \begin{equation}
                  \Graph(H') = \left(V_H, E_H \setminus \{(o_y, FM(f))_f \} \cup
                  \{(o_y, o_z)_f \} \right)
                \end{equation}
                Let $L(y) = o_y$. We prove each part separately. 
                We begin with \eqref{eq:lem2.15a}.  Let
                \begin{equation*}
                  q \in \reachable(\accRoots(FS', H'), \Graph(H')).
                \end{equation*}
                Because of reachability equivalence (proposition
                \ref{prop:reacheq}) this means
                \begin{equation}
                  \begin{aligned}
                    \exists o &\in \accRoots(FS', H'). \\
                    & \exists \text{ path } o_0, \dots, o_n \in \Graph(H')
                    \text{ s.t. } o_0 = o, o_n = q
                  \end{aligned}
                \end{equation}
                Let $f_i$ the associated field names from which the path is
                formed, i.e. $(o_i, o_{i+1})_{f_i} \in \Graph(H')$.
                Clearly $o \in \accRoots(FS, H)$ by \eqref{eq:eassign1}. 

                There are two cases:
                \begin{enumerate}
                  \item For some $i \in \left\{ 0, \dots, n-1 \right\}$ $o_i =
                    o_y, o_{i+1} = o_z$ and $f_i = f$, i.e. the edge $(o_y,
                    o_z)_{f}$ is part of the path.
                  \item There is no such $i$.
                \end{enumerate}
                We prove each case separately. For case 1 let $i$ be the last
                such index. Examine the path 
                \begin{equation}
                  o_{i+1}, \dots, o_n
                \end{equation}
                Clearly this path is in $\Graph(H)$ since we chose $i$ to be the
                last index that fulfills case 1.  Since $o_{i+1} = o_z$ and
                $o_z$ obviously is in $\accRoots(FS, H)$
                \begin{equation}
                  q \in \reachable(\accRoots(FS, H), \Graph(H)).
                \end{equation}

                For case 2 the new edge $(o_y, o_z)_f$ is not part of the path.
                Thus all edges $(o_i, o_{i+1})_{f_i}$ are in $\Graph(H)$ and and
                thus the path $o_0, \dots, o_n$ is a path in $\Graph(H)$ aswell.
                This means $q \in \reachable(\accRoots(FS, H), \Graph(H))$ which
                concludes the proof of the first part of the lemma.

                To prove \eqref{eq:lem2.15b} take any
                \begin{equation}
                  q \in \reachable(\accRoots(HS, H'), \Graph(H')).
                \end{equation}
                Similarly to before
                \begin{equation}
                  \begin{aligned}
                    \exists o &\in \accRoots(HS, H'). \\
                    & \exists \text{ path } o_0, \dots, o_n \in \Graph(H')
                    \text{ s.t. } o_0 = o, o_n = q
                  \end{aligned}
                \end{equation}
                We let $f_i$ be as before.
                We have exactly the same cases. For case 1 we let $i$ be the last
                such index. Then similarly to before we get that $o_{i+1} \in
                \accRoots(FS, H)$ and $q \in \reachable(\accRoots(FS, H),
                \Graph(H))$.
                
                Case 2 is analogous and reaches the conclusion that $q \in
                \reachable(\accRoots(HS, H), \Graph(H))$.

                Lastly we prove that if $a = \ocap$ or $b = \ocap$ then
                \eqref{eq:lem2.15c} holds. To prove this implication we assume
                that either $a = \ocap$ or $b = \ocap$. As before we take any
                \begin{equation}
                  q \in \reachable(\accRoots(HS, H'), \Graph(H')).
                \end{equation}
                Similarly we get that there there must be an $o \in
                \accRoots(HS, H')$ such that
                \begin{equation}
                  \exists \text{ path } o_0, \dots, o_n \in \Graph(H') \text{ s.t. } o_0 = o,
                  o_n = q
                \end{equation}
                with related field names $f_i$. We get the exact same cases as
                above, but we instead let $i$ be the first index with the
                properties described. As noted this means we have the edge
                $(o_y, o_z)_f$ as part of our path. Specifically this means
                \begin{equation}\label{eq:lemma2.15-p3-1}
                  o_y \in \reachable(\accRoots(HS, H), \Graph(H))
                \end{equation}
                since $o_0, \cdot, o_i$ is a path in \Graph from $o$ to $o_y$.
                Moreover
                \begin{equation}\label{eq:lemma2.15-p3-2}
                  o_y \in \reachable(\accRoots(FS, H), \Graph(H))
                \end{equation}
                since $(y \mapsto o_y) \in L$. From the {\sc E-Assign} rule
                definition it is clear that
                \begin{equation}\label{eq:lemma2.15-p3-3}
                  \typeOf(o_y, H) \not\stof \CellType.
                \end{equation}
                Now \eqref{eq:lemma2.15-p3-1}, \eqref{eq:lemma2.15-p3-2} and
                \eqref{eq:lemma2.15-p3-3} contradicts isolation. To see this
                first note that $H \vdash P$ together with $a = \ocap$ or $b =
                \ocap$ implies $\isolated(H, FS, HS)$. Then apply
                proposition~\ref{prop:2.6}. Since this case contradicts
                isolation it is impossible.
                
                For the second case it is clear that $o_0, \dots, o_n$ is also a
                path from $o$ to $q$ in $\Graph(H)$ which immediately implies
                $q \in \reachable(\accRoots(HS, H), \Graph(H))$.

                This concludes the proof of the lemma.
              \end{proof}

              Continuing with the proof of case {\sc E-Assign} first off we use
              the lemma to prove OCAP reachability.  We first take any thread
              $HS|_b^{\iota'} \in Q$. Studying the $\ocr$ rules it is clear that we can
              prove that $H'; b \vdash HS \tsep \ocr$ holds trivially for any
              thread $HS|_b^{\iota'}$ for which $b = \nocap$. Therefore we from now on
              assume that $b = \ocap$.  Using proposition \ref{prop:ocr_eq} we
              note that $H \vdash P \tsep \ocr$ iff $H;b \vdash HS \tsep
              \ocr$ which in using proposition \ref{prop:ocrloc_eq} in turn is
              equivalent to  
              \begin{equation}\label{eq:eassign14}
                \begin{aligned}
                  \forall o &\in \reachable(\accRoots(HS, H), H). \\
                            &\ocap(\typeOf(o, H)).
                \end{aligned}
              \end{equation}
              Similarly it is also clear that $H';b \vdash HS \tsep \ocr$ if
              \begin{equation}
                \begin{aligned}
                  \forall o &\in \reachable(\accRoots(HS, H'), \Graph(H')). \\
                            &\ocap(\typeOf(o, H')).
                \end{aligned}
              \end{equation}
              Equations \eqref{eq:eassign14} and \eqref{eq:eassign2} combined
              with the last statement of lemma~\ref{lem:2.15}  makes it clear
              that this is in fact the case.

              In order to prove $H' \vdash P' \tsep \ocr$ the only thing that
              remains is to show $H';a \vdash FS' \tsep \ocr$. This is anologous
              to the above proof of $H';b \vdash HS~\ocr$ using
              \eqref{eq:lem2.15a} instead. Thus we have
              \begin{equation*}
                H' \vdash P' \tsep \ocr.
              \end{equation*}

              Finally $\isolation(H', P')$ follows immediately from
              proposition~\ref{prop:2.6}, lemma~\ref{lem:2.15} and applying rule
              {\sc ISO-Procs}. We also note that nowhere have we relied on the
              fact that $H~\vdash~P~\tsep~\gsep$ so using
              corollary~\ref{cor:2.9} we also have $H'~\vdash~P'~\tsep~\gsep$.

            % TODO add definition of \default, maybe \fdecls also
            \item[Case {\sc E-New}:] According to rule {\sc E-New}
              \begin{equation} \label{eq:enew1}
                \begin{gathered} 
                  F = \sframe{L, t} \andalso t = \Let{x}{\New{C}}{t'} \\
                  F' = \sframe{L', t'} \andalso L' = L[x \mapsto o_x] \\
                  o_x \text{ fresh object reference } \\
                  FM = [f \mapsto \default(\sigma): (\VarDecl{f}{\sigma}) \in
                  \fdecls(C)] \\
                  H' = H[o_x \mapsto \Obj{C, FM}]
                \end{gathered}
              \end{equation}
              If we let $\Graph(H) = (V_H, E_H)$ it is easy to see that
              \begin{equation}
                \Graph(H') = ( V_H \cup \left\{ o_x \right\}, E_H )
              \end{equation}
              By definition of $\default$
              \begin{equation} \label{eq:enew-h1}
                \begin{aligned}
                  \forall f &\in \fields(C).\\
                  &f \in \dom(FM) \land \typeOf(FM(f), H') \stof \ftype(f, C).
                \end{aligned}
              \end{equation}
              Also
              \begin{gather}
                \forall o \in \dom(H). \: H(o) = H'(o) \label{eq:enew-h2}\\
                \forall o \in \dom(H). \: \typeOf(o, H) = \typeOf(o, H').
                \label{eq:enew-h3}
              \end{gather}
              Using the definition of a well typed heap together with
              \eqref{eq:enew-h1}, \eqref{eq:enew-h2} and \eqref{eq:enew-h3} we
              can easily show that $\vdash H'$.

              By proposition~\ref{prop:2.13}
              \begin{equation} 
                H \vdash Q \andalso H; a \vdash FS.
              \end{equation}
              Propositions \ref{prop:2.13} and \ref{prop:2.14} implies
              \begin{equation} \label{eq:enew-typing1}
                H' \vdash Q.
              \end{equation}
              $H;a \vdash FS$ and {\sc T-FS1} yields
              \begin{equation}
                \begin{gathered}
                  H \vdash \Gamma; L \andalso \TypeRel{\Gamma}{a}{t}{\sigma'}
                  \\
                  \sigma' \stof \sigma \andalso H;a \vdash^{s: \sigma} GS.
                \end{gathered} 
              \end{equation}
              Applying {\sc T-Let} and {\sc T-New} we get
              \begin{equation} \label{eq:enew-typing5}
                \TypeRel{\Gamma, x: C}{a}{t'}{\sigma'} \andalso a = \ocap
                \implies \ocap(C).
              \end{equation}
              Letting $\Gamma' = \Gamma, x: C$ we have
              \begin{equation} \label{eq:enew-typing2}
                \TypeRel{\Gamma'}{a}{t'}{\sigma'}.
              \end{equation}
              We have that $H' \vdash \Gamma';L';x$ since $\typeOf(L'(x), H')
              \stof \Gamma'(x)$. Thus we have
              \begin{equation} \label{eq:enew-typing3}
                H' \vdash \Gamma'; L'.
              \end{equation}
              By proposition \ref{prop:2.14}
              \begin{equation} \label{eq:enew-typing4}
                H';a \vdash^{s: \sigma} GS.
              \end{equation}
              Using {\sc T-FS1} together with $\sigma' \stof \sigma$, \eqref{eq:enew-typing2},
              \eqref{eq:enew-typing3} and \eqref{eq:enew-typing4} we get
              \begin{equation}
                H';a \vdash FS',
              \end{equation}
              and combined with \eqref{eq:enew-typing1} and {\sc T-Procs} we finally
              get
              \begin{equation}
                H' \vdash P'.
              \end{equation}

              Moving on to OCAP reachability. It is easy to prove that
              \begin{equation} \label{eq:enew-ocr1}
                \begin{aligned}
                  &\reachable(\accRoots(HS, H'), \Graph(H')) \subseteq \\
                  &\reachable(\accRoots(HS, H), \Graph(H)) 
                \end{aligned}
              \end{equation}
              and
              \begin{equation} \label{eq:enew-ocr2}
                \begin{aligned}
                  &\reachable(\accRoots(FS', H'), \Graph(H')) \subseteq \\
                  &\reachable(\accRoots(FS, H), \Graph(H)) \cup \left\{ o_x
                  \right\}.
                \end{aligned}
              \end{equation}
              Clearly $\typeOf(o_x, H') = C$. Using this fact together with
              \eqref{eq:enew-h3}, \eqref{eq:enew-typing5}, \eqref{eq:enew-ocr1},
              \eqref{eq:enew-ocr2} and propositions \ref{prop:ocr_eq} and
              \ref{prop:ocrloc_eq} it is easily shown that
              \begin{equation}
                H' \vdash P' \tsep \ocr.
              \end{equation}

              Similarly it is easy to show $\isolation(H', P')$ using
              \eqref{eq:enew-ocr1}, \eqref{eq:enew-ocr2} together with
              proposition~\ref{prop:2.6}. Since the isolation proof never relies
              on $H \vdash P \tsep \gsep$ we can apply
              corollary~\ref{cor:2.9}. Thus we have proved
              \begin{equation}
                \isolation(H', P') \andalso H' \vdash P' \tsep \gsep
              \end{equation}

            \item[Case {\sc E-NewCell}:] We have
              \begin{equation}
                H' = H[o \mapsto \Cell{DEP, \bot_{\LatVals}}] \andalso DEP = \emptyset
              \end{equation}
              for some fresh object reference $o$. 
              Since $DEP$ is empty \eqref{eq:defwth2} holds vacuously and since
              all other heap elements are the same as in $H$ we have
              \begin{equation}
                \vdash H'.
              \end{equation}
              Proving the rest is similar to case {\sc E-New}.

            \item[Case {\sc E-Put}:] According to rule {\sc E-Put}
              \begin{equation}
                \begin{gathered}
                  F = \sframe{L, t} \andalso t = \Let{x}{\Put{y}{z}}{t'} \\
                  F' = \sframe{L', t'} \andalso L' = L[x \mapsto L(y)] \\
                  L(y) = o_y \andalso H(o_y) = \Cell{DEP, l} \\
                  L(z) = l' \andalso H' = H[o_y \mapsto \Cell{DEP, l \sqcup
                  l'}].
                \end{gathered}
              \end{equation}
              Since the only change is $l$ being updated to $l \sqcup l'$, by
              studying the definition for a well typed heap and using $\vdash H$
              it is easy to conclude that
              \begin{equation}
                \vdash H'.
              \end{equation}
              The proof of $H' \vdash P'$ follows similar to case {\sc E-Var}.
              The preconditions of corollary~\ref{cor:2.11} clearly holds and we
              can thus also apply corollary~\ref{cor:2.9} to get
              \begin{equation}
                H' \vdash P' \tsep \ocr \andalso \isolation(H', P') \andalso H'
                \vdash P' \tsep \gsep
              \end{equation}

            \item[Case {\sc E-When}:] From rule {\sc E-When} we know
              \begin{equation}
                \begin{gathered}
                  F = \sframe{L, t} \\ 
                  t = \Let{x}{\When{y}{z}{\CB{\overline{cap}}{w}{t''}}}{t'} \\
                  F' = \sframe{L', t'} \andalso L' = L[x \mapsto L(y)] \\
                  L(y) = o_y \andalso H(o_y) = \Cell{DEP, l} \andalso L(z) = l' \\
                  L_{\text{env}} = [u \mapsto L(u'): (\Capt{u}{u'}) \in
                  \overline{cap}] \andalso cb = (L_{\text{env}}, w \Rightarrow
                  t'') \\
                  d \text{ fresh thread id} \andalso DEP' = DEP \cup (l', cb)^d
                  \\
                  H' = H[o \mapsto \Cell{DEP', l}].
                \end{gathered}
              \end{equation}
              Since only object on heap changed is $H(o_y)$ we have shown
              $\vdash H'$ if we can show that
              \begin{equation}
                \begin{aligned}
                  \forall (l^{\text{cb}}&, (L^{\text{cb}}_{\text{env}}, z^{\text{cb}} \Rightarrow t^{\text{cb}}))^{d^{\text{cb}}} \in
                  DEP'. \\
                  &\forall (x \mapsto k) \in L^{\text{cb}}_{\text{env}}.\: \typeOf{(k, H')} \stof
                  \CellType \: \land \\
                  &\TypeRel{\Gamma_{\CellType{}}(L^{\text{cb}}_{\text{env}}), z^{\text{cb}}:
                  \LatType}{\ocap}{t^{\text{cb}}}{\gamma} .
                \end{aligned}
              \end{equation}
              We note that $\dom(H) = \dom(H')$ and that
              \begin{equation}
                \begin{aligned}
                  \forall o &\in \dom(H).  \\
                    &\typeOf(o, H) = \typeOf(o, H'),
                \end{aligned}
              \end{equation}
              and thus
              \begin{equation}
                \begin{aligned}
                  \forall k \in \Values . \\
                    &\typeOf(k, H) = \typeOf(k, H').
                \end{aligned}
              \end{equation}
              Because of this
              \begin{equation}
                \begin{aligned}
                  \forall (l^{\text{cb}}&, (L^{\text{cb}}_{\text{env}},
                  z^{\text{cb}} \Rightarrow t^{\text{cb}}))^{d^{\text{cb}}} \in
                  DEP' \text{ s.t. }  \\
                  &(l^{\text{cb}}, (L^{\text{cb}}_{\text{env}}, z^{\text{cb}}
                  \Rightarrow t^{\text{cb}}))^{d^{\text{cb}}} \neq
                  (l', cb)^d. \\
                  &\forall (x \mapsto k) \in L^{\text{cb}}_{\text{env}}.\: \typeOf{(k, H')} \stof
                  \CellType \: \land \\
                  &\TypeRel{\Gamma_{\CellType{}}(L^{\text{cb}}_{\text{env}}), z^{\text{cb}}:
                  \LatType}{\ocap}{t^{\text{cb}}}{\gamma}.
                \end{aligned}
              \end{equation}
              Thus what remains to prove $\vdash H'$ is to show that
              \begin{equation} \label{eq:ewhen-h1}
                \forall (x \mapsto k) \in L_{\text{env}}.\: \typeOf{(k, H')} \stof
                \CellType
              \end{equation}
              and
              \begin{equation} \label{eq:ewhen-h2}
                \TypeRel{\Gamma_{\CellType{}}(L_{\text{env}}), w:
                \LatType}{\ocap}{t''}{\gamma}.
              \end{equation}
              Similarly to earlier cases, using typing rules we can easily prove
              that
              \begin{equation}\label{eq:ewhen-h3}
                H \vdash \Gamma; L \andalso \forall (\Capt{u}{u'}) \in
                \overline{cap}. \: \TypeRel{\Gamma}{a}{u'}{\CellType} 
              \end{equation}
              and
              \begin{equation} \label{eq:ewhen-h4}
                \begin{gathered}
                  \Gamma_{\text{cells}} = [u \mapsto \CellType: (\Capt{u}{u'}) \in
                  \overline{cap}] \\
                  \TypeRel{\Gamma_{\text{cells}}, w:
                  \LatType}{\ocap}{t''}{\gamma'}.
                \end{gathered}
              \end{equation}
              \eqref{eq:ewhen-h4} is just a different formulation of
              \eqref{eq:ewhen-h2} since the actual value of $\gamma'$ does not
              matter. Moreover with \eqref{eq:ewhen-h3} we can easily show
              \eqref{eq:ewhen-h2} by using the definitions of {\sc WF-EnvVar}
              and {\sc WF-Env}. Thus we are done and have
              \begin{equation}
                \vdash H'.
              \end{equation}
              Proof of $H' \vdash P'$ is similar to case {\sc E-Var} and
              similarly to other cases we can apply corollaries \ref{cor:2.11}
              and \ref{cor:2.9} to get
              \begin{equation}
                H' \vdash P' \tsep \ocr \andalso \isolation(H', P') \andalso H'
                \vdash P' \tsep \gsep.
              \end{equation}

              {\bf This concludes the case} {\sc E-FProp}.
          \end{description}

        \item[Case {\sc E-Call}:] From rule {\sc E-Call}
          \begin{equation} \label{eq:ecall1}
            \begin{gathered}
              FS = \sframe{L, t} \circ GS \andalso t = \Let{x}{\Call{y}{m}{z}}{t''}\\
              FS' = \xframe{L', t'}^x \circ \sframe{L, t} \circ GS \\
              L(y) = o_y \andalso H(o_y) = \Obj{C, FM} \\
              \mbody(m, C) = u \to t' \\
              L_{\text{base}} =
              \begin{cases}
                \emptyset & \text{ if } a = \ocap \\
                L_0       & \text{ if } a = \nocap
              \end{cases} \\
              L' = L_{\text{base}}[\This \mapsto L(y), u \mapsto L(z)] \\
              H = H'.
            \end{gathered}
          \end{equation}
          Since $H = H'$ 
          \begin{equation*}
            \vdash H'.
          \end{equation*}
          By propositions~\ref{prop:2.13} and~\ref{prop:2.14}, $H = H'$ and
          $H \vdash P$ we have
          \begin{equation}
            \begin{gathered}
              H' \vdash Q \andalso H;a \vdash FS.
            \end{gathered}
          \end{equation}
          Using the second of these and rule {\sc T-FS1} we have
          \begin{equation} \label{eq:ecall-hp0}
            \begin{gathered}
              H \vdash \Gamma; L \andalso \TypeRel{\Gamma}{a}{t}{\sigma'} \\
              \sigma' \stof \sigma \andalso H; a \vdash^{s:\sigma} GS.
            \end{gathered}
          \end{equation}
          Since $H = H'$
          \begin{equation} \label{eq:ecall-hp5}
            H' \vdash \Gamma;L \andalso H'; a \vdash^{s: \sigma} GS.
          \end{equation}
          Using $\TypeRel{\Gamma}{a}{t}{\sigma}$ with rules {\sc T-Let, T-Call} and
          {\sc T-Var} we have that
          \begin{equation} \label{eq:ecall-hp1}
            \begin{gathered}
              \TypeRel{\Gamma}{a}{\Call{y}{m}{z}}{\tau} \andalso \TypeRel{\Gamma,
              x: \tau}{a}{t''}{\sigma'} \\
              \Gamma(y) = C' \andalso \mtype(m, C') = \gamma \mapsto \tau \\
              \Gamma(z) = \gamma' \andalso \gamma' \leq \gamma.
            \end{gathered}
          \end{equation}
          Using {\sc T-FS2} with \eqref{eq:ecall-hp0}, \eqref{eq:ecall-hp5} and
          \eqref{eq:ecall-hp1} we get 
          \begin{equation} \label{eq:ecall-hp6}
            H';a \vdash^{x: \tau} FS.
          \end{equation}
          By $H' \vdash \Gamma;L$ we have
          \begin{equation} \label{eq:ecall-hp2}
            \typeOf(L'(u), H') = \typeOf(L(z), H') \stof \gamma' \stof \gamma,
          \end{equation}
          and similarly
          \begin{equation} \label{eq:ecall-hp3}
            \typeOf(L'(\This), H') = \typeOf(L(y), H') \stof C \stof C'.
          \end{equation}

          Now, $a$ can be either $\nocap$ or $\ocap$. We first assume that $a =
          \nocap$.
          By well typedness of classes together with the above we must have a
          class $C''$ s.t. $C \stof C'' \stof C'$ where such that
          \begin{equation}
            C'' \vdash \MethodDef{m}{u}{\gamma}{\tau}{t'}
          \end{equation}
          which by $C''$ being well typed in turn means that
          \begin{equation} \label{eq:ecall-hp4}
            \TypeRel{\Gamma_0,\This: C'', u: \gamma}{\nocap}{t'}{\tau'}
            \andalso \tau' \stof \tau.
          \end{equation}
          We let
          \begin{equation}
            \Gamma_{\nocap}' = \Gamma_0, \This: C'', u: \gamma.
          \end{equation}
          It is not hard to prove that 
          \begin{equation} \label{eq:ecall-hp7}
            H' \vdash \Gamma_{\nocap}'; L'.
          \end{equation}
          By \eqref{eq:ecall-hp4} we clearly have
          \begin{equation} \label{eq:ecall-hp8}
            \TypeRel{\Gamma_{\nocap}'}{a}{t'}{\tau'} \andalso \tau' \stof \tau.
          \end{equation}
          Using  \eqref{eq:ecall-hp6}, \eqref{eq:ecall-hp7},
          \eqref{eq:ecall-hp8} and rule {\sc T-FS1} we have
          \begin{equation}
            H';a \vdash FS'.
          \end{equation}
          Combining this with $H' \vdash Q$ and {\sc T-Procs} we get
          \begin{equation}
            H' \vdash P'.
          \end{equation}

          Now assume that $a = \ocap$. By $H \vdash P \tsep \ocr$ we have
          \begin{equation}
            \ocrloc(FS, H)
          \end{equation}
          and thus by proposition~\ref{prop:ocrloc_eq}
          \begin{equation}
            \begin{aligned}
              \forall q &\in \reachable(\accRoots(FS, H), \Graph(H)) \\
              & \ocap(\typeOf(q, H)).
            \end{aligned}
          \end{equation}
          
          Of course this means that $\ocap(\typeOf(L(y), H))$ or equivalently
          $\ocap(C)$. The rest is similar to the case where $a = \nocap$
          replacing $\Gamma_{\nocap}'$ with 
          \begin{equation}
            \Gamma'_{\ocap} = \This: C'', u: \gamma.
          \end{equation}
          This can be done because of the OCAP typing rules in
          Figure~\ref{fig:ocap_typing} which demands that: all parent classes of
          $C$ must also be OCAP typed which means $\ocap(C'')$, and thus that
          all $C''$s methods are typed in an environment without $\Gamma_0$.
          Hence the difference between $\Gamma'_{\nocap}$ and $\Gamma'_{\ocap}$.

          We now prove isolation, OCAP reachability and global object
          separation. Instead of working with $P$ directly we instead use
          $\tilde{P}$. It is clear from propositions~\ref{prop:2.8} and
          \ref{prop:ocrtilde_eq} that in order to show 
          \begin{equation}
            H' \vdash P' \tsep \ocr \andalso \isolation(H', P') \andalso H'
            \vdash P' \tsep \gsep
          \end{equation}
          we can instead show
          \begin{equation}
            H' \vdash \tilde{P'} \tsep \ocr \andalso \isolation(H', \tilde{P'}).
          \end{equation}

          To show this we start by using propositions~\ref{prop:2.8} and
          \ref{prop:ocrtilde_eq} to get
          \begin{equation}
            H \vdash \tilde{P} \tsep \ocr \andalso \isolation(H, \tilde{P}).
          \end{equation}
          We also note that 
          \begin{equation}
            \tilde{P} = \tilde{Q} \cup_D \left\{ FS|_a^\iota \right\} \text{ and }
            \tilde{P'} = \tilde{Q} \cup_D \left\{ FS'|_a^\iota \right\}.
          \end{equation}

          We first assume that $a = \ocap$. It is very
          easy to see that
          \begin{equation}
            \accRoots(FS', H') \subseteq \accRoots(FS, H)
          \end{equation}
          and that
          \begin{equation}
            \begin{aligned}
              \forall HS|_b^{\iota'} &\in \tilde{Q}. \\
              &\accRoots(HS, H') = \accRoots(HS, H).
            \end{aligned}
          \end{equation}
          Clearly then the preconditions to corollary~\ref{cor:2.11} holds and
          we have $\isolation(H', \tilde{P'})$ and $H' \vdash \tilde{P'} \tsep
          \ocr$. Thus we are done.

          Now instead assume $a = \nocap$. Since $\isolation(H, \tilde{P})$ we have
          $\isolated(H, HS_1, HS_2)$ for any two distinct $HS_1|_b^{\iota'},
          HS_2|_c^{\iota''}
          \in \tilde{Q}$ such that $b = \ocap$ or $c = \ocap$. Therefore since $H = H'$
          \begin{equation}
            \begin{aligned}
              \forall &\text{ distinct } HS_1|_b^{\iota'}, HS_2|_c^{\iota''} \in \tilde{Q}. \\ 
              & b = \ocap \lor c = \ocap \implies \isolated(H', HS_1, HS_2).
            \end{aligned}
          \end{equation}
          Thus all we need for $\isolation(H', \tilde{P'})$ is to show that
          \begin{equation}
            \begin{aligned}
              \forall HS|_b^{\iota'} &\in \tilde{Q} . \\
              &b = \ocap \implies \isolated(H', FS', HS)
            \end{aligned}
          \end{equation}
          since $a = \nocap$. To prove this implication we take any
          $HS|_b^{\iota'} \in
          \tilde{Q}$ such that $b = \ocap$. Because of $a = \nocap$ and the
          definition of $L'$ in \eqref{eq:ecall1}
          \begin{equation} \label{eq:ecall-iso1}
            \accRoots(FS', H') = \accRoots(FS, H) \cup \left\{ o_g \right\}.
          \end{equation}
          Since $\isolation(H, \tilde{P})$ we have
          \begin{equation} \label{eq:ecall-iso2}
            \isolation(H, HS, FS_g).
          \end{equation}
          By definition
          \begin{equation} \label{eq:ecall-iso3}
            \accRoots(FS_g, H) = \left\{o_g \right\}.
          \end{equation}
          By $\isolation(H, \tilde{P})$ 
          \begin{equation} \label{eq:ecall-iso4}
            \isolated(H, FS, HS).
          \end{equation}
          By inspecting \eqref{eq:ecall-iso1}, \eqref{eq:ecall-iso2},
          \eqref{eq:ecall-iso3} and \eqref{eq:ecall-iso4} and considering
          proposition~\ref{prop:2.6} it is fairly easy to show
          \begin{equation}
            \isolated(H, FS', HS).
          \end{equation}
          To show this, assume for a contradiction that we do not have
          $\isolated(H, FS', HS)$. This would entail that
          \begin{equation}
            \begin{aligned}
              \exists q &\in \dom(H), o \in \accRoots(HS, H), o' \in
              \accRoots(FS', H) . \\
              &\typeOf(q, H) \not\stof \CellType \land \reach(H, o, q) \land
              \reach(H, o', q).
            \end{aligned}
          \end{equation}
          If $o' = o_g$ this immediately contradicts $\isolated(H, HS, FS_g)$ by
          proposition~\ref{prop:2.6}. If we instead have $o' \neq o_g$ then by
          \eqref{eq:ecall-iso1} we must have
          \begin{equation}
            o' \in \accRoots(FS, H).
          \end{equation}
          Similarly this would contradict \eqref{eq:ecall-iso4}.  Since
          $HS|_b^{\iota'}$ was arbitrary and $H = H'$ we have 
          \begin{equation}
            \isolation(H', \tilde{P'}).
          \end{equation}

          By $H \vdash \tilde{P} \tsep \ocr$ and
          proposition~\ref{prop:ocrtilde_eq} we get
          \begin{equation} \label{eq:ecall-ocr1}
            H \vdash \tilde{Q} \tsep \ocr.
          \end{equation}
          Since $a = \nocap$ we have that 
          \begin{equation} \label{eq:ecall-ocr2}
            H'; a \vdash FS' \tsep \ocr 
          \end{equation} 
          follows vacuously. Using rule {\sc OCR-P} in
          Figure~\ref{fig:def_ocapreach} together with \eqref{eq:ecall-ocr1} and
          \eqref{eq:ecall-ocr2} we get
          \begin{equation}
            H' \vdash \tilde{P'} \tsep \ocr.
          \end{equation}  
          Thus we are done with case $a = \nocap$.
          
        \item[Case {\sc E-Ret}:] Clearly from rule {\sc E-Ret} we have
          \begin{equation}
            \begin{gathered}
              FS = \xframe{L, z}^x \circ \sframe{L', t'} \circ GS \\
              FS' = \sframe{L'', t'} \circ GS \\
              L'' = L'[x \mapsto L(z)] \\
              H = H'.
            \end{gathered}
          \end{equation}
          By $H \vdash P$ and proposition~\ref{prop:2.13} we have
          \begin{equation} \label{eq:eret-hp0}
            H \vdash Q \andalso H; a \vdash FS.
          \end{equation}
          $H = H'$ immediately yields
          \begin{equation}
            H' \vdash Q.
          \end{equation}
          $H; a \vdash FS$ together with rule {\sc T-FS1} gives us that
          \begin{equation}
            \begin{gathered}
              H \vdash \Gamma; L \andalso \TypeRel{\Gamma}{a}{z}{\tau'} \\
              \tau' \stof \tau \andalso H; a \vdash^{x: \tau} \sframe{L', t'} \circ GS,
            \end{gathered}
          \end{equation}
          the last of which together with {\sc T-FS2} gives us
          \begin{equation} \label{eq:eret-hp1}
            \begin{gathered}
              H \vdash \Gamma'; L' \andalso \TypeRel{\Gamma', x:
              \tau}{a}{t'}{\sigma'} \\
              \sigma' \stof \sigma \andalso H; a \vdash^{s: \sigma} GS
            \end{gathered}
          \end{equation}
          We let $\Gamma'' = \Gamma', x : \tau$. By $H = H'$, $H \vdash \Gamma;
          L$ and $\tau' \stof \tau$ it should be obvious that
          \begin{equation} \label{eq:eret-hp3}
            \typeOf(L''(x), H') = \typeOf(L(z), H') \stof \tau.
          \end{equation}
          Using $H \vdash \Gamma'; L'$, \eqref{eq:eret-hp3} and rules {\sc WF-EnvVar}, {\sc
          WF-Env} we get
          \begin{equation} \label{eq:eret-hp4}
            H' \vdash \Gamma'', L''.
          \end{equation}
          By definition of $\Gamma''$, \eqref{eq:eret-hp1}, \eqref{eq:eret-hp4}
          and rule {\sc T-FS1} we have that
          \begin{equation} \label{eq:eret-hp2}
            H';a \vdash FS'
          \end{equation}
          Applying {\sc T-Procs} together with \eqref{eq:eret-hp0} and
          \eqref{eq:eret-hp2} we finally get
          \begin{equation}
            H' \vdash P'
          \end{equation}

          Similarly to many previous cases we can apply corollarys
          \ref{cor:2.11} and \ref{cor:2.9} to get
          \begin{equation*}
            H' \vdash P' \tsep \ocr \andalso \isolation(H', P') \andalso H'
            \vdash P' \tsep \gsep
          \end{equation*}
      \end{description}
      {\bf This concludes the {\sc E-FSProp} case.}
      
    \item[Case {\sc E-Spawn}:] By rule {\sc E-Spawn} we have
      \begin{equation}
        \begin{gathered} \label{eq:espawn1}
          o \in \dom(H) \andalso H(o) = \Cell{DEP, l} \\
          l' \sqsubseteq l \andalso (l', cb)^d \in DEP \andalso cb =
          (L_{\text{env}}, z \Rightarrow t') \\
          L' = L_{\text{env}}[z \mapsto l'] \\
          H' = H[o \mapsto \Cell{DEP - (l', cb)^d, l}] \\
          P' = P \cup \left\{ FS'|_{\ocap}^d \right\} \\
          FS' = \xframe{L', t'}^- \circ \varepsilon.
        \end{gathered}
      \end{equation}
      
      By definition of $H'$, $\vdash H$ and equation \eqref{eq:defwth2} from the
      definition of the well typed heap it should be fairly obvious that
      \begin{equation}
        \vdash H'
      \end{equation}
      since we have only removed an element from $DEP$. 


      We note that
      \begin{equation} \label{eq:espawn-hp0}
        \begin{aligned}
          \forall o &\in \dom(H) = \dom(H'). \\
            &\typeOf(o, H) = \typeOf(o, H').
        \end{aligned}
      \end{equation}
      By propositions \ref{prop:2.13} and \ref{prop:2.14} we have
      \begin{equation} \label{eq:espawn-hp4}
        H' \vdash P
      \end{equation}
      From $\vdash H$ we can see that
      \begin{equation} \label{eq:espawn-hp1}
        \begin{aligned}
          \forall (x \mapsto k) &\in L_{\text{env}}. \\
            &\typeOf(k, H) \stof \CellType
        \end{aligned}
      \end{equation}
      and
      \begin{equation} \label{eq:espawn-hp2}
        \TypeRel{\Gamma_{\CellType}(L_{\text{env}}), z:
        \LatType}{\ocap}{t'}{\gamma}.
      \end{equation}
      Letting
      \begin{equation}
        \Gamma' = \Gamma_{\CellType}(L_{\text{env}}), z: \LatType
      \end{equation}
      we can use \eqref{eq:espawn-hp0},\eqref{eq:espawn-hp1} and rules {\sc
      WF-EnvVar, WF-Env} to get
      \begin{equation} \label{eq:espawn-hp3}
        H' \vdash \Gamma', L'
      \end{equation}
      By rules {\sc T-FSEmpty2, T-FS1} and \eqref{eq:espawn-hp2},
      \eqref{eq:espawn-hp3} we get
      \begin{equation}
        H'; \ocap \vdash FS'
      \end{equation}
      which combined with \eqref{eq:espawn-hp4} and {\sc T-Procs} yields
      \begin{equation}
        H' \vdash P'.
      \end{equation}
      
      Next we prove OCAP reachability. It is easy to see that
      \begin{equation} \label{eq:espawn-ocr0}
        \Graph(H') = \Graph(H).
      \end{equation}
      Because of this, \eqref{eq:espawn-hp0} and propositions \ref{prop:ocr_eq}
      and \ref{prop:ocrloc_eq}
      \begin{equation}
        H' \vdash P \tsep \ocr.
      \end{equation}
      This leaves only to prove
      \begin{equation}
        H'; \ocap \vdash FS' \tsep \ocr.
      \end{equation}
      in order to get $H' \vdash P' \tsep \ocr$.
      Inspecting {\sc OCR-FS} we note that we are done if we can prove
      \begin{equation} \label{eq:espawn-ocr1}
        \ocrloc(FS', H').
      \end{equation}
      But this is easy. By definition of $FS'$
      \begin{equation}
        \forall o \in \accRoots(FS', H'). \: \typeOf(o, H') \stof \CellType
      \end{equation}
      which implies
      \begin{equation} \label{eq:espawn-ocr2}
        \begin{aligned}
          \forall o &\in \reachable(\accRoots(FS', H'), \Graph(H')). \\
            &\typeOf(o, H') \stof \CellType
        \end{aligned}
      \end{equation}
      The only objects of type \CellType{} on the heap are cell objects 
      whose exact type are \CellType{}, which is $\ocap$ by {\sc OCAP-Cell}.
      Thus
      \begin{equation}
        \begin{aligned}
          \forall o &\in \reachable(\accRoots(FS', H'), \Graph(H')). \\
            &\ocap(\typeOf(o, H')).
        \end{aligned}
      \end{equation}
      By proposition \ref{prop:ocrloc_eq} we have shown \eqref{eq:espawn-ocr1}.

      Continuing with showing isolation we note that by \eqref{eq:espawn-hp0},
      \eqref{eq:espawn-ocr0} and \ref{prop:2.6} we have that
      \begin{equation}
        \isolation(H', P).
      \end{equation}
      Thus by proposition \ref{prop:2.6} we are done if we can prove 
      \begin{equation}
        HS|_b^{\iota'} \in P. \: \isolated(H', FS', HS).
      \end{equation}
      Again using proposition~\ref{prop:2.6} we are done if we show
      \begin{equation}
        \begin{aligned}
          \forall q \in \: &\reachable(\accRoots(FS', H'), \Graph(H')) \cap \\
            &\reachable(\accRoots(HS, H'), \Graph(H')) . \\
            &\typeOf(q, H') \stof \CellType.
        \end{aligned}
      \end{equation}
      But this follows immediately from \eqref{eq:espawn-ocr2} and 
      by the above we have
      \begin{equation}
        \isolation(H', P').
      \end{equation} 
      Furthermore we never relied on the fact that $H \vdash P \tsep \gsep$ to
      prove this and thus can get
      \begin{equation}
        H' \vdash P' \tsep \gsep 
      \end{equation}
      for free using corollary~\ref{cor:2.9}.

    \item[Case {\sc E-Term}:] Trivial.
  \end{description}
\end{proof}


\section{Proof of Progress}
\label{sec:proof_of_progress}

\begin{theorem*}[Progress]
  Let $S$ be a state such that $\vdash S \tsep \stateok$. Then either 
  \begin{enumerate}
    \item $\exists S'$ s.t. $S \Rrightarrow S'$, 
    \item $S = H, \emptyset$ for some heap $H$ s.t. $\noSpawn{(H)}$ or
    \item $S = \Error$.
  \end{enumerate}
\end{theorem*}

\begin{proof}
  Assume for a contradiction that $\vdash S \tsep \stateok$ but that neither 1,
  2 or 3 from the theorem holds. We must then have that the following three
  statements hold
  \begin{enumerate}
    \item $\not\exists S'$ s.t. $S \Rrightarrow S'$.
    \item $S = H, \emptyset$ but $\noSpawn(H)$ does not hold, or $S = H, P$
      where $P \neq \emptyset$.
    \item $S \neq \Error$.
  \end{enumerate}
  We have two cases: Either $S = H, \emptyset$ and $\noSpawn(H)$ does not hold,
  or $S = H, P$ and $P \neq \emptyset$.

  We begin with the case where $S = H, \emptyset$ and $\noSpawn(H)$ does not
  hold. Looking at the definition of $\noSpawn$ this means that there is at
  least one $o \in \dom(H)$ such that, for $H(o) = \Cell{DEP, l}$
  \begin{equation*}
    \exists (l', cb)^d \in DEP. \: l' \sqsubseteq l.
  \end{equation*}
  This clearly satisfies the preconditions of execution rule {\sc E-Spawn} to
  find another state $S' = H', P'$ such that $S \Rrightarrow S'$. But this
  contradicts statement 1 above.

  We proceed with the case where $S = H, P$ and $P \neq \emptyset$. This means
  that
  \begin{equation*}
    P = Q \cup_D \left\{ FS|_a^\iota \right\} \andalso FS = \sframe{L, t} \circ GS
  \end{equation*}
  By $\vdash S \tsep \stateok$ we must have
  \begin{equation*}
    H \vdash P.
  \end{equation*}
  By proposition \ref{prop:2.13} this means that 
  \begin{equation*}
    H;a \vdash FS.
  \end{equation*}
  Inspecting rule {\sc T-FS1} we see that
  \begin{equation} \label{eq:prog_tbase}
    \begin{gathered}
      H \vdash \Gamma;L \andalso \TypeRel{\Gamma}{a}{t}{\sigma'} \\
      \sigma' \stof \sigma \andalso H;a \vdash^{s: \sigma} GS.
    \end{gathered}
  \end{equation}
  We proceed by cases on the form of $t$. 
  
  \begin{remark}
    Note that we only state which "base rule" is used for each step found below.
    This means that if, e.g., the rule used is a frame reduction rule (one of the
    rules defined in Figure~\ref{fig:frame_red_rules})the application of {\sc
    E-FProp} and {\sc E-FSProp} is implied.
  \end{remark}

  \begin{description}
    \item[Case $t = x$:] 
      If $GS = \varepsilon$ then we can make a step to $S' = H, Q$ by applying
      rule {\sc E-Term}.

      Otherwise we have that $GS = \xframe{L', t'}^{s'} \circ HS$. This means we
      can step to $S' = H, P'$ where
      \begin{equation*}
        \begin{gathered}
          P' = Q \cup_D \left\{ FS'|_a^\iota \right\} \andalso FS' = \xframe{L'',
          t'}^{s'} \circ HS \\ 
          L'' = L'[s \mapsto L(x)],
        \end{gathered}
      \end{equation*}
      by base rule {\sc E-Ret}. \contradiction
    
    \item[Case $t = \Let{x}{e}{t'}$:]
      We proceed on cases of $e$.
      \begin{description}
        \item[Case $e = \NullVal$:] 
          We can step to $S' = H, P'$ where
          \begin{equation*}
            \begin{gathered}
              P' = Q \cup \left\{FS'|_a^\iota \right\} \andalso FS' = \sframe{L',
              t'} \\
              L' = L[x \mapsto \NullVal]
            \end{gathered}
          \end{equation*}
          by base rule {\sc E-Null}. \contradiction

        \item[Case $e = l$:]
          We can step to $S' = H, P'$ where 
          \begin{equation*}
            \begin{gathered}
              P' = Q \cup \left\{FS'|_a^\iota \right\} \andalso FS' = \sframe{L',
              t'} \\
              L' = L[x \mapsto l]
            \end{gathered}
          \end{equation*}
          by rule {\sc E-LVal}. \contradiction

        \item[Case $e = y$:]
          We can step to $S' = H, P'$ where
          \begin{equation*}
            \begin{gathered}
              P' = Q \cup \left\{FS'|_a^\iota \right\} \andalso FS' = \sframe{L',
              t'} \\
              L' = L[x \mapsto L(y)]
            \end{gathered}
          \end{equation*}
          by rule {\sc E-Var}. \contradiction

        \item[Case $e = \FSel{y}{f}$:]
          By \eqref{eq:prog_tbase}, {\sc T-Let}, {\sc T-Select} and {\sc T-Var}
          \begin{equation*}
            \Gamma(y) = C \andalso \ftype(f, C) = \tau.
          \end{equation*}
          By $H \vdash \Gamma;L$
          \begin{equation*}
            \typeOf(L(y), H) \stof C
          \end{equation*}
          which means that either $\typeOf(L(y), H) = C' \stof C$ or \\
          $\typeOf(L(y), H) = \NullType$. 
          
          If $\typeOf(L(y), H) = \NullType$ by definition of $\typeOf$
          \begin{equation*}
            L(y) = \NullVal
          \end{equation*}
          Then we can step to $S' = \Error$ by {\sc E-NullSelect}.
          \contradiction

          If $\typeOf(L(y), H) = C'$ then
          \begin{equation*}
            L(y) = o_y \andalso H(o_y) = \Obj{C', FM}
          \end{equation*}
          Since $C' \stof C$ and $\ftype(f, C) = \tau$
          \begin{equation*}
            f \in \fields(C') \andalso \ftype(f, C') = \tau.
          \end{equation*}
          Then $\vdash H$ implies
          \begin{equation*}
            f \in \dom(FM).
          \end{equation*}
          But then by rule {\sc E-Select} we can step to $S' = H, P'$ where
          \begin{equation*}
            \begin{gathered}
              P' = Q \cup \left\{FS'|_a^\iota \right\} \andalso FS' = \sframe{L',
              t'} \\
              L' = L[x \mapsto FM(f)].
            \end{gathered}
          \end{equation*}
          \contradiction

        \item[Case $e = \FAss{y}{f}{z}$:]
          By \eqref{eq:prog_tbase}, {\sc T-Let}, {\sc T-Assign} and {\sc T-Var}
          \begin{equation*}
            \begin{gathered}
              \Gamma(y) = C \andalso \ftype(f,C) = \tau  \\
              \Gamma(z) = \tau' \andalso \tau' \stof \tau.
            \end{gathered}
          \end{equation*}
          By $H \vdash \Gamma;L$
          \begin{equation*}
            \typeOf(L(y), H) \stof C,
          \end{equation*}
          which means that either $\typeOf(L(y), H) = C' \stof C$ or \\
          $\typeOf(L(y), H) = \NullType$.

          If $\typeOf(L(y), H) = \NullType$ we get
          \begin{equation*}
            L(y) = \NullVal.
          \end{equation*}
          We can then step to $\Error$ by rule {\sc E-NullAssign}.
          \contradiction

          If $\typeOf(L(y), H) = C'$ then
          \begin{equation*}
            L(y) = o_y \andalso H(o_y) = \Obj{C', FM}.
          \end{equation*}
          By $C' \stof C$ and $\ftype(f, C) = \tau$
          \begin{equation*}
            f \in \fields(C') \andalso \ftype(f, C') = \tau.
          \end{equation*}
          Then by $\vdash H$ we have 
          \begin{equation*}
            f \in \dom(FM).
          \end{equation*}
          We can then by {\sc E-Assign} step to $S' = H', P'$ where
          \begin{equation*}
            \begin{gathered}
              P' = Q \cup \left\{FS'|_a^\iota \right\} \andalso FS' = \sframe{L',
              t'} \\
              L' = L[x \mapsto L(z)] \andalso FM' = FM[f \mapsto L(z)] \\
              H' = H[o_y \mapsto \Obj{C', FM'}]
            \end{gathered}
          \end{equation*}
          \contradiction

        \item[Case $e = \New{C}$:]
          By \eqref{eq:prog_tbase}, {\sc T-Let} and {\sc T-New}
          \begin{equation*}
            \TypeRel{\Gamma}{a}{\New{C}}{C}.
          \end{equation*}
          Since the program is well formed the class $C$ exists. 
          \begin{equation*}
            FM = [f \mapsto \default(\tau): (\VarDecl{f}{\tau}) \in \fdecls(C)]
          \end{equation*}
          is therefore well defined. By {\sc E-New} we can therefore step to $S'
          = H', P'$ where
          \begin{equation*}
            \begin{gathered}
              P' = Q \cup \left\{FS'|_a^\iota \right\} \andalso FS' = \sframe{L',
              t'} \\
              L' = L[x \mapsto o] \andalso o \text{ fresh object reference } \\
              H' = H[o \mapsto \Obj{C, FM}].
            \end{gathered}
          \end{equation*}
          \contradiction

        \item[Case $e = \NewCell$:]
          We can immediately step to $S' = H', P'$ where
          \begin{equation*}
            \begin{gathered}
              P' = Q \cup \left\{FS'|_a^\iota \right\} \andalso FS' = \sframe{L',
              t'} \\
              L' = L[x \mapsto o] \andalso o \text{ fresh object reference } \\
              H' = H[o \mapsto \Cell{\emptyset, \bot_{\LatVals}}]
            \end{gathered}
          \end{equation*}
          by rule {\sc E-NewCell}. \contradiction
      
        \item[Case $e = \Call{y}{m}{z}$:]
          By \eqref{eq:prog_tbase}, {\sc T-Let}, {\sc T-Call} and {\sc T-Var}
          \begin{equation*}
            \Gamma(y) = C \andalso \mtype(m, C) = \gamma \to \tau \\
            \Gamma(z) = \gamma' \andalso \gamma' \stof \gamma.
          \end{equation*}
          By $H \vdash \Gamma;L$ 
          \begin{equation*}
            \typeOf(L(y), H) \stof C.
          \end{equation*}
          Then either $\typeOf(L(y), H) = C' \stof C$ or $\typeOf(L(y)) =
          \NullType$.

          If $\typeOf(L(y), H) = \NullType$ then
          \begin{equation*}
            L(y) = \NullVal
          \end{equation*}
          and we can apply rule {\sc E-NullCall} to step to \Error.
          \contradiction

          If $\typeOf(L(y), H) = C'$
          \begin{equation*}
            L(y) = o_y \andalso H(o_y) = \Obj{C', FM}.
          \end{equation*}
          $C' \stof C$ and classes being well typed means that if $\mtype(m, C)$
          is defined then $\mbody(m, C')$ is defined aswell. Thus let
          \begin{equation*}
            \mbody(m, C) = w \to t''.
          \end{equation*}
          By rule {\sc E-Call} we can then step to $S' = H, P'$ where
          \begin{equation*}
            \begin{gathered}
              P' = Q \cup \left\{FS'|_a^\iota \right\} \andalso FS' = \xframe{L'',
              t''}^x \circ \sframe{L, t'} \\
              L_{\text{base}} =
              \begin{cases}
                \emptyset & \text{ if } a = \ocap \\
                L_0       & \text{ if } a = \nocap
              \end{cases} \\
              L'' = L_{\text{base}}[\This \mapsto L(y), w \mapsto L(z)].
            \end{gathered}
          \end{equation*}
          \contradiction

        \item[Case $e = \Put{y}{z}$:]
          By \eqref{eq:prog_tbase}, {\sc T-Let}, {\sc T-Put} and {\sc T-Var}
          \begin{equation*}
            \Gamma(y) = \CellType \andalso \Gamma(z) = \LatType.
          \end{equation*}
          Thus by $H \vdash \Gamma; L$
          \begin{equation*}
            \typeOf(L(y), H) \stof \CellType \andalso \typeOf(L(z), H) \stof
            \LatType.
          \end{equation*}
          By definition of $\typeOf$ and the structure of the type lattice this
          means that $\typeOf(L(z), H) = \LatType$ and that either
          $\typeOf(L(y), H) = \CellType$ or $\typeOf(L(y), H) = \NullType$. \\
          $\typeOf(L(z), H) = \LatType$ implies
          \begin{equation*}
            L(z) = l' \andalso l' \in \LatVals.
          \end{equation*}

          If $\typeOf(L(y), H) = \NullType$
          \begin{equation*}
            L(y) = \NullVal
          \end{equation*}
          and we can apply {\sc E-NullPut} to step to $\Error$. \contradiction

          If $\typeOf(L(y), H) = \CellType$ then 
          \begin{equation*}
            L(y) = o_y \andalso H(o_y) = \Cell{DEP, l}
          \end{equation*}
          Then we can use rule {\sc E-Put} to step to $S' = H', P'$ where 
          \begin{equation*}
            \begin{gathered}
              P' = Q \cup \left\{FS'|_a^\iota \right\} \andalso FS' = \sframe{L',
              t'} \\
              L' = L[x \mapsto L(y)] \andalso c' = \Cell{DEP, l \sqcup l'} \\
              H' = H[o_y \mapsto c']
            \end{gathered}
          \end{equation*}
          \contradiction

        \item[Case $e = \When{y}{z}{(\overline{cap}, w \Rightarrow t'')}$:]
          From \eqref{eq:prog_tbase}, {\sc T-Let}, {\sc T-When} and {\sc T-Var}
          \begin{equation*}
            \begin{gathered}
              \Gamma(y) = \CellType \andalso \Gamma(z) = \LatType \\
              \forall(\Capt{u}{u'}) \in \overline{cap}. \: \Gamma(u') = \CellType \\
              \Gamma_{\text{cells}} = [(u: \CellType): (\Capt{u}{u'}) \in
              \overline{cap}] \\
              \TypeRel{\Gamma_{\text{cells}}, w: \LatType}{\ocap}{t}{\gamma}.
            \end{gathered}
          \end{equation*}
          Combinig the first two with $H \vdash \Gamma;L$
          \begin{gather*}
            \typeOf(L(y), H) \stof \CellType \\
            \typeOf(L(z), H) \stof \LatType.
          \end{gather*}
          Similarly to the previous case we have
          \begin{equation*}
            L(z) = l' \andalso l' \in \LatVals.
          \end{equation*}
          Also
          \begin{equation*}
            \typeOf(L(y), H) = \CellType
          \end{equation*}
          or
          \begin{equation*}
            \typeOf(L(z), H) = \NullType.
          \end{equation*}

          If $\typeOf(L(y), H) = \NullType$ then
          \begin{equation*}
            L(y) = \NullVal.
          \end{equation*}
          By {\sc E-NullWhen} we can step to $\Error$. \contradiction

          If $\typeOf(L(y), H) = \CellType$
          \begin{equation*}
            L(y) = o_y  \andalso H(o_y) = \Cell{DEP, l}
          \end{equation*}
          By {\sc E-When} we can step to $S' = H', P'$ where
          \begin{equation*}
            \begin{gathered}
              P' = Q \cup \left\{FS'|_a^\iota \right\} \andalso FS' = \sframe{L',
              t'} \\
              L' = L[x \mapsto L(y)] \andalso L_{\text{env}} = [u \mapsto L(u'):
              (\Capt{u}{u'}) \in \overline{cap}] \\
              cb = ( L_{\text{env}}, w \Rightarrow t'' ) \andalso DEP' = DEP
              \cup (l', cb)^d \\
              d \text{ fresh thread id } \andalso H' = H[o_y \mapsto \Cell{DEP',
              l}].
            \end{gathered}
          \end{equation*}
          \contradiction
      \end{description}
      {\bf This concludes the case $t = \Let{x}{e}{t'}$.}
  \end{description}
  Clearly all cases for $t$ leads to a contradiction and therefore we are done.
\end{proof}








